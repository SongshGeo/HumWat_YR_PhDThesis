% Table generated by Excel2LaTeX from sheet '数据来源表'
\begin{table}[htbp]
    \centering
    \caption{多主体模型的数据源}
      \begin{tabularx}{\textwidth}{LLLLLL}
      \toprule
      数据名称  & 变量    & 使用数据的模块 & 模型使用描述 & 时空精度  & 数据来源 \\
      \midrule
      中国区域地面气象要素驱动数据集 & 气温、风速、气压、降水、辐射 & 自然模块  & 计算潜在蒸散发 & 1979年至今,0.1度分辨率,逐月 & 阳坤, 何杰. (2019).  \\
      作物物候数据 & 物候生长期 & 自然模块  & 计算作物蒸散发 & 全中国干旱/半干旱区 & Chen等人文献整理(投稿中) \\
      灌溉配额数据 & 灌溉配额及其距平 & 人类模块  & 估算用水配额 & 2014年迄今,逐月 & 黄河水利委员会 \\
      黄河流域用水数据 & 灌溉面积、灌溉用水量 & 模型规则  & 估算主体数量 & 1965年至2013年逐年 & Zhou等人2020 \\
      人口分布数据 & 人口数量  & 模型规则  & 生成主体分布位置的概率 & 仅2010年一期 & GlobPop \\
      \bottomrule
      \end{tabularx}%
    \label{ch6:tab:dataset}%
  \end{table}%
  
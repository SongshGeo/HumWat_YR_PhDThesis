第五章研究表明,黄河流域的分水制度变化会对地区用水产生不同且深远的影响。
其中,“八七”分水方案的制度改革于1987年提出,显著促进了黄河流域增加了约5.75\%的水资源使用;而1998年提出的流域统一调度制度,则解决了黄河断流的生态危机,使流域总用水量开始逐年下降。
尽管第五章通过数据分析,从社会-生态系统治理结构的角度分析制度改变的结果在时间和空间上均存在明显差异,并分析了其原因,但对明晰流域人水系统变化的驱动机制来说,仅从自上而下的路径出发仍有所不足。

省与流域管理部门的结构变化主要反映上层管理者的决策,而这些决策通常反映了下辖区域不计其数的利益相关者的诉求。因此,从具体用水者的角度出发,仍不清楚这种差异产生的原因。
另外,1998年的流域统一调度制度通过行政手段控制地表水开采达到了符合预期的效果,但尚不清楚这是否会带来新的生态问题。例如,有人认为限制地表取水将极大增加地下水的开采。
因此,本章进一步聚焦黄河流域地表用水量最多的农业部门,以模拟制度变化后不同地区灌溉用水量随时间的变化,并探讨上述两个问题。

本章建立了自下而上的多主体模型,从不同尺度(灌溉单元、用水集体、行政单元)建立反馈过程,模拟黄河流域农业灌溉用水者的响应差异。
模型包含了基于公共池塘资源演化博弈的人类模块、模拟作物需水的自然模块,并使用“基于主体的社会-生态系统框架”(ABSESpy)建立人类-自然两模块的耦合关系。
通过该耦合模型,本章可以将小麦、玉米、水稻三种主要粮食作物种植用水来源在月尺度区分为自然降水、地表水、地下水,并分析分水制度变化对各类水资源开采量的影响。

% 第五章研究表明,黄河流域的分水制度变化会深远影响流域不同地区的用水量。
% 但$1987$年提出的“八七”分水方案与限制用水的预期相悖,直至$1998$年“流域统一调度”的制度变化才成功限制用水,解决了黄河的断流问题,且尚不清楚这是否会带来新的生态问题,如替代性地增加地下水开采。
% 此外,上层决策通常反映着下辖区域不计其数利益相关者的诉求,仍需要从具体用水者的角度出发,自下而上深入分析和模拟其中机制。
% 本章进一步聚焦黄河流域地表用水量最多的农业部门,通过模拟各地区不同来源的灌溉用水如何响应制度变化,回答上述具体问题。

本章建立了自下而上的多主体模型,从不同尺度(灌溉单元、用水集体、行政单元)建立反馈过程,模拟黄河流域农业灌溉用水者对$1987$年和$1998$年两次分水制度变化的响应差异。
模型包含了基于公共池塘资源演化博弈的人类模块、模拟作物需水的自然模块,并使用“基于主体的社会-生态系统框架”(ABSESpy)建立人类-自然两模块的耦合关系。
通过该耦合模型,本章可以模拟小麦、玉米、水稻三种主要粮食作物种植用水来源,在月尺度区分为自然降水、地表水、地下水,从而分析分水制度变化对各类水资源开采量的影响。

为了更好地理解用水者对制度变化的响应,本章研究聚焦于用水者决策与环境之间的反馈过程进行建模。自下而上的多主体模型能够仿真流域内部复杂的人水关系,随着计算机算力的提升,其结合复杂环境空间分布数据的能力正不断增强,是揭示流域系统层面人水关系变化机制的理想建模方法。

在基于主体的模型(Agent-based Model, ABM)中,“主体”是对系统主要行为者(或利益相关者)的抽象,主体的交互行为都是在一个由斑块组成的“环境”中进行的。
多主体建模的核心在于设计主体之间的交互规则、主体与环境交互的规则、以及环境变化的方式。本节将使用多主体模型分析黄河农业灌溉主体对流域水资源分配制度的决策过程,并从“主体规则的设计”、“模块构建与耦合”、“数据输入与分析”三个方面介绍模型构建方法。

\subsection{主体规则的设计}

本小节将分别从个体尺度、地区尺度、流域尺度介绍多主体模型的规则设计,三者的关系如图\ref{ch6:fig:framework}所示:
(1)个体尺度的灌溉主体是进行农业生产的基本单位,基于环境信息和制度条件做出具体、实际的用水决策;
(2)地区是灌溉主体接受制度规制的主要尺度,流域系统的制度变化通过此级规则落实到主体间交互之中;
(3)流域尺度则约束了模型允许的环境背景,例如主体分布、水文气候条件、主体交互网络。
% 模型的主要目的是分析流域水资源分配制度如何影响农业灌溉主体的用水来源:用多少地下水?
% 将小麦、玉米、水稻三种主要粮食作物种植用水来源在月尺度区分为自然降水、地表水、地下水,并分析了地表分水制度变化对地下水开采量的影响。

\begin{figure}[htb]
    \centering
    \includegraphics[width=\textwidth]{img/ch6/ch6_framework.png}
    \caption{多主体模型设计框架}\label{ch6:fig:framework}
\end{figure}

\subsubsection*{个体尺度}

对单个灌溉主体来说,灌溉引水决策的本质是对作物在天然降水之外的水亏缺进行人为补给。
水亏缺是考虑灌溉损失后的作物蒸散发与实际降水量之间的差,作物蒸散发则采用国际粮农组织(FAO)推荐的单作物系数法进行计算,由潜在蒸散发和作物蒸散发系数计算得到。

\begin{equation}
    \label{ch6:eq:deficits}
    W_{deficits} = \frac{ET_c - P}{k_{loss}}
\end{equation}

\begin{equation}
    \label{ch6:eq:etc}
    ET_c = ET_{0} * Kc
\end{equation}

其中潜在蒸散发和降水根据主体所处的位置直接从自然模块调用相应属性,作物蒸散发系数$Kc$的同样通过查阅采用国际粮农组织($FAO56$)推荐的作物表得到(表\ref{ch6:tab:crops})。


% Table generated by Excel2LaTeX from sheet '作物物候表'
\begin{table}[htbp]
    \centering
    \caption{模拟黄河流域主要粮食作物物候表}
      \begin{tabularx}{\textwidth}{p{1.5cm} LLLLL p{1cm} p{1cm} p{1cm}}
      \toprule
      作物名称  & \multicolumn{1}{l}{播种期} & \multicolumn{1}{l}{萌发期} & \multicolumn{1}{l}{生长期} & \multicolumn{1}{l}{中期} & \multicolumn{1}{l}{后期} & \multicolumn{1}{l}{$Kc_{ini}$} & \multicolumn{1}{l}{$Kc_{mid}$} & \multicolumn{1}{l}{$Kc_{end}$} \\
      \midrule
      小麦    & 4月5日  & 4月25日 & 5月20日 & 7月19日 & 8月18日 & 0.15  & 1.15  & 0.30  \\
      水稻    & 4月15日 & 5月15日 & 6月14日 & 7月14日 & 8月13日 & 1.00  & 1.20  & 0.70  \\
      玉米    & 4月19日 & 5月9日  & 6月13日 & 7月23日 & 8月22日 & 0.15  & 1.20  & 0.50  \\
      \bottomrule
      \end{tabularx}\label{ch6:tab:crops}%
\end{table}%


但在农民生产实际中,基本不可能做到完全按照作物的水亏缺决定灌溉量,而是根据经验,既往文献中通常在水亏缺基础上额外使用一个系数$\alpha$表示实际决策中的灌溉量,该系数估计了农民对作物水亏缺的经验认知与节水灌溉的影响,即实际灌溉用水量$WU$可表示为:

\begin{equation}
    \label{ch6:eq:WU}
    WU = W_{deficits} * \alpha
\end{equation}

但在本章研究中,由于时间跨度较长($1980 \sim 2010$),区域灌溉的节水升级改造是响应制度变化的重要措施,系数$\alpha$一定是剧烈变化的。
因此,本章研究根据用水量$WU$的年际统计数据将尺度下降到月尺度,$WU_i$代表农民在作物生长季在第$i$个月的实际的用水需求,而经验系数$\alpha$的值则可以作为反映灌溉节水的指标,则有:

\begin{equation}
    \label{ch6:eq:WUi}
    WU_i = WU * \frac{W_{deficits, i}}{\sum_{i} W_{deficits, i}}
\end{equation}

其中$i$为模型的每个时间步长单位(月),因此在每步模拟中,可将每个灌溉主体的作物生长用水来源$W$分为降水来源$W_p$、地表水来源$W_s$、地下水来源$W_g$三部分,具体用水决策及模型计算顺序如下:
(1)根据降水和作物蒸散感知水亏缺$W_{deficits}$,如果$W_{deficits}=0$,说明降水已经满足了需求,则作物生长水源$W$就完全由降水提供,$W=W_{p}$;
(2)如果降水无法满足作物生长需求,就根据公式\ref{ch6:eq:WUi}评估当月的水需求$WU$,并设定降水来源$W_{p} = P$;
(3)对于需求$WU > 0$,假定农民首先考虑从地表水获得,根据自己的官方配额$Q_{0}$和超采的配额$Q_{1}$决定多取用多少配额$Q = Q_{0} + Q_{1}$,若$Q > WU$,则地表水使用$W_s = WU$,否则$W_s = Q$;
(4)最后估计地表水开采量$W_g$,所有降水和地表水仍不被满足的用水需求都被认为从地下水获取,即$W_g = WU - W_s - W_p$。

\subsubsection*{地区尺度}

黄河流域的配额制度通常仅细致到地级市范围,如何在地区尺度分配该地区的可供分配的水资源配额取决于主体之间的自组织。
本章研究中假设地区尺度完全根据作物的水亏缺$WU_{deficits}$分配灌溉水量,作为完全的“按需分配”,这是节约配额的理论最优(也是理论最公平的)情景,这也意味着现实中的资源供需匹配不会比本章研究的模拟结果更好。
在数学上,假设总地表水配额为$Q_{C}$的地区$C$中包含$n$个主体,则主体$j$的配额$Q_j$为:

\begin{equation}
    \label{ch6:eq:quota}
    Q_j = Q_{C} * \frac{W_{deficits, j}}{\sum_{j \in C} W_{deficits, j}}
\end{equation}

由于第五章中介绍的分水制度差异及其变化,地区$C$的配额并非固定不变的,在$1987 \sim 1998$年间水资源配额存在可浮动的特性,使用$Q_{\min}$和$Q_{\max}$两个属性进行表达,允许用水个体的实际地表取用水在其间浮动。
$Q_{\max}$由黄河流域各省在$1983$年自主上报的水资源配额需求基础上降尺度到地区和月份来估算(详见表\ref{ch5:tab:quota}),代表自主取水时期满足该地区主体用水需求的能力上限,通常这是官方建议配额$Q_{\min}$的近两倍。
因此,关于$Q_j$的判定规则如公式\ref{ch6:eq:which_quota}所示,与模型当前模拟的年份$yr$有关:

\begin{equation}
    \label{ch6:eq:which_quota}
    Q_j =
    \begin{cases}
        Q_{\max} & \text{when } yr < 1987 \\
        Q_{\in} \leq Q_j \leq Q_{\max} & \text{when } 1987 \leq yr \leq 1998 \\
        Q_{\min} & \text{when } yr > 1998
    \end{cases}
\end{equation}

在$1987 \leq yr \leq 1998$时期,每个主体在计算配额$Q$时如果有较大的用水需求,其$Q_{0} = Q_{\min}$会保证被优先满足,因为这是分水方案中明确保证的合法配额。
由于这个配额没有被强制执行,该主体拥有一定的灵活性索取获得超越官方建议配额$Q_{\min}$的额外配额$Q_{1}$,即超出制度规定但不违背需求上限的“违背制度”部分,该部分配额的值与主体的社会属性有关,详见\ref{ch6:sec:society}\refname{ch6:sec:society}模块的介绍。
但$Q = Q_{0} + Q_{1}$不会超过该主体能被分配到的最大合理配额$Q_{\max}$。

\subsubsection*{流域尺度}

在流域尺度,需要决定主体生成的数量、位置、以及相互作用网络。
气候、水文、人口的空间数据精度同化至$0.1$度,即每个空间栅格的范围大约为$A_{cell} = 121{km}^2$,这也是主体进行灌溉活动的基本单元。
本章研究的时间精度是月,考虑作物的生长季主要在$4$月到$8$月之间(见表\ref{ch6:tab:crops}),因此根据统计数据中三类主要作物的灌溉面积和模拟的空间精度来估计每个地级市的主体数量。

\begin{equation}
    n_{C} = \sum_{k=R, M, W}\frac{A_{C, k}}{A_{cell}}
\end{equation}

其中$k = R, M, W$分别代表本研究考虑的三种作物:水稻、玉米和小麦,$A_{C, k}$为该种作物在$C$市的灌溉面积,$A_{cell}$则为每个斑块(栅格)代表的面积。
由于各地级市的面积每年都会变化,主体数量也会每年使用新的统计数据来进行更新。
每个栅格只能同时存在一个主体,在保证主体数量与灌溉面积一致的基础上,主体生成的位置是随机的,概率遵循以人口空间分布的加权平均。

主体决策之间会互相影响,每年的主体数量和位置更新后,还需要预设主体之间的交互关系。
以每个主体为一个节点,存在相互影响的主体间存在连边,那么整个流域可形成一个复杂网络,该网络可包括三种潜在连边:
(1)同一地级市的灌溉主体之间
(2)不同地级市的灌溉主体之间
(3)不同省区的灌溉主体间
由于中国农业基本单元的封闭性,本研究假设不同省之间的主体之间不会互相影响(即不会出现跨省互相影响决策的情况),在同一地级市之内和地级市之间的连边遵循具有固定连边概率的ER随机图(Erdős–Rényi random graphs)构建算法(Erdos and Renyi, 1959),即随机图是一个由$C * n$个节点组成的图,其中同一地级市每对节点连接的概率为$p_n$,不同地级市之间则每对节点的连接概率为$p_C$。
因此一旦概率对$p_n$,$p_C$给定,全流域范围内所有灌溉主体之间可能互相影响的网络拓扑关系就被确定了。

\subsection{模块构建与耦合}

在上述规则构架下,本研究包括两个关键模块:
(1)人类模块:流域内的灌溉主体在给定水文气候条件、给定配额上限/下限的情况下如何决定其用水来源及用水量;
(2)自然模块:水文气候条件、流域主体的作物选择和灌溉策略如何影响作物蒸散发。
本节先分别介绍两个模块的运行过程,最后介绍二者间的耦合反馈关系。

\subsubsection*{人类用模块}\label{ch6:sec:society}

在人类模块中,本研究使用演化公共池塘博弈的分析框架对主体面对配额限制的决策响应进行分析,使用遵循配额($C$)与违背配额($D$)来概括主体在面对灵活水配额制度($1987$年至$1998$年间)的基本决策。
博弈论(Game Theory)是研究决策者之间战略互动的数学模型,是研究具有竞争性质现象的理论和方法,对于公共水配额这种竞争性、排他性都很强的的“公共池塘资源”,符合演化公共物品博弈的模型框架\cite{ostrom2009, traulsen2010}。
演化公共池塘博弈框架又称多代理人囚徒困境博弈,本研究中主体$i$选择是否遵循限制地表用水的配额制度($S_i = 1$表示合作$C$,$S_i=0$代表决策$D$)。
如果某个主体同意严格遵循配额制度$Q = Q_{\min}$,那么他将损失潜在收益$c$中的一部分。
如果每个人都遵循配额(选择合作$C$),那么共同投资的总额可能会变成原来的$r$倍,并在所有参与的投资者中平分。
如果有$N$个代理人参与,那么每个人的博弈收入$E$如由两部分组成(见公式\ref{ch6:eq:game}):$-c*S_i$ 代表单个主体对集体贡献的成本,而 $\sum_{j=1}^N S_j$则代表共同利益池给每个贡献的主体分到的利益。

\begin{equation}
    \label{ch6:eq:game}
    E=-c * S_i+\frac{r c}{N} \sum_{j=1}^N S_j=-\left(c-\frac{r}{N}\right) S_i+\frac{r c}{N} \sum_{j \neq 1}^N S_j
\end{equation}

在本研究中,黄河分水制度对于遵循配额的主体没有任何额外的经济补贴(这被视为一种行政上的义务行为),也没有正式的经济上的行政处罚,因此公共池塘的收益系数$r = 1$。
此外,因为黄河流域灌区的地下水取水的成本普遍高于地表取水,潜在成本$c$与地下水开采量$W_g$有关,使用$c \propto W_g = k_e * W_g$表示,本研究中为直观考虑采取经济系数$k_e = 0.3$,即意味着地下水水费每单位比贵$0.3$单位。
这个系数的取值实际上不会影响模型表现,因为本研究后续影响主体决策的是主体间的收入差距,而所有主体的潜在收益$E$与地下水取水量都是同样的线性关系,决定主体收益差距仅由地下水开采量$W_g$的差距决定。
因此理论上可预期的是,在响应制度的经济驱动因素上,遵循配额会产生大量地下水开采成本的地区将拥有更大的动机违背配额。

作为没有任何补贴的制度,影响主体是否遵循配额的另一个主要驱动因素是社会规范,因为即便没有经济惩罚,违背制度也是不被提倡的。
多元理性理论是一个理解个人决策与社会规范影响之间关系的经典框架,指出由“个体/集体主义”与“注重/轻视规范”两个维度,就可以预测不同文化背景下遵循社会规范的倾向\cite{verweij2015},因此可以对复杂的政策问题进行结构化的诊断,因此该方法框架已被应用于解释水治理的成功与失败。
参考 Castilla-Rho 等人基于多元理性(Plural rationality theory PRA)的社会理论的一系列研究,可以使用两个简单的参数$b$和$v$来测量主体遵循制度规范的动因。
其中$b\in[0, 1]$表征某主体“注重/轻视规范”的程度,在模型中也是该主体突破规定配额,做出决策$D$的概率。
$v \in [0, 1]$则表征主体“个体/集体主义”的程度,该参数越大代表主体越重视与他人保持一致,在模型中是对该主体社交网络中违规行为进行负面评判的概率。
考虑到人们不愿意主动得罪别人,也不愿意丧失名誉,这种评价将成为计算社会得分的基础,采用在福利经济学中常用的Cobb-Douglas函数的形式进行表达:
% Cobb-Douglas函数(https://inomics.com/terms/cobb-douglas-production-function-1456726)
% [Happy Planet Index](https://happyplanetindex.org/) 
% 人类发展指数(http://hdr.undp.org/en)
\begin{equation}
    S = {(grid)}^m * {(1 - group)}^n
    \label{ch6:eq:society}
\end{equation}

其中 $m$ 是一个主体发现其它人违规超配额用水的次数,$n$ 则是他被他人发现超配额用水的次数。
即$grid^m$ 这部分表达的是主体因周围人违背制度规范而感到不舒服;$group^n$ 这部分表达的是一个主体违背制度规范而被周围人指指点点而感到不舒服。

在演化博弈中,决策还是可演化的(Evolutionary),因为一些反应因为符合社会规范而受到奖励,而另一些反应则因为背离社会规范而受到惩罚。
人类模块在每一个时间步的最后将计算所有主体的博弈结果——$0 \sim 1$标准化后的经济得分($E$)与社会得分($S$),并将两者平均$Score = \frac{E + S}{2}$作为主体的最终得分。
在社会网络中,表现最成功的主体决策以一定概率$p_{l}$被他的朋友们学习,学习后的主体使用被学习对象的$b$和$v$参数代替学习者原本的参数。

综上所述,在人类模块的每次模拟中,每个主体都将遵循图\ref{ch6:fig:society}所示的流程图:

\begin{figure}[htb]
    \centering
    \includegraphics[width=\textwidth]{hello}
    \caption{绘图测试}\label{ch6:fig:society}
\end{figure}

\subsection{自然模块}

在自然模块中,本研究使用 FAO56 推荐的算法计算潜在蒸散发$ET_0$、作物蒸散系数$Kc$、以及作物蒸散发$ET_c$:

\begin{equation}
    ET_c = ET_0 * Kc * Ks
\end{equation}

其中$Ks$是水应力系数,主要与土壤质地有关。由于在本研究中仅考虑三种主要粮食作物,且主要分布在灌溉条件良好的土壤上,对此系数的空间分异不再作进一步考虑,使用系数$Ks = 0.3$进行估计。  % TODO chen 2023

潜在蒸散发$ET_0$的计算方法使用Penman-Monteith方法,式\ref{ch6:eq:pm}:

\begin{equation}
    \label{ch6:eq:pm}
    PET = \frac{0.408 \Delta (R_{n}-G)+\gamma \frac{900}{T+273}
    (e_s-e_a) u_2}{\Delta+\gamma(1+0.34 u_2)}
\end{equation}

其中$Rn$为作物表面上的净辐射,$MJ/(m^2 day)$;$G$为土壤热通量,$MJ/(m^2 day)$;$T$为$2$米高处日平均气温$^\circ C$;$u2$为$2$米高处的风速,$m/s$;$es$为饱合水汽压,$kPa$;$ea$为实际水汽压$kPa$;$es - ea$为饱和水汽压差,$kPa$;为饱和水汽压曲线的斜率;$r$为湿度计常数,$kPa/^\circ C$。
由于计算的时间尺度(月),土壤热通量可以忽略。根据其它数据可得性,对经验公式\ref{ch6:eq:pm}中的饱和水汽压$es$使用气温数据进行估算:

\begin{equation}
    e^0(T) = 0.6108 * \exp \frac{17.27 T}{T + 237.3}
\end{equation}

\begin{equation}
    es = \frac{e^0(T_{\max}) + e^0(T_{\min})}{2}
\end{equation}

作物生长期间的作物系数$Kc$的曲线是用三个阶段的作物系数来估计的:初期($Kc_{ini}$)、中期($Kc_{mid}$)和后期($Kc_{end}$)。
作物系数的各转折点以表\ref{ch6:tab:crops}中的数据为依据进行定位。
$Kc_{ini}$的计算方法如下\cite{allen1998}。

\begin{equation}
    \label{ch6:eq:kcini}
    K_{\text {c, ini}}=\frac{\mathrm{TEW}-(\mathrm{TEW}-\mathrm{REW}) \exp \left[-\left(\mathrm{t}_w-\mathrm{t}_1\right) E_{so}\left(1+\frac{\mathrm{REW}}{\mathrm{TEW}-\mathrm{REW}}\right)\right]}{\mathrm{t}_w E T_{\mathrm{o}}}
\end{equation}

其中,$TEW$为总的可蒸发水量($mm$);$REW$为易蒸发水量($mm$);$tw$为初始时期湿润事件之间的平均时间(天);$t_1$为易蒸发水量干燥完成所需的时间(天);$Eso$为由于低含水层湿润土壤或由于以前干燥时期储存在表层的热量而增加的蒸发潜力($mm/d$)。其中$E_{so} = 1.15 * ET_0$, $t_w = \frac{L_{ini}}{n_w + 0.5}$, $t_1 = REW / E_{so}$。

当$t_1 < t_w$的时候,$K_{\text {c, ini}} = 1.15$,否则使用\ref{ch6:eq:kcini}进行估计,且用来$TEW_{cor}$和$REW_{cor}$替代$TEW$和$REW$。

对于渗透深度$depth < 10mm$的所有土壤质地,有:

\begin{equation}
    TEW_{cor} = 10mm \\
    REW_{cor} = \min(\max(2.5, 6 / ET_0^{0.5}, 7))
\end{equation}

对于渗透深度$depth >= 40mm$的所有土壤质地,有:

\begin{equation}
    TEW_{cor} = \min(15, 7 {(ET_0)}^{0.5}) \\
    REW_{cor} = \min(6, TEW_{cor} - 0.01)
\end{equation}

适用于渗透深度$depth >= 40mm$的中、细土质,有:

\begin{equation}
    TEW_{cor} = \min(28, 13{(ET_0)}^{0.5}) \\
    REW_{cor} = \min(9, TEW_{cor} - 0.01)
\end{equation}

对于$depth \in (10mm, 40mm)$,有:
\begin{equation}
    K_{c ini} = K_{c ini(I<=10mm)} + \frac{I-10}{40-10}[K_{c ini(I>10mm)} - K_{c ini(I<=10mm)}]
\end{equation}

其中,$K_{c ini(I≤10mm)}$是指入渗深度 $depth<=10 mm$的所有土壤质地的$K_{c ini}$值;$K_{c ini(I > 40 mm)}$是指入渗深度$depth>40 mm$的粗或中/细土壤质地的$K_{c ini}$值。

$K_{c mid}$ 可以由下式估计:
\begin{equation}
    K_{c mid} = K_{c mid(Tab)} + [0.04(u_2 - 2) -0.004(RH_{\min} - 45)]{(\frac{h}{3})}^{0.3}
\end{equation}

$K_{c end}$ 可以由下式估计:
\begin{equation}
    K_{c end} = K_{c end(Tab)} + [0.04(u_2 - 2) -0.004(RH_{\min} - 45)]{(\frac{h}{3})}^{0.3}
\end{equation}

其中 $K_{c mid(Tab)}$ 和 $K_{c end(Tab)}$ 均是 FAO 作物表中推荐的估计值(见表\ref{ch6:tab:crops})。
完成各阶段作物系数计算后,参考作物的生长周期,逐日还原出当年的作物系数曲线,并通过按月平均匹配到本研究的尺度。

\subsubsection*{人地模块的耦合}

本章研究的模型在基于地理数据的多主体建模框架“ABSESpy”基础上搭建,这是一个多主体建模开源框架软件包。
该框架使用真实的地理空间数据集来构建人工社会生态系统,同时充分考虑人类行为因素。
它支持使用地理空间数据,对具有认知、社会影响和反应等行为的主体进行建模,用户可以专注于分别实现人类行为的逻辑和自然过程的仿真,该框架使得两者之间能够轻松地访问和修改变量值。
该框架可以调度任意数量的自然模块和社会模块,并使它们之间松散耦合。
例如在本章研究中,人类用水模块通过作物种植来改变自然模块的作物地表覆被属性,可以简单地使用一句\lstlisting{alter_nature}命令来完成;自然模块则通过降水、地表蒸散发的变化来影响人类用水模块中的决策环境,主体可以根据其所在位置调用相关的自然属性。

\subsection{数据处理}

\subsubsection*{输入数据}

作物物候、气象、灌溉强度、灌溉面积、用水配额等

模型能够模拟各省份绿水(降水+土壤水)/蓝水(地表/地下水)的使用情况

本章研究使用的数据清单如表所示

\subsubsection*{参数选取}

下表\ref{ch6:tab:params}总结了模型中使用到的主要参数:

这个社会效用函数在过往的研究中也有使用,如 Castilla-Rho et al., (2017, 2019) [@castilla-rho2017, @castilla-rho2019]. 如果没有指定这两个参数,我们将参考 [@castilla-rho2017] 的研究从数据库中直接为这两个参数指定值。
这个社会效用函数的参数(Grid 和 Group)可以用各个地区的文化差异来进行量化。

% \subsubsection*{敏感性检验}

\subsubsection*{趋势分析方法}


\subsubsection*{建模和分析工具}

所有模型使用Python3.11进行构建。

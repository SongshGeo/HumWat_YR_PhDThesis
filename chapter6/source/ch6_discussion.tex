\section{自下而上的制度变化响应机制}

在黄河流域,水资源的配额分配存在一定时空错配的问题,大型灌区是使用水的大户,需要进一步深入研究并制定相应的政策和措施,以保障不同地区的作物水资源供应(图\ref{ch6:fig:matches})。
还需要注意到在不同季节和地区,黄河流域的水资源使用情况存在较大的差异,例如在生长季期间,河套灌区的水资源利用效率偏低,大量的水资源配额供给相对作物实际需求过剩,其他地区则存在供需不平衡的情况。
可能的原因是农业生产需水是在时间和空间上高度集中的,但来自黄河干流的水资源供给需要遵从统一调度,有诸多方面需要协调和权衡,无法优先照顾农业生产的分配,本研究也仅考虑了三种粮食作物的农业灌溉水需求。
此外,偏小的灌溉需求得到的配额也过小,偏大的灌溉需求得到的配额则有所富余,这说明灌溉农业具有引水上的“规模效应”,因此“大引大排”的模式在河套灌区为代表的传统灌溉农业区主导,一些零散分布的灌溉区很难得到如此稳定、集中的配额供给。
正如在本章方法部分中所强调的,本研究在所有需要估计的地方都使用了最优分配的原则,因此实际生产生活中配额与需求的时空错配肯定更为严峻。
这表明在制定水资源管理政策和措施时,需要考虑季节性的变化和区域性的差异,采取针对性的措施,以保障黄河流域的水资源可持续利用和发展。

深入研究分水制度变化对地表水、地下水使用量的影响机制,为制定更有效的水资源管理政策提供科学依据。
在$1980 \sim 2010$三十年间,降水供给的比例持续增加,地表水和地下水同时呈现下降趋势,说明黄河流域各省市的主要粮食作物灌溉在响应分水制度时以节水改革为主,而非简单地用地下水替代被限制的地表水。
河南、山东、宁夏这些黄河用水大省选择遵循配额制度的比例在$10$年后仍显著低于制度颁布之初,这与第五章的数据分析结果相一致,用水大省在$1987$年与$1998$年之间违背配额约束是尚未深入节水改革时,保证农业增长的必然选择。
各省区面对配额制度的不同响应与各地的经济发展本底、地下水资源的分布有关,例如黄河上游的宁蒙灌区则依赖于大引大排,在节水农业推广上多采取鼓励种植节水作物,分水制度颁布初期既超配额采用地表水,又考虑了地下水替代,但在$1998$年后迅速推动节水转型。
中游对配额的遵循程度较好,一定程度上因为难以,因此其地下水抽水量常年稳步提升。
黄河下游因地下水盐碱化问题很难有效利用地下水,经济上的优势使其成为最先推动节水转型的区域,目前节水灌溉率已接近$80\%$。
因此,在制定水资源管理政策和措施时,需要充分考虑不同地区的特点和实际情况,并针对性地制定相应的政策和措施。

\section{对流域水资源治理的启示}

黄河流域的水资源分配存在着一定的问题和差异,这对当地的农业生产和经济发展产生了不良影响。
通过分析不同省份的蓝水和绿水使用情况,可以更好地理解不同省份之间的差异和变化,从而制定更好的水资源管理政策和措施。
同时,对于分水制度变化对地下水使用量的影响机制需要进一步深入研究,为制定更有效的水资源管理政策提供科学依据。
综合利用各种水源,提高水资源的利用效率,调整和优化水资源的分配和使用,以满足不同地区的作物生长需求,是保障黄河流域水资源可持续利用的重要措施。

% % 人水匹配 comment mydoc.docx
% 作为实现匹配的一个重要途径,制度分析认为,权利、规则和决策等制度可以引起或解决人与环境互动中的问题。因此,建立匹配的制度可以引导系统功能向理想的结果发展。
% 研究表明,制度匹配在很多方面有利于人类水系统的可持续性。
% 例如,设立流域管理机构可以有效避免水事纠纷,建立水权转换制度有利于资源的有效配置,加强监管可以遏制河流污染。
% 因此,许多大流域的综合治理是建立一套制度的尝试,它以实现人水匹配的一系列制度为核心,包括与之密切相关的文化和技术要素,从而保证流域的可持续发展。
% 这样一个体系的建立庞大而重要,需要自然科学和社会科学的交叉,解耦和理解人水系统中复杂的反馈回路,从而通过制度分析实现人水匹配(图1,PATH 1和路径2)。

% 制度分析往往是在特定背景下进行的,因此,人水关系的动态变化使得宏大的人水系统匹配更加复杂。
% 首先,动态变化是人类社会系统和自然水文系统的共同特征,但只有当关键变量达到临界点时,系统功能的稳态转变才会发生。
% 需要匹配的关键系统功能往往是由流域可持续性的需要决定的,这为理解复杂系统动态的观点提供了价值判断(图1,路径3)。
% 另一方面,当一些可能导致系统稳态转变的变化发生时,也需要重新审视如何在新的人水关系下实现匹配。因此,这些超过阈值的变化可能成为成功过渡到人水匹配的关键驱动力(图1,PATH 4)。
% 然而,实现上述动态匹配还需要对人水系统有更高层次的理解,因为这些动态变化为系统解耦提供了新的科学难题,如变化的弹性、变化之间的因果关系、稳态转化及其级联效应等(图1,PATH 5)。
% 只有通过深度解耦和对反馈循环的理解,才能对人类与水系统的动态进行预测(图1,PATH 6),促进机构分析和过渡匹配(图1,路径1和路径3)。
% 一般来说,人-水系统的反馈循环和动态变化是人-水匹配的基础。由于三者是相互联系的,面对不断变化的流域系统,人水匹配需要动态实现。
\section{自下而上的制度响应机制}

\subsection{灌溉需求与分水配额失配的原因}

在三种主要粮食作物(玉米、水稻、小麦)的生长季期,黄河流域的水资源使用情况存在较大的差异,例如河套灌区的水资源利用效率偏低,大量的水资源配额供给相对作物实际需求过剩,其他地区则存在供需不平衡的情况(图\ref{ch6:fig:matches})。
本章的研究结果表明黄河流域水资源配额存在时空错配的问题,需要进一步深入研究并制定相应的政策和措施,以保障不同地区的作物水资源供应。
农业生产需水是在时间和空间上的高度集中是可能的原因之一,来自黄河干流的水资源供给需要遵从统一调度,有诸多方面(如防洪调沙)需要协调和权衡,无法优先照顾农业生产的分配\cite{hu2015}。
此外,配额富余多出现在灌溉需求偏大的大型灌区,偏小的灌溉需求得到的配额也过小,这说明灌溉农业具有引水上的“规模效应”,因此“大引大排”的模式在河套灌区为代表的传统灌溉农业区主导,一些零散分布的灌溉区可能很难得到如此稳定、集中的配额供给\cite{xiong2021a}。
此外,本研究也仅考虑了三种粮食作物的农业灌溉水需求,这三种粮食作物通常只有在灌区才被供水者重点考虑,在更小型的小农种植单元上,产出更高的经济作物通常能优先得到更多的配额供给。
本章的研究方法也必然存在一定局限,评估的灌溉水需求并非绝对准确,但正如在本章方法部分中所强调的,本研究在所有需要估计的地方都使用了最优分配的原则,因此实际生产生活中配额与需求的时空错配肯定更为严峻。
这表明在制定水资源管理政策和措施时,需要考虑季节性的变化和区域性的差异,采取针对性的措施,以保障黄河流域的水资源可持续利用和发展。


\subsection{灌溉用水来源响应分水制度变化的机制}

在$1980$年至$2010$年间的黄河流域各省粮食作物灌溉用水中,降水供给的比例持续增加,地表水和地下水占比同时呈现下降趋势,说明黄河流域各省市的主要粮食作物灌溉在响应分水制度时以节水改革为主,而非简单地用地下水替代被限制的地表水。
限制来自黄河干流的地表水是分水制度作用的直接方式,但正如第四章已论述的:1987年至1987年之间不仅没有成功限制地表水使用,反而在全流域刺激了约$5.75\%$的地表用水,且对河南、山东、宁夏这些黄河用水大省尤其显著。
% 本章的模拟结果指出,多数省份在“演化公共池塘资源博弈”中选择遵循配额制度决策的比例在此时期显著低于制度颁布之初(图\ref{ch6:tab:compliacne}),这佐证了第五章中数据分析的结果。
此外,对1987、1987年两次制度变化时间节点前、中、后灌溉用水量的分布也可以发现(图\ref{ch6:fig:sources_change}),地表水是三种用水来源中变化最为明显,在1987年与1987年之间,用水量前$50\%$的主体用水量明显增长,但用水量偏小的另一半主体则没有明显的地表水使用量变化。

各地区对配额制度变化的不同响应可能与各地的经济发展水平、生态环境本底都有关,例如黄河上游宁夏、内蒙的灌区天然降水量偏小且经济发展水平相对落后,一向依赖于黄河水的大引大排,同样面对缩减的配额,这里的灌溉主体面对的转型压力大于下游等经济发达、雨水稳定的地区\cite{xiong2021a}。
因此,在节水转型过程中,上游的灌溉农业在分水制度颁布初期很难做到迅速的转型,采用了既超配额采用地表水,又适当采用地下水作替代的策略,但在制度转型的中期(约$1991$至$1994$年)前后也推动了节水转型,以推广、鼓励种植节水作物为主\cite{yin2021}。
不同于配额紧俏的上游和下游,黄河中游地区对配额的遵循程度一向较好,参考前人研究一定程度上因为引水系统难以适应新的水情,而地表水使用受限也可以解释其地下水抽水量在研究关注时段内常年呈提升趋势。
黄河下游具备经济发展水平上的优势,成为最先推动节水转型的区域,目前节水灌溉率已接近$80\%$,没有使用地下水开采作为响应制度的策略可能是因为地下水盐碱化问题导致黄河下游很难有效利用地下水\cite{huangronghan1962}。


\section{对流域制度情景建模的启示}

\subsection{模拟黄河流域分水制度的意义}

本章研究结合灌溉主体的决策,分析了黄河流域不同地区各类灌溉水源的使用情况,加深了理解不同省份之间的差异和变化,为流域治理的分析和评价提供了独特的思路和方法。
通过深入分析地表水、地下水使用量对分水制度变化的响应机制,可以为制定更有效的水资源管理政策提供科学依据,对理解和提升流域社会\textendash{}生态系统的制度弹性有帮助。
本章分析结果表明,黄河流域的配额与粮食作物灌溉需求之间存在失配问题,但黄河流域各省市的主要粮食作物灌溉在响应分水制度时以节水改革为主,而非简单地用地下水替代被限制的地表水,因此在制度变化之初尚未深入节水改革时,违背配额约束可能是保证农业产出的主要响应策略。
可见,未来在制定黄河水资源管理政策和措施时,需要充分考虑不同地区的特点和实际情况,并针对性地制定相应的政策和措施。
此外,综合利用各种水源、提高水资源的利用效率、调整和优化水资源的分配和使用,依然是保障黄河流域水资源可持续利用的重要措施。
在流域系统治理的背景下,深入研究主体决策对各类制度变化的响应机制对明晰人\textendash{}水关系演变机制至关重要,从自下而上的视角为制定更有效的治理政策提供了科学依据。
本研究也存在一定局限,例如黄河流域下游有冬小麦种植,但由于在数据源中未做区分而难以考虑到模型之中,未来可以考虑结合更精细的作物类型和物候数据进行更进一步的研究。

\subsection{建立流域治理多主体模型的意义}

制度分析往往依赖于特定背景,动态变化是人类社会系统和自然水文系统的共同特征,使预测制度对流域系统的影响变得困难,而制度通常只能造福部分利益相关者\cite{epstein2015}。
作为实现人\textendash{}水关系匹配的重要途径,制度分析认为权利、规则和决策等制度可以引起或解决人与环境互动中的问题,因此建立适当的制度可以引导流域系统向理想结果发展\cite{wang2019c}。
但是,只有当关键变量达到临界点时才会发生系统功能的稳态转换,自下而上地模拟利益相关者的涌现现象,更有助于预测制度表现是否符合预期,而自上而下的评估通常难以获得这种预测\cite{reyers2018}。
多主体模型是自下而上进行建模的方法,在系统治理领域有大量应用,其优势在于建立模型后还可以设计制度实验情景,探索利益相关者存在不同响应模式、引导系统走上不同发展路径的可能性。
人水系统的反馈循环和动态变化是人-水匹配的基础,面对不断变化的流域系统,人水匹配需要动态实现,构建流域人与自然耦合的多主体模型有助于在不断变化的流域环境中进行适应性治理。

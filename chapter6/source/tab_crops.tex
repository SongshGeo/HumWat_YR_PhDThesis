
% Table generated by Excel2LaTeX from sheet '作物物候表'
\begin{table}[htbp]
    % \centering
    \caption{黄河流域三种主要粮食作物的物候表$^a$}
      \begin{tabularx}{\textwidth}{p{0.8cm} LLLLL p{0.8cm} p{0.8cm} p{0.8cm}}
      \toprule
      作物  & \multicolumn{1}{l}{播种期} & \multicolumn{1}{l}{萌发期} & \multicolumn{1}{l}{生长期} & \multicolumn{1}{l}{中期} & \multicolumn{1}{l}{后期} & \multicolumn{1}{l}{$Kc_{ini}$} & \multicolumn{1}{l}{$Kc_{mid}$} & \multicolumn{1}{l}{$Kc_{end}$} \\
      \midrule
      小麦    & 4月5日  & 4月25日 & 5月20日 & 7月19日 & 8月18日 & 0.15  & 1.15  & 0.30  \\
      水稻    & 4月15日 & 5月15日 & 6月14日 & 7月14日 & 8月13日 & 1.00  & 1.20  & 0.70  \\
      玉米    & 4月19日 & 5月9日  & 6月13日 & 7月23日 & 8月22日 & 0.15  & 1.20  & 0.50  \\
      \bottomrule
    \end{tabularx}\label{ch6:tab:crops}%
    \footnotesize\\
    $a$ 本模型不考虑冬小麦的种植。
\end{table}%

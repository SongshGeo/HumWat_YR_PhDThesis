本章结合包括了三种主要粮食作物(水稻、玉米、小麦)灌溉水需求的自然模块、以及1987年``八七''分水制度与1998年流域统一调度两次水资源分配制度的人类模块,构建了黄河流域社会-生态系统的多主体模型。
本章结果指出,黄河流域的水资源分配制度所规定的地表水额度与三种主要粮食作物(水稻、玉米、小麦)的灌溉需求之间存在时空不匹配。
利用模型进一步区分黄河流域粮食作物种植的用水来源,发现限制地表取水的水资源分配制度限制并没有直接导致大规模的地下水替代性开采,农业节水转型是响应制度变化的主要方式。

本章研究表明深入理解人-水关系变化的机制需要考虑人-水系统内利益相关者复杂的、差异化的反馈过程。
深入研究分水制度为代表的流域治理措施对流域系统的影响机制,可以为制定更有效的水资源管理政策提供科学依据。

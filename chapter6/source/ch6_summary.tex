本章在第五章研究的基础上,使用自下而上的多主体模型进一步分析了生产水稻、玉米、小麦三种黄河流域主要粮食作物的灌溉用水者对水资源分配制度的响应机制。
结果表明,黄河流域水资源配额与灌溉需求之间仍存在时空不匹配,在制度执行相对灵活的$1987$年至$1998$年,粮食作物灌溉主体(尤其是用水需求大省)倾向于产生违背配额制度的决策。
诸省份蓝/绿水的比例整体呈下降趋势,表明分水制度主要促进农业节水转型,而不是使用地下水替代受到限制的地表水,但不同区域响应分水制度的方式存在一定差异。
下游地区地下水开采量始终维持在较低水平且变化不大;中游地区持续增加地下水的开采量;上游则$1987$年分水制度提出之初期迅速增加了地下水开采,在$1998$年统一调度制度实施前后才出现迅速下降趋势。
产生上述响应差异的原因可能是多种多样的:上游灌区长期依赖地表水配额的“大引大排”,中游地表水取水日益困难,下游地下水难以利用且有相对充足的资本率先进行节水改造。
本章研究表明深入理解人水关系变化的机制需要考虑人水系统内利益相关者复杂的、差异化的反馈过程。
深入研究分水制度为代表的流域治理措施对流域系统的影响机制,可以为制定更有效的水资源管理政策提供科学依据。

本章在第五章研究的基础上,使用自下而上的多主体模型进一步分析了生产水稻、玉米、小麦三种黄河流域主要粮食作物的灌溉用水者对水资源分配制度的响应机制。
结果表明,黄河流域的上述分水制度所规定的地表水额度与三种主要粮食作物(水稻、玉米、小麦)的灌溉需求之间存在时空不匹配,灌溉主体在响应分水制度时以节水改革为主,而非简单地用地下水替代被限制的地表水,但在尚未深入节水改革时,违背配额约束很可能是响应制度变化的主要策略。
结果还表明农业主体响应分水制度的方式存在区域差异:黄河下游的地下水开采量在$1980$年至$2010$年始终维持在较低水平且变化不大;地表引水较困难的中游地区缓慢增加地下水开采;上游仅在$1987$年分水制度提出之初至$1991$年拐点出现前短暂增加了地下水开采。
本章研究表明深入理解人水关系变化的机制需要考虑人水系统内利益相关者复杂的、差异化的反馈过程。
深入研究分水制度为代表的流域治理措施对流域系统的影响机制,可以为制定更有效的水资源管理政策提供科学依据。

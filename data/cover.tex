% !Mode:: "TeX:UTF-8"

\ctitle{流域系统治理视角下的黄河人-水关系演变过程及其机制研究}
\etitle{Co-evolution of human-water interactions from a governing perspective in the Yellow River Basin, China: processes and mechanism}


\makeatother

% 学士学位封面
% \cbuyuanxi{天文系} % 部院系
% \czhuanye{天文学} % 专业
% \cxuehao{201511160109} % 学号
% \cxueshengxingming{某某某} % 学生姓名
% \czhidaojiaoshi{某某} % 指导教师
% \czhidaojiaoshizhicheng{教授} % 指导教师职称
% \czhidaojiaoshidanwei{北京师范大学天文系} % 指导教师单位

% 博士(硕士)学位封面
\czuozhe{宋爽} % 作者
\cdaoshi{傅伯杰\ 教授} % 导师
\cxibienianji{地理学部2018级} % 系别年级
\cxuehao{201831051016} % 学号
\cxuekezhuanye{自然地理} % 学科专业
\cwanchengriqi{2023年4月} % 完成日期


\begin{cabstract}
% 论文的摘要是对论文研究内容和成果的高度概括。摘要应对论文所研究的问题及其研究目的进行描述,对研究方法和过程进行简单介绍,对研究成果和所得结论进行概括。摘要应具有独立性和自明性,其内容应包含与论文全文同等量的主要信息。使读者即使不阅读全文,通过摘要就能了解论文的总体内容和主要成果。

% 论文摘要的书写应力求精确、简明。切忌写成对论文书写内容进行提要的形式,尤其要避免“第 1 章……;第 2 章……;……”这种或类似的陈述方式。

% 本文介绍北京师范大学论文模板 BNUThesis 的使用方法。本模板是在清华大学学位论文模板 THUThesis 的基础上修改而来,以及BNU硕博士模板BNUThesis上修改完成。完全参照《毕业论文写作规范(修订)》(师教文[2007]186 号)、北京师范大学图书馆Word版学士模板、北京师范大学信息科学与技术学院Word版写作模板的格式要求制作而成。

大河流域是人类文化起源和发展的中心,但随着社会的发展,人类活动改变了流域的自然和社会水循环过程,导致大部分流域出现不可持续的发展轨迹,人类与河流的关系变得空前紧张。
治理黄河是中华民族的千年夙愿,重塑黄河人水关系、支持黄河流域生态保护和高质量发展是重大国家战略。
因此,研究黄河流域人水关系演变规律、揭示黄河流域人水关系演变机制,有助于深化对人地关系地域系统结构特征和耦合机理的认识,为黄河流域治理提供理论基础和科学依据。
本研究通过严格定义“流域人水关系”,并以黄河流域为研究区,结合多源数据如水文气象观测、社会经济统计、历史数据重建和遥感反演等,借助统计分析和模型模拟等手段,发展了识别其演变规律的分析框架,分别在历史时期和现代治黄时期,定量划分黄河流域人水关系演变的主要阶段与过程;同时,本研究发展了流域人水关系演变机制分析的因果推断方法,建立了黄河流域社会-生态系统的多主体模型,识别了推动人水关系变化的关键机制,并定量评估了其产生的影响,主要研究结论如下:

(1)过去两千年中黄河流域有两个湿润气候驱动期($900AD\sim1100AD$和$1700AD\sim1900AD$)和两个突出的人类活动驱动期($1350AD \sim 1650AD$ 和 $1900AD$迄今)。其中第一个气候驱动期也被称为中世纪暖期(约$900AD \sim 1100AD$),黄河流域此时期发生的水沙特征变化回退到之前的状态。随后黄土高原农田与人口快速扩张,不断增加的人为压力与潮湿气候共同推动流域系统至稳态转换临界点,因此第二个$1700AD \sim 1900AD$潮湿期结束后各指标反而显著增强,标志着人类活动压力最早可能在$1350AD$后取代周期波动的气候,成为了主导黄河流域人-水关系的驱动因素。

(2)$1965$至$2013$年的黄河流域的治理演变可划分为三个阶段,依据其各自特点可命名为:集中供水时期($1965 \sim 1978$)、治理转变时期($1979 \sim 2001$)、适应增强时期($2002 \sim 2013$)。灌区扩张和经济增长是推动前两个阶段变化的直接原因,环境背景、社会文化、水治理政策等因素对向第三阶段的转型做出主要贡献,水资源供给逼近极限可能是触发这种治理转变的关键。

(3)在$1979 \sim 2001$的制度转变期,黄河流域的分水制度变化是推动转型的关键因素,两次自上而下的水资源分配制度改革对流域社会-生态系统的结构功能产生了不同的影响。$1987$年的“八七”分水方案违背制度预期地促使黄河流域用水量显著增加约$5.75\%$;$1998$年流域统一调度后,大多数区域用水量迅速得到控制,流域总用水以每年$6.6$亿立方米的速度显著下降,结束了长达二十余年的黄河断流。

(4)进一步针对水稻、玉米、小麦三种主要粮食作物灌溉需水构建了耦合人与自然的多主体模型,分析自下而上对流域分水制度的响应,解析了黄河流域诸省份在$1987$年“八七”分水方案制度后仍倾向于违背配额制度的机制。结果发现黄河流域水资源配额与灌溉需求之间仍存在时空不匹配,且农业主体响应分水制度的方式还存在区域差异。但黄河流域农业灌溉主体利用地表、地下水和降水的比例整体呈下降趋势,说明农业粮食灌溉响应分水制度的主要方式是促进节水转型。

本研究提出了流域系统人-水关系的定义及演变分析框架,以黄河为例在人水关系变化上开展了系统研究,定量识别并划分了黄河流域人水关系演变的主要阶段与过程、分析了不同阶段驱动流域发生稳态转换的主要因素、发展了流域人水关系演变机制分析的因果推断方法、建立黄河流域社会-生态系统的多主体模型,明晰了推动人水关系变化的关键制度机制并定量评估其产生的影响。本研究可为黄河这个典型的人类活动主导的流域的高质量、可持续发展提供科学基础和决策依据,也对相关的流域社会-生态系统治理具有借鉴意义。

  % 本文的创新点主要有:
  % \begin{itemize}[$\bullet$]
  %   \item 用例子来解释模板的使用方法;
  %   \item 用废话来填充无关紧要的部分;
  %   \item 一边学习摸索一边编写新代码。
  % \end{itemize}

  % 关键词是为了文献标引工作、用以表示全文主要内容信息的单词或术语。关键词不超过5个,每个关键词中间用分号分隔。(模板作者注:关键词分隔符不用考虑,模板会自动处理。英文关键词同理。)
\end{cabstract}
\ckeywords{黄河流域, 人水关系, 稳态转换, 水治理, 多主体建模}



\begin{eabstract}
  River basins have been at the center of human culture's origin and development. However, the development of society has caused changes in the natural and social water cycle processes of these basins, leading to unsustainable development trajectories and straining relationships between humans and rivers. Restoring the relationship between people and water in the Yellow River Basin has long been an aspiration of the Chinese nation, and it is a major national strategy for supporting ecological protection and high-quality development. Therefore, studying the evolution law and mechanism of man-water relationship in the Yellow River Basin is crucial for a deeper understanding of the structural characteristics and coupling mechanism of the man-land relationship regional system, as well as for providing a theoretical and scientific basis for the governance of the Yellow River Basin.

  This study strictly defined the ``human-water relationship in the Yellow River Basin'' and developed an analytical framework for identifying its evolution law through statistical analysis and model simulation. The study combined multi-source data, including hydro-meteorological observation, socio-economic statistics, historical data reconstruction, and remote sensing inversion, for both the historical period and the modern Yellow River control period. The study quantitatively divided the main stages and processes of the evolution of man-water relationship in the Yellow River Basin and developed a causal inference method to analyze the evolution mechanism of human-water relationship. Furthermore, a multi-agent model of the social-ecosystem in the Yellow River Basin was established to identify the key mechanism driving the change of human-water relationship and to quantitatively assess its impact. The main conclusions are as follows.

  (1) Over the past 2000 years, the Yellow River Basin has experienced two wet-driven periods ($900AD \sim 1100AD$ and $1700AD \sim 1900AD$) and two significant human-driven periods ($1350AD \sim 1650AD$ and $1900AD$ to the present day). The first climate driving period, also known as the Medieval Warm Period (about $900AD \sim 1100AD$), caused the water and sediment characteristics in the Yellow River Basin to revert to their previous state. Following this period, rapid expansion of farmland and population in the Loess Plateau, along with increasing anthropogenic pressure and humid climate, pushed the watershed system towards a stable transition critical point. Thus, after the second humid period from $1700AD \sim 1900AD$, all indicators significantly improved, indicating that human activity pressure may have replaced periodic fluctuating climate as the driving factor of the dominant human-water relationship in the Yellow River Basin from as early as $1350AD$.

  (2) The governance evolution of the Yellow River Basin between $1965$ and $2013$ can be divided into three stages, each with its unique characteristics: centralized water supply period ($1965 \sim 1978$), governance transformation period ($1979 \sim 2001$), and adaptation enhancement period ($2002 \sim 2013$). The first two stages were primarily driven by the expansion of irrigation areas and economic growth. However, factors such as the environmental background, social culture, and water management policies contributed significantly to the transformation to the third stage. The approaching limit of water resources supply may have played a key role in triggering the transformation of governance.

  (3) From $1979 \sim 2001$, the institutional transformation period in the Yellow River Basin was mainly driven by changes in the water distribution system. The two top-down water resources allocation system reforms had different impacts on the structure and function of the basin's socio-ecological system. The ``August 7'' Water Allocating Scheme, implemented in $1987$, significantly increased water consumption in the Yellow River Basin by approximately $5.75\%$, contrary to institutional expectations. After the unified operation of the river basin in August, water consumption in most areas was quickly controlled, leading to a significant reduction in total water consumption in the river basin at a rate of $660$ million cubic meters per year. This helped end the interruption of the Yellow River flow since 1972.

  (4) The study constructed an agent-based model that incorporates both human and natural elements to analyze the irrigation water requirements of rice, maize, and wheat, and their response to the bottom-up water division system in the Yellow River Basin. The results indicate that the provinces in the basin still tend to violate the quota system, even after the implementation of the ``August 7'' Water Allocating Scheme in $1987$. 
  The study also revealed a spatial-temporal mismatch between water resource quotas and irrigation demands in the Yellow River Basin, and there are regional differences in how agricultural entities respond to the water distribution system. However, the proportion of surface, groundwater, and precipitation used for agricultural irrigation in the Yellow River Basin has decreased, indicating that promoting water-saving transformation is the primary way to respond to the water separation system for agricultural and grain irrigation.

  This study proposes a framework for analyzing the evolution of the human-water relationship in a watershed system. Using the Yellow River as an example, the research systematically examines the changes in the human-water relationship. 
  It identifies and categorizes the main stages and processes of human-water relationship evolution in the Yellow River Basin, analyzes the key factors that drive the transformation of the basin in different stages, develops a causal inference method to understand the evolution mechanism of the human-water relationship, and establishes a multi-agent model of the socio-ecological system in the Yellow River Basin. 
  The study clarifies the institutional mechanisms that drive the change of human-water relationship and assesses their impacts quantitatively. The findings of this study can provide a scientific and decision-making basis for the high-quality and sustainable development of the Yellow River basin, which is a typical human-dominated basin, and has reference significance for governance of other basin social and ecological systems.

\end{eabstract}
\ekeywords{Yellow River Basin, Human-water interactions, Regime shift, Water governance, Agent-based model}



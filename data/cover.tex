% !Mode:: "TeX:UTF-8"

\ctitle{流域治理视角下黄河人-水关系演变过程及驱动机制}
\etitle{Co-evolution of human-water interactions from a governing perspective in the Yellow River Basin, China: processes and mechanism}


\makeatother

% 学士学位封面
% \cbuyuanxi{天文系} % 部院系
% \czhuanye{天文学} % 专业
% \cxuehao{201511160109} % 学号
% \cxueshengxingming{某某某} % 学生姓名
% \czhidaojiaoshi{某某} % 指导教师
% \czhidaojiaoshizhicheng{教授} % 指导教师职称
% \czhidaojiaoshidanwei{北京师范大学天文系} % 指导教师单位

% 博士(硕士)学位封面
\czuozhe{宋爽} % 作者
\cdaoshi{傅伯杰\ 教授} % 导师
\cxibienianji{地理学部2018级} % 系别年级
\cxuehao{201831051016} % 学号
\cxuekezhuanye{自然地理} % 学科专业
\cwanchengriqi{2023年4月} % 完成日期


\begin{cabstract}
% 论文的摘要是对论文研究内容和成果的高度概括。摘要应对论文所研究的问题及其研究目的进行描述,对研究方法和过程进行简单介绍,对研究成果和所得结论进行概括。摘要应具有独立性和自明性,其内容应包含与论文全文同等量的主要信息。使读者即使不阅读全文,通过摘要就能了解论文的总体内容和主要成果。

% 论文摘要的书写应力求精确、简明。切忌写成对论文书写内容进行提要的形式,尤其要避免“第 1 章……;第 2 章……;……”这种或类似的陈述方式。

% 本文介绍北京师范大学论文模板 BNUThesis 的使用方法。本模板是在清华大学学位论文模板 THUThesis 的基础上修改而来,以及BNU硕博士模板BNUThesis上修改完成。完全参照《毕业论文写作规范(修订)》(师教文[2007]186 号)、北京师范大学图书馆Word版学士模板、北京师范大学信息科学与技术学院Word版写作模板的格式要求制作而成。

大河流域是人类起源和发展的中心,随着人类活动改变了流域的自然和社会水循环过程,许多大河流域出现了不可持续的发展态势。
治理黄河是中华民族的千年夙愿,重塑黄河人水关系、实现黄河流域生态保护和高质量发展也是国家的重大战略。
因此,研究黄河流域人水关系演变规律、揭示演变机制,可为黄河流域治理提供理论基础和科学依据。
本研究结合水文气象观测、社会经济统计、历史数据重建和遥感反演等多源数据,借助统计分析、因果推断、与多主体建模等手段,发展了流域人水关系演变的识别与分析框架及机制分析方法,在历史时期和现代治黄时期定量划分黄河流域人水关系演变的主要阶段与过程、识别了推动人水关系变化的关键机制,主要研究结论如下:

(1)历史时期黄河流域存水沙变化的在两个湿润气候驱动期($900AD\sim1100AD$和$1700AD\sim1900AD$)和两个人类活动驱动期($1350AD \sim 1650AD$ 和 $1900AD$迄今)。其中第一个气候驱动期位于“中世纪暖期”(约$900AD \sim 1100AD$),此时黄河的水沙变化变化仍由气候因素主导。随后中游黄土高原地区发生农田与人口的快速扩张,不断增加的人为压力与另一次潮湿气候共同推动水沙特征在$1700AD \sim 1900AD$越过变化的临界点。上述结果表明人类活动带来的影响最早追溯至$1350AD$才开始取代周期变化的气候,逐步成为历史时期主导黄河流域人-水关系的主要因素。

(2)本研究使用综合水治理指数(IWGI)将当代治黄时期的流域水治理演变过程划分为三个阶段,并依据其各自特点命名为:集中供水时期($1965 \sim 1978$年)、治理转变时期($1979 \sim 2001$年)、以及适应增强时期($2002 \sim 2013$年)。灌区扩张和水库修建的放缓,是黄河从集中供水时期向治理转变时期过渡的主要特征。在治理转变时期流域的非工程治理措施迅速增加,过渡至适应增强时期后保持稳定,并着重提升用水效率。经讨论,上述治理模式转变可能在全世界流域系统中普遍存在,而本研究的分析指出水资源供给趋近极限可能是触发转变的关键。

(3)在上述治理转变期,$1987$年提出的黄河水资源配额制度及其在$1998$年的改革,是对流域影响深远的典型非工程治理政策。本研究指出这两次制度变化以不同的方式重塑了黄河流域的社会-生态系统结构,因而较政策初衷而言产生了不同的治理效果。其中,$1987$年通过的“八七”分水方案违背制度预期地使黄河流域用水量显著增加约$5.75\%$;而$1998$年参照该分水方案施行流域统一调度之后,大多数省份地区的用水量迅速得到控制,流域总用水在接下来十年间以每年$6.6$亿立方米的速度显著下降,成功治理了长达二十余年的黄河断流问题。

(4)黄河水资源配额制度在$1987$年与$1998$年两次变化对流域用水产生的影响,是用水者主体自下而上响应并适应制度变化的表现。本研究耦合了反映水资源配额制度的人类社会模块与计算三种主要粮食作物(水稻、玉米、小麦)灌溉用水需求的自然模块,发展了黄河流域农业灌溉用水者响应分水制度变化的多主体模型。模型仿真结果表明黄河流域的粮食生产灌溉需求与水资源配额之间存在时空不匹配,配额在$1987$至$1998$年间对用水的约束效果不明显,而$1998$年之后,配额制度在约束地表取水的同时也没有导致大规模的地下水开采。

本研究定量识别并划分了黄河流域人水关系演变过程的主要阶段、发展了分析其驱动机制的因果推断方法、开发了黄河流域社会-生态系统的多主体模型、自下而上解析了农业用水主体对制度变化的响应如何导致不同的流域治理效果。
本研究从流域系统治理和人水关系演变的视角为黄河流域的高质量、可持续发展提供科学基础和决策依据,重点解析了过去常常被忽视的流域非工程治理措施对人水关系演变的驱动作用,为流域治理制度的设计提供科学评估方法,这在世界大河流域非工程治理措施日益增多的当今显得尤为重要。

  % 本文的创新点主要有:
  % \begin{itemize}[$\bullet$]
  %   \item 用例子来解释模板的使用方法;
  %   \item 用废话来填充无关紧要的部分;
  %   \item 一边学习摸索一边编写新代码。
  % \end{itemize}

  % 关键词是为了文献标引工作、用以表示全文主要内容信息的单词或术语。关键词不超过5个,每个关键词中间用分号分隔。(模板作者注:关键词分隔符不用考虑,模板会自动处理。英文关键词同理。)
\end{cabstract}
\ckeywords{黄河流域, 人水关系, 稳态转换, 水治理, 多主体模型}



\begin{eabstract}
  Large river basins have been crucial to human society and its development. However, human activities have altered the natural and social water cycles in these basins, leading to unsustainable situation and unprecedented tension relationship between humans and rivers. Controlling the Yellow River has long been an aspiration of the Chinese nation, and a major national strategy aims to restore the human-water relationship in the Yellow River Basin, ensuring ecological protection and high-quality development.

  To facilitate this goal, it is vital to study the evolution of the human-water relationship in the Yellow River Basin and reveal its processes and underlying mechanisms. This will deepen our understanding of the regional human-earth interactions and provide a theoretical and scientific foundation for the governance of the Yellow River Basin. In this study we focus human-water interactions of the Yellow River Basin, China. We develop identification and analysis frameworks for the evolution of the human-water relationship by using statistical analysis and other methods, drawing upon multi-source data such as hydro-meteorological observations, socio-economic statistics, historical data reconstruction, and remote sensing. Our frameworks allow for quantitative divisions of the main stages and processes in the evolution of the human-water relationship in the Yellow River Basin during historical and modern periods.
  
  Furthermore, this study develops a causal inference method for analyzing the evolution mechanisms of the human-water relationship in the Yellow River Basin. A agent-based model of the socio-ecological system in the Yellow River Basin is established, and the key mechanisms driving the changes in the human-water relationship are identified. The main conclusions are as follows:

  (1) The historical period of the Yellow River Basin witnessed two wet climate-driven phases ($900 AD \sim 1100 AD$ and $1700 AD \sim 1900 AD$) and two human activity-driven phases ($1350 AD \sim 1650 AD$ and $1900 AD$ to present). The first climate-driven phase, also known as the Medieval Warm Period (approximately $900 AD \sim 1100 AD$), was dominated by climate-induced changes in the comprehensive index. Subsequently, rapid expansion of farmland and population on the Loess Plateau led to increasing human pressure, which, combined with a humid climate, drove water and sediment changes beyond a critical point during the second wet phase ($1700 AD \sim 1900 AD$). This suggests that human activity began to replace cyclical climate fluctuations as the primary driver of the human-water relationship in the Yellow River Basin after $1350 AD$.

  (2) In the contemporary period of Yellow River management, the water management evolution process can be divided into three stages: the centralized water supply period ($1965 \sim 1978$), the management transition period ($1979 \sim 2001$), and the adaptation enhancement period ($2002 \sim 2013$). Irrigation expansion and economic growth were the direct causes of changes in the first two stages. Environmental background, social culture, and water management policies had varying degrees of influence in each stage and were the main driving forces behind the transition from the second stage to the third stage. Such a management transition model is common in global river systems, with water resource supply approaching its limit as a potential key trigger for change.
  
  (3) The Yellow River Basin's surface water allocation system experienced two significant changes, which constituted far-reaching policy interventions during the management transition period. These interventions reshaped the social-ecological system structure of the Yellow River Basin and produced distinctly different management outcomes. The $1987$ water allocation plan unexpectedly led to an approximately $5.75\%$ increase in water usage in the Yellow River Basin. However, after the basin-wide unified scheduling in $1998$, water usage in most areas was quickly controlled, resulting in a significant annual decrease of $660$ million cubic meters in total water usage, effectively ending the Yellow River's over two-decade-long dry-up.
  
  (4) There was a spatial-temporal mismatch between the surface water quota stipulated by the water resources allocation system in the Yellow River Basin and the irrigation demand of three main food crops (rice, maize and wheat) in the basin. By developing a multi-agent model of responsive water allocation system for grain irrigation in the Yellow River Basin, we find that the water resource allocation system that restricts surface water intake does not directly lead to large-scale underground water substitute exploitation, and water-saving transformation is the main way to respond to system changes in each region.

  This study reveals the evolution process of human-water relationship in the Yellow River Basin, analyzes the main reasons driving the evolution of human-water relationship in different stages, develops a causal analysis method for the mechanism of human-water relationship change in the Yellow River Basin, establishes an agent-based model of the socio-ecological system in the Yellow River Basin, clarifies the mechanism of institution-driven human-water relationship change and evaluates its impact quantitatively. 
  From the perspective of river basin governance and human-water relationship evolution, this study provides scientific basis and decision-making basis for the high-quality and sustainable development of the Yellow River Basin. This study also provide innovative methods for the design of water governance and the analysis of institutional response mechanism, which is particularly important in today's world where attempts of non-engineering governance practices are growing.

\end{eabstract}
\ekeywords{Yellow River Basin, Human-water interactions, Regime shift, Water governance, Agent-based model}



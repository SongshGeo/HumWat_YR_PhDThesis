% !Mode:: "TeX:UTF-8"

\ctitle{黄河流域人-水关系演变过程及机制研究}
\etitle{Human-water co-evolution in the Yellow River Basin, China: processes and mechanism}


\makeatother

% 学士学位封面
% \cbuyuanxi{天文系} % 部院系
% \czhuanye{天文学} % 专业
% \cxuehao{201511160109} % 学号
% \cxueshengxingming{某某某} % 学生姓名
% \czhidaojiaoshi{某某} % 指导教师
% \czhidaojiaoshizhicheng{教授} % 指导教师职称
% \czhidaojiaoshidanwei{北京师范大学天文系} % 指导教师单位

% 博士(硕士)学位封面
\czuozhe{宋爽} % 作者
\cdaoshi{傅伯杰\ 教授} % 导师
\cxibienianji{地理学部2018级} % 系别年级
\cxuehao{201831051016} % 学号
\cxuekezhuanye{自然地理} % 学科专业
\cwanchengriqi{2023年4月} % 完成日期


\begin{cabstract}
论文的摘要是对论文研究内容和成果的高度概括。摘要应对论文所研究的问题及其研究目的进行描述,对研究方法和过程进行简单介绍,对研究成果和所得结论进行概括。摘要应具有独立性和自明性,其内容应包含与论文全文同等量的主要信息。使读者即使不阅读全文,通过摘要就能了解论文的总体内容和主要成果。

论文摘要的书写应力求精确、简明。切忌写成对论文书写内容进行提要的形式,尤其要避免“第 1 章……;第 2 章……;……”这种或类似的陈述方式。

本文介绍北京师范大学论文模板 BNUThesis 的使用方法。本模板是在清华大学学位论文模板 THUThesis 的基础上修改而来,以及BNU硕博士模板BNUThesis上修改完成。完全参照《毕业论文写作规范(修订)》(师教文[2007]186 号)、北京师范大学图书馆Word版学士模板、北京师范大学信息科学与技术学院Word版写作模板的格式要求制作而成。

  本文的创新点主要有:
  \begin{itemize}[$\bullet$]
    \item 用例子来解释模板的使用方法;
    \item 用废话来填充无关紧要的部分;
    \item 一边学习摸索一边编写新代码。
  \end{itemize}

  关键词是为了文献标引工作、用以表示全文主要内容信息的单词或术语。关键词不超过5个,每个关键词中间用分号分隔。(模板作者注:关键词分隔符不用考虑,模板会自动处理。英文关键词同理。)
\end{cabstract}
\ckeywords{\TeX, \LaTeX, CJK, 模板, 论文}



\begin{eabstract}
An abstract of a dissertation is a summary and extraction of research work and contributions. Included in an abstract should be description of research topic and research objective, brief introduction to methodology and research process, and summarization of conclusion and contributions of the research. An abstract should be characterized by independence and clarity and carry identical information with the dissertation. It should be such that the general idea and major contributions of the dissertation are conveyed without reading the dissertation.

An abstract should be concise and to the point. It is a misunderstanding to make an abstract an outline of the dissertation and words ``the first chapter'', ``the second chapter'' and the like should be avoided in the abstract.
 
Key words are terms used in a dissertation for indexing, reflecting core information of the dissertation. An abstract may contain a maximum of 5 key words, with semi-colons used in between to separate one another.
\end{eabstract}
\ekeywords{\TeX, \LaTeX, CJK, template, thesis}



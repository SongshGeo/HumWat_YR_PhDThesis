% !Mode:: "TeX:UTF-8"

\ctitle{流域治理视角下黄河人-水关系演变过程及驱动机制}
\etitle{Co-evolution of human-water interactions from a governing perspective in the Yellow River Basin, China: processes and mechanism}


\makeatother

% 学士学位封面
% \cbuyuanxi{天文系} % 部院系
% \czhuanye{天文学} % 专业
% \cxuehao{201511160109} % 学号
% \cxueshengxingming{某某某} % 学生姓名
% \czhidaojiaoshi{某某} % 指导教师
% \czhidaojiaoshizhicheng{教授} % 指导教师职称
% \czhidaojiaoshidanwei{北京师范大学\天文系} % 指导教师单位

% 博士(硕士)学位封面
\czuozhe{宋爽} % 作者
\cdaoshi{傅伯杰\ 教授} % 导师
\cxibienianji{地理科学学部2018级} % 系别年级
\cxuehao{201831051016} % 学号
\cxuekezhuanye{自然地理学} % 学科专业
\cwanchengriqi{2023年4月} % 完成日期


\begin{cabstract}
% 论文的摘要是对论文研究内容和成果的高度概括。摘要应对论文所研究的问题及其研究目的进行描述,对研究方法和过程进行简单介绍,对研究成果和所得结论进行概括。摘要应具有独立性和自明性,其内容应包含与论文全文同等量的主要信息。使读者即使不阅读全文,通过摘要就能了解论文的总体内容和主要成果。

% 论文摘要的书写应力求精确、简明。切忌写成对论文书写内容进行提要的形式,尤其要避免“第 1 章……;第 2 章……;……”这种或类似的陈述方式。

% 本文介绍北京师范大学论文模板 BNUThesis 的使用方法。本模板是在清华大学学位论文模板 THUThesis 的基础上修改而来,以及BNU硕博士模板BNUThesis上修改完成。完全参照《毕业论文写作规范(修订)》(师教文[2007]186 号)、北京师范大学图书馆Word版学士模板、北京师范大学信息科学与技术学院Word版写作模板的格式要求制作而成。

大河流域是人类起源和发展的中心,随着人类活动改变了流域的自然和社会水循环过程,许多大河流域出现了不可持续的发展态势。
治理黄河是中华民族的千年夙愿,重塑黄河人-水关系、实现黄河流域生态保护和高质量发展也是国家的重大战略。
因此,研究黄河流域人-水关系演变规律、揭示演变机制,可为黄河流域治理提供理论基础和科学依据。
本研究首先在历史时期和现代治黄时期定量划分黄河流域人-水关系演变的主要阶段与过程,其中识别历史时期的演变过程是理解人类活动如何主导流域人-水关系的基础,理解人类主导的现代水治理转型则是从工程治理走向综合治理的关键。
针对水治理转型期典型的非工程治理政策,本研究接下来分别从自上而下和自下而上的视角,识别并解析了驱动人-水关系变化的关键机制。主要研究结论如下:

(1)历史时期黄河流域存在水沙变化的在两个湿润气候驱动期(900AD - 1100AD和1700AD - 1900AD)和两个人类活动驱动期(1350AD - 1650AD 和 1900AD迄今)。其中第一个气候驱动期位于“中世纪暖期”(约900AD - 1100AD),此时黄河的水沙变化仍由气候因素主导。随后中游黄土高原地区发生农田与人口的快速扩张,不断增加的人为压力与另一次温暖潮湿的气候驱动期共同驱使水沙特征在1700AD - 1900AD越过变化的临界点。上述结果表明人类活动带来的影响最早追溯至1350AD才开始超越气候变化,逐步成为历史时期主导黄河流域人-水关系的主要因素。

(2)本研究使用综合水治理指数(IWGI)将当代治黄时期的流域水治理演变过程划分为三个阶段,并依据其各自特点命名为:集中供水时期(1965 - 1978年)、治理转变时期(1979 - 2001年)、以及适应增强时期(2002 - 2013年)。灌区扩张和水库修建的放缓,是黄河从集中供水时期向治理转变时期过渡的主要特征。在治理转变时期流域的非工程治理措施迅速增加,过渡至适应增强时期后保持稳定,并着重提升用水效率。经讨论,上述治理模式转变可能在全世界流域系统中普遍存在,而本研究的分析指出水资源供给趋近极限可能是触发转变的关键。

(3)在上述治理转变期,1987年提出的黄河水资源配额制度及其在1998年的改革,是对流域影响深远的典型非工程治理政策。本研究指出这两次制度变化以不同的方式重塑了黄河流域的社会-生态系统结构,因而较政策初衷而言产生了不同的治理效果。其中,1987年通过的``八七''分水方案违背制度预期地使黄河流域用水量显著增加约$5.75\%$;而1998年参照该分水方案施行流域统一调度之后,大多数省份地区的用水量迅速得到控制,流域总用水在接下来十年间以每年6.6亿立方米的速度显著下降,成功解决了长达二十余年的黄河断流问题。

(4)黄河水资源配额制度在1987年与1998年两次变化对流域用水产生的影响,是用水者主体自下而上响应并适应制度变化的表现。本研究耦合了反映水资源配额制度的人类社会模块与计算三种主要粮食作物(水稻、玉米、小麦)灌溉用水需求的自然模块,发展了黄河流域农业灌溉用水者响应分水制度变化的多主体模型。模型仿真结果表明黄河流域的粮食生产灌溉需求与水资源配额之间存在时空不匹配,配额在1987至1998年间对用水的约束效果不明显,而1998年之后,配额制度在约束地表取水的同时也没有导致大规模的地下水开采。

通过结合水文气象观测、社会经济统计、历史数据重建和遥感反演等多源数据,借助统计分析、因果推断、与多主体建模等手段,本研究发展了流域人-水关系演变的识别与分析框架及制度分析方法,定量识别并划分了黄河流域人-水关系的主要演变过程,解析了人-水关系变化的制度驱动机制。
本研究从流域系统治理和人-水关系演变的视角为黄河流域高质量、可持续发展提供科学基础和决策依据,重点解析了过去常常被忽视的流域非工程治理措施对人-水关系演变的驱动作用,为流域治理制度的设计提供了科学评估方法,这在世界大河流域非工程治理措施日益增多的当今显得尤为重要。

  % 本文的创新点主要有:
  % \begin{itemize}[$\bullet$]
  %   \item 用例子来解释模板的使用方法;
  %   \item 用废话来填充无关紧要的部分;
  %   \item 一边学习摸索一边编写新代码。
  % \end{itemize}

  % 关键词是为了文献标引工作、用以表示全文主要内容信息的单词或术语。关键词不超过5个,每个关键词中间用分号分隔。(模板作者注:关键词分隔符不用考虑,模板会自动处理。英文关键词同理。)
\end{cabstract}
\ckeywords{黄河, 流域, 人-水关系, 社会水文, 多主体模型}



\begin{eabstract}
  The river basin is the center of human origin and development. As human activities change the natural and social water cycle processes in the basin, many large river basins show unsustainable development trends. Managing the Yellow River has been the long-standing aspiration of the Chinas, and improving the relationship between humans and water in the Yellow River Basin (YRB), as well as achieving ecological protection and high-quality development of the basin, is a major national strategy. 
  Therefore, studying the evolution of the human-water relationship in the YRB and revealing the evolutionary mechanisms can provide a theoretical basis and scientific foundation for the management of the YRB.\

  This study first quantitatively divides the main stages and processes of the evolution of human-water relations in the YRB during the historical period and the modern period. Identifying the evolution process of the historical period is the basis for understanding how human activities dominate the human-water relations in the river basin, and understanding the modern water management transformation led by humans is the key to moving from engineering management to integrated management.
  Focusing on the typical non-engineering management policies during the water management transformation stage, this study then identifies and analyzes the key mechanisms driving the changes in human-water relations from both top-down and bottom-up perspectives. The main research conclusions are as follows:

  (1) In the historical period, the changes in water and sediment in the YRB occurred during two wet climate-driven periods (900AD - 1100AD and 1700AD - 1900AD) and two human activity-driven periods (1350AD - 1650AD and 1900AD to the present). The first climate-driven period was during the ``Medieval Warm Period'' (approximately 900AD - 1100AD), when the changes in water and sediment of the Yellow River were still dominated by climatic factors. Subsequently, rapid expansion of farmland and population occurred in the middle reaches of the Loess Plateau, with increasing anthropogenic pressure and another humid climate jointly driving the characteristics of water and sediment across the critical point of change between 1700AD - 1900AD.\ The above results indicate that the impact of human activities can be traced back to 1350AD, gradually replacing climate changes and becoming the main factor dominating the human-water relationship in the YRB during the historical period.

  (2) This study uses the Integrated Water Management Index (IWGI) to divide the contemporary Yellow River management period into three regimes, which are named according to their respective characteristics: massive water supply regime (1965 - 1978), governance transforming regime (1979 - 2001), and adaptation oriented regime (2002 - 2013). The slowdown in the expansion of irrigation areas and the construction of reservoirs are the main features of the transition from the massive water supply regime to the governance transforming regime. During the governance transforming regime, non-engineering management measures in the basin increased rapidly, and after the transition to the adaptation oriented regime, they remained stable and focused on improving water use efficiency. The analysis in this study points out that the key trigger for the transition may be the approach of water resource deficits, which could be a common phenomenon in river basin systems worldwide.

  (3) During the above governance transforming regime, the Yellow River water resource allocation scheme proposed in 1987 and its reform in 1998 were typically far-reaching non-engineering management policies for the basin. 
  This study points out that the two institutional shifts reshaped the structure of the social-ecological system in the YRB in different ways, thus producing different management effects compared to the original intentions. Specifically, the ``87 water allocation scheme'' passed in 1987 inadvertently increased water use in the YRB by approximately $5.75\%$; while after the implementation of the unified basinal regulating of the basin in 1998 based on the water allocation scheme, the water use in most provinces and regions was rapidly limited, and the total water use in the basin significantly decreased at a rate of 0.66 billion cubic meters per year over the next decade, successfully addressing the problem of the Yellow River's runoff drying up that had persisted for more than twenty years.

  (4) The impact of the Yellow River water resource allocation scheme on water use in 1987 and 1998 is a manifestation of stakeholders' bottom-up response and adaptation to institutional changes. 
  This study couples a human social module reflecting the water resource allocation scheme with a natural module calculating the irrigation water demand of three major food crops (rice, maize, and wheat), and then develops an agent-based model for agricultural irrigation water users in the YRB to respond to the institutional changes. 
  Simulation results show that there is a spatiotemporal mismatch between crops production irrigation demand and water resource allocation in the YRB, and the constraint effect of the allocation policy on water use is not significant between 1987 and 1998. 
  In addition, the allocation scheme constrained surface water withdrawal without leading to large-scale groundwater exploitation.

  By combining multi-source data such as hydro-meteorological observations, socio-economic statistics, historical reconstruction data, and remote sensing data, utilizing of statistical analysis, causal inference, and agent-based modeling, this study developed an analysis framework for the evolution of human-water relations and institutional analysis approach. 
  We quantitatively identified and divided the main evolutionary processes of human-water relations in the YRB and analyzed the institutional driving mechanisms of changing human-water relation.
  This research provides a scientific basis and decision-making support for the high-quality and sustainable development of the YRB from the perspective of human-water relations. 
  It focuses on the driving role of non-engineering governance practices in the evolution of human-water relations, which have often been overlooked in the past. 
  This study provides a scientific assessment approach for the design of river basin governance institutions, which is particularly important today as non-engineering measures are increasingly being adopted in major river basins worldwide.
\end{eabstract}
\ekeywords{Yellow River, river basin, human-water interactions, socio-hydrology, water governance, agent-based model}



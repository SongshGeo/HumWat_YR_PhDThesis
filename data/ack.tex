% !Mode:: "TeX:UTF-8"


\begin{ack}

    转眼间,我研究了五年的人与河。河有四时,有盈亏,有荣枯,亦有喜乐哀愁。河没有名字,我在河中打着水漂,所有想说的话河便替我说了——

    人顺河而下,可期百川入海。
    
    感谢恩师傅伯杰院士,学研若沧茫大海,取扁舟一叶横渡,他是“大渡海”的引路人。傅老师的教诲总是简单直接又具有启发性,帮助我这位五年前还徘徊在海边「面朝西边,指向北方」的少年找到了渡海的起点。感谢王帅教授,他包容着我的自由随性,倾听我的理想主义,给我开卷阅读的空间,给我自主探索的时间。渡海途中,我不止一次想对他说「Oh, captain, my captain」!
    
    感谢北京师范大学李小雁老师、赵文武老师、刘世梁老师、王佩老师、李琰老师、周沙老师、沈妙根老师、缪驰远老师、刘焱序老师;中国科学院生态环境研究中心吕一河老师、周伟奇老师、高光耀老师,中国科学院地理科学与资源研究所张扬建老师,北京大学城市与环境学院彭建老师,中国水利水电科学研究院杜龙江老师,天津师范大学刘见波老师和三位匿名评审专家在论文开题、预答辩、外审和正式答辩过程中提出的宝贵意见和无私指导。

    特别感谢北京师范大学的武旭同师兄,从五年前野外相识起,他见证并影响着我的学术思想演变,几乎每一篇研究工作我都曾向他请教,并总是能得到极有价值的建议。感谢中山大学一直同我保持联系的杜建会老师和胡亮老师,杜老师的敢想敢做与亮哥的认真严谨,都是我学术路途中后知后觉的宝贵财富。
    
    人沿河而行,可辨成长足迹。
    
    感谢卢品言,她是不断让我清醒地认识到“人生不止有一个奖章”的灵魂伴侣,淬炼我的天真与敏锐,使我的好奇心与求知欲常驻。感谢我博士期间的两位舍友卢文路与骆玉川,与他们同处的日常绝不会有一丝枯燥无趣。感谢王奕佳师妹,从中山大学到北师大,从青藏高原到维也纳,她一直是值得信赖的战友。感谢其他同院的挚友们,陈鹏、宋佳熙、张疋亥、冯思远,在学术和生活中他们都是能与之畅所欲言的伙伴。感谢课题组各位同仁,尤其是张萌萌、潘宁、李彤、王亚萍、李子栋、张皓宇、焦晨泰、叶冲冲、高德新、谭子敏,他们为我提供了举不胜举的帮助与支持,使科研生活的点滴得以汇聚成流。
    
    感谢香港大学的李研师兄,五年来他恰到好处的点拨让本不会编程的我逐步将短板补长。感谢北京大学的王文宇与黄永源,中国人民大学的牛昱尧、张炼、与温慧瑜,同他们在经济、历史、考古、社会学等领域的定期探讨极大地开拓了我的视野。感谢我的桥牌搭档、北京大学的肖逸南,他在许多数学问题上给予了我莫大帮助,对我而言好似数次雪中送炭。感谢所有为我提供过资料、信息、修改意见的老师同学乃至陌生审稿人们,我想对你们付出的答谢已毋需言表——我从来没有忽视过任何有价值的建议。
    
    人溯河回望,可见源头活水。
    
    感谢我的父母,他们一直努力为我营造着独立自主、无忧无虑的学习环境。他们默默操持着我们的家,即便来探望我,也会说着「我们看过北京了,我们看过珠海了,我们回家吧」之类的话,只因不想从学业中分走我太多精力。文章我自甘沦落,不觅封侯但觅诗,也许我永远无法成为一个世俗意义上的成功者,但多亏他们的支持,我才能在自己喜欢的道路上坚定前行,我走过的每一段路,笔下的每一个字符,都饱含他们的理解与厚爱。
    
    感谢互联网世界为我提供过帮助的众多开源社区和开源工具,你们使我能够自由勾勒出脑海里的玉宇琼楼,而不必从「先造一颗轮子」开始。功不唐捐无须先磨细铁杵,我心中本有的针才不会荒芜。工具的快速迭代也督促着我保持终身学习,哪怕世上的一切终将腐朽,拥抱开源的人也不必在一天内吃掉五十罐凤梨罐头。
    
    最后,致敬我的精神导师们——爱德华·威尔逊,段义孚,大卫·爱登堡,你们的作品是指引我前进的光,去追寻那个人与自然和谐共处的世界。你们用一份了不起的优雅,维持住了那个世界的幻觉,尽管那里的一切要么未曾到来,要么已经逝去\ldots
    
    悠悠五载换拙文百页,忽忽万绪聚笔尖一言。读博乃极尽自私之事,沿途固多承善意,然答报时光益疏,纵有穷尽感恩之思,终渺沧海之一粟。未尽之人事,今朝情谊铭心坊,来日重逢待报梅。
\end{ack}

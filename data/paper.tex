% !Mode:: "TeX:UTF-8"
% chktex-file 8

\begin{paper}
\begin{enumerate}
	\item \textbf{宋爽}, 王帅, 傅伯杰, 陈海滨, 刘焱序, 赵文武. 社会—生态系统适应性治理研究进展与展望. 地理学报, 74, 2401–2410 (2019). (IF = 10.14) 
	\item Wang, S., \textbf{Song, S.}, Zhang, J., Wu, X. \& Fu, B. Achieving a fit between social and ecological systems in drylands for sustainability. Curr. Opin. Environ. Sustain. 48, 53–58 (2021). (IF = 7.96) 
	\item \textbf{Song, S.} et al. Sediment transport under increasing anthropogenic stress: Regime shifts within the Yellow River, China. Ambio 49, 2015–2025 (2020). (IF = 6.94) 
	\item \textbf{Song, S.}, Wang, S., Wu, X., Huang, Y. \& Fu, B. Decreased virtual water outflows from the Yellow River basin are increasingly critical to China. Hydrology and Earth System Sciences 26, 2035–2044 (2022). (IF = 6.62) 
	\item \textbf{Song, S.} et al. Improving representation of collective memory in socio‐hydrological models and new insights into flood risk management. J Flood Risk Management 14, (2021). (IF = 4.01) 
	\item \textbf{Song, S.} et al. The responses of Spinifex littoreus to sand burial on the coastal area of Pingtan Island, Fujian Province, South China. Écoscience 1–10 (2021). (IF = 1.34) 
	% 接下来是合著
	\item Wu, X., Fu, B. J., Wang S., \textbf{Song S.}, et al. Decoupling of SDGs followed by re-coupling as sustainable development progresses. Nature Sustainability (2022) doi:10.1038/s41893- 022-00868-x. (IF = 27.2)
	\item 王奕佳, 刘焱序, \textbf{宋爽}, 姚莹, 傅伯杰. 社区尺度社会-生态系统适应途径述评. 地理科学进展, 41, 935–944 (2022). (IF = 6.05) 
	\item 王奕佳, 刘焱序, \textbf{宋爽}, 傅伯杰. 水—粮食—能源—生态系统关联研究进展. 地球科学进展, 36, 684–693 (2021). (IF = 6.04) 
	\item Jiao, C., Wang, S., \textbf{Song, S.}, Fu B., Long-term and Seasonal Variation of Open-surface Water Bodies in the Yellow River Basin during 1990–2020. Hydrological Processes, e14846 (2023).
	\item Chen, P., Wang, S., \textbf{Song, S.}, Wang Y., Wang Y., Gao D., Li Z. Ecological restoration intensifies evapotranspiration in the Kubuqi Desert. Ecological Engineering 175, 106504 (2022). (IF = 4.37) 
	\item Yao, Y., Fu, B., Liu, Y., Wang, Y., \& \textbf{Song, S.} The contribution of ecosystem restoration to sustainable development goals in Asian drylands: A literature review. Land Degrad Dev ldr.4065 (2021). (IF = 4.38)
	\item Zhang, M., Wang S., Fu, B.J., Wei, X. H., Wang, C., \textbf{Song, S.} Structure Disentanglement and Effect Analysis of the Arid Riverscape Social-Ecological System Using a Network Approach. Sustainability 11, 5159 (2019). (IF = 3.89) 
	\item Gao, D. Wang, S. Li, Z.D., Wei. F. L., Chen, P., \textbf{Song, S.}. Threshold of vapour– pressure deficit constraint on light use efficiency varied with soil water content. Ecohydrology (2021) doi:10.1002/eco.2305. (IF = 3.17) 
	\item Li, Z., Wang, S., \textbf{Song, S.}, Wang, Y. \& Musakwa, W. Detecting land degradation in Southern Africa using Time Series Segment and Residual Trend (TSS-RESTREND). Journal of Arid Environments 184, 104314 (2021). (IF = 2.76) 

	% \item Yang Y, Ren T L, Zhang L T, et al. Miniature microphone with silicon-based ferroelectric thin films. Integrated Ferroelectrics, 2003, 52:229-235. (SCI 收录, 检索号:758FZ.)
	% \item 杨轶, 张宁欣, 任天令, 等. 硅基铁电微声学器件中薄膜残余应力的研究. 中国机械工程, 2005, 16(14):1289-1291. (EI 收录, 检索号:0534931 2907.)
	% \item 杨轶, 张宁欣, 任天令, 等. 集成铁电器件中的关键工艺研究. 仪器仪表学报,2003, 24(S4):192-193. (EI 源刊.)
	% \item Yang Y, Ren T L, Zhu Y P, et al. PMUTs for handwriting recognition. Inpress. (已被 Integrated Ferroelectrics 录用. SCI 源刊.)
	% \item Wu X M, Yang Y, Cai J, et al. Measurements of ferroelectric MEMS microphones. Integrated Ferroelectrics, 2005, 69:417-429. (SCI 收录, 检索号:896KM.)
	% \item 贾泽, 杨轶, 陈兢, 等. 用于压电和电容微麦克风的体硅腐蚀相关研究. 压电与声光, 2006, 28(1):117-119. (EI 收录, 检索号:06129773469.)
	% \item 伍晓明, 杨轶, 张宁欣, 等. 基于MEMS技术的集成铁电硅微麦克风. 中国集成电路, 2003, 53:59-61.
\end{enumerate}
\end{paper}

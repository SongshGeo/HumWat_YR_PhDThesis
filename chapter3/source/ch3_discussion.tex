
\subsection{人类活动主导黄河流域的影响}\label{ch3:sec:human-activity}

本章研究关注的重点,是黄河流域在2000余年的历史跨度内,不断增加的人为压力如何取代周期波动的气候,成为主导流域人-水关系的主要驱动力的。
不断增加人为活动压力与气候变化一起,推动系统跨越临界条件并触发稳态转换,被广泛认为是世界范围内触发稳态转换的主要原因 \cite{scheffer2001,scheffer2003}。
本章研究表明,虽然黄河在历史上随气候周期波动下也产生了许多明显的特征变化,但直到人为压力接近临界点,流域系统才出现了稳态转换。
与全球各地报告的许多其他稳态转换类似,推动临界点的人为干扰是伴随人口增长的中游农业开垦与粮食生产,因为历史时期黄土高原社会-生态系统管理策略的局域性,这种人为干扰带来的影响带来的级联效应如多米诺骨牌般,成为黄河稳态转换的触发因素之一\cite{rocha2018}。
因此,随着人类世人口的增长粮食生产需要也在迅速增长,此类人类干扰更容易促使生态系统超过临界点,这也是黄土高原地区通过“退耕还林”工程对黄河中游土地的复垦进行控制的原因之一。
事实上,这类人类对生态系统的管理已经在短短60年内将黄河的高泥沙量降低到原始水平,这种剧烈的变化曾是数百年至数千年的气候和人为强迫共同塑造的\cite{wang2016e, ji2018}。
当然,泥沙含量回到原始状态并不代表流域系统的稳态也回到了原始状态,相反可能标志着人水关系进入新的稳态,这种人水关系稳态下,人为干预为社会-水文二元循环结构带来了诸多剧烈变化,将对黄河流域的人类社会产生更大的影响。
而黄河人-水关系突破临界点并发生稳态转换的过程,也是非常有助于理解稳态转换内在机制的有用案例,因为在整个地球历史上,水生、陆地和界面生态系统都发生过这样巨大的变化\cite{hughes2013, rocha2018}。
通过在长时间尺度上通过进一步了解这种驱动力及其影响之间的关系,本章研究可以更好地评估社会-生态系统的弹性和稳态转变,从而制定可持续的管理战略\cite{scheffer2003}。

\subsection{理解黄河历史稳态转变过程的意义}

% TODO 这里要注意查重
由于研究时段较长,本章使用的数据集是历史记录、历史时期重建数据、估算数据的组合(表\ref{tab:data_source})。
历史记录和重建数据高度依赖相关典籍和文献的准确性,但由于隐匿和漏报等问题,一些历史档案中可能存在错误或不确定性(葛全胜等, 2003)。
而本章用到的历史时期估算数据也是来自其他研究根据 相关文献典籍或相关资料(如沉积速率等)的估算,虽然存在一定不确定性,时间分辨率也有限,但指标的变化趋势基本可信。
本章用到的大部分历史数据集是相应指标的仅有的数据源,虽然对这些数据的可靠性持谨慎态度,本章研究还是相信本研究基本体现了黄河流域系统人水关系变化的整体过程。

\section{人类主导水沙变化的时间}

黄河中游的毁林开荒及其带来的历史输沙量增加,是历史自然地理的经典研究问题,也是本研究路线中将水沙变化和人\textendash{}水关系演变之间建立联系的关键,有必要就此问题对前人贡献和本章研究进行详细的对比总结。
任美锷总结的黄河泥沙沉积数据一直被视为是黄河输沙量增加的直接证据\cite{renmeie2006},该数据在500AD至1400AD年之间样本数量不够丰富,因此关于人类活动主导黄河水沙特征发生在宋朝还是在明清之后,是需要集百家之言认真讨论的。
首先,此时期都存有公认明确记载的人类开垦活动,但明朝强度更甚:北宋时期黄土高原已开始驻扎军队进行屯田,开垦坡地、砍伐树木、修筑堡寨;明代不仅“驻守大量军队”在黄土高原地区,并鼓励移民垦种大量坡地,以至于晋西和陕北的黄土高原丘陵区“山之悬崖峭壁,无尺寸不耕”~\cite{renmeie2006,wu2020a,shinianhai1985}。
其次,任美锷的研究时间尺度更长、精度更低,认为人类强烈影响黄河流域的时间点出现于5000BP,因而缺乏对宋、明时期气候的细致对比,也缺乏对两个时期变化差异的区分\cite{renmeie2006, mei-e1994}。
但本章结果已经表明,在明以前的温暖气候时期(CDP1)与之后的温暖气候期(CDP2)之间,黄河流域“沉积、洪泛、航运”综合构成的水沙特征指标变化呈明显差异,辨别此差异出现的时段对人\textendash{}水关系演变识别意义重大。

从$1100$年到$1750$年被认为是黄土高原历史时期中的“耕种快速扩张”阶段,人口从600万增加到1600万,耕地面积从$2.33 \times 10^6~ha$增加到$8.57 \times 10^6~ha$,而森林覆盖率从$33\%$下降到$18\%$\cite{wu2020a}。
在这段时间内,已有研究对黄河中游植被覆盖度的记录不仅差异较小,且存在很大争议,而农牧交错带的北界扩张相对可靠且更为显著,在温暖的宋朝结束后向南移动,但在清朝重新向北移动,指示着宋朝前后气候仍是主导因素\cite{shinianhai1985,GeQuanSheng2011}。
此外,宋代有许多决溢洪水的记载,这在一些研究中被认为是黄河在宋朝开始进入高输沙的佐证\cite{chen2012}。
但在此时期恰逢中世纪暖期降水量增多,河事也是朝政热点议题因而留下丰富的记载,因此本章研究认为将宋代与其他相对较干旱、记载较少的其它时段以统一标准对比有失公允。
综上所述,本章研究没有选用中游的植被覆盖面积数据,而选用了农牧交错带数据和水量变化的航运数据,有助于剥离宋朝气候暖期对水沙特征的气候影响因素,结果表明最晚至公元$1450$年,人类活动还没能超越气候影响主导黄河流域的人\textendash{}水关系。


\section{对流域弹性治理的启示}

研究昨日之黄河有助于理解今日之黄河,回望水沙关系的长时间尺度变化有助于在决策时看得更远。
本章研究的重点在于分析黄河流域在$2000$多年的历史时期内,人类活动逐渐取代了气候成为主要的流域系统驱动力。
虽然气候变化在历史上也造成了许多明显的变化,但是直到人类活动的压力在1400AD之后才迫使流域系统接近稳态转换的临界点。
这种不断增加人为活动压力与气候变化一起,推动系统跨越临界条件并触发稳态转换,被广泛认为是世界范围内触发稳态转换的主要原因~\cite{scheffer2001,scheffer2003}。
推动临界点的人为干扰是伴随人口增长的中游农业开垦与粮食生产,因为历史时期黄土高原社会\textendash{}生态系统管理策略的局域性,这种人为干扰带来的影响,作为一种级联效应成为黄河稳态转换的触发因素之一\cite{rocha2018,wu2020a}。
因此,随着人类世人口的增长粮食生产需要也在迅速增长,此类人为干扰更容易促使生态系统超过临界点,这也是黄土高原地区通过“退耕还林”工程对黄河中游土地的复垦进行控制的原因之一~\cite{wu2020a}。

% % 江恩慧 2022本子
全球气候变化和人类活动深刻影响着水文泥沙过程、生态环境演变和社会经济发展进程。众多大型河流的水沙通量发生了显著变化,尤以黄河最为突出\cite{best2019, best2020}。
近年来,对黄河流域水沙变化趋势的研究基本取得了共识,认为气候变化和人类活动共同影响使得水沙变化更加复杂,人类活动是水沙变化的主要驱动原因\cite{wang2016a, ma2020}。
通过$60$年的努力,对黄河流域的大规模治理,将高泥沙含量降低到了原始水平,这是历经数百年至数千年的气候和人类活动共同塑造的剧烈变化之后的结果\cite{wang2016a, ji2018}。
然而,这种回归并不代表流域系统的稳态也回到了原始状态,相反,这可能标志着人\textendash{}水关系进入了一个新的稳态,这种新的稳态下,人类干预为社会\textendash{}水文二元循环结构带来了许多剧烈变化,将对黄河流域的人类社会产生更大的影响。
而黄河人\textendash{}水关系突破临界点并发生稳态转换的过程,也是有助于理解稳态转换内在机制的有用案例,因为在整个地球历史上,水生、陆地和界面生态系统都经历过这样巨大的变化\cite{hughes2013, rocha2018}。
通过长时间尺度上进一步了解这种驱动力及其影响之间的关系,本章研究可以更好地评估社会\textendash{}生态系统的弹性和稳态转变,从而制定可持续的管理战略\cite{scheffer2003}。
在人类活动成为主导之前,厘清黄河水沙特征变化逐渐与气候周期“脱钩”的过程,就是流域人\textendash{}水关系发生稳态转换的过程。本章研究通过对黄河流域历史时期水沙稳态转换驱动、影响因素指标的变化,为了解人类活动对流域系统的影响及其协同演变提供了研究框架。
通过对过去两千年黄河人\textendash{}水关系变化的研究,可以更好地认识全球气候变化和人类活动对水文泥沙过程、生态环境演变和社会经济发展进程的深刻影响,为未来的生态环境保护和可持续发展提供参考和借鉴。

由于研究时段较长,本章使用的数据集是历史记录、历史时期重建数据和估算数据的组合(表\ref{tab:data_source})。历史记录和重建数据高度依赖相关典籍和文献的准确性,但由于隐匿和漏报等问题,一些历史档案不可避免存在不确定性\cite{wu2020a}。
本章用到的大部分历史数据集是相应指标的唯一数据源,这些历史时期数据大多根据相关文献典籍(如正史记载)或相关资料(如沉积速率等)的估算,虽然存在一定的不确定性,时间分辨率也有限,但指标的变化趋势基本可信。

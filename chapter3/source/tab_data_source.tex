% Table generated by Excel2LaTeX from sheet 'Sheet1'
\begin{table}[hbt]
  % \centering
  % \begin{minipage}[t]{0.8\linewidth}
  \caption{历史时期黄河稳态转换识别的数据来源}
    % \begin{tabular}{llllll}
    \begin{tabularx}{\textwidth}{ p{2cm} L p{5cm} L p{3cm} L}
    \toprule
    数据集   & 类型    & 描述    & 原始材料  & 可信度   & 参考 \\
    \midrule
    干旱和洪涝频率 & 气候驱动力 & 每五十年为一个时期,分别统计中原、华北平原地区的干旱、洪涝灾害的频次 & 官方和地方的史料记录 & 根据历史材料的丰富程度,在不同时期的可信度是不一致的 & 葛全胜,2011~\cite{GeQuanSheng2011} \\
    湿润指数累计距平 & 气候驱动力 & 根据史料中与湿度相关的记录,评估历史时期气候湿润程度的等级 & 官方和地方的史料记录 & 根据历史材料的丰富程度,在不同时期的可信度是不一致的 & Zheng,2006~\cite{zheng2006} \\
    黄河中游人口数量 & 人类活动驱动力 & 利用史料记录推断的黄河中游人口 & 历史户籍注册信息 & 数字不精确但可以反映趋势以供纵向比较 & Chen et al., 2012~\cite{chen2012} \\
    农牧交错带的北界位置 & 混合驱动力 & 农牧交界线距离潼关所在维度的平均距离 & 中国历史地图集 & 时间采样点比较少,但每个点的位置数据较为可信 & 谭其骧,1996~\cite{TanQiXiang1996} \\
    三门峡峡谷航运数据 & 影响$^*$    & 三门峡航运通行难易程度的等级 & 官方和地方的史料记录 & 数据从可靠的史料来源获得,但存在数据缺失情况 & 王守春,1993~\cite{WangShouChun1993} \\ %TODO 这里还需要参考,查证王老师那的书
    黄河下游决溢数据 & 影响    & 黄河下游堤防崩溃的次数 & 官方和地方的史料记录 & 根据历史材料的丰富程度,在不同时期的可信度是不一致的 & Chen et al., 2012~\cite{chen2012} \\
    黄河古河道沉积速率 & 影响    & 结合历史地图和取样测定的黄河古河道沉积速率 & 沉积样本  & 样本所在的古河道时间跨度越长样本越精确 & Xu 2003~\cite{xu2003a} \\
    \bottomrule
    % \end{tabular}%
  \end{tabularx}\\[2pt]
  % \leftalighn
  \footnotesize 注:*影响:指黄河关键特征变化带来的影响。\\\label{tab:data_source}%
  % \end{minipage}
\end{table}%

识别历史时期不断增强的人类活动压力如何超越气候周期变化的驱动力的时间,分析两者推动黄河流域水沙特征突破临界点并发生稳态转换过程,对于理解黄河历史人水关系及其演变至关重要。
本章研究重点包括识别黄河流域在历史时期典型稳态转换的发生过程,以及分析在该稳态转换发生前黄河流域人水关系的模式。

通过历史时期重建数据和相关文献资料,本章首先识别不同驱动力(潮湿气候/人类活动)所主导的时段,发现过去两千年中黄河流域有两个数据可靠的湿润气候驱动期($900AD\sim1100AD$和$1700AD\sim1900AD$),不断增长的人类活动驱动力也在两个时期内尤为突出($1350AD \sim 1650AD$和$1900AD$之后)。
通过分析驱动力时段前、中、后能反映关键水砂特征的影响指标变化,本章研究发现在中世纪暖期(约$900AD \sim 1100AD$)发生的变化在此时期结束后几乎都回退到潮湿期前的状态,而$1700AD \sim 1900AD$潮湿期结束后,各指标不仅没有回退反而显著增强。
上述现象为历史时期黄河流域勾勒出如下的稳态转换过程:不断增加的人为压力与潮湿气候在$1700AD \sim 1900AD$之间共同推动流域系统至稳态转换临界点,而人类活动压力最早可能在$1350AD$后取代周期波动的气候,成为了主导黄河流域人-水关系的驱动因素。

本章研究分析了黄河流域在历史时期发生稳态转换并逐步走向人类主导水沙过程的状态,由于在整个地球历史的各类水生、陆地和界面生态系统都发生过这样巨大的变化,本章研究有助于更好地理解和评估社会-生态系统的稳态转变,从而制定可持续的流域管理战略。

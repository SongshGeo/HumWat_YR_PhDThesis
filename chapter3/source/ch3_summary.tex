厘清水沙特征变化与气候周期的关系,是明晰黄河流域人水关系是否发生变化的关键。
本章研究通过识别黄河流域在过去两千年中的水沙特征变化,总结历史时期人水关系变化的发生过程。
研究表明历史时期的黄河流域水沙变化存在两个湿润气候驱动期($900AD\sim1100AD$和$1700AD\sim1900AD$)和两个人类活动驱动期($1350AD \sim 1650AD$ 和 $1900AD$迄今)。其中第一个气候驱动期也被称为中世纪暖期(约$900AD \sim 1100AD$),此时期的综合指标变化仍由气候主导。随后黄土高原农田与人口快速扩张,不断增加的人为压力与潮湿气候共同推动水沙变化在$1700AD \sim 1900AD$的第二个潮湿期已越过临界点,说明人类活动压力最早在$1350AD$后开始取代周期波动的气候,逐步成为主导黄河流域人-水关系的驱动因素。
本章研究梳理了黄河流域在历史时期发生稳态转换并逐步走向人类主导水沙过程的状态,有助于更好地制定流域社会-生态系统的可持续治理战略。



中国历史上长达三千余年的时间里,黄河流域一直是全国政治、经济和文化中心,治理“善淤、善决”的黄河是历代中央王朝主导下人水关系的主要特征,思想虽先后有“堵”、“疏”、“分流”、“束水攻沙”等,但始终着眼于河道\cite{WangWeiJing2009}。
直到近百年的综合治理让黄河的年均泥沙排放量急剧下降至$0.1\sim0.2 Gt/yr$,恰好与约$2000$年前黄河较原始的泥沙输送量接近,以“泥沙淤积”、“洪泛决溢”、“治水应对”为主的历史时期人水关系才发生根本性变化~\cite{wang2007,song2020, wang2016a}。
对于受人类主导黄河流域而言,梳理水沙关系变化对理解人水关系至关重要,因为水沙关系只是近代黄河流域治理的一部分,却是维系历史时期“泥沙淤积-洪泛决溢-治水应对”流域人水关系的核心。
但是,对于历史时期两千年间,黄河水沙的变化过程何时发生、如何发生却尚缺乏系统认识。

% 黄河在距今$2000$年前仍处于较为原始的状态,其输沙量约为$0.1 \sim 0.2 Gt/yr$,仅为20世纪均值的十分之一\cite{xu2003a},黄河河道也相对较平稳,决溢灾害和治河活动相对较少(图\ref{fig:ch3:why_regime_shift})\cite{WangWeiJing2009, chen2012}。
% 但在$20$世纪以降,黄河流域已成为学界公认的、由人类主导“自然-社会二元水循环”的典型流域系统,“行洪输沙、生态环境、社会经济”多过程、多功能并重,由众多利益相关者共同组成了影响流域水循环的合力\cite{jiang2020b}。
% 但在历史时期,人类活动影响从微弱逐步变强大的过程中,黄河流域的人水关系一定经历了离开其原始稳态的过程,因为$2000$原始时期流域的生态环境与社会经济几乎不会在系统层面影响黄河,保障“行洪输沙”和社会安定是主要——甚至是利益相关者唯一的考虑。
% 在千年时间尺度上,除却气候的周期性波动,人类活动的影响虽总体呈增加趋势,也存在诸多非线性变量,因此识别“不可逆的结构重组”是识别稳态转换发生的关键标志\cite{GeQuanSheng2011}。

本章发展了黄河流域历史时期水沙特征的稳态转换分析框架,稳态转换可能是由系统内部的渐进或累积变化所引起,驱动力可能来自于由人类干扰或气候波动。
在历史时期,气候的变化和不断积累的人类活动双重作用对黄河水沙特征的变化均有贡献,梳理其稳态转换何时、因何原因主导而发生,是长时间尺度下理解人水关系变化的重要基础。
% 在$2000$余年前黄河流域的土地已开始被广泛地开垦使用,这一历史时期降水量主要由气候周期规律,但不断增强的人类活动导致黄河的泥沙排放总体呈现上升趋势\cite{wang2007}。
本章通过历史重建数据分析了过去两千年间黄河水沙排放的变化与其对流域的影响,特别关注历史时期黄河水沙特征的稳态转换,解析黄河流域在历史时期人水关系变化的过程与驱动力。

% 黄河的频繁泛滥对中央王朝的治理带来严峻挑战,也积累有关黄河成灾的“集体记忆”。
% 史书累牍的黄泛之殇,是新中国成立伊始毛主席感慨“把黄河的事情办好”的动力,也可能是黄河人-水关系长期维持“洪水-响应”模式的内在机理。
% 这种有关长时间尺度的洪水灾害模型中,集体记忆已被认为是影响社会响应方式的核心变量之一~\cite{dibaldassarre2015, dibaldassarre2019,ciullo2017}。
% 因此,在识别稳态转换的基础上,使用集体记忆模型在气候周期波动变化下,预测人类社会对洪水的响应模式,是解释黄河长时间尺度上人-水关系演变的关键。

% 本章的研究首先从稳态转换的视角梳理了黄河从低输沙到高输沙的演变过程,从而区分周期性的气候驱动与不断积累的人为因素,继而利用洪水记忆模型解释了人类活动未突破稳态临界点之前,气候灾害的周期及集体记忆变化,是黄河人水关系在历史时期维持中央政府主导“洪水-响应”模式的内在机制。


历览长河,“黄河”除了留下气势磅礴的诗文,也留下了多灾多难的记载。面对善决善淤、泛滥无度的黄河,治水是历代中央王朝主导下,黄河人水关系的主旋律,更有甚者认为中华文明就是黄河的治理史。前人对“黄河人水关系”的研究已总结了历代几个重要的治水思想先后是:堵、疏导、分流、束水攻沙、水土保持等。但对于为什么在千年尺度上,黄河人-水关系始终维持这种模式,尚缺乏流域系统的层面的理解。

黄河频繁决溢的主要症结是丰富的泥沙引起的下游沉积。作为世界上泥沙曾经最丰富的河流,其经历了两次大规模的泥沙稳态变化\cite{wang2007, wei2016}。黄河泥沙的峰值在1960s,随后由于现代化的大规模人为治理,短短半个世纪便降低至$0.1\sim0.2 Gt/$。这个数字恰好与大约2000年前,黄河的泥沙含量相对较低时期相接近 \cite{song2020}。但在千年尺度上,气候的变化和不断积累的人类活动双重作用对黄河由低输沙转化为高输沙状态的变化均有贡献,梳理其稳态转换何时、因何原因主导而发生,是理解人-水关系长时间尺度变化的重要基础。

黄河的频繁泛滥对中央王朝的治理带来严峻挑战,也积累有关黄河成灾的“集体记忆”。史书累牍的黄泛之殇,是新中国成立伊始毛主席感慨“把黄河的事情办好”的动力,也可能是黄河人-水关系长期维持“洪水-响应”模式的内在机理。这种有关长时间尺度的洪水灾害模型中,集体记忆已被认为是影响社会响应方式的核心变量之一 \cite{dibaldassarre2015, dibaldassarre2019,ciullo2017}。因此,在识别稳态转换的基础上,使用集体记忆模型在气候周期波动变化下,预测人类社会对洪水的响应模式,是解释黄河长时间尺度上人-水关系演变的关键。

过去、现在、未来的黄河有着不同的人水关系,研究昨日之黄河有助于理解今日之黄河,回望人水关系的长时间尺度变化有助于在决策时看得更远。本章的研究首先从稳态转换的视角梳理了黄河从低输沙到高输沙的演变过程,从而区分周期性的气候驱动与不断积累的人为因素,继而利用洪水记忆模型解释了人类活动未突破稳态临界点之前,气候灾害的周期及集体记忆变化,是黄河人水关系在千年尺度上维持中央政府主导“洪水-响应”模式的内在机制。

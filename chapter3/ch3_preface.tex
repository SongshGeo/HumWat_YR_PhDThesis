

中国历史上长达3000多年的时间里,黄河流域一直是全国政治、经济和文化中心。
有说法认为“中国历史在某种程度上就是一部“治黄”史”,因为治理“善淤、善决”的黄河是历代中央王朝主导下人水关系的主旋律。
前人总结千年间历朝“黄河治理”的思想,认为虽先后有“堵”、“疏”、“分流”、“束水攻沙”等,但始终是着眼于河道的“治水为主”\cite{WangWeiJing2009}。
这种以“泥沙淤积”、“洪泛决溢”、“治水应对”所主导的人水关系,直到过去近百年间“从河道过渡至流域”的治理才得到改变。
近现代“水沙关系”的面上治理让黄河的年均泥沙排放量急剧下降至$0.1\sim0.2 Gt/$,有研究认为这个数字恰好与大约$2000$年前黄河原始的泥沙输送量相接近~\cite{wang2007}。
最近百年间的“水沙减少”的变化趋势已得到广泛研究\cite{wei2016, song2020, wang2016a},但对于过去$2000$年间,黄河水沙的变化过程何时发生、如何发生却尚缺乏系统认识。
于受人类主导之前黄河流域而言,梳理水沙关系变化对理解人水关系至关重要,因为水沙关系只是近代黄河流域治理的一部分,却是维系历史时期“泥沙淤积-洪泛决溢-治水应对”流域人水关系的核心。

本章将流域系统“稳态转换”现象的分析框架应用在黄河,这些转换可能是由系统内部的渐进或累积性变化所引起,也可能是由人类或自然干扰所引起。
在$2000$余年前黄河流域的土地已开始被广泛地开垦使用,这一历史时期降水量主要由气候周期规律决定,但不断增强的人类活动导致黄河的泥沙排放总体呈现上升趋势\cite{wang2007}。
在历史时期,气候的变化和不断积累的人类活动双重作用对黄河由低输沙转化为高输沙状态的变化均有贡献,梳理其稳态转换何时、因何原因主导而发生,是理解人-水关系长时间尺度变化的重要基础。
研究昨日之黄河有助于理解今日之黄河,回望水沙关系的长时间尺度变化有助于在决策时看得更远。
本章就结合历史重建数据分析了过去两千年间黄河水沙排放的变化与其对流域的影响,特别关注历史时期黄河从原始水平到沉积物丰富水平的稳态转换,解析黄河流域脱离“泥沙淤积-洪泛决溢-治水应对”人水关系反馈关系的过程与机制。

% 黄河的频繁泛滥对中央王朝的治理带来严峻挑战,也积累有关黄河成灾的“集体记忆”。
% 史书累牍的黄泛之殇,是新中国成立伊始毛主席感慨“把黄河的事情办好”的动力,也可能是黄河人-水关系长期维持“洪水-响应”模式的内在机理。
% 这种有关长时间尺度的洪水灾害模型中,集体记忆已被认为是影响社会响应方式的核心变量之一~\cite{dibaldassarre2015, dibaldassarre2019,ciullo2017}。
% 因此,在识别稳态转换的基础上,使用集体记忆模型在气候周期波动变化下,预测人类社会对洪水的响应模式,是解释黄河长时间尺度上人-水关系演变的关键。

% 本章的研究首先从稳态转换的视角梳理了黄河从低输沙到高输沙的演变过程,从而区分周期性的气候驱动与不断积累的人为因素,继而利用洪水记忆模型解释了人类活动未突破稳态临界点之前,气候灾害的周期及集体记忆变化,是黄河人水关系在历史时期维持中央政府主导“洪水-响应”模式的内在机制。

\chapter{绪论}\label{cha:introduction}


% 江恩慧 2022本子
长期的研究与探索,流域系统概念逐渐明晰。程国栋和李新(2015)在“黑河流域生态-水文过程集成研究”中提出了流域科学的概念,将流域视为地球系统的缩微,考虑水文和生态系统的自组织性如何影响流域系统的功能,以及人的因素如何被集成到流域水文学和流域生态学中。傅伯杰(2017)指出亟需聚焦人地系统耦合机理与调控途径,揭示黄河流域人水关系演变及社会-水文-生态系统动态。

\section{研究背景与意义}\label{sec:background}
\subsection{人地关系:地理学研究的核心}

% 帮王老师写的综述:人地系统结构与可持续V2
在人类活动的强烈影响下,地球进入“人类世”的新纪元,这意味着人类对地球改造的程度与后果可以与传统意义上的地质营力产生的影响相匹敌,成为环境变化的重要的驱动力\cite{lenton2019, lewis2015, lewis2018}。
这种来自人类的改造和控制在过去数世纪以来,已让气候变化、生物多样性损失和氮循环等关键地球系统生态过程超越了危险的“地球界限”,导致了众多资源、生态与环境问题\cite{steffen2015}。
如何在满足人类发展需求的同时,持续地保障地球生命支持系统的基本结构和功能,实现可持续发展,已成为学术界和社会各界广泛关注的重大科学和决策问题\cite{wu2014}。
人与自然地理环境密不可分,理解现代环境变化机理、持续地保障地球生命支持系统的基本结构和功能就需要发展人地系统整体的方法,耦合自然与人文过程,探讨变化环境下的系统耦合机制\cite{fu2015}。

在涉及人地关系综合研究的学科中,地理学以地域为单元着重研究地球表层人与自然的相互影响与反馈作用,对人地关系的认识素来是地理学的研究核心\cite{wu1991}。
无论是钱学森先生倡导的“地球表层学”、吴传钧先生提出的“人地关系地域系统”、黄秉维先生倡导“陆地表层系统科学”,均强调人与自然相互作用形成的人地关系复合系统,也就是人与地在特定的地域中相互联系、相互作用而形成的一种动态结构。
面对环境问题的复杂性,不同空间尺度上人类活动与自然环境的耦合关系也正在成为国际学界的主流话题\cite{fu2015}:美国科学基金会(NSF)于2001年就开展了自然与人类耦合系统动力学(Dynamics of Coupled Natural and Human Systems-CNH)研究计划,2019年进一步发展为了CNH2-社会-环境综合系统动力学计划;“未来地球”科学计划旨在推动自然科学与社会科学研究成果共同为可持续发展服务。
在人类活动影响力不断增强的背景下,要制定区域可持续发展战略,就必须深入了解人类赖以生存的地球环境系统与人类系统之间相互作用的基本过程,也就是关注人地系统演变及其机制的人地系统动力学\cite{fu2022}。

\subsection{流域系统:人水关系的研究单元}
% 开题报告
水是连接自然系统和社会系统的纽带,水循环过程是生态过程和社会发展的重要驱动力。人-水系统是以水循环为纽带将人文系统与水系统联系在一起的复合系统,是在流域尺度紧密相连的开放巨系统,也是人与自然耦合系统的典型代表[5,6]。
% “人”和“地”这两方面的要素按照一定的规律相互交织在一起,交错构成的复杂开放的巨系统内部具有一定的结构和功能机制,在空间上具有一定的地域范围,便构成了人地关系地域系统。

% 开题报告
人-水系统是社会和水文协同演化的复杂系统[8,11],既有以路径依赖为代表的演化特性,也有不确定性等复杂系统特征,这意味着对人水关系的理解与预测需要同时考虑演化的历史进程与未来变化的不确定性。
流域是人-水关系研究的完整单元,流域演化由快变量(如水文、经济、工程)与慢变量(如生态与社会文化)共同作用导致,社会和生态系统的慢变量经过长期累积决定了人-水系统的演化进程[12]。
例如水资源管理作为流域特定社会文化、经济水平和政治体制的直接产物,通过调整水量的分配进而作用于生态系统,决定生态系统的健康状况和社会系统的人类福祉。长期以来,流域水资源管理往往通过调控快变量,旨在短期内提高水的利用规模和效益,较少考虑系统状态的长期演变,对生态和社会慢变量的积累变迁更是缺乏反馈机制,限制了维持流域长期可持续的能力。此外,尽管因流域而异的演化轨迹是人-水系统复杂性的体现,但世界主要大河流域也展现出了相似的关键变量与作用路径。例如,在干旱-半干旱区的墨累-达令河流域、科罗拉多河流域以及黄河流域,水资源量的限制都促使水资源分配制度的诞生,该制度又以相似的作用方式影响了河流的水文状态。长期以来这些作用路径被认为是人-水系统演化中的突发政策性影响,因而忽略了其背后的一般性相互作用机制,制约了人-水关系调控的系统性和前瞻性。因此,识别流域人-水关系的长期演化,并理解复杂系统关键变量对演变过程的核心作用机制,对流域的可持续、高质量发展至关重要。


% 王帅 2023
大河流域一直是人类文化起源和发展的中心,通过粮食生产、水力发电和水源供给等给人类社会带来巨大收益,支持着众多的人口,具有显著的社会重要性并构成多样化的生态系统(Best, 2019)。
但是,人口增长以及对水、电、粮食和土地需求的增加(Crutzen et al., 2006),人类对水循环过程的影响已从外部动力演变为系统内力,给大河流域生态系统完整性和可持续性带来了前所未有的挑战(Sivapalan et al., 2019)。

% 王帅 2023
黄河流域生态保护和高质量发展是重大国家战略。流域生态环境脆弱、水资源保障形势严峻,上游局部地区生态系统退化、中游水土流失威胁依然严峻、下游生态流量偏低带来多重生态胁迫,这些问题都可归结为人地关系不协调,是我国人地矛盾最为突出和复杂的区域之一(傅伯杰等,2021)。
选取黄河流域为对象,研究流域人地系统结构特征及其时空演变,揭示流域社会-生态-水沙协同演变规律与耦合机理,以多主体模型集成人地系统耦合模型,将有助于:深化认识人地系统结构特征与耦合机理,为流域协调人地关系和促进协同治理提供理论框架和科学依据。

% 自己写的
人类社会系统与水文系统之间的互馈机制是人水关系研究的关键。

\subsection{人水关系:流域高质量发展之根基}


% 人水关系 匹配 comment
幸运的是,人类与水的关系并非完全难以捉摸。事实证明,社会发展与人水关系的变化之间存在着规律,这使得流域系统产生了可以解释和预测的共性问题,如过度和低效开发导致的水资源短缺。这表明,尽管我们有能力预测社会对水的需求和自然界水循环的变化,但仍然缺乏有效的理论和方法来匹配这两者。由于人类社会系统和流域水文系统的结构和功能具有不同的尺度和动态,匹配是指两者之间的良好关系。因此,我们需要深入了解人与水的关系及其动态变化,并找到实现匹配的方法,从而将流域引向可持续发展的轨道。

% 开题报告
黄河是中华民族的母亲河,“黄河宁,天下平”,治理黄河是千年夙愿。黄河流域大部分属干旱半干旱地区,流域面积占全国陆地面积的8.3\%,年径流量只占2\%,但却承担着全国12\%的人口、15\%的耕地和13个国家能源化工基地的供水任务[13,14]。水资源开发利用率接近80\%,为我国十大一级流域中最高,远超一般流域40\%的生态警戒线。可见,由于社会经济发展与自然生态系统过程严重失调,黄河流域已成为我国人-水矛盾最为突出和复杂的区域之一[15]。当前黄河流域面临的社会、生态问题既是在人-水关系长期演化中产生的现象,也是长期忽视人-水关系演化规律而进行水资源管理的后果。因此,无论是仅着眼于自然子系统或社会子系统的模型,还是局限于支持水资源年内和年际调度的规划,都难以满足黄河流域长期、可持续利用水资源的需要。缺乏对人-水关系演变过程与机制的深入理解正严重制约着流域的高质量发展。
综上所述,人类对自然水循环的影响已从外部动力演变为系统内力,人-水系统耦合机理是地球系统科学的前沿,也是应对环境变化挑战、保持人水和谐、实现可持续发展的重要科学基础。在流域尺度理解人-水关系演变的过程与机制是解耦人-水系统这个复杂开放巨系统的关键,阐明黄河流域历史和当代人-水关系的演变过程,识别人-水关系演变的核心机制,能够为黄河流域可持续发展提供关键科学依据。

% 人水关系 匹配 comment
水不仅是地表生态过程的一个重要组成部分,也是人类社会发展所依赖的重要资源。目前,大江大河流域支撑着xx\%的重要生态系统,为世界上xx人提供了生存的水资源。因此,可以毫不夸张地说,在社会与生态系统高度相关的大河流域,河流是文明发展的血液。然而,随着社会的发展,人类活动改变了流域的自然和社会水循环过程,导致大部分流域出现不可持续的发展轨迹,人类与河流的关系变得空前紧张。

% 人水关系 匹配 comment
人类的水关系取决于我们如何对待河流,以及我们从流域中得到什么。就像恒河的信徒相信河流能净化他们的灵魂,尼罗河的农民期待着河流的丰收,胡佛站在大坝前骄傲地宣布他已经 "征服 "了科罗拉多河。社会越来越多地为了自己的福祉而改造流域,尽管水经常提供许多其他关键功能,如调节气候、支持生物质生产、物质运输和净化环境。这些功能对于可持续发展可能更加重要,以保持流域社会生态系统的弹性,使其能够适应外部变化或在转型中摆脱危机。然而,在许多情况下,人类改造流域的活动作为干扰超过了系统的临界点,导致维持核心水功能的反馈回路发生变化,引发社会生态系统的稳定转变。为了避免可能导致系统崩溃的稳态转变的级联效应,必须将人水关系的变化视为流域社会生态系统的内部原因,特别是在人类影响的人类世。



\section{研究进展}\label{sec:introduction}
\subsection{流域系统的理论框架}\label{ch1:sec:theories}
\subsubsection{水的自然与社会属性}

水具有自然属性和社会属性两种不同的属性\cite{ning2004}:自然属性包括水的形态与组成、物理性质与生态功能,以及水参与的地球系统过程等自然特征;而社会属性则包括水的社会功能和社会影响,例如水对个体或集体的认知影响以及水的社会经济功能。

% 开题报告
水在自然界中具有重要的生态功能,不仅参与水文地质循环和生态水文过程,还是大多数陆地生态系统的主要限制因素和物质运移的重要介质,对于生态系统的调节、支持和净化具有至关重要的作用,维持着流域社会生态系统的弹性,使其能够应对外部变化并实现可持续发展\cite{gleeson2020a}。
但是随着人类活动的干预,大河流域的水循环过程受到了严重的破坏,导致其所承担的社会、生态功能已经接近“地球边界”的安全界限\cite{gleeson2020}。
除了自然属性,水还具有重要的社会属性。水不仅是人类生活必需品,而且贯穿了自然、养育、实践和象征等多个方面,具有非常丰富的社会功能和影响\cite{zhangyahui2008}。
水的社会属性可以从“实践理性”和“文化理性”两个方面出发,前者是指在水的生产实践中因控制、竞争、分配、排斥而产生的人水互动;后者是指因水观念、水文化而产生的人水互动\cite{zhangyahui2008}。
除了物质本身的存在,水还具有符号、历史、政治等多种意义,影响着人们的日常生活和流域的政治经济体制\cite{ballestero2019}。

水资源管理是流域特定水文条件、社会文化、经济水平和政治体制的综合产物,通过调整水量的社会分配进而影响生态系统,对于生态系统健康和人类社会福祉具有至关重要的作用。
根据 Stephanie 的总结\cite{scarrow2021},当今人\textendash{}水关系演变的研究可以分为“水的社会性”和“水的技术性”两类。
例如,《大河与大国》\cite{MaDing2021}与《征服自然》\cite{DaWei2019}两项著作分别以美国和德国为例,从“水的社会性”与“社会控制水”两个角度系统梳理了两国主要大河流域的人\textendash{}水关系演变史。前者高度强调“水的社会性”是塑造当代美国社会性质的重要自然因素;后者则指出德国出于征服和控制的考虑,利用技术永远重塑了自然水文景观。
长期以来,流域水资源管理往往通过工程措施短期内提高水的利用规模和效益,而较少考虑系统状态的长期演变,并缺乏反馈机制来处理生态和社会慢变量的积累变迁,限制了维持流域长期可持续的能力\cite{falkenmark2021}。
而可持续水的利用和管理不仅需要考虑其自然特性,还应兼顾其社会功能和经济效益,这需要相关研究耦合水的自然和社会属性,以实现流域可持续发展的目标。

\subsubsection{自然\textendash{}社会二元水循环}

王浩等人提出的“自然\textendash{}社会二元水循环”理论指出,流域水循环受到人类活动影响,呈现出“天然\textendash{}人工”二元特性\cite{wang2006}(图~\ref{ch1:fig:two_water_cycle}~A)。
该理论强调需要以“蒸散发管理”和“耗水管理”为核心来管理自然水循环和人工水循环,提出“以蒸散耗管理为核心、七大总量控制为约束”的水资源管理理念,在水资源评价工作中兼顾了水的自然和社会属性\cite{wang2010}。
该理论的发展也得到了不少学者的补充和完善,例如王浩与贾仰文指出水循环的演变效应也是该理论的研究重点\cite{wang2016},邓铭江等则提出了“自然\textendash{}社会-贸易”三元水循环模式,解释西北干旱区内陆河流域水循环的机理\cite{deng2020}。
该理论的提出强调了水资源管理需要考虑自然和社会属性,应对强烈人类活动干预为水资源评价和管理带来的挑战。

自然\textendash{}社会二元水循环理论以平衡态为基本假设,同时考虑了许多人类活动的影响。在水量估算与建模评价上具有理论优势,主要研究手段是原型观测、物理模型和数学模型。国内学者基于二元水循环的概念模式主要在水资源和水生态方面开展评价管理研究,包括识别循环结构、多尺度多过程分析、演变规律、未来预测与调控等\cite{wang2016}。
例如,刘家宏等人在海河流域应用该理论,构建水平衡方程厘清“自然\textendash{}人工”二元水循环结构,借助数据定量识别了该结构中各部分的数量关系\cite{liu2010}。
王浩\cite{wang2004}、周祖昊\cite{zhou2022}等人在长达近二十年的时间里,将二元水循环理论从黄河的无定河小流域拓展到整个黄河流域,从初步的二元水循环要素到综合考虑气象、下垫面、人类取用水、水利水保工程、水库调度等诸多要素,不断拓展理论应用的时空尺度。
黄强等人采用小波分析的方法对黄河二元模式的逐年演变规律进行了探索\cite{huang2002};裴源生等人采用该理论改进了水量、水质、水效的联合调控方法\cite{pei2020}。
可见,“自然\textendash{}社会二元水循环”通过考虑人类社会系统对水循环的影响和对水资源的消耗,对指导现代流域水资源管理实践大有裨益。但是,其底层仍基于平衡态的工程思想,在人与水的互馈作用研究上有所不足。

\begin{figure}[!ht]
    \centering
    \includegraphics[width=\textwidth]{img/ch1/ch1_two_water_cycle.png}
    \caption[流域人水耦合系统的典型理论示意图]{流域人\textendash{}水系统耦合的典型理论示意图\cite{wang2006,dibaldassarre2015}。蓝色代表自然水循环的典型过程;黄色代表社会水循环的典型过程。}\label{ch1:fig:two_water_cycle}
\end{figure}

\subsubsection{社会水文学}

% 开题报告
社会水文学旨在理解人-水之间协同演化规律和循环互馈机制,多年来在现象检测、机理分析、模型预测等方面都获得了长足发展\cite{sivapalan2012, blair2016, srinivasan2016}。
社会水文学在诞生之初便以流域系统为单元分析人与水的互动反馈过程(图~\ref{ch1:fig:two_water_cycle}~B),其最根本的特征是将社会\textendash{}水文系统视为动态系统,并将人类社会相关变量作为系统内部的驱动力,而非像传统水文学那样假设水文系统是在人类外部干扰下处于平衡态的\cite{sivapalan2012}。
Konar等人将社会水文学的主要研究方向总结为四个:水循环与水资源利用、人与干旱之间的相互作用、人与洪水之间的相互作用、人与政策制度的相互作用\cite{konar2019}。
Yu等人则指出该学科经多年发展后呈现出三个特征:在水循环的不同时空尺度下开展研究、将人类文化的演化特征纳入研究、将基础设施建设对水循环的干扰纳入分析\cite{yu2020}。
在这种超越传统水文学的思想指导下,一些社会\textendash{}水文现象得到揭示,例如增大用水效率却常常无法节约流域水资源的“用水效率悖论”\cite{grafton2018, xiong2021},以及流域管理策略常在开发和保护之间周期性摆动的“钟摆效应”\cite{kandasamy2014, roobavannan2017, mostert2018}。

社会水文学的发展同时也面临着诸多挑战。
Troy 等人通过文献分析,认为社会水文学研究尚集中在数据收集与整理、数据观察与推测、理论模型建立三个阶段,还缺乏对成熟的参数率定与模型预测,因此其预测能力相对较差,对指导政策制定还相去甚远\cite{troy2015}。
Sanderson 认为这种糟糕的模型表现很大程度上因为社会水文学并没有真正将社会因素纳入考量,因而发出了“社会水文学需要社会科学”的呼吁\cite{sanderson2017}。
然而,社会科学常常没有统一、公认的理论,且人的主观能动性与文化变量均难以被模型捕捉。复杂系统的科学思想能够将复杂的人类行为纳入分析框架,被认为是社会水文学未来重要的前进方向之一\cite{ahlstrom2021}。
综上所述,社会水文学的发展让人们对水问题背后的机制有了更深入的了解。分析与水相连接的社会动态是对传统的水文学的补充。但仍需要进一步在方法学上突破,结合多学科背景和复杂系统思想来分析社会\textendash{}水文系统,以突破上述瓶颈。


\subsubsection{流域社会\textendash{}生态系统}

% Handbook 什么是社会\textendash{}生态系统
社会\textendash{}生态系统(Social-ecological system, SES)的概念最早诞生于20世纪90年代中期,由生态经济学研究公共池塘资源问题的学者通过结合系统科学方法和适应性管理提出的跨学科概念\cite{biggs2021},此概念有助于整合水的自然/社会属性研究\cite{fowler2022}。
它是一个典型的复杂适应性系统(complex adaptive system),由许多互相独立的部分组成,并以涌现的方式相互作用,系统层面的格局难以由某部分的属性来预测\cite{schluter2019}。
因此,社会\textendash{}生态系统不等同于社会系统与生态系统的简单加和,而是由社会和生态组分之内/之间反馈所塑造的有机整体\cite{biggs2021}。
社会\textendash{}生态系统理论发展至今已经产生了许多流行的框架,包括 Folke 和 Berkes提出的社会\textendash{}生态系统概念框架\cite{berkes2008};将系统弹性描述为不同尺度适应性循环结果的扰沌框架\cite{gunderson2001};Liu 等人提出的远程耦合框架\cite{liu2018};Ostrom 分析公共池塘资源的社会\textendash{}生态系统诊断框架\cite{ostrom2009};Schluter 等人提出的社会\textendash{}生态系统行动情景框架等\cite{schluter2019}。

随着可持续发展目标的提出,社会\textendash{}生态系统理论在21世纪得到广泛认可,作为跨学科概念反哺社会水文学,逐渐形成了“流域社会\textendash{}生态系统”的概念。
Huggins对全球流域社会\textendash{}生态系统在水资源压力下的脆弱性进行了评估,分析了淡水资源压力不断加剧对社会\textendash{}生态系统的潜在影响,识别了全球的热点流域\cite{huggins2022}。
Varis综合考虑了社会系统的三种适应力和生态系统的三种脆弱性因素,在全球尺度评估流域社会\textendash{}生态系统在弹性和适应之间的平衡\cite{varis2019}。
国内也逐渐接受流域系统是社会\textendash{}生态系统或复杂系统的观点,重视系统科学在流域综合研究中的重要性。
程国栋和李新在“黑河流域生态-水文过程集成研究”中提出了流域科学的概念,将流域视为地球系统的缩微,考虑水文和生态系统的自组织性如何影响流域系统的功能,以及人的因素如何被集成到流域水文学和流域生态学中\cite{cheng2015}。
傅伯杰等人指出亟需聚焦人地系统耦合机理与调控途径,揭示黄河流域人\textendash{}水关系演变及社会\textendash{}水文-生态系统动态\cite{fu2021}。
一些实证研究也开始从不同角度分析流域社会\textendash{}生态系统,包括结构变化\cite{song2022}、制度变化\cite{wang2019c}、社会意识\cite{liu2023}等对系统局部、流域系统、外部系统等产生的不同影响,证明了思想与社会\textendash{}生态系统理论对流域研究的重要性。

\subsubsection{人\textendash{}水关系与人\textendash{}水系统}

正如中文语境的“人地关系”在英语世界一般等价于“human-environment interactions”\cite{li2016c, liu2023},“人\textendash{}水关系”直译的“human-water relationship”也并非英文世界的常见关键词,取而代之的是通常不限定时空尺度的术语“human-water systems”,即人\textendash{}水系统\cite{konar2019}。
人\textendash{}水系统是以水循环为纽带,将人文系统和自然系统联系起来并组成的复杂系统,能依靠自身循环动力和经济发展动力而演变\cite{zuo2007},也是一类典型的社会\textendash{}生态系统,因此“流域社会\textendash{}生态系统”与“流域人\textendash{}水系统”通常同义\cite{yu2020}。
左其亭认为中文语境常用的概念“人水和谐”不曾有严谨的学术定义\cite{zuo2007},指出“人\textendash{}水关系”应指人\textendash{}水系统中“人文系统”与“水系统”之间的关系,“人水和谐”则是对人\textendash{}水系统要素间关系的评价,并在此基础上提出了与“人\textendash{}水关系和谐论”与“人\textendash{}水关系学”\cite{zuoqiting2022, zuo2016a}。
但是,“人\textendash{}水关系学”侧重对人\textendash{}水系统进行整体评价,以指导水资源分配与流域管理,因此“人\textendash{}水关系”仍需被具体化为人类与水资源的互动关系\cite{zuo2016, zuo2020a}。

综上所述,许多理论框架,如“自然\textendash{}社会二元水循环”、“社会水文学”、“流域社会\textendash{}生态系统”和“人\textendash{}水关系学”,从不同侧重点出发重视水的自然\textendash{}社会双重属性研究。
其中,“自然\textendash{}社会二元水循环”基于平衡态假设,已对多个流域在人类影响下的水平衡模式进行了计算与建模,服务于水资源管理工程;“社会水文学”关注“人”与“水”之间的协同演化,在非平衡态假设下揭示了流域演变规律;“流域社会\textendash{}生态系统”或“流域人\textendash{}水系统”的概念结合了社会水文学与系统科学,强调整体性和协同演化的复杂性,是目前流域系统研究的前沿。
然而,目前流域尺度上仍缺乏对“人\textendash{}水关系”的具体定义,相关研究在不同时空尺度下均使用过于宽泛的“人\textendash{}水关系”概念,这给分析流域人\textendash{}水系统的演变过程与演变机制带来了阻碍。
鉴于在实际研究中,研究者绝无可能穷尽流域人\textendash{}水系统的全部作用关系,本研究将“人\textendash{}水关系”的涵义界定为:人类活动直接改变水圈要素、过程,或人类决策反之受到水圈要素、过程不可被忽略的干预或影响时,人与地球表层系统的相互作用。


\subsection{流域系统的演变规律}\label{ch1:sec:process}
系统演变路径是社会-生态系统研究的核心内容,演化多样性是人水系统复杂性的体现,人水系统的演变过程更是流域社会-水文研究热点。根据侧重点的不同,当前人水系统与社会-生态系统演变过程的研究可以大致分为“概念模式变化”、“关键指标变化”、“结构功能变化”三个方面。

其中概念模式便是对许多共性规律的高度抽象或理论总结,以便研究人员和决策者在宏观层面把握人水系统的演变过程;关键指标是通过能够表征流域系统的关键变量或综合指标的变化规律把握流域演变过程;结构功能变化则是借由流域系统内重要组分的相互关系变化、以及于这种变化相关的系统功能改变来把握流域演变过程。

\subsubsection*{概念模式的变化}

首先,此类人水系统演变研究可以通过借鉴其它学科或其它领域的已有理论或概念提出。
例如张家诚(2006)从科学哲史和科学哲学的角度出发回顾人水关系的演变历史,指出人水关系从古代时“天-地-人的平衡中庸模式”,到近代工业社会变为在追求自己的发展时忽略自然环境的“数学模式”,同时展望了信息时代人水关系的“工程调控模式”\cite{zhang2006}。
Gleick 和 Palaniappan 借鉴“石油峰值”为流域提出了“水峰值”的概念,借助该概念可将流域潜在水供应分为三个变化阶段:阶段一随着用水需求的增加可用水资源的供应(新修水坝、水库、泵);一旦达到最大成本效益的地表水和地下水开采;有一个最终转移到一个更高的成本支撑的供应水如海水淡化或转移等各种来源的增量增加供应\cite{gleick2010}。

另一方面,人水系统演变的概念模式还可以通过流域的实际发展规律总结得出,这类理论模式通常有更活跃的理论生命力,也更能有助于指导后续研究和决策。
Turton 在1999年提出了影响深远的“流域适应能力”概念框架,指出根据水资源的供需关系,流域人水系统随着发展可能依次在“获取更多水资源供应”、“提高用水效率”、“提高分配效率”、“适应水短缺状态”四种原型模式间演变\cite{turton1999}。
该框架在澳大利亚流域水治理改革案例中的应用不仅佐证了其重要指导意义,还暗示了流域演变过程可能还存在水需求下降的“第五阶段”\cite{loch2020}。
另一个典型的案例是通过澳大利亚东部 Murrumbidgee 河流域总结出的“钟摆效应”模式,它说明了流域人水关系在取水用于粮食生产和努力缓解流域环境退化之间保持动态演进,并在摇摆间将流域人水系统演变划分为四个时期\cite{kandasamy2014, roobavannan2017},而这一规律随后也在中国和欧洲等更多区域得到了复现\cite{han2017, mostert2018}。

\subsubsection*{关键指标变化}

由于人水系统的复杂性,研究常从不同角度切入构建综合指标,用以表征人水关系的变迁。
刘海猛等人基于复杂自组织系统理论,在辨析人水系统基本内涵的基础上提出了人水关系演化的概念模型,将人水系统的演变过程概念化为社会经济、生态环境、水资源开发利用三个变量组的函数关系\cite{liu2014}。
Zuo 等人将“和谐人水关系”指标分为健康度、发展度、协调度\cite{zuo2008},该综合指标可将中国的人-水关系分为二十世纪中期以前、1950年至1980年、1980年至1990年、1990年之后四个时间阶段,各阶段的流域人水关系的主要特征分别是:应对用水需求和水灾、管理供水和去中心化、依法管理水资源、高度重视人水和谐\cite{zuo2016a}。

这种通过关键指标识别人水关系的方法须广泛收集来自不同领域的数据,在时间序列数据收集和数据同化上存在诸多挑战,但其优势在于一旦数据可用,能同时对全球各大流域进行大规模计算与分析。
Varis 等人(2019)基于三个社会系统的适应性指标和三种生态系统的脆弱性指标,构建了综合指数评估流域社会-生态系统在弹性和适应之间的平衡,分析了各流域通过发展提升适应性的演变过程\cite{varis2019}。
Huggins(2022)则更侧重流域在水资源压力下的脆弱性,通过全球各流域的水资源可用性指标和人类面对压力的适应指标,综合分析了不断加剧的淡水资源压力对流域社会-生态系统的潜在影响路径\cite{huggins2022}。
Qin 等人(2019)提出的稀缺性-韧性-易变性(SFV)指标,考虑了管理措施(如水库的建设)和用水结构变化(有些用水方式如能源用水是难以被短期替代的),选择了三个子指标对流域人类活动情况下水资源短缺情况做出评估,并分析了不同发展程度的地区在面对水资源短缺时的可能发展路径\cite{qin2019}。

\subsubsection*{结构功能变化}

流域系统的结构变化是其功能改变的必要条件,二者都是流域系统层面的要素特征,流域系统的变化常常同时伴随着结构和功能的变化,但在研究途径上有很大不同。

通过结构变化识别流域系统变化时,最简单直接的方法是寻找决定系统功能的关键变量,如Wang 等人提出了“矛盾统一”的人水关系理论框架,通过识别人与水之间是否是“对抗”关系,在两千年时间尺度上拟合了中国的人水关系变化\cite{wang2017}。
水-粮食-能源是另一组常被用来表征流域人水关系演变的要素,但 Rollason (2021)经研究指出,三者间的关系可能比许多研究中所表达的更为复杂\cite{rollason2021},这表明在利用关键要素间关系进行识别时,依赖于研究者对关键变量的精准把握,包括需要纳入分析的要素有哪些,以及如何建构它们对关系。 % todo: \cite{水-粮-能...}

改进上述问题的一个思路是使用复杂网络分析方法,尽可能穷尽当前人水系统中所有需要考虑的因素,进而使用复杂网络指标分析系统演变。 % todo: \cite{network...}
例如 Song 等人(2022)参考经济复杂性概念,使用全国的年均虚拟水足迹数据集(1978年至2008年)构建了农产品-虚拟水转移量的二分网络,通过复杂网络分析计算了网络复杂性随时间的变化,进而分析黄河流域水资源的重要性演变规律\cite{song2022}。
Sayles 等人(2017)结合生态和生态修复政策的数据,构建了流域的社会-生态网络,发现存在潜在结构问题的区域合作网络的密度和生产力都最弱的\cite{sayles2017}。
识别网络结构需要收集相对全面的关系数据集,对数据质量要求较高,这一定程度上限制了此种研究方法的广泛应用。

最后,还可以直接从系统功能的角度出发寻找表征系统演变的证据,该方法易于在不同流域系统中使用,但依赖对人水系统功能的正确认识,因此得到广泛认可的方法指标相对较少。
Falkenmark 等人提出的“蓝水”和“绿水”概念\cite{falkenmark2006},以及后来提出的“灰水”,都是典型的从系统功能角度出发,研究流域结构-功能变化的理论框架。 % todo \cite{灰水...}
传统水资源规划和管理针对的重点仅是河道的液态水(蓝水),大多数在流域面上以降水和蒸散的形式参与循环(绿水),此外还有少部分经人类利用后又参与循环的水资源(灰水),三者在流域人水系统中承担着截然不同的功能\cite{craswell2007}
随着流域社会经济发展和气候变化,流域水循环各部分的绿水和蓝水比例都会发生变化,这将对人水系统功能产生极为深远的影响,如降低弹性、削弱流域系统的生态系统服务、甚至导致系统崩溃\cite{falkenmark2019}。


\subsection{流域系统演变的机制}\label{ch1:sec:mechanism}
社会系统和生态系统的互馈机制是社会-生态系统研究的核心内容,也是当前社会 -生态系统研究的热点和前沿。根据侧重点的不同,当前社会-生态系统互馈机制的研究 可以分为“时”、“空”、“构”、“阈”四个方面


\subsubsection*{结构-功能}

\subsubsection*{人水匹配}
% 人水关系 匹配 comment
由于人类社会系统和流域水文系统的结构和功能具有不同的尺度和动态,匹配是指两者之间的良好关系。

\subsubsection*{动力学}

\subsubsection*{稳态转换}

总体来说,尽管人-水系统协同演化的理论不断发展,相应的量化分析工具也在不断丰富,但多见诸于解耦自然过程与社会过程对人-水系统的影响,对其系统的演变机制鲜有探讨。而且,这种解耦常常建立在人-水关系已经发生变化的经验主义基础之上,因而在黄河流域多集中于人类强烈干预的人-水沙关系等问题的研究,忽略了人-水资源关系的量化分析及理论探讨。

\subsection{流域系统人水关系模型}\label{ch1:sec:model}


% 开题报告
模型的建立是为了理解、描述和预测自然界,应当在真实性、预测性、普遍性之间达到均衡[27],流域系统模型通常旨在表征水的地球物理过程的动力学,以及人类用于管理系统的元素,如基础设施、机构和治理。 %\cite{hadjimichael2020}。
流域人水系统既有空间尺度依赖的地表生态过程,也有系统层次变量反馈的动力过程,还有多尺度的复杂人水相互作用[47–49]。
相应地,流域人水系统的建模主要有基于传统分布式水文模型耦合人类活动模块发展而来的分布式社会-水文模型、在系统或区域层次上耦合来自社会、自上而下模拟生态系统关键变量的系统动力学模型、自下而上对流域内复杂人-水互动进行仿真的多主体模型。

\subsubsection*{分布式社会-水文模型}

% 焦 本研一体 本子
与传统的集总式水文模型相比,分布式流域水文模型不再将流域视作均匀的整体,充分地考虑了流域内水文过程的异质性[207],是流域研究的主流工具,常见的分布式水文模型如如SWAT模型、新安江模型、陕北模型、布式时变增益水循环模型等(徐宗学,2019)[208-210],在国内外都得到了大量应用[211-213]。 % 焦 本子
% [207]王中根,刘昌明,吴险峰.基于DEM的分布式水文模型研究综述[J].自然资源学报,2003(02):168-173.
% [208]J. G. Arnold,R. Srinivasan,R. S. Muttiah,J. R. Williams. LARGE AREA HYDROLOGIC MODELING AND ASSESSMENT PART I: MODEL DEVELOPMENT[J]. JAWRA Journal of the American Water Resources Association,1998,34(1):73-89.
% [209]王中根,刘昌明,黄友波.SWAT模型的原理、结构及应用研究[J].地理科学进展,2003(01):79-86.
% [210]夏军,王纲胜,吕爱锋,谈戈.分布式时变增益流域水循环模拟[J].地理学报,2003(05):789-796.
% [211]Karim C. Abbaspour,Jing Yang,Ivan Maximov,Rosi Siber,Konrad Bogner,Johanna Mieleitner,Juerg Zobrist,Raghavan Srinivasan. Modelling hydrology and water quality in the pre-alpine/alpine Thur watershed using SWAT[J]. Journal of Hydrology,2006,333(2):413-430.
% [212]Darren L. Ficklin,Yuzhou Luo,Eike Luedeling,Minghua Zhang. Climate change sensitivity assessment of a highly agricultural watershed using SWAT[J]. Journal of Hydrology,2009,374(1):16-29.
% [213]Gangsheng Wang,Jun Xia,Ji Chen. Quantification of effects of climate variations and human activities on runoff by a monthly water balance model: A case study of the Chaobai River basin in northern China[J]. Water Resources Research,2009,45(7).
% 江 本子
流域分布式模型通过耦合生态过程,可用于描述大尺度流域陆地生态演变过程的生态水文模型,如SWIM模型(Krysanova等,2005)、RHESSys模型(Tague和Band,2004)、Budyko–Choudhury–Porporato模型(Donohue等,2012)、EHSM模型(Viola等,2014)、HYMOD-BGM模型(Tang等,2018)等。
国内的包括EcoHAT模型(刘昌明等,2009)、WEP-IBIS模型(Cao等,2015)、CLM-GBHM模型(Jiao等,2017)、GBEHM模型(Qin等,2017)、BEPS-TerrainLab模型(Chen等,2007)、HEIFLOW(Tian等,2018)等。

% 焦 本子
现有的分布式水文模型由于构建原理及最初率定区域不同,导致模型侧重点有所不同。如SWAT模型侧重描述产流过程[209],LISTFLOOD模型侧重模拟水动力过程、洪水过程等[217],但现有的分布式模型仍较少将人为干预水文过程的因素作为模拟重点,
% [209]王中根,刘昌明,黄友波.SWAT模型的原理、结构及应用研究[J].地理科学进展,2003(01):79-86.
% [217]曾照洋,王兆礼,吴旭树,赖成光,陈晓宏.基于SWMM和LISFLOOD模型的暴雨内涝模拟研究[J]
贾仰文等人WEP-L分布式流域二元水循环模型(简称 WEP-L 模型)是具有物理机制的流域分布式水循环模型,考虑了人类取用水和水利水保工程等因素对水循环过程的影响,实现“自然-社会二元水循环”过程耦合模拟和分析。 % todo citation
2019年,国际应用系统分析研究所(International Institute for Applied Systems Analysis, IIASA)开发了基于社区的水文模型模型(Community Water Model, CWatM)模型,将水库调度等水资源管理要素也纳入了模型[218]。
但迄今为止,仍鲜有将水资源治理制度(如法律法规)等人类活动要素的影响作为流域分布式模型模拟的重点。
% [218]Peter Burek, Yusuke Satoh, Taher Kahil,et al. Development of the Community Water Model (CWatM v1.04)- a high-resolution hydrological model for global and regional assessment of integrated water resources management[J].GEOSCIENTIFIC MODEL DEVELOPMENT,2020,13(7):3267-3298.

\subsubsection*{自上而下的系统动力学模型}

% 江本子
系统动力学模型能够解析流域系统层面的要素关联、反馈与演化(Jaeger等,2017;Jiang等,2022),可预测变化环境下流域系统关键变量及其反馈过程的变化(Vaighan等,2017)。

\cite{muneepeerakul2017}提出了一个框架和正式的风格化模型,以探讨在什么情况下稳定的治理结构可以在由共享的自然、社会和建筑基础设施组成的耦合基础设施系统中内生地出现。

V等人

流域人水系统的系统动力学模型是一种用于研究人类活动与水资源系统间相互关系的模型。该模型建立在系统动力学理论的基础上,结合社会经济、环境、生态等多个领域的相关知识,通过分析系统间的相互作用、内部结构变化等现象,揭示流域人水关系的动态特征。

% 开题报告
系统动力学模型被用于解释和预测人类社会面对流域频发的洪水灾害时,水文和社会系统组件(如公众应对、风险文化、经济发展等)的互馈作用[43]。

\subsubsection*{自下而上的多主体模型}
% 开题报告
涌现(Emergence)指系统实体会产生其所有组成部分本身没有的属性,人-水系统作为开放的复杂巨系统,广泛存在的相互作用就是宏观演化属性涌现的关键[51]。
% 来自 chat GPT
流域人水系统的多主体模型是指在研究人类活动对水资源的影响时,将不同的相关主体,如政府、水用户、环境保护者等作为系统的不同主体分别进行考虑,并综合考虑它们之间的相互影响关系,以达到更加全面、系统地研究人水关系的目的。这种模型在人水关系研究领域越来越受到关注,为了更好地研究人类活动对水环境的影响,并为流域水资源管理提供有力的指导。

自组织(Self-organization)是指一种起源于初始无序系统的部分元素之间的局部相互作用、所产生出某种形式的整体秩序的过程。这与复杂系统建模中自下而上的基于主体的建模(Agent-based model) 思想类似。
因此Castilla-Rho等人[61–63]利用多主体建模的复杂系统模拟方法将Ostrom发现的机制应用于地下水流域的管理模型中,发现可以解释地下水资源治理模式的涌现。但对于大河流域来说,由于人能直接观察到流动的水资源变化,且存在明显的时空分异性,尚缺乏较好的复杂系统建模以探索其中人-水关系的核心演化机制。

自下而上模拟流域水资源使用时上中下游不同行业利益相关者的冲突、合作,及其相互作用下治理体系的整理结构与功能,逐渐成为流域可持续治理的重要基础[53,54]。


刻画人水关系的模型和计量手段日益丰富,如应用多主体模型揭示了逐水而居的本能可能是人类早期迁徙演化的重要驱动力[42];

由于人类社会系统和流域水文系统的结构和功能具有不同的尺度和动态,匹配是指两者之间的良好关系。
Sayles 2017 研究了流域尺度社会-生态的匹配

总的来说,基于复杂系统的建模已成为研究人-水系统的重要技术手段,能够基于水的资源属性对利益相关者的人-水互动进行理论机制上的探索。这种自下而上的建模思路与SES的自组织管理机制一脉相承,但目前比较成熟的模型主要关注地下水流域和,对河流的水资源治理,尤其是水资源稀缺的大河尚缺乏泛用性较强的机制模型。


\subsection{黄河流域人水关系研究}\label{ch1:sec:yellow_river}

% 开题报告
黄河流域是世界上人-水关系变化最频繁、最剧烈、影响最为深远的大河之一,人类社会与黄河相互作用的研究历史源远流长。
% 从水的社会性上看:大禹治水的故事对中华文明的脉络有着深远的影响[30];三门峡与下游运河的运输功能对历代王朝的国祚可谓牵一发而动全身[30];近年来水资源的区域间分配也对经济发展有着制约性影响[31]。
% 从社会控制水的角度上看,历代中央王朝的黄河洪泛治理对水文系统变化影响深远;引黄灌溉则是这个干旱-半干旱区流域的农业经济发展命脉;近年来水库的调水调沙更是彻底重塑了黄河下游的自然生态环境[32]。
% 黄河流域这种频繁的人-水关系变化常常为梳理人-水系统复杂的的反馈循环与协同演化带来了更大的困难[12,33]。
如魏夫特考察黄河流域的社会发展历史后提出了著名的“水利社会理论”,认为这样大规模的治水活动是中央封建王朝皇权诞生的重要因素;但更多中国学者认为魏氏倒转了复杂人-水关系中的因果,正是中央皇权的大力发展才使得大规模治水活动变得可能[35,36]。
治水方略的选择通常被认为是决定黄河河道演变的重要人为因素,同时河道的自然条件又是选择治理方略的重要依据,由两者共同决定的治理结果则会造成路径依赖[34]。
因此,为了梳理黄河流域的人-水关系变化,也不乏着眼于总结演变过程的著述。
于宏瑞将黄河流域的人-水关系演变的典型事件按区域(上、中、下游)进行了细致梳理[19],但对跨时空的事件关联着墨不多。
也有研究倾向于按照黄河流域人-水关系的关键主题梳理演变进程,力求展示出其时间纵深和因果联系,如葛剑雄从黄河与中华文明互动的角度[30],王渭泾从黄河治理的角度[34],分别对近两千年来黄河人-水关系的演化史进行梳理。
但既有研究的常对人水系统进行“专题梳理”,且以描述概念模式为主,无法对流域人水系统的演变进行定量分析,难以满足对未来演变方向作出解释的科学需要。

% 于璐 本子
黄河流域还是社会-水文学研究人水系统协同演化机制的热点案例。
黄河流域长时间受人类活动强烈干扰[48],流域系统内变量关系复杂存在诸多子系统(如水沙子系统、社会经济子系统、社会-生态子系统等),素来是稳态转换研究的代表性区域。
目前黄河流域系统人水关系演变机制的研究主要集中在评估人类活动影响(Shi et al., 2022)、人为干预治理或资源管理的效果与影响(Feng et al., 2016; Zhou et al., 2021)等方面,包括水库调控和人类用水对流域人水关系的影响等\cite{wang2019c}。
在分析人-水系统关系演变机制的方法上,既有研究多集中于指标和结构层面;指标评估如耦合协调度(李波等,2022;赵良仕等,2022)、承载力(Wang et al., 2022)、多维指标评价(陈莉等,2022)等;结构层面如网络(Song et al., 2022; Zhi et al., 2020; 张伟丽等,2022)、供需(Wang et al., 2019; Yin et al., 2021; Zhang et al., 2021; 孙久文等,2022)等,这些方法多依赖于对长期、结构化时间序列数据的收集,因此难以应用于非工程性质的、非连续的流域治理制度等人水系统驱动力的机制分析。
整体上,黄河流域人水关系演变机制的研究对非工程的水资源治理措施的影响机制严重不足,难以对流域规划中政策影响下的未来情景进行预测,对流域治理政策的指导意义不足。

针对黄河流域人水系统关系复杂的特点,也诞生了一些同时考虑人类活动与水文生态的模型。
% 焦 本子
贾仰文等基于WEP模型开发了WEP-L模型,通过划分子流域及等高带的方式实现了黄河历史径流复现[214,215],杨大文等构建了分布式水文模型复现或表征了黄河流域多个水文参数[216]。
% 江 本子
岳瑜素等(2020)应用系统动力学模型优化了黄河下游滩区自然-经济-社会协同的可持续发展模式;Jiang等(2021)建立系统动力学模型,从社会、经济、资源、生态、文化五个方面预测了在未来政策情景下黄河流域高质量发展水平;黄昌硕等(2021)建立了基于“经济社会-水资源-生态环境”系统动力学模型,为动态预测区域水资源承载力的变化趋势和制定优化调控方案提供了参考依据。
% 自己写的
相较发展和应用均较为广泛的分布式水文模型和系统动力学模型,多主体模型研究相对较少。
xxx使用多主体模型模拟了流域主体对水资源调度的影响,优化黄河流域的水资源分配,但模拟的主体数量较少且主体间互动仅考虑工程要素,与水资源管理的联合多目标优化相似。

综上所述,既有研究的常按主题梳理黄河人水系统演变的概念模式,缺乏从流域人水系统视角出发的定量分析;对演变机制的定量分析和建模仿真也多集中在水库等工程因素影响上,忽视流域等水治理政策对流域人水关系变化的重要驱动作用。


\section{研究目标与内容及关键科学问题}\label{sec:contents}

\subsection{研究目标}
% 武旭同话术,待修改
基于上述研究进展,本研究面向社会-水文学研究前沿和黄河流域高质量发展的国家需求,以黄河流域为研究区,结合水文气象观测、社会经济统计、历史数据重建、遥感反演等获取多源数据,借助统计分析和模型模拟等手段,分析黄河流域人水关系的演变过程和驱动机制。具体研究目标如下:

(1)定义便于研究的“流域人水关系”,发展识别其演变规律的分析框架,分别在历史时期和现代治黄时期,定量划分黄河流域人水关系演变的主要阶段与过程。
(2)发展流域人水关系演变机制分析的因果推断方法,建立黄河流域社会-生态系统的多主体模型,识别推动人水关系变化的关键机制,定量评估其产生的影响。


\subsection{研究内容}

针对流域这个特定的人地关系地域系统(或社会-生态系统)给出“人水关系”的概念定义,为满足不同时间尺度的分析需要,建立流域系统层面的人水关系变化分析框架。在此框架的基础上,本研究具体包含以下主要内容:

% 开题报告
(1)黄河流域历史时期流域水沙特征演变:识别历史时期不断增强的人类活动压力超越气候周期变化的驱动力的时间,以及两者主导并推动黄河流域水沙特征突破临界点的稳态转换过程,分析该稳态转换发生之前黄河的人水关系状态。
% 接下来利用建立的分析框架对黄河流域历史上的大事记进行分析,梳理出黄河流域人-水关系演变的主要脉络。最后利用系统回顾法,整合近代有研究以来对黄河人-水关系变化的定量分析成果,建立黄河流域人-水关系变化数据库,综合分析有研究以来人-水系统的主要驱动因素及演化路径。 

(2)现代治黄时期黄河流域的水治理演变:分析上世纪六十年代有实测数据以来,黄河流域“有多少水、怎么用水、怎么分水”如何随时间变化,以及三者如何共同体现流域水治理变化,并解释这些变化背后流域系统如何被人类活动所主导。

(3)黄河流域分水制度改革的影响:识别黄河流域在上世纪八十年代治理转变期间,先后于1987年和1998年两次改革流域水资源分配制度对流域系统的作用机制,分析其如何自上而下重塑社会-生态系统结构、以及对不同地区用水量的影响。

(4)黄河灌溉水分配的多主体建模:识别黄河流域在上世纪八十年代治理转变期间,流域用水的利益相关者对系统环境和制度变化的响应机制,分析其如何通过挑战水资源分配,自下而上地对流域转型做出适应,以及对水资源利用方式的影响。

基于上述研究内容,本研究的技术路线如图\ref{ch1:fig:workflow}所示:

\begin{figure}[htb] % use float package if you want it here
    \includegraphics[width=\textwidth]{img/ch1/ch1_workflow.png}
    \caption{研究技术路线图}\label{ch1:fig:workflow}
\end{figure}

首先,发展流域系统人水关系演变的概念框架,明晰流域人水关系的概念内涵,厘定怎样的变化可以被识别为发生了“流域人水关系演变”,以及有哪些潜在的驱动机制会导致这些变化。接下来利用研究概念框架,分别在历史时期和现代治黄时期识别流域人水关系的演变过程:
(1)人类主导流域水沙特征变化是黄河人水关系变化的关键标志,黄河流域的水沙特征在历史时期发生从气候周期驱动向人类活动主导的过程,需要识别这一稳态转换的发生过程。
(2)水治理是现代流域人水关系变化的主要驱动力,黄河流域的“自然-社会二元水循环”在现代治黄时期则已由人类活动主导,需要识别流域水治理系统发生稳态转换的过程。
通过整合历史时期和人民治黄时期的人水关系演变,可以回答“黄河流域的人水关系有哪些主要演变阶段?”的科学问题,并总结各阶段黄河流域人水关系的演变特征。

针对现代治黄时期流域治理转变由人类制度主导的特点,接下来的两章以制度变化为落脚点,以对黄河流域影响最为深远的水资源分配制度为例,分别从自上而下和自下而上两个路径分析流域人水关系变化的驱动机制:
(1)自上而下:使用主成分分析、社会-生态系统网络构件识别、因果推断模型,分析制度改革对黄河流域系统结构及不同地区用水量的影响。
(2)自下而上:使用由人类模块和自然模块耦合构成的多主体模型,分析农业用水决策者的如何响应环境和制度变化,改变其水资源利用的过程。

\subsection{关键科学问题}

(1)黄河流域的人水关系有哪些主要演变阶段?

(2)不同时期人水关系变化的驱动机制是什么?


\section{研究区概况}\label{sec:study_area}

\subsection{黄河流域概况}

% 空天计划 PPT
黄河流域地貌多样、地理与生态过程复杂、人水关系紧张,具有无以伦比的独特性,其生态保护与高质量发展迫切需要科学基础。

% 空天计划 PPT
黄河以2\%径流承担着全国15\%的耕地和12\%人口,人均仅为全国平均水平的27\%,水资源短缺是重要限制条。
土壤侵蚀强度大,水土流失严重,水少沙多、水沙异源、水沙关系不协调
上游天然草地生态功能退化、水源涵养功能降低,中游植被恢复引致径流下降,下游河口三角洲湿地萎缩、生态系统退化
黄河流域气候和生态系统类型复杂,跨越3个气候带,上中下游地理条件相差极大,是全球受人类活动最为强烈的地区之一:

\subsection{黄河流域治理史}



\section{章节安排}\label{sec:chapters_summary}


\subsection{黄河流域概况}



% 空天计划 PPT
黄河以2\%径流承担着全国15\%的耕地和12\%人口,人均仅为全国平均水平的27\%,水资源短缺是重要限制条。
水沙关系不协调

黄河流域气候和生态系统类型复杂,跨越3个气候带,上中下游地理条件相差极大,是:

% 来自 WAInstitution_YRB_2021 文章
% 它支撑着中国$35.63\%$的灌溉和$30\%$的人口,却只拥有中国 $2.66\%$的水资源(数据来自\href{http://www.黄河水利委员会.gov.cn}{http://www.黄河水利委员会.gov.cn},最后访问:\today)。

% 来自人水匹配 commentary
黄河以其第二大泥沙输运量而闻名,也是一个典型的大流域,长期以来,人类的水关系经历了许多快速而剧烈的变化(图2-A)。在人类活动和气候变化的双重影响下,黄河的泥沙输送量自历史时期(约公元1500年)以来增加了数倍,并在上世纪中期达到惊人的XXX(图2-b)。然而,随着人类一系列惊天动地的变革,黄河的年平均泥沙输运量在短短60年内就变为原来的十分之一,与XXX年前的情况大致相同(图2-b)。这一过程伴随着流域内人水关系在许多方面的迅速转变,包括水资源利用、土地利用、水库和大坝建设、三角洲形态变化等(图2:C至E)。这些变化往往伴随着特定系统结构和功能的重新组合,因此被认为是由流域内人类水系统的累积压力或突然干扰引发的制度转变。对于黄河流域来说,这种稳态转变涉及植被、社会、水文、泥沙和气候等多个子系统的相互作用,并从各个方面改变人水关系。
在这种背景下,人水关系的动态变化往往导致新系统状态下流域内人水系统的不匹配。公元XX年,为了提高粮食产量和扩大人口,国家鼓励在黄河流域的中游地区开荒,并承诺免税。结果,中游地区的人口、耕地和植被之间的关系进入一个新的稳定状态。在新的稳定状态下,植被迅速减少,而人口继续增加。日益强烈的水土流失使黄河的泥沙输送量在这一时期迅速增加,导致下游地区洪涝灾害频繁发生。在灾害发生时,人们急需生计,这反而消耗了国家大量的钱粮用于河道作业和救灾。20世纪80年代,随着灌溉和水库技术的成熟和农业技术的发展,流域内的农业开发达到了极限,黄河从一个脾气暴躁的野兽变成了一个造福国家和人民的粮仓。然而,新的危机又出现了。由于此时抽取的水总量接近地表径流的80\%,XX年以来,黄河频繁断流,呈现出越来越严重的水资源危机。在科学技术尚不发达的历史时期,人们对人与水的关系缺乏系统的认识,造成流域系统的不匹配。然而,在上世纪科学实施节水措施后,由于忽视了水资源与社会发展的匹配,出现了新的错配问题。现在通过一系列的综合治理措施,黄河流域已经基本走出了水危机。但是,随着工业和服务业用水需求的增加,人-水-食物-能源之间的耦合关系更加复杂,人与水的关系出现了新的变化。因此,在了解人-水系统反馈循环的基础上,需要以长远和动态的眼光实现人水匹配和流域的可持续发展。


% 开题报告
黄河流域大部分属干旱半干旱地区,流域面积占全国陆地面积的8.3\%。可见,由于社会经济发展与自然生态系统过程严重失调,黄河流域已成为我国人-水矛盾最为突出和复杂的区域之一[15]。因此,无论是仅着眼于自然子系统或社会子系统的模型,还是局限于支持水资源年内和年际调度的规划,都难以满足黄河流域长期、可持续利用水资源的需要。缺乏对人-水关系演变过程与机制的深入理解正严重制约着流域的高质量发展。
阐明黄河流域历史和当代人-水关系的演变过程,识别人-水关系演变的核心机制,能够为黄河流域可持续发展提供关键科学依据。

\subsection{黄河流域治理史}

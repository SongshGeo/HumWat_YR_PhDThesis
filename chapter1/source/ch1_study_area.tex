\subsection{黄河流域概况}

% todo 数字均来自焦晨泰的毕业论文,需要查证参考文献,修改表述
黄河流域($95˚52'37” \sim 119˚3'56”E$,$32˚9'38” \sim 41˚51'37”N$)跨越三个气候带,气候和生态系统类型复杂,干流流程$5.4 \times 103km$、流域面积$76.6 \times 104km^2$,约为中国国土面积的$8.3\%$,是中国第二大流域。
黄河流域大部分位于我国干旱半干旱地区,年平均温度约为$6.4^{\circ}C$,潜在蒸发量超过$800mm$[102],年均降水量不足$500mm$且季节差异明显[101],中游以上流域夏季降水量可占全年降水量的$85\%$[503]。
黄河流域地形变化较大,流域海拔变化超过$6200m$,水面海拔高度差达$4480m$,流域内多样复杂的地形让黄河流域上、中、下游地理条件相差极大,且具有无以伦比的独特性:上游是世界上最年轻/快速隆升的青藏高原;中游是唯一正在堆积/强烈剥蚀的黄土高原;下游供给着人口密集/水资源需求极大的华北平原。
上游与源区最大的生态系统功能是水源涵养,黄河$534.79$亿立方米[405]的多年平均地表水资源量中,有$60\%$以上来水来自兰州以上的源区[408]。
中游黄土易垦区在人类活动和气候变化的双重影响下,因植被破坏和水土流失产生的泥沙让黄河曾以泥沙输运量最高的河流而闻名于世\cite{best2019}。
而在花园口站以下的黄河下游区域,因泥沙沉积抬高高程而形成了地上悬河,流域面积仅为$3*10^4 km^2$,产流量低,且主河道在历史时期因频繁的洪泛决溢而在华北平原上频繁摆动,自西向东形成不计其数的辫状古河道。

\begin{figure}[htb] % use float package if you want it here
    \includegraphics[width=\textwidth]{hello}
    \caption{黄河流域研究区示意图}\label{ch1:fig:study_area}
\end{figure}

经过三千多年的历史开发和近现代的高强度经济活动影响,黄河流域的人类活动与资源环境间要素关系不断变化\cite{fu2021a}。
随着黄河流域人口压力的增大,过去近千年来中游黄土高原地区的耕种面积迅速扩大,坨垦甚至发展到丘陵和陡坡地区,日益严重的水土流失导致黄河的泥沙输送量迅速增加\cite{wu2020a}。
大量的泥沙沉积在下游产生地上悬河,导致洪泛决溢的频繁发生,反复为历史时期黄河中下游的社会经济发展带来灾难,如宋朝和清朝都花费了极大人力物力在黄河流域的赈灾上。
自二十世纪中期以来,随着灌溉和水库技术的成熟和农业技术的发展,黄河下游逐渐从泛滥成灾的变成了造福国家和人民的粮仓,但“水少”的新问题也悄然出现,
流域内的农业灌溉水资源开发逼近极限,抽取的水总量接近地表径流的$80\%$,导致黄河自$1972$年以来频繁断流,表现出日益严重的水资源危机。
如今黄河流域仍然是中国重要的农业和工业生产基地,人口稠密,社会经济活动密集,随着工业和服务业用水需求的增加,人-水-食物-能源之间的耦合关系更加复杂,更需要在了解人-水系统反馈循环的基础上,以长远和动态的眼光实现人水关系和谐的流域高质量、可持续发展。

\subsection{黄河流域治理简史}

黄河流域的水治理素来是人水关系不断变化的重要内部驱动力,自有记录的历史时期以来,中央王朝便为治理黄河积累了成篇累牍的文献资料,近现代“治黄”更是被视为关乎国家长治久安的民族大计。
西汉中后期随着河道淤积与河患增多,各种治河思想就已经非常活跃,史料表明在著名的“王景治河”时期,黄河下游已经普遍建有较完备的堤防建设。
王景治河之后黄河安流数世纪,在北宋至元代重新进入了有史以来水患最为严重的一段时期,频繁的河道变迁与堤防决溢迫使宋朝政府广泛征集治河水利人才,并建立岁修制度和治河责任制,但治河总方针在“恢复东流”和“北流入海”间争执不下,消耗了国家大量的钱粮用于河道作业和救灾。
明代黄河泛滥趋势有所缓解,河流治理为主要为保护京杭大运河漕运服务,采用潘季驯“束水工沙”的思想,初步认识到水沙关系的重要性。
清朝前期国力强盛时也有“汰沙澄源”重视水保的提议,但未受到重视,恪尽职守继续沿用古人思想进行治理,只是重新将大运河与黄河分离,漕运与治河并重;但清王朝中后期随着国力日渐衰微,不再能负担不起这延续数千年的黄河治理,最终随着又一次决堤,黄河离开了七百年来南流夺淮的河道。


进入现代中国,科学认识的加深为黄河流域带来“治黄先治沙”的治理方针,其后随着水库、淤地坝、渠道等一系列工程措施,黄河的年平均泥沙输运量在短短$60$年内就变为原来的十分之一,极大缓解了困扰数千年来高输沙量、高淤积量的问题。
面对以水资源短缺为代表“治黄新挑战”,黄河流域实施了几项雄心勃勃的治理措施来缓解压力,既包括如水库联合调控、南水北调等工程性措施,也包括水资源配额制度、最严格水资源管理、黄河流域保护法等制度法规。
其中水资源配额制度对流域的影响最为深远,该制度自1980年首次讨论,1987年正式提出,此后三十余年间都是指导和限制流域各地区用水的关键,包括联合调度、最严格水资源管理等都是在该方案的基础上执行的,尽管多个省份认为此方案正愈发脱离实际情况。
在2019年9月的黄河流域生态保护和高质量发展座谈会上,黄河流域生态保护和高质量发展被正式提升为重大国家战略,强调黄河流域发展需要“重塑人水关系”并“守住生态保护红线”;同年,对黄河水资源分配制度进行适当调整被提上日程;2022年,《黄河保护法》正式通过 \ldots。
这些新时代的新举措表明,“治好黄河”的方向和决心都始终没有变,而理解黄河流域人水关系演变与机制,分析流域治理实践对人水关系的作用机制,有助于在未来全球变化家加剧的环境下,成为动态解决黄河高质量、可持续发展的科学基础。

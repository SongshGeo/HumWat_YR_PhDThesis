本论文共分为六章。
第一章绪论主要介绍人\textendash{}水关系研究的背景和选题意义,梳理了人\textendash{}水关系演变与机制研究相关的理论,及其在社会\textendash{}生态系统、社会\textendash{}水文学、稳态转换研究中的相关进展,凝练当前研究的关键科学问题,提出本研究的主要研究目标和研究内容,简要说明本研究的研究区概况。
% 第二章为“流域人\textendash{}水关系”等研究核心概念提出可操作的严格定义,凸显本研究旨在对黄河流域的人\textendash{}水关系变化产生宏观整体性认识,而不桎梏于细节琐碎的系统局部,同时提出通过稳态转换理论来识别流域人\textendash{}水关系变化的总体思路框架。
历史时期的黄河治理主要依赖于中央王朝对黄河变化的响应,第二章分析黄河历史时期人\textendash{}水关系的演变过程,着眼于不断增强的人类活动驱动力何时超越气候周期驱动力,并推动黄河流域发生水沙特征的稳态转换。
现代治黄的主要特征是强调流域系统综合治理,第三章定量识别二十世纪六十年代以来现代治黄时期的水治理变化,并通过整合统计资料、水文数据、文献记录等多源数据,分析近七十年间水治理稳态变化的关键转型期,并分析相关驱动因素的作用路径。
针对第三章识别的治理转变期,第四章和第五章将分别从自上而下、自下而上两条作用路径分析典型治理政策推动流域人\textendash{}水关系演变的内在机制。
第四章梳理了自上而下的两次水资源分配制度变化对流域社会\textendash{}生态系统结构的重塑,并使用“差分合成控制”因果推断模型,构建“如果没有发生制度变化”的反事实推断,评估这种变化对流域用水情况的影响及其作用机制。
第五章构建了自下而上的多主体模型,模拟流域治理转变期农业用水利益相关者对气候与环境变化、社会与制度变化的响应,通过对主体的用水决策进行仿真,分析黄河流域人\textendash{}水关系在治理转变期的驱动机制及后续影响。
第六章总结本研究的主要结论,梳理主要创新点,展望未来可能的研究方向。
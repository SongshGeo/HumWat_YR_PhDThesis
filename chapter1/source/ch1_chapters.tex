本论文共分为七章。
第一章绪论主要介绍人-水关系研究的背景和选题意义,梳理了人水关系演变与机制研究相关的理论,及其在社会-生态系统、社会水文学、稳态转换研究中的相关进展,凝练当前研究的关键科学问题,提出本研究的主要研究目标和研究内容,简要说明本研究的研究区概况。
% TODO 梳理一遍章节安排
第二章为“流域人水关系”等研究核心概念提出可操作的严格定义,凸显本研究旨在对黄河流域的人水关系变化产生宏观整体性认识,而不桎梏于细节琐碎的系统局部,同时提出通过稳态转换理论来识别流域人水关系变化的总体思路框架。
第三章通过整合历史时期重建数据,从千年时间尺度上分析黄河人水关系的演变过程,着眼于不断增强的人类活动驱动力何时超越气候周期驱动力,并推动黄河流域发生历史第一次流域稳态的转换的机制。
第四章在第三章的基础上,借助断点识别方法重点对现代治黄时期人类主导黄河“自然-社会二元水循环”时期的水治理变化进行定量识别,并通过整合统计资料、水文数据、文献记录等多源数据,分析近七十年间水治理稳态变化的关键转型期,并分析相关驱动因素的作用路径。
针对第四章识别的转型期和驱动因素,第五章和第六章将分别从自上而下、自下而上两条作用路径分析转型期推动水治理稳态演变的内在机制。
第五章梳理了自上而下的两次水资源分配制度改革对流域社会-生态系统结构的重塑,并使用“差分合成控制”因果推断模型,构建“如果没有发生制度改革”的反事实推断,评估这种结构变化对流域用水情况的影响及其作用机制。
第六章构建了自下而上的多主体模型,模拟流域治理转变期农业用水利益相关者对气候与环境变化、社会与制度变化的响应,通过对主体的用水决策进行仿真,分析黄河流域人水关系在治理转变期的演变的机制及后续影响。
第七章总结本研究的主要结论,梳理主要创新点,展望未来可能的研究方向。
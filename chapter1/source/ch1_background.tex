\subsection{人地关系:地理学研究的核心}

% 帮王老师写的综述:人地系统结构与可持续V2
在人类活动的强烈影响下,地球进入“人类世”的新纪元,这意味着人类对地球改造的程度与后果可以与传统意义上的地质营力产生的影响相匹敌,成为环境变化的重要的驱动力\cite{lenton2019, lewis2015, lewis2018}。
这种来自人类的改造和控制在过去数世纪以来,已让气候变化、生物多样性损失和氮循环等关键地球系统生态过程超越了危险的“地球界限”,导致了众多资源、生态与环境问题\cite{steffen2015}。
如何在满足人类发展需求的同时,持续地保障地球生命支持系统的基本结构和功能,实现可持续发展,已成为学术界和社会各界广泛关注的重大科学和决策问题\cite{wu2014}。
人与自然地理环境密不可分,理解现代环境变化机理、持续地保障地球生命支持系统的基本结构和功能就需要发展人地系统整体的方法,耦合自然与人文过程,探讨变化环境下的系统耦合机制\cite{fu2015}。

在涉及人地关系综合研究的学科中,地理学以地域为单元着重研究地球表层人与自然的相互影响与反馈作用,对人地关系的认识素来是地理学的研究核心\cite{wu1991}。
无论是钱学森先生倡导的“地球表层学”、吴传钧先生提出的“人地关系地域系统”、黄秉维先生倡导“陆地表层系统科学”,均强调人与自然相互作用形成的人地关系复合系统,也就是人与地在特定的地域中相互联系、相互作用而形成的一种动态结构。
面对环境问题的复杂性,不同空间尺度上人类活动与自然环境的耦合关系也正在成为国际学界的主流话题\cite{fu2015}:美国科学基金会(NSF)于2001年就开展了自然与人类耦合系统动力学(Dynamics of Coupled Natural and Human Systems-CNH)研究计划,2019年进一步发展为了CNH2-社会-环境综合系统动力学计划;“未来地球”科学计划旨在推动自然科学与社会科学研究成果共同为可持续发展服务。
在人类活动影响力不断增强的背景下,要制定区域可持续发展战略,就必须深入了解人类赖以生存的地球环境系统与人类系统之间相互作用的基本过程,也就是关注人地系统演变及其机制的人地系统动力学\cite{fu2022}。

\subsection{流域系统:人水关系的研究单元}

% 开题报告
水是连接自然系统和社会系统的纽带,水循环过程是生态过程和社会发展的重要驱动力,人-水系统是以水循环为纽带将人文系统与水系统联系在一起的复合系统,是在流域尺度紧密相连的开放巨系统,也是典型的人与自然耦合系统\cite{li2007}。
% 于璐 本子
早在2013年,国际水文科学协会就启动2013-2022十年科学计划“万物皆流(Panta Rhei)”,旨在解析水文过程与人类社会的连接和动态演变过程(Montanari et al., 2013)。
近年来研究者对于人水关系的认识逐渐由外部动力演变为复杂的内在过程,人类活动也进入了水循环地图(Abbott et al., 2019; USGS, 2022)

作为一个复杂的人地关系地域系统,流域系统是在水循环上完整的地理单元,其演化路径由快变量(如水文、经济、工程)与慢变量(如生态与社会文化)共同作用导致,社会和生态系统的慢变量经过长期累积决定了人-水系统的演化进程\cite{falkenmark2021}。
近十余年,程国栋、傅伯杰等国内专家学者组织多学科交叉的团队,围绕黑河流域系统、黄河流域系统的人地关系开展了大量集成研究,发展了多学科交叉、多种观测与模型结合的流域系统研究方法,使流域成为人地关系研究的重点突破单元\cite{cheng2014, fu2021a}。
在全球变化和人类活动驱动下,流域系统的动态变化仍在不断加速,理解其演变规律和作用机制是推动人水关系和谐共生,实现可持续高质量发展的科学基础,也是当前研究的主要难点\cite{reyers2018}。

大河流域一直是人类文化起源和发展的中心,通过粮食生产、水力发电和水源供给等给人类社会带来巨大收益,支持着众多的人口,具有显著的社会重要性并构成多样化的生态系统\cite{best2019}。
但随着社会的发展,人类活动改变了流域的自然和社会水循环过程,导致大部分流域出现不可持续的发展轨迹,人类与河流的关系变得空前紧张\cite{best2019, best2020}。
人口增长以及对水、电、粮食和土地需求的增加\cite{crutzen2006},人类对水循环过程的影响已从外部动力演变为系统内力,给大河流域生态系统完整性和可持续性带来了前所未有的挑战\cite{dibaldassarre2019}。
以流域为单元开展人-水系统耦合动态与机理研究,阐明的人-水关系的演变过程并识别其机制,既是地球系统科学的前沿,也是应对环境变化挑战、保持人水和谐、实现可持续发展的重要科学基础。

\subsection{重塑黄河人水关系:流域高质量发展之本}

% 空天计划 PPT
黄河流域地貌多样、地理与生态过程复杂、人水关系紧张,具有无以伦比的独特性,其生态保护与高质量发展迫切需要科学基础。
长期以来,黄河以只占全国$2\%$的年径流量承担着全国$12\%$的人口、$15\%$的耕地和$13$个国家能源化工基地的供水任务,水资源开发利用率一度接近$80\%$,为我国十大一级流域中最高,远超国际认定的$40\%$的流域生态警戒线\cite{fu2021a}。
作为我国人地矛盾最为突出和复杂的区域之一,黄河流域还存在上游天然草地生态功能退化、水源涵养功能降低;中游土壤侵蚀强度大、水土流失严重;下游河口三角洲湿地萎缩、生态系统退化等一系列问题\cite{mazhuguo2020}。
这些,这些黄河流域生态保护与可持续发展面临的瓶颈,既是流域人地系统协同演化产生的现象,也是人水关系长期不协调的体现\cite{fu2021a}。

黄河是中华民族的母亲河,“黄河宁,天下平”,治理黄河是千年夙愿,黄河流域生态保护和高质量发展是重大国家战略,亟需在发展过程中平衡经济、生态、以及社会效益。
研究黄河流域人水关系演变规律、揭示黄河流域人水关系演变机制,有助于深化认识人地关系地域系统的结构特征与耦合机理,为协调黄河流域人水关系、促进高质量发展提供理论框架和科学依据。

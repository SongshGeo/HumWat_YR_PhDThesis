
\subsection{人地关系研究:地理学研究的基本命题}

% 帮王老师写的综述:人地系统结构与可持续V2
地理学着重研究地球表层人与自然的相互影响与反馈作用,对人地关系的认识,素来是地理学的研究核心(吴传钧 1991)。巴罗斯认为地理学是研究人与其赖以生存的自然 环境的相互影响,可称人类生态学。吴传钧先生(1991)认为涉及人地关系综合研究的学科不限于地理学,但以地域为单元,着重研究人地关系地域系统的唯有地理学,因此,地理学的研究核心是人地关系地域系统,是以地球表层一定地域为基础的人地关系系统,也就是人与地在特定的地域中相互联系、相互作用而形成的一种动态结构。“人”和“地”这两方面的要素按照一定的规律相互交织在一起,交错构成的复杂开放的巨系统内部具有一定的结构和功能机制,在空间上具有一定的地域范围,便构成了一个人地关系地域系统。主要研究内容包括人地关系地域系统的形成过程、结构特点和发展趋向;各子系统相互作用强度、潜力、后效和风险;人与地两大系统间相互作用和物质、能量传递与转换的机理、功能、结构和整体调控途径;地域的人口承载力分析;动态仿真模拟、地域分异规律和类型分析以及优化调控模型。樊杰认为地域功能性、系统结构化、时空变异有序过程、以及人地系统效应的差异性及可调控性,是该理论的精髓,也是综合研究地理格局形成与演变规律的理论基础。

% 帮王老师写的综述:人地系统结构与可持续V2
史培军(史培军 1997)认为在人类活动影响力不断增强的背景下,要制定区域可持续发展战略,就必须了解人类赖以生存的地球环境系统与人类日益发展的生产系统 (基于地球之资源开发)之间相互作用的基本过程,也就是人地系统动力学,主要包括(1)人类活动的环境效应,即人类活动驱动下的自然生态系统的响应机制与反馈机理;(2)确保自然生态系统演替(良性循环)下的人类改造自然界的可能途径。方创琳(方创琳 2010)总结中国人地关系研究在理论、方法、实践应用等方面取得的进展,认为是以全新视角研究人地关系,深入分析人地系统的基本特征,科学划分人地系统的学科分支,尝试揭示人地关系演进趋势与基本规律,新型人地关系理论不断出现,正在逐步形成理论体系,人地系统研究方法趋于多元化,定性研究方法、数学模拟方法、3S 技术方法和综合集成方法等新方法与新手段不断应用,人地系统理论在实践中广泛应用于解决不同时空尺度的区域发展问题和不同领域的可持续发展问题。

\subsection{人水关系演变:人地耦合系统的监测器}

% 开题报告
在人类活动的强烈影响下,全球资源环境面临前所未有的压力,地球进入“人类世”的新纪元[1–3]。理解现代环境变化机理、预测未来变化趋势就需要发展系统整体的方法,耦合自然与人文过程,探讨变化环境下的系统耦合机制[2,4]。水是连接自然系统和社会系统的纽带,水循环过程是生态过程和社会发展的重要驱动力。人-水系统是以水循环为纽带将人文系统与水系统联系在一起的复合系统,是在流域尺度紧密相连的开放巨系统,也是人与自然耦合系统的典型代表[5,6]。对自然生态系统而言,从水文地质循环到生态水文过程,从旱区陆地到河湖水体,水是大多数陆地生态系统的主要限制因素,也是物质运移和环境演化的重要介质[3,7,8]。对于人类社会系统而言,从饮食洗涮到田间地头再到江河治理,从民间祭祀到哲学论述再到神话传说,水是一种贯穿了自然与养育、实践与象征的物质[9]。理解人-水关系就是理解人-水复杂系统内部人类社会系统与自然水系统间的相互作用模式[10],重塑人水关系就是通过调整这一互动模式来重塑人与自然的关系。
% 开题报告
人-水系统是社会和水文协同演化的复杂系统[8,11],既有以路径依赖为代表的演化特性,也有不确定性等复杂系统特征,这意味着对人水关系的理解与预测需要同时考虑演化的历史进程与未来变化的不确定性。流域是人-水关系研究的完整单元,流域演化由快变量(如水文、经济、工程)与慢变量(如生态与社会文化)共同作用导致,社会和生态系统的慢变量经过长期累积决定了人-水系统的演化进程[12]。例如水资源管理作为流域特定社会文化、经济水平和政治体制的直接产物,通过调整水量的分配进而作用于生态系统,决定生态系统的健康状况和社会系统的人类福祉。长期以来,流域水资源管理往往通过调控快变量,旨在短期内提高水的利用规模和效益,较少考虑系统状态的长期演变,对生态和社会慢变量的积累变迁更是缺乏反馈机制,限制了维持流域长期可持续的能力。此外,尽管因流域而异的演化轨迹是人-水系统复杂性的体现,但世界主要大河流域也展现出了相似的关键变量与作用路径。例如,在干旱-半干旱区的墨累-达令河流域、科罗拉多河流域以及黄河流域,水资源量的限制都促使水资源分配制度的诞生,该制度又以相似的作用方式影响了河流的水文状态。长期以来这些作用路径被认为是人-水系统演化中的突发政策性影响,因而忽略了其背后的一般性相互作用机制,制约了人-水关系调控的系统性和前瞻性。因此,识别流域人-水关系的长期演化,并理解复杂系统关键变量对演变过程的核心作用机制,对流域的可持续、高质量发展至关重要。


% 王帅 2023
大河流域一直是人类文化起源和发展的中心,通过粮食生产、水力发电和水源供给等给人类社会带来巨大收益,支持着众多的人口,具有显著的社会重要性并构成多样化的生态系统(Best, 2019)。但是,人口增长以及对水、电、粮食和土地需求的增加(Crutzen et al., 2006),人类对水循环过程的影响已从外部动力演变为系统内力,给大河流域生态系统完整性和可持续性带来了前所未有的挑战(Sivapalan et al., 2019)。围绕流域人地系统耦合研究,已经开展了生态水文过程、流域综合管理、社会水文学以及水-粮食-能源关联等方面的大量研究()。进一步协调流域人地关系促进协同治理,迫切需要:定量解析社会生态多要素相互作用的人地系统整体结构特征;揭示流域参与主体与水沙及生态过程之间的互馈机理;发展能够集成自然过程与社会响应的人地系统耦合模型。

% 王帅 2023
黄河流域生态保护和高质量发展是重大国家战略。流域生态环境脆弱、水资源保障形势严峻,上游局部地区生态系统退化、中游水土流失威胁依然严峻、下游生态流量偏低带来多重生态胁迫,这些问题都可归结为人地关系不协调,是我国人地矛盾最为突出和复杂的区域之一(傅伯杰等,2021)。选取黄河流域为对象,研究流域人地系统结构特征及其时空演变,揭示流域社会-生态-水沙协同演变规律与耦合机理,以多主体模型集成人地系统耦合模型,将有助于:深化认识人地系统结构特征与耦合机理,为流域协调人地关系和促进协同治理提供理论框架和科学依据。

% 自己写的
人类社会系统与水文系统之间的互馈机制是人水关系研究的关键。根据侧重点不同,当前的人-水互馈研究的可分为xxx方面。

\subsection{人水关系重塑:黄河高质量发展之核心}

% 开题报告
黄河是中华民族的母亲河,“黄河宁,天下平”,治理黄河是千年夙愿。黄河流域大部分属干旱半干旱地区,流域面积占全国陆地面积的8.3\%,年径流量只占2\%,但却承担着全国12\%的人口、15\%的耕地和13个国家能源化工基地的供水任务[13,14]。水资源开发利用率接近80\%,为我国十大一级流域中最高,远超一般流域40\%的生态警戒线。可见,由于社会经济发展与自然生态系统过程严重失调,黄河流域已成为我国人-水矛盾最为突出和复杂的区域之一[15]。当前黄河流域面临的社会、生态问题既是在人-水关系长期演化中产生的现象,也是长期忽视人-水关系演化规律而进行水资源管理的后果。因此,无论是仅着眼于自然子系统或社会子系统的模型,还是局限于支持水资源年内和年际调度的规划,都难以满足黄河流域长期、可持续利用水资源的需要。缺乏对人-水关系演变过程与机制的深入理解正严重制约着流域的高质量发展。
综上所述,人类对自然水循环的影响已从外部动力演变为系统内力,人-水系统耦合机理是地球系统科学的前沿,也是应对环境变化挑战、保持人水和谐、实现可持续发展的重要科学基础。在流域尺度理解人-水关系演变的过程与机制是解耦人-水系统这个复杂开放巨系统的关键,阐明黄河流域历史和当代人-水关系的演变过程,识别人-水关系演变的核心机制,能够为黄河流域可持续发展提供关键科学依据。

% 人水关系 匹配 comment
水不仅是地表生态过程的一个重要组成部分,也是人类社会发展所依赖的重要资源。目前,大江大河流域支撑着xx\%的重要生态系统,为世界上xx人提供了生存的水资源。因此,可以毫不夸张地说,在社会与生态系统高度相关的大河流域,河流是文明发展的血液。然而,随着社会的发展,人类活动改变了流域的自然和社会水循环过程,导致大部分流域出现不可持续的发展轨迹,人类与河流的关系变得空前紧张。

% 人水关系 匹配 comment
人类的水关系取决于我们如何对待河流,以及我们从流域中得到什么。就像恒河的信徒相信河流能净化他们的灵魂,尼罗河的农民期待着河流的丰收,胡佛站在大坝前骄傲地宣布他已经 "征服 "了科罗拉多河。社会越来越多地为了自己的福祉而改造流域,尽管水经常提供许多其他关键功能,如调节气候、支持生物质生产、物质运输和净化环境。这些功能对于可持续发展可能更加重要,以保持流域社会生态系统的弹性,使其能够适应外部变化或在转型中摆脱危机。然而,在许多情况下,人类改造流域的活动作为干扰超过了系统的临界点,导致维持核心水功能的反馈回路发生变化,引发社会生态系统的稳定转变。为了避免可能导致系统崩溃的稳态转变的级联效应,必须将人水关系的变化视为流域社会生态系统的内部原因,特别是在人类影响的人类世。

% 人水关系 匹配 comment
幸运的是,人类与水的关系并非完全难以捉摸。事实证明,社会发展与人水关系的变化之间存在着规律,这使得流域系统产生了可以解释和预测的共性问题,如过度和低效开发导致的水资源短缺。这表明,尽管我们有能力预测社会对水的需求和自然界水循环的变化,但仍然缺乏有效的理论和方法来匹配这两者。由于人类社会系统和流域水文系统的结构和功能具有不同的尺度和动态,匹配是指两者之间的良好关系。因此,我们需要深入了解人与水的关系及其动态变化,并找到实现匹配的方法,从而将流域引向可持续发展的轨道。


% 开题报告
黄河流域是世界上人-水关系变化最频繁、最剧烈、影响最为深远的大河之一,人类社会与黄河相互作用的研究历史源远流长。
% 从水的社会性上看:大禹治水的故事对中华文明的脉络有着深远的影响[30];三门峡与下游运河的运输功能对历代王朝的国祚可谓牵一发而动全身[30];近年来水资源的区域间分配也对经济发展有着制约性影响[31]。
% 从社会控制水的角度上看,历代中央王朝的黄河洪泛治理对水文系统变化影响深远;引黄灌溉则是这个干旱-半干旱区流域的农业经济发展命脉;近年来水库的调水调沙更是彻底重塑了黄河下游的自然生态环境[32]。
% 黄河流域这种频繁的人-水关系变化常常为梳理人-水系统复杂的的反馈循环与协同演化带来了更大的困难[12,33]。
如魏夫特考察黄河流域的社会发展历史后提出了著名的“水利社会理论”,认为这样大规模的治水活动是中央封建王朝皇权诞生的重要因素;但更多中国学者认为魏氏倒转了复杂人-水关系中的因果,正是中央皇权的大力发展才使得大规模治水活动变得可能[35,36]。
治水方略的选择通常被认为是决定黄河河道演变的重要人为因素,同时河道的自然条件又是选择治理方略的重要依据,由两者共同决定的治理结果则会造成路径依赖[34]。
因此,为了梳理黄河流域的人-水关系变化,也不乏着眼于总结演变过程的著述。
于宏瑞将黄河流域的人-水关系演变的典型事件按区域(上、中、下游)进行了细致梳理[19],但对跨时空的事件关联着墨不多。
也有研究倾向于按照黄河流域人-水关系的关键主题梳理演变进程,力求展示出其时间纵深和因果联系,如葛剑雄从黄河与中华文明互动的角度[30],王渭泾从黄河治理的角度[34],分别对近两千年来黄河人-水关系的演化史进行梳理。
但既有研究的常对人水系统进行“专题梳理”,且以描述概念模式为主,无法对流域人水系统的演变进行定量分析,难以满足对未来演变方向作出解释的科学需要。

% 于璐 本子
黄河流域还是社会-水文学研究人水系统协同演化机制的热点案例。
黄河流域长时间受人类活动强烈干扰[48],流域系统内变量关系复杂存在诸多子系统(如水沙子系统、社会经济子系统、社会-生态子系统等),素来是稳态转换研究的代表性区域。
目前黄河流域系统人水关系演变机制的研究主要集中在评估人类活动影响(Shi et al., 2022)、人为干预治理或资源管理的效果与影响(Feng et al., 2016; Zhou et al., 2021)等方面,包括水库调控和人类用水对流域人水关系的影响等\cite{wang2019c}。
在分析人-水系统关系演变机制的方法上,既有研究多集中于指标和结构层面;指标评估如耦合协调度(李波等,2022;赵良仕等,2022)、承载力(Wang et al., 2022)、多维指标评价(陈莉等,2022)等;结构层面如网络(Song et al., 2022; Zhi et al., 2020; 张伟丽等,2022)、供需(Wang et al., 2019; Yin et al., 2021; Zhang et al., 2021; 孙久文等,2022)等,这些方法多依赖于对长期、结构化时间序列数据的收集,因此难以应用于非工程性质的、非连续的流域治理制度等人水系统驱动力的机制分析。
整体上,黄河流域人水关系演变机制的研究对非工程的水资源治理措施的影响机制严重不足,难以对流域规划中政策影响下的未来情景进行预测,对流域治理政策的指导意义不足。

针对黄河流域人水系统关系复杂的特点,也诞生了一些同时考虑人类活动与水文生态的模型。
% 焦 本子
贾仰文等基于WEP模型开发了WEP-L模型,通过划分子流域及等高带的方式实现了黄河历史径流复现[214,215],杨大文等构建了分布式水文模型复现或表征了黄河流域多个水文参数[216]。
% 江 本子
岳瑜素等(2020)应用系统动力学模型优化了黄河下游滩区自然-经济-社会协同的可持续发展模式;Jiang等(2021)建立系统动力学模型,从社会、经济、资源、生态、文化五个方面预测了在未来政策情景下黄河流域高质量发展水平;黄昌硕等(2021)建立了基于“经济社会-水资源-生态环境”系统动力学模型,为动态预测区域水资源承载力的变化趋势和制定优化调控方案提供了参考依据。
% 自己写的
相较发展和应用均较为广泛的分布式水文模型和系统动力学模型,多主体模型研究相对较少。
xxx使用多主体模型模拟了流域主体对水资源调度的影响,优化黄河流域的水资源分配,但模拟的主体数量较少且主体间互动仅考虑工程要素,与水资源管理的联合多目标优化相似。

综上所述,既有研究的常按主题梳理黄河人水系统演变的概念模式,缺乏从流域人水系统视角出发的定量分析;对演变机制的定量分析和建模仿真也多集中在水库等工程因素影响上,忽视流域等水治理政策对流域人水关系变化的重要驱动作用。

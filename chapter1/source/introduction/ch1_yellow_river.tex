
% 开题报告
黄河是全球人与水关系变化最频繁、最剧烈、影响最深远的大河之一。
人类社会与黄河相互作用的研究历史悠久,例如魏特夫对黄河流域的社会发展历史进行了考察,提出了著名的“水利社会理论”,认为这样的大规模的治水活动是中央封建王朝皇权诞生的重要因素\cite{weitefu1989}。
但许多中国学者认为,魏氏的观点是颠倒了复杂人\textendash{}水关系中的因果关系,正是中央皇权的强劲发展使得大规模治水活动成为可能\cite{jizhaoding1981}。
治水策略的选择是决定黄河河道演变的重要人为因素,同时河道的自然条件也是选择治理策略的重要依据,由两者共同决定的治理结果会形成路径依赖\cite{WangWeiJing2009}。
因此,为了总结黄河流域人与水关系的变化,不乏有着眼于总结演变过程的著作。
于瑞宏详细地梳理了黄河流域人\textendash{}水关系演变的典型事件,并将其分为上游、中游和下游区域,但未能将其串联在一起\cite{yuruihong2011}。
同样,有一些研究倾向于按照黄河流域人\textendash{}水关系的关键主题来梳理演变进程,并努力展示其时间深度和因果关系,例如,葛剑雄从黄河与中华文明互动的角度进行了梳理\cite{gejianxiong2020},王渭泾则从黄河治理的角度进行了梳理\cite{WangWeiJing2009},以总结近两千年来黄河人\textendash{}水关系的演变历史。
然而,这些研究通常仅限于对人水系统进行“专题梳理”,并以描述性概念模型为主,无法对流域人水系统的演变进行定量分析,因此难以对未来演变方向作出科学解释。

% 于璐 本子
黄河流域是人\textendash{}水关系和社会-水文学研究的重要案例。
长期以来,该流域受到人类活动的强烈影响,其人\textendash{}水关系演变具有复杂性,因此是稳态转换研究的代表区域\cite{zuo2022, wang2014}。
目前的研究主要集中在评估人类活动对流域的影响、评估人为干预治理或资源管理的效果与影响等方面\cite{wang2016a, WuXuTong2021, wang2019c}。
在分析人水系统关系演变机制方面,一些研究集中在指标和结构层面,如耦合协调度\cite{libo2022}、承载力\cite{wang2022d}、多维指标评价\cite{li2020}、网络\cite{song2022}和蓝水与绿水等\cite{zhuo2016a}。
但这些方法依赖于对长期、结构化时间序列数据的收集,因此很难应用于分析非工程性质的、非连续的流域治理制度等人水系统驱动力的机制。
总的来说,黄河流域人\textendash{}水关系演变机制的研究对于非工程的水资源治理措施的影响机制探讨不足,因此很难对流域规划中政策影响下的未来情景进行预测。

针对黄河流域人与水文系统的复杂关系,也有一些研究采用了同时考虑人类活动和水文过程的模型。
例如,贾仰文等人基于WEP模型开发了WEP-L模型,通过划分子流域和等高带的方法实现了黄河历史径流的复现\cite{jiayangwen2005}。
杨大文等人构建了分布式水文模型,以表示黄河流域的多个水文过程\cite{yangdawen2004}。
岳瑜素等人应用了系统动力学模型,优化了黄河下游滩区自然、经济和社会协同的可持续发展模式\cite{yueyusu2020}。
Jiang等人建立了系统动力学模型,从社会、经济、资源、生态和文化五个方面预测了在未来政策情景下黄河流域的高质量发展水平\cite{jiang2021b}。
黄昌硕等人建立了基于“经济社会-水资源-生态环境”系统动力学模型,为动态预测区域水资源承载力的变化趋势和制定优化调控方案提供了参考依据\cite{huangchangshuo2021}。
% 自己写的
相比于广泛发展和应用的分布式水文模型和系统动力学模型,仿真黄河复杂人水系统相互作用和揭示涌现现象的多主体模型研究相对较少。
Cai 在水资源调度方面使用了多主体模型,以优化黄河流域的水资源分配。然而,他模拟的主体数量较少,且主体间的互动仅考虑了工程要素,与水资源管理的联合多目标优化相似\cite{cai2011}。
Du 等人在黑河这个黄河的子流域实现了分布式水文模型和多主体模型的耦合,分析了水政策、农民用水和水文条件之间的关系,并指出了区域水文地质条件(如地下水位深度和与河流的距离)是影响人类与水文相互作用的关键因素\cite{du2020}。
但是,在黄河流域研究中,仍缺乏考虑多级利益相关者的、从下而上的、耦合分布式水文模型的多主体模型,限制了对不同政策情境下黄河人\textendash{}水关系演变机制的分析与预测。

总的来说,现有的研究主要集中于梳理黄河人水系统演变的概念模型,缺乏从流域人水系统视角出发的定量分析;尽管对演变机制进行了定量分析和建模仿真,但多集中在水库等工程因素的影响上,忽视了流域水治理政策对人\textendash{}水关系变化的重要驱动作用。

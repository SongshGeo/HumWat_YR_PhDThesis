黄河流域长时间受人类活动强烈干扰[48],流域系统内变量关系复杂存在诸多子系统(如水沙子系统、社会经济子系统、社会-生态子系统等),素来是稳态转换研究的代表性区域。

黄河是受人为活动强烈影响的大河,黄河流域是研究社会-水文学的天然实验田。从1922年魏特夫的“水利社会理论”以来,关于人-水关系演化理论机制的探讨便未曾中断,对关键系统过程进行量化分析的尝试也不乏进展。近年来借助社会-水文学的技术手段,以水沙关系调控为代表的人为驱动下黄河人-水系统变化更是成了社会-水文学研究的重点。人与黄河流域水资源的关系有着更长的互动历史与更具一般性的演化过程,与其它干旱区的大河流域更具相似性和可比性,如科罗拉多河流域和墨累-达令河流域[44,45]。王铭铭就将中国水利社会分成了三种类型[46],分别是以都江堰为代表的丰水型社会,一种是以西北地区为代表的缺水型社会,另一种则是以内河航运及海运为代表的水运型社会。这种类型划分本质是对水在中国不同区域对人类社会最大的价值体现在何处的基本认识,黄河流域水资源是社会发展的天然制约性因素,因而显然属于缺水型社会。当前,在全流域已经完全被人工调控的黄河,日益严重的人-水资源矛盾成为多方利益相关者关注的重点,亟需发展人-水关系变化理论和相应的量化方法。

% 江恩慧 2022本子
上世纪80年代以来,随着社会经济的快速发展,人类活动成为影响流域地表过程的重要因素。在水文水资源研究方面,面对水沙自然资源供需与全球生态环境危机、社会经济可持续发展之间的强烈冲突,仅从河流/流域自然层面研究已经不能满足现实需求,厘清人与自然耦合系统运行规律和演化机制,受到国际学界高度重视(Gleeson等,2020)。国际地圈生物圈计划(IGBP)框架强调植被-生态过程与水文系统的耦合。国际水文科学协会(IAHS)提出了“万物皆流”(Panta Rhei-Everything Flows)研究计划,强调流域水-生态-经济系统的研究思路。Sivapalan等(2012)提出了社会水文学(Socio-hydrology),致力于探索水资源视角下水文系统与人类系统的双向互馈机制,旨在更加深刻地表达人-水耦合系统协同演化过程,以实现水资源的可持续利用和管理。王浩等(2006,2010)构建了自然-社会二元水循环理论体系,充分考虑了人类活动对水文过程和水资源循环过程的影响。李少华等(2007)提出了水资源复杂巨系统的概念,将水资源、人口、社会、经济、生态、环境等均视为水资源系统的子系统。在泥沙研究方面,随着大规模的人类活动改变了流域下垫面条件,进而影响流域侵蚀产沙过程与进入河流的水沙关系,使得人为因素对流域水沙关系的影响成为当前研究重要内容之一(傅伯杰等,2010)。

% 江恩慧 2022本子
全球气候变化和人类活动深刻影响着水文泥沙过程、生态环境演变和社会经济发展进程(Best,2019)。世界上众多大型河流的水沙通量发生了显著变化(Li等,2020),尤以黄河最为突出。近年来,黄河的径流、泥沙分别减少40\%和80\%以上,远超世界平均水平(刘晓燕,2020;胡春宏和张晓明,2018)。对水沙变化趋势性、周期性的研究基本取得了共识,认为气候变化和人类活动共同影响使得水沙变化更加复杂,人类活动是水沙趋少的主要驱动原因(Wang和Sun,2020;Wang等,2016,2020;Ma等,2020;Zhang等,2021);沙量减幅大于水量减幅,输沙量集中度和不均匀程度大于径流量(Tian等,2019)。

% 江恩慧 2022本子
随着黄河流域生态保护和高质量发展重大国家战略的深入实施,黄河流域社会经济从“高速发展”进入了“高质量发展”阶段,需从过去对水资源的过度开发和粗放式利用转向产业结构优化调整下的节约集约利用,对流域水资源配置提出了更高要求。然而,黄河水资源开发利用程度高达80\%,甚至在西北部分地区达到了100\%,已触及水资源供给的“天花板”(王建华等,2020);沿黄城市发展高度依赖黄河水资源配置,且已有90\%以上城市处于水资源超载状态(李原园等,2021);未来水资源供需矛盾有可能进一步加剧,据王煜等(2019)预测,河南黄河供水区2025年缺水14亿m3、2035年缺水17亿m3。


% 开题报告
2.1 黄河流域的人-水关系演变研究
社会系统与水文系统都是动态变化的系统,在不同尺度上有各自的变化规律,人-水关系也会因此随之发生演变。人-水关系的演变是理解人与自然的关系的重要切入点,因为其常常与人-水系统内跨区域、多尺度、且常常与不可逆的系统结构-功能重大变化相联系[26–29]。国际水文科学研究协会(International Association of Hydrological Sciences, IAHS)提出2013—2022年的“科学十年”研究主题为“万物皆流”(Panta Rhei—Everything Flows),该计划强调发展先进的监测和分析技术,广泛寻求水文学与社会经济的跨学科联系,将解析人-水系统的内在变化过程作为水文学发展蓝图的关键[18],但对人-水关系宏观演变的分析视角尚未形成共识。如于宏瑞认为人-水关系是随着社会发展而变化的,因此经历了自然崇拜、趋势利用、盲目开发、持续协调四个阶段[19]。Di Baldassarre 等人则着眼于人类根据应对水文系统挑战的不同模式,区分“适应自然”与“对抗自然”两种演变模式[20,21]。Kandasamy等人则提出人对水的观念演变,始终存在悬于开发和保护之间来回摇摆的“钟摆效应”[22]。因此,Stephanie(2021)总结认为[23]当今人-水系统研究可分为“水的社会性”和“水的技术性”两类。如《大国大河》[24]与《征服自然》[25]两项著作就分别以美国和德国为例,从“水的社会性”与“社会控制水”两个角度系统梳理了两国主要大河流域的人-水关系演变史。前者高度强调“水的社会性”是塑造当代美国社会性质的重要自然因素;后者则指出德国出于征服和控制的考虑,利用技术永远重塑了自然水文景观。

黄河流域同样是世界上人-水关系变化最频繁、最剧烈、影响最为深远的大河流域之一。从水的社会性上看:大禹治水的故事对中华文明的脉络有着深远的影响[30];三门峡与下游运河的运输功能对历代王朝的国祚可谓牵一发而动全身[30];近年来水资源的区域间分配也对经济发展有着制约性影响[31]。从社会控制水的角度上看,历代中央王朝的黄河洪泛治理对水文系统变化影响深远;引黄灌溉则是这个干旱-半干旱区流域的农业经济发展命脉;近年来水库的调水调沙更是彻底重塑了黄河下游的自然生态环境[32]。但是,黄河流域这种频繁的人-水关系变化常常为梳理人-水系统复杂的的反馈循环与协同演化带来了更大的困难[12,33]。如魏夫特考察黄河流域的社会发展历史后提出了著名的“水利社会理论”,认为这样大规模的治水活动是中央封建王朝皇权诞生的重要因素;但更多中国学者认为魏氏倒转了复杂人-水关系中的因果,正是中央皇权的大力发展才使得大规模治水活动变得可能[35,36]。又如治水方略的选择通常被认为是决定黄河河道演变的重要人为因素,同时河道的自然条件又是选择治理方略的重要依据,由两者共同决定的治理结果则会造成路径依赖[34]。因此,为了梳理黄河流域无处不在的、长时间尺度的、影响深远的人-水关系变化,也不乏着眼于总结演变过程的著述。于宏瑞将黄河流域的人-水关系演变的典型事件按区域(上、中、下游)进行了细致梳理[19],但对跨时空的事件关联着墨不多。也有研究倾向于按照黄河流域人-水关系的关键主题梳理演变进程,力求展示出其时间纵深和因果联系,如葛剑雄从黄河与中华文明互动的角度切入[30],王经纬从黄河治理的角度切入[34],各自梳理了长时间尺度黄河的人-水关系演化。但由于切入主题常常对应于人-水系统的单一或部分关键功能,若缺乏统一的分析框架对其进行整合,仍难以满足流域人-水系统模拟和预测对系统性和可解释性的要求。

综上所述,目前对黄河流域人-水关系演变的分析以案例研究居多,常落于列表式梳理或分功能的阐释,缺乏系统性分析框架,因而难以突破时间、空间、主题的限制,无法对人-水关系的演变过程及机制产生系统性认识。

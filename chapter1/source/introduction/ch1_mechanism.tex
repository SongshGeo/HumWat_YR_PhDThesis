社会系统和生态系统的互馈机制是社会-生态系统研究的核心内容,也是当前社会 -生态系统研究的热点和前沿。根据侧重点的不同,当前社会-生态系统互馈机制的研究 可以分为“时”、“空”、“构”、“阈”四个方面


\subsubsection*{结构-功能}

\subsubsection*{人水匹配}
% 人水关系 匹配 comment
由于人类社会系统和流域水文系统的结构和功能具有不同的尺度和动态,匹配是指两者之间的良好关系。

\subsubsection*{动力学}

\subsubsection*{稳态转换}

总体来说,尽管人-水系统协同演化的理论不断发展,相应的量化分析工具也在不断丰富,但多见诸于解耦自然过程与社会过程对人-水系统的影响,对其系统的演变机制鲜有探讨。而且,这种解耦常常建立在人-水关系已经发生变化的经验主义基础之上,因而在黄河流域多集中于人类强烈干预的人-水沙关系等问题的研究,忽略了人-水资源关系的量化分析及理论探讨。


% 开题报告
模型的建立是为了理解、描述和预测自然界,应当在真实性、预测性、普遍性之间达到均衡[27],流域系统模型通常旨在表征水的地球物理过程的动力学,以及人类用于管理系统的元素,如基础设施、机构和治理。 %\cite{hadjimichael2020}。
流域人水系统既有空间尺度依赖的地表生态过程,也有系统层次变量反馈的动力过程,还有多尺度的复杂人水相互作用[47–49]。
相应地,流域人水系统的建模主要有基于传统分布式水文模型耦合人类活动模块发展而来的分布式社会-水文模型、在系统或区域层次上耦合来自社会、自上而下模拟生态系统关键变量的系统动力学模型、自下而上对流域内复杂人-水互动进行仿真的多主体模型。

\subsubsection*{分布式社会-水文模型}

% 焦 本研一体 本子
与传统的集总式水文模型相比,分布式流域水文模型不再将流域视作均匀的整体,充分地考虑了流域内水文过程的异质性[207],是流域研究的主流工具,常见的分布式水文模型如如SWAT模型、新安江模型、陕北模型、布式时变增益水循环模型等(徐宗学,2019)[208-210],在国内外都得到了大量应用[211-213]。 % 焦 本子
% [207]王中根,刘昌明,吴险峰.基于DEM的分布式水文模型研究综述[J].自然资源学报,2003(02):168-173.
% [208]J. G. Arnold,R. Srinivasan,R. S. Muttiah,J. R. Williams. LARGE AREA HYDROLOGIC MODELING AND ASSESSMENT PART I: MODEL DEVELOPMENT[J]. JAWRA Journal of the American Water Resources Association,1998,34(1):73-89.
% [209]王中根,刘昌明,黄友波.SWAT模型的原理、结构及应用研究[J].地理科学进展,2003(01):79-86.
% [210]夏军,王纲胜,吕爱锋,谈戈.分布式时变增益流域水循环模拟[J].地理学报,2003(05):789-796.
% [211]Karim C. Abbaspour,Jing Yang,Ivan Maximov,Rosi Siber,Konrad Bogner,Johanna Mieleitner,Juerg Zobrist,Raghavan Srinivasan. Modelling hydrology and water quality in the pre-alpine/alpine Thur watershed using SWAT[J]. Journal of Hydrology,2006,333(2):413-430.
% [212]Darren L. Ficklin,Yuzhou Luo,Eike Luedeling,Minghua Zhang. Climate change sensitivity assessment of a highly agricultural watershed using SWAT[J]. Journal of Hydrology,2009,374(1):16-29.
% [213]Gangsheng Wang,Jun Xia,Ji Chen. Quantification of effects of climate variations and human activities on runoff by a monthly water balance model: A case study of the Chaobai River basin in northern China[J]. Water Resources Research,2009,45(7).
% 江 本子
流域分布式模型通过耦合生态过程,可用于描述大尺度流域陆地生态演变过程的生态水文模型,如SWIM模型(Krysanova等,2005)、RHESSys模型(Tague和Band,2004)、Budyko–Choudhury–Porporato模型(Donohue等,2012)、EHSM模型(Viola等,2014)、HYMOD-BGM模型(Tang等,2018)等。
国内的包括EcoHAT模型(刘昌明等,2009)、WEP-IBIS模型(Cao等,2015)、CLM-GBHM模型(Jiao等,2017)、GBEHM模型(Qin等,2017)、BEPS-TerrainLab模型(Chen等,2007)、HEIFLOW(Tian等,2018)等。

% 焦 本子
现有的分布式水文模型由于构建原理及最初率定区域不同,导致模型侧重点有所不同。如SWAT模型侧重描述产流过程[209],LISTFLOOD模型侧重模拟水动力过程、洪水过程等[217],但现有的分布式模型仍较少将人为干预水文过程的因素作为模拟重点,
% [209]王中根,刘昌明,黄友波.SWAT模型的原理、结构及应用研究[J].地理科学进展,2003(01):79-86.
% [217]曾照洋,王兆礼,吴旭树,赖成光,陈晓宏.基于SWMM和LISFLOOD模型的暴雨内涝模拟研究[J]
贾仰文等人WEP-L分布式流域二元水循环模型(简称 WEP-L 模型)是具有物理机制的流域分布式水循环模型,考虑了人类取用水和水利水保工程等因素对水循环过程的影响,实现“自然-社会二元水循环”过程耦合模拟和分析。 % todo citation
2019年,国际应用系统分析研究所(International Institute for Applied Systems Analysis, IIASA)开发了基于社区的水文模型模型(Community Water Model, CWatM)模型,将水库调度等水资源管理要素也纳入了模型[218]。
但迄今为止,仍鲜有将水资源治理制度(如法律法规)等人类活动要素的影响作为流域分布式模型模拟的重点。
% [218]Peter Burek, Yusuke Satoh, Taher Kahil,et al. Development of the Community Water Model (CWatM v1.04)- a high-resolution hydrological model for global and regional assessment of integrated water resources management[J].GEOSCIENTIFIC MODEL DEVELOPMENT,2020,13(7):3267-3298.

\subsubsection*{自上而下的系统动力学模型}

% 江本子
系统动力学模型能够自上而下地解析系统层面的要素关联、反馈与演化(Jaeger等,2017;Jiang等,2022),可预测变化环境下关键变量及其反馈过程的变化(Vaighan等,2017)。
许多大尺度的评估模型都是系统动力学模型,有 ANEMI3 和在中国区域建立的 T21 China 模型,均包含自然和社会系统的多个部门,反映其相互作用。
因此,流域人水系统的系统动力学模型可以通过分析系统间的相互作用、内部结构变化等现象,揭示流域人水关系的动态特征,如 ANEMI Yangtze 是在 ANEMI3 的基础上在长江流域建立的系统动力学模型。
将空间范围限制在流域尺度的系统动力学模型可以在建模过程中考虑研究区特征,使模拟结果更符合流域系统的实际情况,例如 ANEMI Yangtze 将长江的渔业作为模型的外生变量纳入考虑,分析了十年禁渔政策对流域系统的影响。 % todo citation

人类活动还可以作为系统内部反馈循环的关键变量被纳入模型。
% 开题报告
Viglione 和 Baldassarre 等人在流域尺度构建了用于解释和预测人类社会与洪水互相反馈、协同演化的系统动力学模型,其中水文要素(洪水)和社会组分(如公众意识、风险文化、经济发展等)都是反馈过程的核心变量[43]。
Muneepeerakul 等人(2017)则提出了一个流域社会-生态系统的发展轨迹模型,以探讨在什么情况下稳定的治理结构可以在促使堤坝等公共基础设施在系统中内生地出现\cite{muneepeerakul2017}。
系统动力学的劣势也很明显,首先是缺乏对空间特征的识别,这对流域系统是至关重要的;其次是通常需要自上而下对系统要素间关系给出基于数据或经验的函数假设,对于较为复杂且尚缺乏充分研究的流域系统变化,建模即便不是不可能,也将是非常困难的。

\subsubsection*{自下而上的多主体模型}
% 开题报告
自下而上的多主体模型(Agent-based model)在研究人类活动对水资源的影响时,将不同的利益相关者,如管理者、水用户等作为系统的不同主体分考虑它们的相互影响关系,研究他们通过自组织在流域人水系统层面的涌现。
自组织(Self-organization)是指一种起源于初始无序系统的部分元素之间的局部相互作用、所产生出某种形式的整体秩序的过程,涌现(Emergence)则指系统实体会产生其所有组成部分本身没有的属性,人-水系统作为开放的复杂巨系统,广泛存在的相互作用就是宏观演化属性涌现的关键[51]。
Manson 等人(2016)研究在制度、社会网络、其他农户行为影响下,涌现出怎样的土地利用格局\cite{manson2016}。
Castilla-Rho等人[61–63]利用多主体建模应用于地下水资源管理之中,指出社会规范的参数可解释地下水资源可持续治理模式的涌现。

受限于验证困难和计算量大的特性,过去多主体模型常以简单机制模型的形式探索系统层面的涌现与发展轨迹,对复杂环境条件的刻画并不好。
复杂多变的时空变化是流域人水关系演变的重要驱动力,随着近年来计算机算力提升和各类仿真算法的增强,越来越多的多主体模型开始尝试结合带有空间分布的实际数据进行多主体建模。
Grêt-Regamey 等人(2019)结合调查统计数据,使用多主体模型证明了行为者多样性有助于提升社会-生态系统对全球变化的韧性。
借助非洲早期的水资源分布数据,多主体模型被用以证明“逐水而居”可能是人类早期迁徙演化的重要驱动力[42];
Sayles(2017)结合多主体模型与网络分析,研究了流域尺度社会-生态的制度匹配,证明制度与生态连接脆弱的区域容易产生生态问题。 % todo citation

综上所述,建模已成为研究人-水系统演变机制的重要手段,但目前比较成熟的分布式水文模型严重忽略了政策制度等非工程人为因素,系统动力学模型缺乏重要的空间信息。
自下而上的多主体模型能够仿真流域内部复杂的人水关系,且随着计算机算力的提升正不断补强其结合复杂环境空间分布数据的能力,是揭示流域系统层面人水关系变化机制的理想建模方法。

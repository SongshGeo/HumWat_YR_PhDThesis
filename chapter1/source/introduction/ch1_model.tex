
刻画人水关系的模型和计量手段日益丰富,如应用多主体模型揭示了逐水而居的本能可能是人类早期迁徙演化的重要驱动力[42];系统动力学模型被用于解释和预测人类社会面对流域频发的洪水灾害时,水文和社会系统组件(如公众应对、风险文化、经济发展等)的互馈作用[43]。 社会-水文学的快速发展既是对“人-水系统耦合”理论需求的回应,也推动了对人-水关系量化分析与未来变化预测的前沿探索[20,21]。

\subsubsection*{数据驱动的统计模型}


\subsubsection*{分布式社会-水文模型}

WEP-L 分布式流域二元水循环模型( 简称 WEP-L 模型) 是具有物理机制的流域分布式水循环模型,可以 综合考虑气象、下垫面、人类取用水、水利水保工程等因素对水循环过程的影响,实现“自然-社会”二元水 循环过程耦合模拟和分析,给出水循环要素时间和空间变化过程以及流域水循环通量。

\subsubsection*{自上而下的系统动力学模型}


\subsubsection*{自下而上的多主体模型}
% Hadjimichael 2020
水系统模型通常旨在表现水的地球物理过程的动力学,以及人类的元素系统用于管理:基础设施、机构和治理(劳克斯,1992)。

% 开题报告
2.3 基于复杂系统的人-水关系建模
作为耦合系统,人-水系统既有系统尺度的动力学结构,也有多尺度系统的相互作用[47–49]。如何确定不同子系统的尺度和表达各个子系统间的耦合关系,对更好的理解和预测地球系统至关重要。例如2021年的诺贝尔物理学奖得主Syukuro Manabe和Klaus Hasselmann就发现,从子系统相互作用的角度去考虑大尺度天气变化甚至全球气候这样的混沌系统时,我们不仅能够预测全球大尺度气候系统的宏观行为,甚至还可以评估人类的碳排放怎样对全球气候造成影响[50]。涌现(Emergence)指的就是系统实体会产生其所有组成部分本身没有的属性[51],如大尺度气候系统的宏观行为就是对子系统交互带来的涌现结果[52]。涌现的属性或行为只有当各个部分在一个更广泛的整体中相互作用时才会涌现,是系统组分相互作用的宏观体现。人-水系统作为开放的复杂巨系统,广泛存在的相互作用就是宏观演化属性涌现的关键。

对人-水资源关系来说,流域的利益相关者与水资源的关系就是复杂系统最基本的互动要素。大型流域的人-水系统通常涉及多个利益相关者,利益相关者之间以及利益相关者与流域系统之间存在依赖、竞争、关联等复杂的相互作用,塑造了复杂的、动态的、非线性的系统动态过程。在传统的流域管理模式中,通常忽略了资源环境承载力对于政策的非线性响应,对在人地关系变化过程中流域管理机构、利益相关者的结构、功能与动态也未加以足够重视[53]。理解流域水资源使用时上中下游不同行业利益相关者的冲突、合作,及其相互作用下治理体系的整理结构与功能,逐渐成为流域可持续治理的重要基础[53,54]。例如在澳大利亚的马兰比吉河流域,新兴的水管理体系要求动态考虑粮食生产和水资源开采的关系,且可以通过经济活动的多样化来促进社区利益相关者参与水资源治理的意愿[22,26]。气候变化和人类活动在改变流域水循环的同时也影响着自然和农业生态系统水分利用效率,进而影响着流域资源环境承载力[55]。在以水定地、以水定人、以水定产的过程中,需要从水资源和生态系统承载底线、粮食需求上限以及三者的空间关系入手,发挥人地系统耦合模型的情景预测能力,并考虑流域内不同利益相关者之间形成的治理结构关系[56]——如水资源使用、粮食生产和生态保护之间的协作,实现满足整体性、区域性和动态性需求的大型流域资源环境承载能力的动态预警[57,58]。因此,随着人类改造自然的能力增强,利益相关者之间的协作、竞争、改造与控制等行为如何通过涌现带来人-水关系的宏观演化,成为亟待使用复杂系统建模解释其发生机制。

模型的建立是为了理解、描述和预测自然界,这几个目的相互重叠但绝不相同,一个模型应当在真实性、预测性、普遍性之间达到一个均衡[27]。尽管社会水文学模型的最终目标是预测人水耦合系统[28],但早期的研究主要在探索利用模型解释的普遍性规律[16-18]。Ostrom 借助社会-生态系统(Social-ecological system, SES)框架提出的“公共池塘资源(Common-pool resources)”的自组织(Self-organization)管理中就着重考虑水资源的非排他性和竞争性,提出了基于博弈论的演化模型以揭示此类系统的普遍性机制,并将其与SES的可持续性相联系[59,60]。这个获得诺贝尔经济学奖的成果随后成为包括人-水复杂系统在内的。自组织(Self-organization)是指一种起源于初始无序系统的部分元素之间的局部相互作用、所产生出某种形式的整体秩序的过程。这与复杂系统建模中自下而上的基于主体的建模(Agent-based model) 思想类似,因此Castilla-Rho等人[61–63]利用多主体建模的复杂系统模拟方法将Ostrom发现的机制应用于地下水流域的管理模型中,发现可以解释地下水资源治理模式的涌现。但对于大河流域来说,由于人能直接观察到流动的水资源变化,且存在明显的时空分异性,尚缺乏较好的复杂系统建模以探索其中人-水关系的核心演化机制。

总的来说,基于复杂系统的建模已成为研究人-水系统的重要技术手段,能够基于水的资源属性对利益相关者的人-水互动进行理论机制上的探索。这种自下而上的建模思路与SES的自组织管理机制一脉相承,但目前比较成熟的模型主要关注地下水流域和,对河流的水资源治理,尤其是水资源稀缺的大河尚缺乏泛用性较强的机制模型。
\subsubsection*{统计模型}

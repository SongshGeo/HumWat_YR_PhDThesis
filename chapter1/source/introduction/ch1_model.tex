

% 开题报告
模型的建立是为了理解、描述和预测自然界,应当在真实性、预测性、普遍性之间达到均衡[27],流域系统模型通常旨在表征水的地球物理过程的动力学,以及人类用于管理系统的元素,如基础设施、机构和治理。 %\cite{hadjimichael2020}。
流域人水系统既有空间尺度依赖的地表生态过程,也有系统层次变量反馈的动力过程,还有多尺度的复杂人水相互作用[47–49]。
相应地,流域人水系统的建模主要有基于传统分布式水文模型耦合人类活动模块发展而来的分布式社会-水文模型、在系统或区域层次上耦合来自社会、自上而下模拟生态系统关键变量的系统动力学模型、自下而上对流域内复杂人-水互动进行仿真的多主体模型。

\subsubsection*{分布式社会-水文模型}

% 焦 本研一体 本子
与传统的集总式水文模型相比,分布式流域水文模型不再将流域视作均匀的整体,充分地考虑了流域内水文过程的异质性[207],是流域研究的主流工具,常见的分布式水文模型如如SWAT模型、新安江模型、陕北模型、布式时变增益水循环模型等(徐宗学,2019)[208-210],在国内外都得到了大量应用[211-213]。 % 焦 本子
% [207]王中根,刘昌明,吴险峰.基于DEM的分布式水文模型研究综述[J].自然资源学报,2003(02):168-173.
% [208]J. G. Arnold,R. Srinivasan,R. S. Muttiah,J. R. Williams. LARGE AREA HYDROLOGIC MODELING AND ASSESSMENT PART I: MODEL DEVELOPMENT[J]. JAWRA Journal of the American Water Resources Association,1998,34(1):73-89.
% [209]王中根,刘昌明,黄友波.SWAT模型的原理、结构及应用研究[J].地理科学进展,2003(01):79-86.
% [210]夏军,王纲胜,吕爱锋,谈戈.分布式时变增益流域水循环模拟[J].地理学报,2003(05):789-796.
% [211]Karim C. Abbaspour,Jing Yang,Ivan Maximov,Rosi Siber,Konrad Bogner,Johanna Mieleitner,Juerg Zobrist,Raghavan Srinivasan. Modelling hydrology and water quality in the pre-alpine/alpine Thur watershed using SWAT[J]. Journal of Hydrology,2006,333(2):413-430.
% [212]Darren L. Ficklin,Yuzhou Luo,Eike Luedeling,Minghua Zhang. Climate change sensitivity assessment of a highly agricultural watershed using SWAT[J]. Journal of Hydrology,2009,374(1):16-29.
% [213]Gangsheng Wang,Jun Xia,Ji Chen. Quantification of effects of climate variations and human activities on runoff by a monthly water balance model: A case study of the Chaobai River basin in northern China[J]. Water Resources Research,2009,45(7).
% 江 本子
流域分布式模型通过耦合生态过程,可用于描述大尺度流域陆地生态演变过程的生态水文模型,如SWIM模型(Krysanova等,2005)、RHESSys模型(Tague和Band,2004)、Budyko–Choudhury–Porporato模型(Donohue等,2012)、EHSM模型(Viola等,2014)、HYMOD-BGM模型(Tang等,2018)等。
国内的包括EcoHAT模型(刘昌明等,2009)、WEP-IBIS模型(Cao等,2015)、CLM-GBHM模型(Jiao等,2017)、GBEHM模型(Qin等,2017)、BEPS-TerrainLab模型(Chen等,2007)、HEIFLOW(Tian等,2018)等。

% 焦 本子
现有的分布式水文模型由于构建原理及最初率定区域不同,导致模型侧重点有所不同。如SWAT模型侧重描述产流过程[209],LISTFLOOD模型侧重模拟水动力过程、洪水过程等[217],但现有的分布式模型仍较少将人为干预水文过程的因素作为模拟重点,
% [209]王中根,刘昌明,黄友波.SWAT模型的原理、结构及应用研究[J].地理科学进展,2003(01):79-86.
% [217]曾照洋,王兆礼,吴旭树,赖成光,陈晓宏.基于SWMM和LISFLOOD模型的暴雨内涝模拟研究[J]
贾仰文等人WEP-L分布式流域二元水循环模型(简称 WEP-L 模型)是具有物理机制的流域分布式水循环模型,考虑了人类取用水和水利水保工程等因素对水循环过程的影响,实现“自然-社会二元水循环”过程耦合模拟和分析。 % todo citation
2019年,国际应用系统分析研究所(International Institute for Applied Systems Analysis, IIASA)开发了基于社区的水文模型模型(Community Water Model, CWatM)模型,将水库调度等水资源管理要素也纳入了模型[218]。
但迄今为止,仍鲜有将水资源治理制度(如法律法规)等人类活动要素的影响作为流域分布式模型模拟的重点。
% [218]Peter Burek, Yusuke Satoh, Taher Kahil,et al. Development of the Community Water Model (CWatM v1.04)- a high-resolution hydrological model for global and regional assessment of integrated water resources management[J].GEOSCIENTIFIC MODEL DEVELOPMENT,2020,13(7):3267-3298.

\subsubsection*{自上而下的系统动力学模型}

% 江本子
系统动力学模型能够解析流域系统层面的要素关联、反馈与演化(Jaeger等,2017;Jiang等,2022),可预测变化环境下流域系统关键变量及其反馈过程的变化(Vaighan等,2017)。

\cite{muneepeerakul2017}提出了一个框架和正式的风格化模型,以探讨在什么情况下稳定的治理结构可以在由共享的自然、社会和建筑基础设施组成的耦合基础设施系统中内生地出现。

V等人

流域人水系统的系统动力学模型是一种用于研究人类活动与水资源系统间相互关系的模型。该模型建立在系统动力学理论的基础上,结合社会经济、环境、生态等多个领域的相关知识,通过分析系统间的相互作用、内部结构变化等现象,揭示流域人水关系的动态特征。

% 开题报告
系统动力学模型被用于解释和预测人类社会面对流域频发的洪水灾害时,水文和社会系统组件(如公众应对、风险文化、经济发展等)的互馈作用[43]。

\subsubsection*{自下而上的多主体模型}
% 开题报告
涌现(Emergence)指系统实体会产生其所有组成部分本身没有的属性,人-水系统作为开放的复杂巨系统,广泛存在的相互作用就是宏观演化属性涌现的关键[51]。
% 来自 chat GPT
流域人水系统的多主体模型是指在研究人类活动对水资源的影响时,将不同的相关主体,如政府、水用户、环境保护者等作为系统的不同主体分别进行考虑,并综合考虑它们之间的相互影响关系,以达到更加全面、系统地研究人水关系的目的。这种模型在人水关系研究领域越来越受到关注,为了更好地研究人类活动对水环境的影响,并为流域水资源管理提供有力的指导。

自组织(Self-organization)是指一种起源于初始无序系统的部分元素之间的局部相互作用、所产生出某种形式的整体秩序的过程。这与复杂系统建模中自下而上的基于主体的建模(Agent-based model) 思想类似。
因此Castilla-Rho等人[61–63]利用多主体建模的复杂系统模拟方法将Ostrom发现的机制应用于地下水流域的管理模型中,发现可以解释地下水资源治理模式的涌现。但对于大河流域来说,由于人能直接观察到流动的水资源变化,且存在明显的时空分异性,尚缺乏较好的复杂系统建模以探索其中人-水关系的核心演化机制。

自下而上模拟流域水资源使用时上中下游不同行业利益相关者的冲突、合作,及其相互作用下治理体系的整理结构与功能,逐渐成为流域可持续治理的重要基础[53,54]。


刻画人水关系的模型和计量手段日益丰富,如应用多主体模型揭示了逐水而居的本能可能是人类早期迁徙演化的重要驱动力[42];

由于人类社会系统和流域水文系统的结构和功能具有不同的尺度和动态,匹配是指两者之间的良好关系。
Sayles 2017 研究了流域尺度社会-生态的匹配

总的来说,基于复杂系统的建模已成为研究人-水系统的重要技术手段,能够基于水的资源属性对利益相关者的人-水互动进行理论机制上的探索。这种自下而上的建模思路与SES的自组织管理机制一脉相承,但目前比较成熟的模型主要关注地下水流域和,对河流的水资源治理,尤其是水资源稀缺的大河尚缺乏泛用性较强的机制模型。



随着大河流域的水循环过程受到强烈的人为改造,其承担的社会、生态功能业已逼近安全运行的“地球边界”[3,7,8]。当前研究中人-水关系理论正在不断完善,王浩等提出“自然-社会二元水循环”理论,指出流域水循环的驱动力、循环结构、循环参数在人类活动的影响下发生了二元演变效应[37,38];邓铭江等在此基础上构建“自然—社会—贸易”三元水循环模式,解释西北干旱区内陆河流域的水循环联系、双向反馈机制和协同进化机理[39]。从水的社会性出发,张亚辉指出人-水关系研究分为“实践理性”和“文化理性”[9],前者指的是在生产实践使用水的过程中因控制、竞争、分配、排斥而产生的人-水互动;后者则是指社会中因水观念、水文化而产生的人-水互动。这些理论是指导大河流域人-水关系分析的重要基础。

\subsubsection*{水的自然-社会属性}

% 开题报告
对自然生态系统而言,从水文地质循环到生态水文过程,从旱区陆地到河湖水体,水是大多数陆地生态系统的主要限制因素,也是物质运移和环境演化的重要介质[3,7,8]。对于人类社会系统而言,从饮食洗涮到田间地头再到江河治理,从民间祭祀到哲学论述再到神话传说,水是一种贯穿了自然与养育、实践与象征的物质[9]。理解人-水关系就是理解人-水复杂系统内部人类社会系统与自然水系统间的相互作用模式[10],重塑人水关系就是通过调整这一互动模式来重塑人与自然的关系。
及其动态变化,并找到实现匹配的方法,从而将流域引向可持续发展的轨道。

% 人水关系 匹配 comment
人类的水关系取决于我们如何对待河流,以及我们从流域中得到什么。就像恒河的信徒相信河流能净化他们的灵魂,尼罗河的农民期待着河流的丰收,胡佛站在大坝前骄傲地宣布他已经 "征服 "了科罗拉多河。社会越来越多地为了自己的福祉而改造流域,尽管水经常提供许多其他关键功能,如调节气候、支持生物质生产、物质运输和净化环境。这些功能对于可持续发展可能更加重要,以保持流域社会生态系统的弹性,使其能够适应外部变化或在转型中摆脱危机。然而,在许多情况下,人类改造流域的活动作为干扰超过了系统的临界点,导致维持核心水功能的反馈回路发生变化,引发社会生态系统的稳定转变。为了避免可能导致系统崩溃的稳态转变的级联效应,必须将人水关系的变化视为流域社会生态系统的内部原因,特别是在人类影响的人类世。
% 开题报告
例如水资源管理作为流域特定社会文化、经济水平和政治体制的直接产物,通过调整水量的分配进而作用于生态系统,决定生态系统的健康状况和社会系统的人类福祉。长期以来,流域水资源管理往往通过调控快变量,旨在短期内提高水的利用规模和效益,较少考虑系统状态的长期演变,对生态和社会慢变量的积累变迁更是缺乏反馈机制,限制了维持流域长期可持续的能力。此外,尽管因流域而异的演化轨迹是人-水系统复杂性的体现,但世界主要大河流域也展现出了相似的关键变量与作用路径。例如,在干旱-半干旱区的墨累-达令河流域、科罗拉多河流域以及黄河流域,水资源量的限制都促使水资源分配制度的诞生,该制度又以相似的作用方式影响了河流的水文状态。长期以来这些作用路径被认为是人-水系统演化中的突发政策性影响,因而忽略了其背后的一般性相互作用机制,制约了人-水关系调控的系统性和前瞻性。因此,识别流域人-水关系的长期演化,并理解复杂系统关键变量对演变过程的核心作用机制,对流域的可持续、高质量发展至关重要。


\subsubsection*{流域社会-生态系统}
% Handbook 什么是社会-生态系统
社会-生态系统是一个新兴的概念,用相互联系和相互依存的方式来理解人类与自然系统相互交织的本质。这一概念最早诞生于20世纪90年代中期,通过生态经济学和公共池塘资源系统学者间的跨领域学者发展而来(如xxx)。特别是《联系社会与生态系统:韧性构建的管理实践与社会机制》一书将系统科学方法和适应性管理结合起来,重点关注了动态的制度和不同的产权制度。书中14个案例研究分析了复原力和地域特色的生态系统管理。社会生态系统秉承“人类区分社会系统与自然系统的区分是武断的”这一理念,强调人与自然是相互交织的,大自然不仅仅是社会互动的舞台,人类也不仅仅是大自然变化的外驱力。因此,社会-生态系统不仅仅是生态系统与社会系统的简单加和,而是由社会和生态组分之内/之间反馈所塑造的有机整体。

% Handbook 什么是社会-生态系统
这意味着社会-生态系统是一个典型的复杂适应性系统(complex adaptive system)。这类系统由很多互相独立的部分组成,但互相之间以涌现的方式相互作用,系统层面的格局很难由某部分的属性来预测。此外,这类系统层面的格局也会反过来影响某个组分的行为和与其它组分的交互方式,创造出适应变化环境的、随时间演化的反馈过程。这种微观的持续相互作用涌现出的宏观格局,是社会生态系统不等于两者简单加和的原因。此外,这意味着社会生态系统可以适应环境的变化,通过学习和自组织初来响应内部和外部压力。这种动态变化的例子是个体在面对危机时选择交互与合作,围绕共同的愿景和叙事来连接和创建社会网络,导致自组织治理的涌现。这一来,随着已有组织和新的制度之间的桥梁出现,与其它不同层次的治理相联系和互相影响,经验表明社会-生态系统便可能转向新的演化路径。例如瑞典的景观管理,澳大利亚的大堡礁保护,以及南大洋区域资源的全球适应治理体系。

\subsubsection*{自然-社会二元水循环}


\subsubsection*{流域社会-水文学}

% 开题报告
在这一背景下,旨在理解人-水之间协同演化规律和循环互馈机制的社会-水文学应运而生并不断发展成熟[16,17]。社会-水文学(Socio-hydrology)的诞生[16]在人-水关系变化的现象检测、机理分析、模型预测等方面都获得了长足发展。新学科的发展通常需要经历数据收集与整理、数据观察与推测、理论模型建立、参数率定与模型修正、模型应用与预测五个阶段[40],Troy等人通过文献分析,认为社会水文学研究尚未突破前三个阶段[41]:历史数据与观测数据的收集尚不充分;案例积累不足以开展广泛的对比研究;相关理论模型仍在探索当中。
刻画人水关系的模型和计量手段日益丰富,如应用多主体模型揭示了逐水而居的本能可能是人类早期迁徙演化的重要驱动力[42];系统动力学模型被用于解释和预测人类社会面对流域频发的洪水灾害时,水文和社会系统组件(如公众应对、风险文化、经济发展等)的互馈作用[43]。 社会-水文学的快速发展既是对“人-水系统耦合”理论需求的回应,也推动了对人-水关系量化分析与未来变化预测的前沿探索[20,21]。

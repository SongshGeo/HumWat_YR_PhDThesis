\subsubsection*{概念模式的演变}

此外,尽管因流域而异的演化轨迹是人-水系统复杂性的体现,但世界主要大河流域也展现出了相似的关键变量与作用路径。例如,在干旱-半干旱区的墨累-达令河流域、科罗拉多河流域以及黄河流域,水资源量的限制都促使水资源分配制度的诞生,该制度又以相似的作用方式影响了河流的水文状态。长期以来这些作用路径被认为是人-水系统演化中的突发政策性影响,因而忽略了其背后的一般性相互作用机制,制约了人-水关系调控的系统性和前瞻性。因此,识别流域人-水关系的长期演化,并理解复杂系统关键变量对演变过程的核心作用机制,对流域的可持续、高质量发展至关重要。

\subsubsection*{关键指标变化过程}

\subsubsection*{变量间关系的演变}

\subsubsection*{系统结构-功能的演变}

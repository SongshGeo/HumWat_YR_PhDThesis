系统演变路径是社会-生态系统研究的核心内容,演化多样性是人水系统复杂性的体现,人水系统的演变过程更是流域社会-水文研究热点。根据侧重点的不同,当前人水系统与社会-生态系统演变过程的研究可以大致分为“概念模式变化”、“关键指标变化”、“结构功能变化”三个方面。

其中概念模式便是对许多共性规律的高度抽象或理论总结,以便研究人员和决策者在宏观层面把握人水系统的演变过程;关键指标是通过能够表征流域系统的关键变量或综合指标的变化规律把握流域演变过程;结构功能变化则是借由流域系统内重要组分的相互关系变化、以及于这种变化相关的系统功能改变来把握流域演变过程。

\subsubsection*{概念模式的变化}

首先,此类人水系统演变研究可以通过借鉴其它学科或其它领域的已有理论或概念提出。
例如张家诚(2006)从科学哲史和科学哲学的角度出发回顾人水关系的演变历史,指出人水关系从古代时“天-地-人的平衡中庸模式”,到近代工业社会变为在追求自己的发展时忽略自然环境的“数学模式”,同时展望了信息时代人水关系的“工程调控模式”\cite{zhang2006}。
Gleick 和 Palaniappan 借鉴“石油峰值”为流域提出了“水峰值”的概念,借助该概念可将流域潜在水供应分为三个变化阶段:阶段一随着用水需求的增加可用水资源的供应(新修水坝、水库、泵);一旦达到最大成本效益的地表水和地下水开采;有一个最终转移到一个更高的成本支撑的供应水如海水淡化或转移等各种来源的增量增加供应\cite{gleick2010}。

另一方面,人水系统演变的概念模式还可以通过流域的实际发展规律总结得出,这类理论模式通常有更活跃的理论生命力,也更能有助于指导后续研究和决策。
Turton 在1999年提出了影响深远的“流域适应能力”概念框架,指出根据水资源的供需关系,流域人水系统随着发展可能依次在“获取更多水资源供应”、“提高用水效率”、“提高分配效率”、“适应水短缺状态”四种原型模式间演变\cite{turton1999}。
该框架在澳大利亚流域水治理改革案例中的应用不仅佐证了其重要指导意义,还暗示了流域演变过程可能还存在水需求下降的“第五阶段”\cite{loch2020}。
另一个典型的案例是通过澳大利亚东部 Murrumbidgee 河流域总结出的“钟摆效应”模式,它说明了流域人水关系在取水用于粮食生产和努力缓解流域环境退化之间保持动态演进,并在摇摆间将流域人水系统演变划分为四个时期\cite{kandasamy2014, roobavannan2017},而这一规律随后也在中国和欧洲等更多区域得到了复现\cite{han2017, mostert2018}。

\subsubsection*{关键指标变化}

由于人水系统的复杂性,研究常从不同角度切入构建综合指标,用以表征人水关系的变迁。
刘海猛等人基于复杂自组织系统理论,在辨析人水系统基本内涵的基础上提出了人水关系演化的概念模型,将人水系统的演变过程概念化为社会经济、生态环境、水资源开发利用三个变量组的函数关系\cite{liu2014}。
Zuo 等人将“和谐人水关系”指标分为健康度、发展度、协调度\cite{zuo2008},该综合指标可将中国的人-水关系分为二十世纪中期以前、1950年至1980年、1980年至1990年、1990年之后四个时间阶段,各阶段的流域人水关系的主要特征分别是:应对用水需求和水灾、管理供水和去中心化、依法管理水资源、高度重视人水和谐\cite{zuo2016a}。

这种通过关键指标识别人水关系的方法须广泛收集来自不同领域的数据,在时间序列数据收集和数据同化上存在诸多挑战,但其优势在于一旦数据可用,能同时对全球各大流域进行大规模计算与分析。
Varis 等人(2019)基于三个社会系统的适应性指标和三种生态系统的脆弱性指标,构建了综合指数评估流域社会-生态系统在弹性和适应之间的平衡,分析了各流域通过发展提升适应性的演变过程\cite{varis2019}。
Huggins(2022)则更侧重流域在水资源压力下的脆弱性,通过全球各流域的水资源可用性指标和人类面对压力的适应指标,综合分析了不断加剧的淡水资源压力对流域社会-生态系统的潜在影响路径\cite{huggins2022}。
Qin 等人(2019)提出的稀缺性-韧性-易变性(SFV)指标,考虑了管理措施(如水库的建设)和用水结构变化(有些用水方式如能源用水是难以被短期替代的),选择了三个子指标对流域人类活动情况下水资源短缺情况做出评估,并分析了不同发展程度的地区在面对水资源短缺时的可能发展路径\cite{qin2019}。

\subsubsection*{结构功能变化}
但关系需要准确识别,哪些建立关键指标。
水、粮食、能源,以促进对这些超大规模的工程干预如何融入水管理战略进行更全面的评估。\cite{rollason2021}

Wang 等人提出了矛盾对立统一理论,该通过变量间关系,人水关系系统\cite{wang2017}。
wu 等人使用\cite{wu2020a}

流域匹配\cite{sayles2017}。

功能:Falkenmark 提出“蓝水”和“绿水”概念,将人类、水与自然的关系分为

song2022
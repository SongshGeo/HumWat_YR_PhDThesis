协同演化路径的多样性体现了人水系统的复杂性,其演变过程也是流域社会-水文研究的热点。从侧重点不同可将目前研究分为“概念模式变化”、“关键指标变化”和“结构功能变化”三个方面。
概念模式是对许多共性规律的高度抽象或理论总结,以便研究人员和决策者在宏观层面把握人水系统的演变过程。关键指标则通过能够表征流域系统的关键变量或综合指标的变化规律来把握流域演变过程。结构功能变化则是通过流域系统内重要组分的相互关系变化,以及与这种变化相关的系统功能改变来把握流域演变过程。

\subsubsection{人水系统的概念模式变化}

在人水系统演变研究中,可以借鉴其他学科或领域的已有理论或概念,总结其概念模式的演变。
例如张家诚从科学哲史和科学哲学的角度出发回顾人水关系的演变历史,指出人水关系从古代时“天-地-人的平衡中庸模式”,到近代工业社会变为在追求自己的发展时忽略自然环境的“数学模式”,同时展望了信息时代人水关系的“工程调控模式”\cite{zhang2006}。
于宏瑞认为人-水关系是随着社会发展而变化的,因此经历了自然崇拜、趋势利用、盲目开发、持续协调四个阶段\cite{yuruihong2011}。
Gleick 和 Palaniappan 借鉴“石油峰值”为流域提出了“水峰值”的概念,借助该概念可将流域潜在水供应分为三个变化阶段:阶段一随着用水需求的增加可用水资源的供应(新修水坝、水库、泵);一旦达到最大成本效益的地表水和地下水开采;有一个最终转移到一个更高的成本支撑的供应水如海水淡化或转移等各种来源的增量增加供应\cite{gleick2010}。

人水系统演变的概念模式还可以通过流域的实际发展规律总结得出,这类模式通常有更活跃的理论生命力,也更能有助于指导后续研究和决策。
1999年,Turton提出了影响深远的“流域适应能力”概念框架,该框架指出,根据水资源的供需关系,流域人水系统随着发展可能依次在“获取更多水资源供应”、“提高用水效率”、“提高分配效率”、“适应水短缺状态”四种原型模式间演变\cite{turton1999}。
该框架在澳大利亚流域水治理改革案例中的应用不仅佐证了其重要指导意义,还暗示了流域演变过程可能还存在水需求下降的“第五阶段”\cite{loch2020}。
另一个典型的案例是通过澳大利亚东部 Murrumbidgee 河流域总结出的“钟摆效应”模式,它说明了流域人水关系在取水用于粮食生产和努力缓解流域环境退化之间保持动态演进,并在摇摆间将流域人水系统演变划分为四个时期\cite{kandasamy2014, roobavannan2017},而这一规律随后也在中国和欧洲等更多区域得到了复现\cite{han2017, mostert2018}。

\subsubsection{人水系统的关键指标变化}

由于人水系统的复杂性,研究常从不同角度切入构建综合指标,用以表征人水关系的变迁。
刘海猛等人基于复杂自组织系统理论,在辨析人水系统基本内涵的基础上提出了人水关系演化的概念模型,将人水系统的演变过程概念化为社会经济、生态环境、水资源开发利用三个变量组的函数关系\cite{liu2014}。
Zuo 等人将“和谐人水关系”指标分为健康度、发展度、协调度\cite{zuo2008},该综合指标可将中国的人-水关系分为二十世纪中期以前、1950年至1980年、1980年至1990年、1990年之后四个时间阶段,各阶段的流域人水关系的主要特征分别是:应对用水需求和水灾、管理供水和去中心化、依法管理水资源、高度重视人水和谐\cite{zuo2016a}。

这种通过关键指标识别人水关系的方法须广泛收集来自不同领域的数据,因此在时间序列数据收集和数据同化上存在诸多挑战,但其优势在于一旦数据可用,能同时对全球各大流域进行大规模计算与分析。
Varis等人基于三个社会系统的适应性指标和三种生态系统的脆弱性指标,构建了综合指数评估流域社会-生态系统在弹性和适应之间的平衡,分析了各流域通过发展提升适应性的演变过程\cite{varis2019}。
Huggins等人则更侧重流域在水资源压力下的脆弱性,通过全球各流域的水资源可用性指标和人类面对压力的适应指标,综合分析了不断加剧的淡水资源压力对流域社会-生态系统的潜在影响路径\cite{huggins2022}。
Qin等人提出的稀缺性-韧性-易变性(SFV)指标,考虑了管理措施(如水库的建设)和用水结构变化(有些用水方式如能源用水是难以被短期替代的),选择了三个子指标对流域人类活动情况下水资源短缺情况做出评估,并分析了不同发展程度的地区在面对水资源短缺时的可能发展路径\cite{qin2019}。

\subsubsection{人水系统的结构功能变化}

流域系统的结构变化是实现功能改变的必要条件,人水关系变化通常伴随着系统结构功能变化。
例如,Wang等人提出了“矛盾统一”的人水关系理论框架,通过识别人与水之间是否是“对抗”关系,在两千年时间尺度上拟合了中国的人水关系变化\cite{wang2017}。
另一组常用于表征流域人水关系演变的要素是水、粮食和能源,但 Rollason 的研究指出,这三者之间的关系可能比许多研究中所表达的更为复杂\cite{rollason2021}。
因此,在利用关键要素间的关系进行识别时,需要精确把握关键变量,包括哪些要素需要纳入分析,以及如何构建它们之间的关系\cite{zhangzongyong2020, wang2021}。

改进上述问题的一个思路是使用复杂网络分析方法,尽可能考虑当前人水系统中所有需要考虑的因素,然后使用复杂网络指标分析系统演变\cite{sayles2019, bodin2017b}。
例如,Song 等人基于经济复杂性概念,使用全国的年均虚拟水足迹数据集(1978年至2008年)构建了农产品-虚拟水转移量的二分网络,通过复杂网络分析计算了网络复杂性随时间的变化,进而分析黄河流域水资源的重要性演变规律\cite{song2022}。
Sayles 等人结合生态和生态修复政策的数据,构建了流域的社会-生态网络,发现存在潜在结构问题的区域合作网络的密度和生产力都是最弱的\cite{sayles2017}。
要识别网络结构,需要收集相对全面的关系数据集,并且对数据质量要求较高,这在一定程度上限制了此种研究方法的广泛应用。

最后,还可以直接从系统功能的角度出发寻找表征系统演变的证据,该方法易于在不同流域系统中使用,但依赖对人水系统功能的正确认识,因此得到广泛认可的方法指标相对较少。
Falkenmark 等人提出的“蓝水”和“绿水”概念\cite{falkenmark2006},以及后来提出的“灰水”,都是典型的从系统功能角度出发,研究流域结构-功能变化的理论框架\cite{mekonnen2011}。
蓝水是河道与地下的液态水传统水资源规划和管理的重点,绿水是大多数在流域面上以降水和蒸散的形式参与循环,灰水则是少部分经人类利用后又参与循环的水资源,三者在流域人水系统中承担着截然不同的功能\cite{craswell2007}。
随着流域社会经济发展和气候变化,流域水循环各部分的绿水和蓝水比例都会发生变化。
这将对人水系统功能产生极为深远的影响,如降低弹性、削弱流域系统的生态系统服务、甚至导致系统崩溃\cite{falkenmark2019}。

综上所述,当前人水系统演变过程可以通过“概念模式变化”、“关键指标变化”、“结构功能变化”三个方面进行分析和总结,但仍需要加强三类框架之间的互相联系。当流域人水系统的变化足够显著时,上述三方面研究都能反映人水关系的演变:即通过识别关键指标变化的阈值,结构-功能表征系统整体变化,并对变化前后的系统状态做概念化总结。

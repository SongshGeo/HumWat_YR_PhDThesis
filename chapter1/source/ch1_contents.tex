
\subsection{研究目标}
% 武旭同话术,待修改
基于上述研究进展,本研究面向社会-生态系统研究前沿和可持续发展的国家需求, 以黄土高原为研究区,结合实地调查、遥感反演、水文气象观测、社会经济统计和历史 数据重建等获取多源数据,借助统计分析和模型模拟等手段,分析黄土高原社会-生态 系统的演变过程和驱动机制,研究黄土高原社会-生态系统的互馈机制。具体研究目标 如下:

% 开题报告
(1)建立一般性的人-水关系演变分析框架,借助该框架对黄河流域人-水关系演变历史有系统性、整体性认识,从而对其历史演化的主要阶段进行划分。
(2)重点考察人-水资源利用关系,定量分析黄河流域人-水资源利用关系的演化过程,找到关系变化的关键时期并对其中推动关系变化的核心机制产生基本认识。
(3)利用复杂系统建模对人-水资源关系的一般演化过程及其核心机制进行复现,再进行合理的情景假设,对黄河流域未来人-水关系变化趋势及可持续性进行模拟。

\subsection{研究内容}

% 开题报告
(1)黄河流域人-水关系演变的历史过程分析(图2-a)。为满足不同时间尺度的分析需要,首先建立统一的、古今适用的人-水关系分析框架,使不同时空尺度、不同人-水关系主题之间的横向对比分析变得可能。接下来利用建立的分析框架对黄河流域历史上的大事记进行分析,梳理出黄河流域人-水关系演变的主要脉络。最后利用系统回顾法,整合近代有研究以来对黄河人-水关系变化的定量分析成果,建立黄河流域人-水关系变化数据库,综合分析有研究以来人-水系统的主要驱动因素及演化路径。 
(2)黄河流域人-水资源关系的关键变化阶段(图2-b)。由于人-水关系的演变体现在水的多种属性上,本研究强调各级利益相关者都紧密联系的资源属性,拟选取人-水资源关系进行更深入一步的定量分析。在近60年的时间尺度上分别对水资源的分配、水资源的治理过程、水资源产生的效应进行定量分析。旨在找到人-水资源关系演变过程中最重要的变化阶段,并对推动演化的核心机制建立初步的科学猜想。
(3)大河流域人-水关系演化的多主体建模(图2-c)。针对上一部分内容中提出的科学假设,首先使用简单的指标对人-水资源利用关系中存在的非线性变化进行拟合,尝试给出猜想的一般表达形式。接下来,建立合理的模型框架并为主体设置基本的运行规则,使用多主体建模来反演上述机制的涌现过程,并对模型进行校验。最后利用建立好的模型进行情景模拟,分别就“气候变化情景”、“制度变化情景”、“适应性治理”[64]等未来情景进行分析。

% 帅王 2023
(1)黄河流域人地系统结构特征及其时空演变
整合社会-生态要素及其相互作用关系构建人地系统网络结构图,利用网络分析方法刻画要素相互作用特征,发展网络指标量化人地系统结构特征并分析上中下游及过去四十年来的时空演变规律,识别其中关键社会-生态要素。
(2)黄河流域社会-生态-水沙协同演变规律与耦合机理
系统研究黄河流域上游自然保护与生态恢复、中游退耕还林还草、下游水沙调控及湿地保育等综合措施对水沙和生态过程的影响,量化人地系统网络结构变化的效应,辨识水沙-社会经济-生态的协同演变规律。
(3)黄河流域人地系统耦合模型
发展包含关键人类活动的生态水文过程模型,建立流域多主体模型,研究流域自然过程与社会经济过程的尺度匹配与转换方法,发展生态水文过程模型与多主体模型的耦合器,建立黄河流域人地系统耦合模型。
(4)情景模拟与人地系统统筹优化
集成气候变化情景和人类活动情景,结合流域和政策情景构建黄河流域未来发展情景集,借助耦合模型模拟不同情景下流域社会-生态-水沙多过程协同演化趋势,探索流域协同治理的机制与路径。

\subsection{关键科学问题}

% 帅王 2023
科学问题:
(1)流域人地系统网络结构特征与互馈过程
(2)人地系统耦合模型发展与协同形成机理
创新性:
(1)通过复杂网络分析方法揭示流域人地系统的结构特征,定量分析结构变化对水沙、生态和经济的影响。
(2)构建链接水文过程模型与主体模型的流域人地系统耦合模型,识别并分析人地匹配网络结构特征,阐释多主体协同过程形成机制与调控路径。

% 开题报告
(1)黄河流域历史人-水关系演变过程的主要阶段。
描述人-水关系及其演变是一个复杂的问题,试图寻找某个指标、建立指标体系或者研制出某种定量数学模型描述黄河人水关系也是困难的。此外, 黄河各河段自然和社会经济状况相差很大,很难对各河段的人-水关系进行统一的描述。但是,为不同主题、不同河段之间建立一个较为统一的分析框架是可能的。利用统一的分析框架对不同主题(如水资源利用、水旱灾防治等)、不同河段(源区、上游、中游、下游)进行综合分析后,可以对黄河流域历史人-水关系演化过程的主要阶段进行粗略划分。
(2)大河流域人-水资源互馈关系变化的核心机制。
人类和生态系统的健康都是以液态水的存在为前提的,而人作为陆生动物严重依赖于地表流域的水资源。因此人-水关系中,为了获取、竞争、保护这种宝贵的自然资源,人类社会与水资源形成了复杂的互馈关系。理解互馈关系是人-水系统模型建立的基础,而模型的建立则是预测未来的保障。在人-水关系发生演变的过程中,人与水资源的互馈关系也发生重大变化, 因此找到这种重大变化的发生机制并利用模型进行解释和模拟,再进一步结合未来情景进行预测。


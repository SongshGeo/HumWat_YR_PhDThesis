

% 人水关系 匹配 comment
幸运的是,人类与水的关系并非完全难以捉摸。事实证明,社会发展与人水关系的变化之间存在着规律,这使得流域系统产生了可以解释和预测的共性问题,如过度和低效开发导致的水资源短缺。这表明,尽管我们有能力预测社会对水的需求和自然界水循环的变化,但仍然缺乏有效的理论和方法来匹配这两者。由于人类社会系统和流域水文系统的结构和功能具有不同的尺度和动态,匹配是指两者之间的良好关系。因此,我们需要深入了解人与水的关系及其动态变化,并找到实现匹配的方法,从而将流域引向可持续发展的轨道。

% 人水关系 匹配 comment
人类的水关系取决于我们如何对待河流,以及我们从流域中得到什么。就像恒河的信徒相信河流能净化他们的灵魂,尼罗河的农民期待着河流的丰收,胡佛站在大坝前骄傲地宣布他已经 "征服 "了科罗拉多河。社会越来越多地为了自己的福祉而改造流域,尽管水经常提供许多其他关键功能,如调节气候、支持生物质生产、物质运输和净化环境。这些功能对于可持续发展可能更加重要,以保持流域社会生态系统的弹性,使其能够适应外部变化或在转型中摆脱危机。然而,在许多情况下,人类改造流域的活动作为干扰超过了系统的临界点,导致维持核心水功能的反馈回路发生变化,引发社会生态系统的稳定转变。为了避免可能导致系统崩溃的稳态转变的级联效应,必须将人水关系的变化视为流域社会生态系统的内部原因,特别是在人类影响的人类世。
% 开题报告
例如水资源管理作为流域特定社会文化、经济水平和政治体制的直接产物,通过调整水量的分配进而作用于生态系统,决定生态系统的健康状况和社会系统的人类福祉。长期以来,流域水资源管理往往通过调控快变量,旨在短期内提高水的利用规模和效益,较少考虑系统状态的长期演变,对生态和社会慢变量的积累变迁更是缺乏反馈机制,限制了维持流域长期可持续的能力。此外,尽管因流域而异的演化轨迹是人-水系统复杂性的体现,但世界主要大河流域也展现出了相似的关键变量与作用路径。例如,在干旱-半干旱区的墨累-达令河流域、科罗拉多河流域以及黄河流域,水资源量的限制都促使水资源分配制度的诞生,该制度又以相似的作用方式影响了河流的水文状态。长期以来这些作用路径被认为是人-水系统演化中的突发政策性影响,因而忽略了其背后的一般性相互作用机制,制约了人-水关系调控的系统性和前瞻性。因此,识别流域人-水关系的长期演化,并理解复杂系统关键变量对演变过程的核心作用机制,对流域的可持续、高质量发展至关重要。

% 帮王老师写的综述:人地系统结构与可持续V2
方创琳(方创琳 2010)总结中国人地关系研究在理论、方法、实践应用等方面取得的进展,认为是以全新视角研究人地关系,深入分析人地系统的基本特征,科学划分人地系统的学科分支,尝试揭示人地关系演进趋势与基本规律,新型人地关系理论不断出现,正在逐步形成理论体系,人地系统研究方法趋于多元化,定性研究方法、数学模拟方法、3S 技术方法和综合集成方法等新方法与新手段不断应用,人地系统理论在实践中广泛应用于解决不同时空尺度的区域发展问题和不同领域的可持续发展问题。

% 帮王老师写的综述:人地系统结构与可持续V2
主要研究内容包括人地关系地域系统的形成过程、结构特点和发展趋向;各子系统相互作用强度、潜力、后效和风险;人与地两大系统间相互作用和物质、能量传递与转换的机理、功能、结构和整体调控途径;地域的人口承载力分析;动态仿真模拟、地域分异规律和类型分析以及优化调控模型。樊杰认为地域功能性、系统结构化、时空变异有序过程、以及人地系统效应的差异性及可调控性,是该理论的精髓,也是综合研究地理格局形成与演变规律的理论基础。


\subsection{流域社会-生态系统研究}

% Handbook 什么是社会-生态系统
社会-生态系统是一个新兴的概念,用相互联系和相互依存的方式来理解人类与自然系统相互交织的本质。这一概念最早诞生于20世纪90年代中期,通过生态经济学和公共池塘资源系统学者间的跨领域学者发展而来(如xxx)。特别是《联系社会与生态系统:韧性构建的管理实践与社会机制》一书将系统科学方法和适应性管理结合起来,重点关注了动态的制度和不同的产权制度。书中14个案例研究分析了复原力和地域特色的生态系统管理。社会生态系统秉承“人类区分社会系统与自然系统的区分是武断的”这一理念,强调人与自然是相互交织的,大自然不仅仅是社会互动的舞台,人类也不仅仅是大自然变化的外驱力。因此,社会-生态系统不仅仅是生态系统与社会系统的简单加和,而是由社会和生态组分之内/之间反馈所塑造的有机整体。

% Handbook 什么是社会-生态系统
这意味着社会-生态系统是一个典型的复杂适应性系统(complex adaptive system)。这类系统由很多互相独立的部分组成,但互相之间以涌现的方式相互作用,系统层面的格局很难由某部分的属性来预测。此外,这类系统层面的格局也会反过来影响某个组分的行为和与其它组分的交互方式,创造出适应变化环境的、随时间演化的反馈过程。这种微观的持续相互作用涌现出的宏观格局,是社会生态系统不等于两者简单加和的原因。此外,这意味着社会生态系统可以适应环境的变化,通过学习和自组织初来响应内部和外部压力。这种动态变化的例子是个体在面对危机时选择交互与合作,围绕共同的愿景和叙事来连接和创建社会网络,导致自组织治理的涌现。这一来,随着已有组织和新的制度之间的桥梁出现,与其它不同层次的治理相联系和互相影响,经验表明社会-生态系统便可能转向新的演化路径。例如瑞典的景观管理,澳大利亚的大堡礁保护,以及南大洋区域资源的全球适应治理体系。

\subsection{流域社会-水文二元循环}

% 开题报告
对自然生态系统而言,从水文地质循环到生态水文过程,从旱区陆地到河湖水体,水是大多数陆地生态系统的主要限制因素,也是物质运移和环境演化的重要介质[3,7,8]。对于人类社会系统而言,从饮食洗涮到田间地头再到江河治理,从民间祭祀到哲学论述再到神话传说,水是一种贯穿了自然与养育、实践与象征的物质[9]。理解人-水关系就是理解人-水复杂系统内部人类社会系统与自然水系统间的相互作用模式[10],重塑人水关系就是通过调整这一互动模式来重塑人与自然的关系。

% Hadjimichael 2020
水系统模型通常旨在表现水的地球物理过程的动力学,以及人类的元素系统用于管理:基础设施、机构和治理(劳克斯,1992)。

\subsection{流域社会-水文互馈机制}

% 开题报告
2.1 黄河流域的人-水关系演变研究
社会系统与水文系统都是动态变化的系统,在不同尺度上有各自的变化规律,人-水关系也会因此随之发生演变。人-水关系的演变是理解人与自然的关系的重要切入点,因为其常常与人-水系统内跨区域、多尺度、且常常与不可逆的系统结构-功能重大变化相联系[26–29]。国际水文科学研究协会(International Association of Hydrological Sciences, IAHS)提出2013—2022年的“科学十年”研究主题为“万物皆流”(Panta Rhei—Everything Flows),该计划强调发展先进的监测和分析技术,广泛寻求水文学与社会经济的跨学科联系,将解析人-水系统的内在变化过程作为水文学发展蓝图的关键[18],但对人-水关系宏观演变的分析视角尚未形成共识。如于宏瑞认为人-水关系是随着社会发展而变化的,因此经历了自然崇拜、趋势利用、盲目开发、持续协调四个阶段[19]。Di Baldassarre 等人则着眼于人类根据应对水文系统挑战的不同模式,区分“适应自然”与“对抗自然”两种演变模式[20,21]。Kandasamy等人则提出人对水的观念演变,始终存在悬于开发和保护之间来回摇摆的“钟摆效应”[22]。因此,Stephanie(2021)总结认为[23]当今人-水系统研究可分为“水的社会性”和“水的技术性”两类。如《大国大河》[24]与《征服自然》[25]两项著作就分别以美国和德国为例,从“水的社会性”与“社会控制水”两个角度系统梳理了两国主要大河流域的人-水关系演变史。前者高度强调“水的社会性”是塑造当代美国社会性质的重要自然因素;后者则指出德国出于征服和控制的考虑,利用技术永远重塑了自然水文景观。
黄河流域同样是世界上人-水关系变化最频繁、最剧烈、影响最为深远的大河流域之一。从水的社会性上看:大禹治水的故事对中华文明的脉络有着深远的影响[30];三门峡与下游运河的运输功能对历代王朝的国祚可谓牵一发而动全身[30];近年来水资源的区域间分配也对经济发展有着制约性影响[31]。从社会控制水的角度上看,历代中央王朝的黄河洪泛治理对水文系统变化影响深远;引黄灌溉则是这个干旱-半干旱区流域的农业经济发展命脉;近年来水库的调水调沙更是彻底重塑了黄河下游的自然生态环境[32]。但是,黄河流域这种频繁的人-水关系变化常常为梳理人-水系统复杂的的反馈循环与协同演化带来了更大的困难[12,33]。如魏夫特考察黄河流域的社会发展历史后提出了著名的“水利社会理论”,认为这样大规模的治水活动是中央封建王朝皇权诞生的重要因素;但更多中国学者认为魏氏倒转了复杂人-水关系中的因果,正是中央皇权的大力发展才使得大规模治水活动变得可能[35,36]。又如治水方略的选择通常被认为是决定黄河河道演变的重要人为因素,同时河道的自然条件又是选择治理方略的重要依据,由两者共同决定的治理结果则会造成路径依赖[34]。因此,为了梳理黄河流域无处不在的、长时间尺度的、影响深远的人-水关系变化,也不乏着眼于总结演变过程的著述。于宏瑞将黄河流域的人-水关系演变的典型事件按区域(上、中、下游)进行了细致梳理[19],但对跨时空的事件关联着墨不多。也有研究倾向于按照黄河流域人-水关系的关键主题梳理演变进程,力求展示出其时间纵深和因果联系,如葛剑雄从黄河与中华文明互动的角度切入[30],王经纬从黄河治理的角度切入[34],各自梳理了长时间尺度黄河的人-水关系演化。但由于切入主题常常对应于人-水系统的单一或部分关键功能,若缺乏统一的分析框架对其进行整合,仍难以满足流域人-水系统模拟和预测对系统性和可解释性的要求。
综上所述,目前对黄河流域人-水关系演变的分析以案例研究居多,常落于列表式梳理或分功能的阐释,缺乏系统性分析框架,因而难以突破时间、空间、主题的限制,无法对人-水关系的演变过程及机制产生系统性认识。

% 开题报告
2.2 大河流域人-水关系的量化分析
随着大河流域的水循环过程受到强烈的人为改造,其承担的社会、生态功能业已逼近安全运行的“地球边界”[3,7,8]。量化人-水关系演变过程的关键路径,是分析流域人-水系统耦合机制的重要出发点。当前研究中人-水关系理论正在不断完善,王浩等提出“自然-社会二元水循环”理论,指出流域水循环的驱动力、循环结构、循环参数在人类活动的影响下发生了二元演变效应[37,38];邓铭江等在此基础上构建“自然—社会—贸易”三元水循环模式,解释西北干旱区内陆河流域的水循环联系、双向反馈机制和协同进化机理[39]。从水的社会性出发,张亚辉指出人-水关系研究分为“实践理性”和“文化理性”[9],前者指的是在生产实践使用水的过程中因控制、竞争、分配、排斥而产生的人-水互动;后者则是指社会中因水观念、水文化而产生的人-水互动。这些理论是指导大河流域人-水关系分析的重要基础。
在这一背景下,旨在理解人-水之间协同演化规律和循环互馈机制的社会-水文学应运而生并不断发展成熟[16,17]。社会-水文学(Socio-hydrology)的诞生[16]在人-水关系变化的现象检测、机理分析、模型预测等方面都获得了长足发展。新学科的发展通常需要经历数据收集与整理、数据观察与推测、理论模型建立、参数率定与模型修正、模型应用与预测五个阶段[40],Troy等人通过文献分析,认为社会水文学研究尚未突破前三个阶段[41]:历史数据与观测数据的收集尚不充分;案例积累不足以开展广泛的对比研究;相关理论模型仍在探索当中。刻画人水关系的模型和计量手段日益丰富,如应用多主体模型揭示了逐水而居的本能可能是人类早期迁徙演化的重要驱动力[42];系统动力学模型被用于解释和预测人类社会面对流域频发的洪水灾害时,水文和社会系统组件(如公众应对、风险文化、经济发展等)的互馈作用[43]。 社会-水文学的快速发展既是对“人-水系统耦合”理论需求的回应,也推动了对人-水关系量化分析与未来变化预测的前沿探索[20,21]。
黄河是受人为活动强烈影响的大河,黄河流域是研究社会-水文学的天然实验田。从1922年魏特夫的“水利社会理论”以来,关于人-水关系演化理论机制的探讨便未曾中断,对关键系统过程进行量化分析的尝试也不乏进展。近年来借助社会-水文学的技术手段,以水沙关系调控为代表的人为驱动下黄河人-水系统变化更是成了社会-水文学研究的重点。人与黄河流域水资源的关系有着更长的互动历史与更具一般性的演化过程,与其它干旱区的大河流域更具相似性和可比性,如科罗拉多河流域和墨累-达令河流域[44,45]。王铭铭就将中国水利社会分成了三种类型[46],分别是以都江堰为代表的丰水型社会,一种是以西北地区为代表的缺水型社会,另一种则是以内河航运及海运为代表的水运型社会。这种类型划分本质是对水在中国不同区域对人类社会最大的价值体现在何处的基本认识,黄河流域水资源是社会发展的天然制约性因素,因而显然属于缺水型社会。当前,在全流域已经完全被人工调控的黄河,日益严重的人-水资源矛盾成为多方利益相关者关注的重点,亟需发展人-水关系变化理论和相应的量化方法。
总体来说,尽管人-水系统协同演化的理论不断发展,相应的量化分析工具也在不断丰富,但多见诸于解耦自然过程与社会过程对人-水系统的影响,对其系统的演变机制鲜有探讨。而且,这种解耦常常建立在人-水关系已经发生变化的经验主义基础之上,因而在黄河流域多集中于人类强烈干预的人-水沙关系等问题的研究,忽略了人-水资源关系的量化分析及理论探讨。

\subsection{人水关系的机制建模}

% 开题报告
2.3 基于复杂系统的人-水关系建模
作为耦合系统,人-水系统既有系统尺度的动力学结构,也有多尺度系统的相互作用[47–49]。如何确定不同子系统的尺度和表达各个子系统间的耦合关系,对更好的理解和预测地球系统至关重要。例如2021年的诺贝尔物理学奖得主Syukuro Manabe和Klaus Hasselmann就发现,从子系统相互作用的角度去考虑大尺度天气变化甚至全球气候这样的混沌系统时,我们不仅能够预测全球大尺度气候系统的宏观行为,甚至还可以评估人类的碳排放怎样对全球气候造成影响[50]。涌现(Emergence)指的就是系统实体会产生其所有组成部分本身没有的属性[51],如大尺度气候系统的宏观行为就是对子系统交互带来的涌现结果[52]。涌现的属性或行为只有当各个部分在一个更广泛的整体中相互作用时才会涌现,是系统组分相互作用的宏观体现。人-水系统作为开放的复杂巨系统,广泛存在的相互作用就是宏观演化属性涌现的关键。
对人-水资源关系来说,流域的利益相关者与水资源的关系就是复杂系统最基本的互动要素。大型流域的人-水系统通常涉及多个利益相关者,利益相关者之间以及利益相关者与流域系统之间存在依赖、竞争、关联等复杂的相互作用,塑造了复杂的、动态的、非线性的系统动态过程。在传统的流域管理模式中,通常忽略了资源环境承载力对于政策的非线性响应,对在人地关系变化过程中流域管理机构、利益相关者的结构、功能与动态也未加以足够重视[53]。理解流域水资源使用时上中下游不同行业利益相关者的冲突、合作,及其相互作用下治理体系的整理结构与功能,逐渐成为流域可持续治理的重要基础[53,54]。例如在澳大利亚的马兰比吉河流域,新兴的水管理体系要求动态考虑粮食生产和水资源开采的关系,且可以通过经济活动的多样化来促进社区利益相关者参与水资源治理的意愿[22,26]。气候变化和人类活动在改变流域水循环的同时也影响着自然和农业生态系统水分利用效率,进而影响着流域资源环境承载力[55]。在以水定地、以水定人、以水定产的过程中,需要从水资源和生态系统承载底线、粮食需求上限以及三者的空间关系入手,发挥人地系统耦合模型的情景预测能力,并考虑流域内不同利益相关者之间形成的治理结构关系[56]——如水资源使用、粮食生产和生态保护之间的协作,实现满足整体性、区域性和动态性需求的大型流域资源环境承载能力的动态预警[57,58]。因此,随着人类改造自然的能力增强,利益相关者之间的协作、竞争、改造与控制等行为如何通过涌现带来人-水关系的宏观演化,成为亟待使用复杂系统建模解释其发生机制。
模型的建立是为了理解、描述和预测自然界,这几个目的相互重叠但绝不相同,一个模型应当在真实性、预测性、普遍性之间达到一个均衡[27]。尽管社会水文学模型的最终目标是预测人水耦合系统[28],但早期的研究主要在探索利用模型解释的普遍性规律[16-18]。Ostrom 借助社会-生态系统(Social-ecological system, SES)框架提出的“公共池塘资源(Common-pool resources)”的自组织(Self-organization)管理中就着重考虑水资源的非排他性和竞争性,提出了基于博弈论的演化模型以揭示此类系统的普遍性机制,并将其与SES的可持续性相联系[59,60]。这个获得诺贝尔经济学奖的成果随后成为包括人-水复杂系统在内的。自组织(Self-organization)是指一种起源于初始无序系统的部分元素之间的局部相互作用、所产生出某种形式的整体秩序的过程。这与复杂系统建模中自下而上的基于主体的建模(Agent-based model) 思想类似,因此Castilla-Rho等人[61–63]利用多主体建模的复杂系统模拟方法将Ostrom发现的机制应用于地下水流域的管理模型中,发现可以解释地下水资源治理模式的涌现。但对于大河流域来说,由于人能直接观察到流动的水资源变化,且存在明显的时空分异性,尚缺乏较好的复杂系统建模以探索其中人-水关系的核心演化机制。
总的来说,基于复杂系统的建模已成为研究人-水系统的重要技术手段,能够基于水的资源属性对利益相关者的人-水互动进行理论机制上的探索。这种自下而上的建模思路与SES的自组织管理机制一脉相承,但目前比较成熟的模型主要关注地下水流域和,对河流的水资源治理,尤其是水资源稀缺的大河尚缺乏泛用性较强的机制模型。

\subsection{黄河的人-水关系演变研究}

% 江恩慧 2022本子
上世纪80年代以来,随着社会经济的快速发展,人类活动成为影响流域地表过程的重要因素。在水文水资源研究方面,面对水沙自然资源供需与全球生态环境危机、社会经济可持续发展之间的强烈冲突,仅从河流/流域自然层面研究已经不能满足现实需求,厘清人与自然耦合系统运行规律和演化机制,受到国际学界高度重视(Gleeson等,2020)。国际地圈生物圈计划(IGBP)框架强调植被-生态过程与水文系统的耦合。国际水文科学协会(IAHS)提出了“万物皆流”(Panta Rhei-Everything Flows)研究计划,强调流域水-生态-经济系统的研究思路。Sivapalan等(2012)提出了社会水文学(Socio-hydrology),致力于探索水资源视角下水文系统与人类系统的双向互馈机制,旨在更加深刻地表达人-水耦合系统协同演化过程,以实现水资源的可持续利用和管理。王浩等(2006,2010)构建了自然-社会二元水循环理论体系,充分考虑了人类活动对水文过程和水资源循环过程的影响。李少华等(2007)提出了水资源复杂巨系统的概念,将水资源、人口、社会、经济、生态、环境等均视为水资源系统的子系统。在泥沙研究方面,随着大规模的人类活动改变了流域下垫面条件,进而影响流域侵蚀产沙过程与进入河流的水沙关系,使得人为因素对流域水沙关系的影响成为当前研究重要内容之一(傅伯杰等,2010)。

% 江恩慧 2022本子
全球气候变化和人类活动深刻影响着水文泥沙过程、生态环境演变和社会经济发展进程(Best,2019)。世界上众多大型河流的水沙通量发生了显著变化(Li等,2020),尤以黄河最为突出。近年来,黄河的径流、泥沙分别减少40\%和80\%以上,远超世界平均水平(刘晓燕,2020;胡春宏和张晓明,2018)。对水沙变化趋势性、周期性的研究基本取得了共识,认为气候变化和人类活动共同影响使得水沙变化更加复杂,人类活动是水沙趋少的主要驱动原因(Wang和Sun,2020;Wang等,2016,2020;Ma等,2020;Zhang等,2021);沙量减幅大于水量减幅,输沙量集中度和不均匀程度大于径流量(Tian等,2019)。

% 江恩慧 2022本子
随着黄河流域生态保护和高质量发展重大国家战略的深入实施,黄河流域社会经济从“高速发展”进入了“高质量发展”阶段,需从过去对水资源的过度开发和粗放式利用转向产业结构优化调整下的节约集约利用,对流域水资源配置提出了更高要求。然而,黄河水资源开发利用程度高达80\%,甚至在西北部分地区达到了100\%,已触及水资源供给的“天花板”(王建华等,2020);沿黄城市发展高度依赖黄河水资源配置,且已有90\%以上城市处于水资源超载状态(李原园等,2021);未来水资源供需矛盾有可能进一步加剧,据王煜等(2019)预测,河南黄河供水区2025年缺水14亿m3、2035年缺水17亿m3。

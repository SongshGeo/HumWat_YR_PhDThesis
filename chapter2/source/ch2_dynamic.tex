\subsection{社会-生态系统稳态转换理论}

% 江恩慧 2022本子
长期的研究与探索,流域系统概念逐渐明晰。程国栋和李新(2015)在“黑河流域生态-水文过程集成研究”中提出了流域科学的概念,将流域视为地球系统的缩微,考虑水文和生态系统的自组织性如何影响流域系统的功能,以及人的因素如何被集成到流域水文学和流域生态学中。傅伯杰(2017)指出亟需聚焦人地系统耦合机理与调控途径,揭示黄河流域人水关系演变及社会-水文-生态系统动态。

就像两个人之间的关系,复杂的、难以,但关系的阶段确是可以定性的,一个人难以用函数估量一段关系的深浅,但关系的阶段确实存在,它发生在一个之后。


\begin{figure}[htb] % use float package if you want it here
    \centering
    \includegraphics{hello}
    \caption[社会-生态系统状态循环]{社会-生态系统状态循环与转型治理、协作治理的关系
    Fig.4  The relationship between social-ecological system adaptive cycle and transition / collaborative governance
    ①SES状态循环包括:开发阶段r,保护阶段K,释放阶段Ω,更新阶段α;图中展示旧的SES因转型治理而进入新的状态循环,并因协作治理的实现而延长开发保护阶段的过程;②转型治理可受其它尺度SES的影响[64,66];③协作治理的主要实现过程包括F-F-D(Face to Face Dialogue, 当面沟通),T-B(Trust-Building, 建立信任),C-P(Commitment to Process, 过程承诺),S-U(Shared Understanding, 信息对称),I-O(Intermediate Outcomes, 阶段成果)五步骤的循环[65,68,58]。}
    \label{fig:xfig0}
\end{figure}

% 地理学报
社会-生态系统能稳定于多个不同状态,并自发历经开发(Growth)、保护(Senescence)、释放(Collapse)、更新(Renewal)四阶段的适应性循环[63],后有学者根据该循环将社会-生态系统框架下的适应性治理体系总结为实现(Emergence)、制度化(Institution)、更新(Renewal)三阶段(Chaffin and Gunderson, 2016)。由于建立适应性治理体系是希望社会-生态系统状态维持在社会所需范围内,因此会产生“主动改变不良的社会-生态系统状态”与“调节并维持良好的社会-生态系统状态”两种不同的需要,这也是适应性治理的两种相关理论:转型治理(Transformative Governance)与协作治理(Collaborative Governance)之间产生构建思路差异的主要原因(图4)。转型治理关注社会-生态系统社会-生态系统的释放和更新阶段,强调适应性治理的实现,以及主动促使社会-生态系统完成状态的更新[64];协作治理则强调适应性治理制度化过程,旨在通过利益相关者间自组织的协作模式来实现社会-生态系统的开发与保护[65]。由于上述出发点的差异,转型治理和协作治理具有不同的定义与实现框架。转型治理是通过触发制度结构与过程的转变,积极将多尺度社会-生态系统转向所期待的状态[64]。实现转型治理需要知识、规则、价值共同服务于决策,以确定是否需要转型并逐步进行转型决策[66]。协作治理则是一个或多个公共机构直接参与利益相关者的集体决策过程,以正式协商一致为导向,旨在制定或执行公共政策或管理公共资产[65,67]。实现协作治理需要通过自下而上的手段,强调协作网络与协商制度的设计,从而建立具适应性的协作机制[68-70]。案例研究同时表明,为通过制度转变来更新社会-生态系统状态,配置转型治理需明确区分制度的步骤[71];为通过协作来维系良性社会-生态系统状态,配置协作治理需建立利益相关者间的协商体系[62]。两种治理理论与适应性治理具有共同的理论基础(即通过调节人类行为来改变社会-生态系统状态)与长远目标(即建立社会-生态系统自组织管理体系),针对不同社会-生态系统状态及目标选择理论框架,有助于提升面对变化环境的系统适应性。

\subsection{流域人水关系变化的定义}
% 地理学报
此处借助“三明治”概念框图梳理其发展模式(图5)。面a内是通过相互作用形成的社会网络,节点越大表明社会资本越大;面c为社会-生态系统状态集合面,其中当前状态正受到FSES的盆地吸引力。制定规则将改变社会网络结构及其对社会-生态系统的作用力,并通过路径α与路径β实现人与环境的动态适应。假设平面b为理论管理决策,韧性管理理论表明,通过设立规则(如图5:B1, B2, B3)作用于社会-生态系统(如图5:Fc1, Fc2, Fc3)从而调节系统状态是可能的[46]。但传统的管理规则相对独立,看似凌驾于社会与自然之上,实则常常失效或偏离[31]。自组织的管理思路则指出,社会网络(面a)的相互作用能够调节规则(如图5:建立Fb1, Fb2, Fb3)使其更加行之有效,且能随社会网络的变化而自适应调节。治理的理念则指出面a上的社会网络能够形成协作的组织结构,使各利益相关者都能影响规则但又不能独自左右规则[24],有利于实现自组织与适应性治理,其中根据治理目标的不同,协作治理强调维持系统状态(如图5:SES),而转型治理强调将社会-生态系统调整至更符合需求的状态(如图5:SES’)。综上所述,适应性治理理论提供了实现社会-生态系统可持续性的综合途径:从社会驱动力开始(面a),自组织并调节的管理规则(面b),进而调节社会-生态系统状态(面c);而通过适应性治理提升适应能力,旨在提升动态系统(曲面a、b通过α-β自适应)的韧性。因此,在环境变化背景下,适应性治理具有应对社会-生态系统复杂性和不确定性的潜力。

% 人水匹配 comment mydoc.docx
作为实现匹配的一个重要途径,制度分析认为,权利、规则和决策等制度可以引起或解决人与环境互动中的问题。因此,建立匹配的制度可以引导系统功能向理想的结果发展。研究表明,制度匹配在很多方面有利于人类水系统的可持续性。例如,设立流域管理机构可以有效避免水事纠纷,建立水权转换制度有利于资源的有效配置,加强监管可以遏制河流污染。因此,许多大流域的综合治理是建立一套制度的尝试,它以实现人水匹配的一系列制度为核心,包括与之密切相关的文化和技术要素,从而保证流域的可持续发展。这样一个体系的建立庞大而重要,需要自然科学和社会科学的交叉,解耦和理解人水系统中复杂的反馈回路,从而通过制度分析实现人水匹配(图1,PATH 1和路径2)。
制度分析往往是在特定背景下进行的,因此,人水关系的动态变化使得宏大的人水系统匹配更加复杂。首先,动态变化是人类社会系统和自然水文系统的共同特征,但只有当关键变量达到临界点时,系统功能的稳态转变才会发生。需要匹配的关键系统功能往往是由流域可持续性的需要决定的,这为理解复杂系统动态的观点提供了价值判断(图1,路径3)。另一方面,当一些可能导致系统稳态转变的变化发生时,也需要重新审视如何在新的人水关系下实现匹配。因此,这些超过阈值的变化可能成为成功过渡到人水匹配的关键驱动力(图1,PATH 4)。然而,实现上述动态匹配还需要对人水系统有更高层次的理解,因为这些动态变化为系统解耦提供了新的科学难题,如变化的弹性、变化之间的因果关系、稳态转化及其级联效应等(图1,PATH 5)。只有通过深度解耦和对反馈循环的理解,才能对人类与水系统的动态进行预测(图1,PATH 6),促进机构分析和过渡匹配(图1,路径1和路径3)。一般来说,人-水系统的反馈循环和动态变化是人-水匹配的基础。由于三者是相互联系的,面对不断变化的流域系统,人水匹配需要动态实现。

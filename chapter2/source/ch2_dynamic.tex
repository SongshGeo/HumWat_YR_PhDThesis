根据本研究对流域人水关系的定义,以及对稳态转换框架的认识,识别流域系统的稳态转换是分析人水关系变化过程的基础,分析其驱动力则是明晰变化机制的关键。
本研究对流域人水系统状态的定义植根于社会-生态系统的“稳态”概念,识别稳态需要从“驱动-现象-效应”三要素中多个要素入手,分析复合特征的变化。
换言之,本研究中识别流域稳态时,须使用特征变化的识别方法对稳态转换的驱动力和现象/效应都进行分析或检验,确定流域系统内既存在明显的突变现象,又存在相应的稳态转换驱动力。
因此本章以黄河流域符合上述定义的稳态变化的实证研究为典型案例,总结了稳态转换研究中三种常见的分析路径如图\ref{ch2:fig:identifying}所示。

最常见的稳态转换识别,是聚焦于稳态转换过程中所表征出的现象,寻找合适的指标与标准识别系统互馈变量的趋势突变或其它相关解释变量的不连续性,对直接相关的变量进行检测。
当分析缺乏完整的现象变量数据资料而驱动力明确的稳态转换现象时,从驱动因素的阈值效应入手建立指标和标准,也可有效分析稳态转换过程。

结合本章先前对“人水关系”进行的严格定义,本研究识别人水关系变化过程及其机制的总体思路是:识别主导流域“自然-社会”二元水循环稳态的关联要素及其稳态变化现象,结合稳态转换的驱动机制(气候变化或人类活动?自上而下或自下而上?)分析利益相关者的与水圈要素过程的联系状态如何变换、因何变化。

\begin{figure}[!htb] % use float package if you want it here
    \includegraphics[width=\textwidth]{img/ch2/ch2_identifying.png}
    \caption[三种分析稳态转换的常见路径的实证研究案例]{三种分析稳态转换的常见路径的实证研究案例。
    蓝色背景椭圆框为可获取数据的变量,白色背景椭圆形框为不可获取数据的变量;双箭头线连接了两个变量,箭头线上蓝色背景方框表示产出效应的具体意义;粉色背景矩形框为判断系统多稳态和稳态转换过程的依据,即通过关键现象(a和b)、驱动因素的阈值效应(c)、系统效应的改变(d)识别并分析稳态转换;实线框为稳态转换的社会-生态系统边界。
    图中展示了四个典型研究:
    a)通过关键现象识别稳态转换后探寻驱动因素(改自文献[49]);
    b)通过关键现象识别稳态转换后分析系统产出效应 (改自文献[83],备注:该案例中驱动因素仅仅作用于NDVI而非全部现象变量);
    c)通过驱动因素的阈值识别稳态转换过程,并结合外溢效应验证(改自文献[48])。
    d)通过产出关系改变识别稳态转换过程,进一步分析驱动和外溢效应变化 (改自文献[65])。}\label{ch2:fig:identifying}
\end{figure}

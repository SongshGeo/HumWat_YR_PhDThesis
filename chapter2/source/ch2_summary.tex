本章主要对“流域人-水关系”这个人-地关系研究中核心的、常用的、却看似无所不包的概念给出了便于研究的定义。
结合人地系统耦合研究中“利益相关者”的概念;本章研究首先为“人-水关系实体”给出如下定义:
在给定的时空尺度下,利益相关者(个体、组织、或集体)通过影响自身利益的感知或决策而与地球表层系统中水圈的要素过程相联系的状态。
尽管该定义中的“人-水关系”是可以被客观描述的,但由于该概念存在尺度敏感性,在不同的时间、空间精度下应用该概念将导致截然不同的研究方法与难以整合的研究结论。

因此,结合流域这一完整地理单元的“人-水关系”研究需要,本研究指出流域系统的“人-水关系”研究应把握:
(1)“人-水关系是指导流域研究的宏观整体性概念”
与(2)“关系的变化是流域系统反馈循环模式的改变”
两个关键思想。
借由地球系统科学中的“适应性循环”与“涌现”等核心概念;本章为“流域系统人-水关系”给出如下便于研究的定义:
在流域尺度下,主导人水复杂系统“自然-社会二元循环结构”的利益相关者与地球表层系统中水圈要素过程的联系状态。
% 这种模式可以来源于系统内拥有决定权的利益相关者以自上而下的方式对自然水文过程采取的干预和控制;也可以来源于大多数利益相关者以自下而上地方式对
最后,结合社会-生态系统“稳态转换”的研究前沿,本章将“流域系统人-水关系的变化”定义为引起流域系统“社会-水文循环结构”发生大规模重组的认知与行动模式改变。
本章的定义为后文分析人-水关系的模式、变化发生的时间、发生变化的原因等均提供了关键的理论指导。
% 基于本章研究中对“人-水关系”提出的定义,并

% 本章研究为人-水关系定义
流域研究绝无可能穷尽系统的全部人水关系,若没有对基本定义、研究范围、分析框架等加以明确,研究便易陷入笼统或琐碎,缺乏对流域系统整体的指导意义。
基于“人-水关系是指导流域研究的宏观整体性概念”这一原则,本章对“流域系统人-水关系”这一存在尺度敏感性的概念给出如下便于研究的定义:
在流域尺度下,主导人水复杂系统“自然-社会二元循环结构”的利益相关者与地球表层系统中水圈要素过程的联系状态。

在识别流域系统人水关系变化方面,本章基于社会-生态系统稳态转换的概念,强调“人水关系的变化是流域系统反馈循环模式的重组”,因此识别稳态转换是分析人水关系变化的基础,明晰其驱动力则是解析变化机制的关键。
人水关系变化的驱动机制除了源于外部干扰或参数变化外,还有来自系统内部的驱动力,它既可能在自上而下的变革中产生,也可能从自下而上的过程中涌现。
本章的概念定义与分析框架为后文研究人-水关系的特征、变化过程、驱动因素等均提供了关键的理论指导。


% 开题报告
\begin{figure}[htb] % use float package if you want it here
    \centering
    \includegraphics{hello}
    \caption[主要研究内容与相关概念]{图2. 主要研究内容与相关概念。本研究主要关注的人-水系统(Human-water system)与概念范围更广的社会-生态系统(Social-ecological system, SES)都是典型的复杂系统(Complex system),因此基于复杂系统的建模也常常以人-水系统为研究对象。(a)本研究首先以黄河为研究案例,梳理人-水系统演变过程;(b)在其基础上重点分析水的资源属性下发生变化的人-水关系。(c)最后在SES分析框架下,利用基于复杂系统建模的技术手段,对资源属性引导下的人-水关系演变过程进行机制模拟。}\label{ch2:fig:definitions}
\end{figure}

% 开题报告
利益相关者是一个系统中或一个事件中存在利益关联的个人、团体、组织,可以来自系统内部或外部,并受到系统变化的直接或间接影响[66]。在人-水系统中,利益相关者可能因自身利益驱动成为人水关系变化的触发者,也可能成为人-水关系变化的受害者,这种变化还可能同时伴随着旧的利益相关者脱离系统或新的利益相关者产生。由于利益相关者在人-水关系的变化中常常会发生自身利益的增加或损失,从该视角切入能让我们对人-水关系变化产生新的认识。

% 开题报告
社会交换理论强调关系中存在报酬和代价,而人际交往的关系就可以通过这种报酬和代价的交换来完成[67]。在人际关系互动中,参与互动的两方在互动中投入代价并获得报酬,如果两者都不期望在关系中付出代价或获得报酬,那么这种关系的密切程度自然不及互相付出并获得报酬的关系。通过将人-水关系视为一种人与水的交往过程,我们可以参考基于社会交换理论的“依存关系”,根据利益相关者付出报酬和获得代价的不同路径来分析人-水关系。由于水文系统总是按照自己的意愿(客观规律)行事,所以人-水关系的互动模式主要取决于人如何付出代价并获得报酬:(1)当利益相关者完全不为水文系统付出代价以期获得报酬时,相互作用的双方均按照自己的意愿行事,人与水文系统仍然存在交互,但这种交互并不基于人类主动去“维护关系”,所以被称为“虚假相依”(图3b)。(2)当利益相关者不仅愿意为自身意图付出代价,还愿意为干预水文系统付出代价,以期两者共同给予自身更多报酬时,这种关系可被视为“非对称相依”(图3c )。(3)当利益相关者仅靠对水文系统的关系中付出代价才能获得报酬时,这种关系可以被视为“反应性相依”(图3d)。

% 开题报告
\begin{figure}[htb] % use float package if you want it here
    \centering
    \includegraphics{hello}
    \caption[基于利益相关者的人-水关系分析框架]{图4.1. 基于利益相关者的人-水关系分析框架。(a.)基于社会交换理论的人际关系互动,人与人之间通过付出代价与获得报酬来维持关系。(b.)“虚假相依”,利益相关者不期为人-水关系付出代价,仅按照自己意图行事以获得报酬。(c.)非对称相依,利益相关者愿意在维持自身意图的情况下为人-水关系付出代价,以期获得更多的报酬。(d.)“反应性相依”,利益相关者付出代价并仅依赖于人-水关系获得报酬。
    该框架不同于前人从不同人-水关系主题或不同区域的角度出发所作的分类,这里基于利益相关者提出的人-水关系分析框架仅考虑了人(利益相关者)在人-水关系中的行事意图,因此针对不同类型、不同结果的人-水关系都具有分析参考意义。}
    \label{ch2:fig:reaction}
\end{figure}

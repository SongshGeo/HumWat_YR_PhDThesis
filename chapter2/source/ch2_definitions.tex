\subsection{从人地关系到人水关系}

人地关系是人类活动与地球表层系统的相互影响与反馈作用\cite{wu1991,li2016d},但人水关系却不能被简单定义为“人类活动与水圈的相互影响与反馈作用”。
地球表层系统作为开放复杂巨系统,四大圈层之内或之间有着复杂的能量交换和物质流动,人类无法脱离其它任何圈层,人类活动对其它圈层的影响也会随着地球表层过程而深远影响水圈。
如从这个角度出发,人地关系与人水关系将完全等价,且都指向人类与地球表层系统所共同构成的人-地复合系统。
但在人地关系研究中,无论是人类与地球表层要素间的单向影响还是反馈循环,研究者约定俗成会将着眼点放在某个“影响/反馈作用实体”上,给出$Y = f(X_1, X_2, \dots)$的显式表达,而绝无可能穷尽此复合系统的全部作用关系。
因此出于研究需要,本章为“人水关系”给出如下广义定义:
% TODO 定义
水圈要素过程作为自变量或因变量时,人类活动与地球表层系统的相互影响与反馈作用。
根据此定义,若人类活动不直接改变水圈要素过程、且其间接导致的水圈变化也不会被人类活动所考虑时,便不再是人水关系的范畴。

根据上述广义定义,水也是塑造人类世界观及行为模式的重要元素,相关理论或经验研究(如社会科学中“水的社会性”研究)也属于广义的人水关系,绝不是从某一时刻才产生的事物,因此本研究题目强调其“演变过程”。
水圈或水文系统被人类活动改造则是人水关系的一种表现,相关理论或经验研究(如流域“自然-社会”二元水循环理论\cite{wang2006, wang2016})所强调的多是近现代人类活动影响水圈要素过程后的一种人水关系表现形式。
因本研究侧重“流域可持续性”层面,不再考虑人类对自然水过程影响微乎其微的史前时期,仅将时间尺度聚焦在人类有能力改造水圈要素过程,因而存在“自然-社会”二元水循环的时期。

% 开题报告
\begin{figure}[htb] % use float package if you want it here
    \centering
    \includegraphics[width=\textwidth]{img/ch2/ch2_definitions.png}
    \caption[主要研究内容与相关概念]{主要研究内容与相关概念。本研究主要关注的人-水系统(Human-water system)与概念范围更广的社会-生态系统(Social-ecological system, SES)都是典型的复杂系统(Complex system),因此基于复杂系统的建模也常常以人-水系统为研究对象。(a)本研究首先以黄河为研究案例,梳理人-水系统演变过程;(b)在其基础上重点分析水的资源属性下发生变化的人-水关系。(c)最后在SES分析框架下,利用基于复杂系统建模的技术手段,对资源属性引导下的人-水关系演变过程进行机制模拟。}\label{ch2:fig:definitions}
\end{figure}

\subsection{人水关系实体的定义}

关系的定义是“连接两个或多个概念、对象的方式或状态”,人水关系的研究形式也与“关系”连接的实体有关,已有研究中既存在定性描述也存在定量表达;既存在单向影响也存在双向反馈;更存在许多概念层面的探讨。
如前文所述,本章研究已将“人水关系”术语中的水定义为“地球表层系统中水圈的要素过程”,而该术语中的“人”的概念还须进一步明晰。
“人”的概念通常指代任意时空尺度的人类活动,在既往研究中根据研究目的不同,可以是具体的个人、组织、社会,也可以是经这些实体生产并存留的其它事物(如水库大坝、污染物、法律制度等)。
本研究将人水关系中的“人”定义为利益相关者,即一个系统中或一个事件中存在利益关联的个人、团体、组织,可以来自系统内部或外部,并受到系统变化的直接或间接影响~\cite{bonnafous-boucher2016}。
简而言之,水库大坝、污染物、法律制度等人类活动生产的事物本身并不属于本研究关注的“人水关系”,但如果它们影响了水圈要素过程,则应关注利益相关者生产它们的决策过程。
因此,利益相关者可能因自身利益驱动成为人水关系变化的触发者,也可能受到人-水关系变化的影响而增加或损失利益。

“人水关系”属于人地关系,拥有人地关系的所有特征,包括多重性、异时相关性、异地相关性、人的主动性、动态性、和多重决定性\cite{fang2004}。
这意味着,即便对具体的某个利益相关而言,他可能受到上千年前的水过程影响(如地下水),也可能影响未来的水过程(如修建取水设施改变水的流向),用显式表达穷尽其与水圈要素过程的全部关系即便不是不可能,也是极度困难的。
因此,本研究主要关注“人”与“水”通过决策过程互相连接的状态实体,识别其变化并解释其机制,而非在具体的时空尺度下为某两个或多个相关变量给出精确的函数表达。
利益相关者的决策过程通常与其成本-收益有关:例如利益相关者认知到潜在损失时,便会组织修堤防洪等治河工程;感知到潜在收益后,便愿意为获取水资源而修建水库和渠道;这种人与水之间的连接模式便是本研究定义并关注的“人水关系状态实体”。
% todo 定义
综上所述,研究为“人-水关系的状态实体”给出如下严格定义:在给定的时空尺度下,利益相关者(个体、组织、或集体)通过影响自身利益的感知或决策而与地球表层系统中水圈的要素过程相联系的状态。

% 开题报告
\begin{figure}[htb] % use float package if you want it here
    \centering
    \includegraphics{hello}
    \caption[基于利益相关者的人-水关系分析框架]{图4.1. 基于利益相关者的人-水关系分析框架。(a.)基于社会交换理论的人际关系互动,人与人之间通过付出代价与获得报酬来维持关系。(b.)“虚假相依”,利益相关者不期为人-水关系付出代价,仅按照自己意图行事以获得报酬。(c.)非对称相依,利益相关者愿意在维持自身意图的情况下为人-水关系付出代价,以期获得更多的报酬。(d.)“反应性相依”,利益相关者付出代价并仅依赖于人-水关系获得报酬。
    该框架不同于前人从不同人-水关系主题或不同区域的角度出发所作的分类,这里基于利益相关者提出的人-水关系分析框架仅考虑了人(利益相关者)在人-水关系中的行事意图,因此针对不同类型、不同结果的人-水关系都具有分析参考意义。}\label{ch2:fig:reaction}
\end{figure}

% % 开题报告
% 社会交换理论强调关系中存在报酬和代价,而人际交往的关系就可以通过这种报酬和代价的交换来完成[67]。在人际关系互动中,参与互动的两方在互动中投入代价并获得报酬,如果两者都不期望在关系中付出代价或获得报酬,那么这种关系的密切程度自然不及互相付出并获得报酬的关系。
% 通过将人-水关系视为一种人与水的交往过程,我们可以参考基于社会交换理论的“依存关系”,根据利益相关者付出报酬和获得代价的不同路径来分析人-水关系。
% 由于水文系统总是按照自己的意愿(客观规律)行事,所以人-水关系的互动模式主要取决于人如何付出代价并获得报酬:(1)当利益相关者完全不为水文系统付出代价以期获得报酬时,相互作用的双方均按照自己的意愿行事,人与水文系统仍然存在交互,但这种交互并不基于人类主动去“维护关系”,所以被称为“虚假相依”(图3b)。(2)当利益相关者不仅愿意为自身意图付出代价,还愿意为干预水文系统付出代价,以期两者共同给予自身更多报酬时,这种关系可被视为“非对称相依”(图3c )。(3)当利益相关者仅靠对水文系统的关系中付出代价才能获得报酬时,这种关系可以被视为“反应性相依”(图3d)。


\subsection{流域人水关系的定义}

人地关系地域系统是人类活动与地理环境相互作用、相互耦合形成的具有地域特色的系统\cite{tan2021},流域是水文过程完整的地理单元,其中所有利益相关者通过地表水过程而产生联通,因此在人水关系研究中有着得天独厚的优越性。
也正因如此,从个人到企业,从行政区到流域管理部门,所有的利益相关者都与流域系统的某些水文要素或过程存在某种程度的连接。
尽管已基于利益相关者的概念对“人水关系实体”给出了严格定义,但在流域这一特定尺度下,哪些利益相关者需要在研究中考虑仍然是需要做出取舍的。
本研究认为流域系统的“人-水关系”研究应把握“人-水关系是指导流域研究的宏观整体性概念”这一原则,识别并针对主导流域“自然-社会”二元水循环模式的利益相关者,将其在当前阶段的“人水关系状态实体”作为流域系统的人水关系。

流域系统是典型的社会-生态系统,复杂自适应性是其关键特征(图\ref{ch2:fig:complexity}),因此在流域尺度上,人水关系绝非所有利益相关者人水关系的简单加和(机械系统观),而是系统内外所有人水关系涌现产生的动态平衡(复杂适应性系统观)。
涌现是复杂系统的核心概念之一,指不存在于系统某单独组分的整体现象或集体行为,具有尺度依赖性,而在忽略对涌现没有影响的冗余局部,从而找到流域系统主导因素的过程也被称为“粗粒化”。
因此,本研究基于上述理论与概念,对“流域系统人-水关系”给出如下便于研究的定义:
% todo:定义
在流域尺度下,主导人水复杂系统“自然-社会二元循环结构”的利益相关者与地球表层系统中水圈要素过程的联系状态。
根据流域复杂系统在演变不同阶段类型不同,流域特征可以由中央管理者自上而下控制(控制系统),或由尺度更小的利益相关者自下而上涌现(自组织系统),因此主导流域的利益相关者数量并不确定。
因此根据该定义,本研究对人水关系演变过程及机制的分析,重点便是这种自上而下的驱动力和自下而上的驱动力如何改变利益相关者与水过程的关系,决定流域尺度的系统联系状态。

\begin{figure}[htb] % use float package if you want it here
    \includegraphics[width=\textwidth]{hello}
    \caption[流域系统作为社会-生态系统的概念图式]{绘图测试}\label{ch2:fig:complexity}
\end{figure}

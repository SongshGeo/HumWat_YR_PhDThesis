
人地关系是地理学的传统研究议题,是地理学重要的学科基石,地理学的研究对象便是人地关系地域系统\cite{zhang2022c}。
水圈既与人类活动紧密联系,也耦合着地球表层系统其它各圈层的要素与过程,人类所有对水过程的感知或对水圈的直接/间接影响都属于广义的人水关系\cite{falkenmark2021}。
但流域是复杂社会-生态系统,并非所有的人水关系变化都会在流域尺度对生态可持续性与社会发展产生影响,若不对“人水关系”相关术语概念的基本定义、研究范围、分析框架等加以明确,人水关系研究便易过于笼统或过于琐碎,缺乏对流域系统的科学指导意义。

流域是研究人水关系的完整地域系统,同时也是典型的社会-生态系统\cite{huggins2022},其人水关系的演变在不同时空尺度上呈现出尺度依赖性、非线性、弹性等重要特征\cite{reyers2018},需从复杂适应系统理论的视角加以理解\cite{wang2019d}。
本章研究从“人水关系”的概念基础切入,发展流域系统人水关系演变的概念框架,明晰流域人水关系的概念内涵,厘定怎样的变化可以被识别为发生了“流域人水关系演变”,以及有哪些潜在的驱动机制会导致这类变化。
为便于后文其它章节研究,本章试图解决的关键理论问题有两个:
1)对于特定的流域系统,如何在难以穷尽的关系实体之上定义系统层面的人-水关系;
2)对于特定的时间尺度,如何识别和分析流域系统层面的人-水关系变化及其机制;
简单来说,本章的铺垫旨在让“人水关系”的概念从含混到具体;使黄河流域系统“人水关系”的分析从局部到系统、从静态到动态。

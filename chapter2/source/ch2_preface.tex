
人地关系是地理学重要的传统研究议题~\cite{zhang2022c},是地理学的学科基石。
但地球表层的几乎一切过程都已受到了人类活动的影响或干预,因此人地关系研究也可以说是无所不包的。
水是与人类活动联系最紧密的地球表层要素之一,从乡村到城市,从干旱到洪涝,从个人到集体,一切有意或无意影响、改造了天然水循环的人类活动都与人水关系紧密相连~\cite{falkenmark2021}。
但是,并非所有的人水关系变化都会对流域尺度的可持续、高质量发展产生影响,若没有对基本定义、研究范围、分析框架等加以明确,人水关系演变研究便极易陷入笼统或琐碎,缺乏对流域系统整体的指导意义。

流域研究是完整地理单元构造的地域系统,人水关系是该人地关系地域系统内最重要的人与环境相互作用,其演变过程与机制在不同时空尺度上呈现出迥然不同的特征~\cite{wang2019d}。
同时,作为典型的社会-生态系统,流域是复杂适应系统~\cite{huggins2022},有着非线性、动态性等特点,其整体特征应从系统论加以理解~\cite{reyers2018}。
本章研究从基础概念和基本问题切入,逐层深入人地关系地域系统与流域人水关系研究的理论问题,总结提出“人水关系的分析框架”,挈领哪些黄河流域的人-水关系将是本论文研究的分析重点,以及哪些系统动态是重塑黄河人水关系需要关注的重要信号。
本章试图解决的关键理论问题有两个:
1)对于特定的地域人地系统(本研究中为黄河流域),如何在关系实体之上定义流域系统层面的人-水关系;
2)对于特定的时间尺度,如何识别和分析流域系统层面的人-水关系变化。
简单来说,如何使黄河流域人-水关系的分析从微观到宏观,从静态到动态。


人地关系是地理学的重要研究领域,是该学科的核心内容\cite{zhang2022}。
几乎所有的地球表层过程都受到了人类活动的影响,因此研究人地关系几乎是无所不包的。
水是与人类活动最密切相关的地球表层要素之一,不管是在农村还是城市,不管是干旱还是洪涝,不管是个人还是集体,所有有意或无意改变天然水循环的人类活动都与人-水关系有关~\cite{falkenmark2021}。
并不是所有的人-水关系变化都会对流域的可持续、高质量发展产生影响,要想对人-水关系演变研究有明确的指导意义,就必须对基本定义、研究范围、分析框架等进行明确定义,避免研究陷入笼统或琐碎。

流域研究是以完整地理单元构造的地域系统为研究对象,人-水关系是该地域系统中最重要的人与环境相互作用关系,其随时间演变呈现出不同的特征~\cite{wang2019c}。
流域还是一个典型的社会-生态系统,具有非线性、动态等特征,必须从系统论的角度加以理解~\cite{reyers2018,huggins2022}。
本章研究从基础概念和基本问题切入,逐层深入人地关系地域系统与流域人-水关系研究的理论问题,总结提出“人-水关系的分析框架”,挈领哪些黄河流域的人-水关系将是本论文研究的分析重点,以及哪些系统动态是重塑黄河人-水关系需要关注的重要信号。
本章从基础概念和基本问题出发,深入探讨流域人-水关系研究的理论基础,总结出“人-水关系的分析框架”,明确本研究分析人-水关系的概念边界与侧重点,试图解决的关键理论问题主要有两个:
1)如何在流域人-水系统层面定义人-水关系;
2)如何识别和分析黄河流域人-水关系的变化。
简而言之,如何使黄河流域人-水关系的研究从静态到动态,从微观到宏观。

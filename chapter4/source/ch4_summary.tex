流域水治理是对“有多少水、谁用水、怎么使用”的综合决策考虑,是当今人类活动主导“自然-社会二元水循环”情况下决定人水关系的关键因素。
本章整合了水文数据、统计数据、历史文献材料等多源数据,开发流域水治理综合指数,并利用断点识别方法量化识别了上世纪中期以来的黄河流域治理的阶段变化,同时深入分析了转变发生的驱动因素。

本章研究表明,两个突变点可以将$1965 \sim 2013$ 年的黄河流域水治理演变历史划分为明显的三个阶段,依据其各自特点可命名为:集中供水时期($1965 \sim 1978$,P1)、治理转型时期($1979 \sim 2001$,P2)、适应增强时期($2002 \sim 2013$,P3),而“多少水(S)”、“怎么用(P)”和“怎么分(A)”的三个方面的指标在不同时期为黄河流域水治理整体特征的变化做出不同贡献。
进一步探讨了致使IWGI变化的原因,灌溉区扩张和工业和服务业的经济增长是推动“集中供水时期”和“治理转型时期”两个阶段变化的关键,环境背景、社会文化、水治理政策等因素则为三个时期的指标变化都做出了贡献。

黄河流域的水治理演化过程是“增大供给、治理转型、适应增强”的突出的案例,而其中变化的内在机制已在世界人类主导流域社会-水循环的过程中广泛出现。
理解黄河流域水治理的变化规律,可以为快速变化的大河流域提供实现可持续性的重要理论依据。

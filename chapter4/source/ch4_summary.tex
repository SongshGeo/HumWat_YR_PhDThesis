本章整合了水文数据、统计数据、历史文献材料等多源数据,开发流域水治理综合指数,利用断点识别方法量化识别了上世纪中期以来的黄河流域治理的阶段变化,并分析了转变发生的驱动因素。

本章研究表明,两个突变点可以将$1965 \sim 2013$ 年的黄河流域水治理演变历史划分为明显的三个阶段,依据其各自特点可命名为:集中供水时期($1965 \sim 1978$,P1)、治理转变时期($1979 \sim 2001$,P2)、适应增强时期($2002 \sim 2013$,P3),而“稀缺情况(S)”、“使用目的(P)”和“分配方式(A)”的三个方面的指标在各阶段为黄河流域水治理的特征变化做出了不同贡献。
进一步探讨致使IWGI变化的原因,灌溉区扩张和工业和服务业的经济增长是推动“集中供水时期”和“治理转变时期”两个阶段变化的主要原因,环境背景、社会文化等因素对不同阶段的指标变化都做出了贡献,水治理政策是将“治理转变时期”推向“适应增强时期”的主要驱动力。

黄河流域的水治理演化过程是“增大供给、治理转变、适应增强”的突出的案例,而其中变化的内在机制已在世界人类主导流域社会-水循环的过程中广泛出现。
理解黄河流域水治理的变化规律,可以为快速变化的大河流域提供实现可持续性的重要理论依据。

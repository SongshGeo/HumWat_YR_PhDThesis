\chapter{百年尺度的黄河流域人水关系演变}\label{cha:4}
第三章研究表明,人为压力和气候变化因素仅触发了黄河泥沙量的稳态转换,黄河流域在千年尺度上是由层级制度主导的“洪泛-响应”关系。除此之外,非中央政府主导的开发利用、生产贸易、综合治理等因素,尚没有显著影响人水关系稳态。而现代黄河流域人水关系出现的关键变化之一就是对黄河流域全面的综合开发、利用、以及治理。黄河流域的战略瓶颈已经由保护黄河泛滥转向应对高质量发展面临的水资源短缺、人水关系不协调问题之上。第三章研究已经表明,这是19世纪以降短短百余年间经济社会高速发展带来的问题,而新中国成立以来,高强度的黄河流域综合治理,使得近六十年间黄河流域人水关系的演变模式与过去千年相比显得截然不同。

首先在流域内部,积累的取水压力让这个位于干旱-半干旱区的流域迅速不堪重负,取水量一度接近年径流的$80\%$,其中近$90\%$均为农业取水。因此,黄河自1970年代开始频繁断流,这引起中央高度重视,开始采取一系列水资源管理的工程、非工程措施来协调水资源供需矛盾。与此同时,大规模的经济建设和城市发展也让宝贵水资源的流域内分配成为了挑战。而跨区域贸易带来了水资源以农产品形式向流域外的转移,让水资源分配的问题变得更加复杂严峻。因此,黄河流域已从过去的调水调沙与防洪等河道内外的“工程问题”,转向了“谁、在何时得到水”的流域层面的“非工程问题”,这是近现代人水关系演变的体现。而这种稳态转变何时发生、如何发生,仍需要定量研究。本章在黄河流域市级统计数据的支持下,利用断点识别、复杂网络分析等方法量化分析了上世纪七十年代以来的黄河人水关系演变过程及驱动因素。


\section{研究方法与数据来源}\label{ch4:methods}
在人类活动主导下的人-水关系中,根据我们的定义,人类社会的“水治理”深入影响了自然,
制定一个全面和直接的方法来确定水治理机制。

\subsection{人类主导时期的水资源治理分框架}

水治理是指影响水的使用和管理的政治、社会、经济和行政系统,本质上是关于“谁获得水,何时获得水,如何获得水”。
因此,联合国开发计划署(UNDP)提出,由水治理相应地决定用水的三个核心方面:“什么时候用、用什么水?”“(重音),‘水如何为人类福祉提供不同的服务?“(目的)”和“谁能平等有效地使用水?””(分配)
\cite{undpwatergovernancefacility2016}。
首先,水资源压力不仅取决于气候(在许多地区日益稀缺和不确定性),还取决于灌溉和工业等经济活动日益无法满足的需求;蓄水可以解决部分问题,但不能解决全部问题
\cite{qin2019,wada2014,huang2021}。
其次,水如何服务于人类福祉的目的是考虑消费用途(例如,饮用和食品生产)和非消费用途(例如,能源生产)之间的权衡。
\cite{liu2017,florke2018,jaeger2019}。
第三,整个流域的水资源分配不仅取决于区域的社会经济和环境背景,还受到系统监管的影响
\cite{schmandt2021,speed2013}。
由于向人类主导体制的过渡导致了三个相互关联方面(压力、目的和分配)的实质性变化,单独考虑它们会导致水治理的系统性失败。

要理解水治理的成功与失败,关键的第一步是确定支撑水治理的不同制度。
水治理机制产生于相互关联的人-水系统(基于管理、制度和开发),以在社会-生态结构和功能中创造局部平衡
\cite{falkenmark2021,bressers2013,loch2020}。
例如,在人类主导的制度下,水库的灵活性使水资源压力更容易缓解;不断增长的能源和工业需求使供水服务向非供应部门倾斜;输水系统使水分配更有计划性(图~\ref{ch4:fig:framework}~A)
然而,缺乏一种全面而直接的方法来确定水治理制度的变化,这对加强水资源利用可持续性的努力构成了挑战。
填补这一空白是本文的目的,这对于人类和水系统的适当调整至关重要。

% 这里我们整合了三个方向,提出了描绘流域人水关系的指数
在这里,我们用相应的指标(见方法)描述了水治理的三个方面(压力、目的和分配),从而通过对它们进行同等加权,得出了综合水治理指数(IWGI),以表明水治理的结果(见图\ref{ch4:fig:framework}~B)。
% 使用案例研究
然后,通过将该指数应用于一个典型的快速变化的大流域(YRB),我们展示了IWGI如何帮助全面而直接地检测和描述复杂的水治理机制。
在综合分析了水的需求、供应、经济结果和制度的变化之后,我们解释了政权转变的主要原因。
% 最后总结出一般性框架
最后,我们提出了一个普遍的政权过渡模式,为探索大流域治理面临的挑战提供了一个协调方法的实践指南。


首先,本章研究从压力(Stress)、目的(Purpose)和分配(Allocation)三个方面构建了综合水治理指数(Integrated Water Governance Index, IWGI),见图~\ref{ch4:fig:framework}。
然后,本章研究利用变化点检测方法分析了$1965\sim2013$年IWGI的变化。
每个维度的指标经标准化后,分析其对IWGI趋势变化的影响和贡献。


\begin{figure}[!ht]
\centering
\includegraphics[width=\textwidth]{img/ch4/framework.png}
\caption[利用综合水治理指数(IWGI)识别水社会循环转型中的水治理机制]{
    利用综合水治理指数(IWGI)识别水社会循环转型中的水治理机制。水资源压力(S)、供水服务的目的(P)和水资源分配(A)是需要考虑的三个方面(\textbf{A.})。随着水系社会循环的转变,以人为主导的制度影响着水治理的这些方面。例如,水库的建设(1)旨在缓解水资源紧张;能源和工业增长(2);水铅密集型农业(3);输水系统(4)控制水的分配。因此,该方法是结合三个方面的相应指标,然后IWGI的突变可以表明水治理的政权转移(\textbf{B.})。
}
\label{ch4:fig:framework}
\end{figure}

\subsection{研究区域定义}
\label{ch4:sec:region}

本章研究将黄河流域划分为四个区域,以计算考虑社会经济和自然条件的指标。该划分与出版物和YRCC \cite{shuilibuhuangheshuiliweiyuanhui,wang2019c,wang2016e}的习惯模式一致,因此可以区分四个重要的水文站(见图~\ref{fig:YRB})。
\begin{itemize}
\item \textbf{黄河源区(SR):}超过$50\%$的自然径流来自这个地区。这里人口稀少,经济不发达,最生态的功能是产水。
\item \textbf{黄河上游(SR):}这个地区人均灌溉土地面积最高,有大量的大型灌溉土地。然而,灌溉效率相对较下游低得多。
\item \textbf{黄河源区(MR):}黄河流经著名的富沙区黄土高原,这里是黄河淤积最多的地区,也是土壤侵蚀风险最高的地区。“退耕还林”工程显著改变了这里的水资源利用,扭转了这一局面。
\item \textbf{黄河下游(LR):}下游地区人口密集,传统农业发展轨迹,曾是最大的用水区。然而,随着产业转型的进行,农业的比重不断下降,但LR仍是各方面用水最多的地区。
\end{itemize}

% 补充图片1:研究区示意图
\begin{figure*}[hbtp!]
\centering
\includegraphics[width=\textwidth]{img/ch4/s1_study_area.jpg}
\caption[黄河流域子区域划分]{黄河流域子区域划分。
    \textbf{A.}黄河下游与盆地划分示意图(SR:源区,UR:上区,MR:中区,DR:下游区)
    \textbf{B.}黄河主河道剖面图。水文站控制SR、UR、MR和DR。
    \textbf{C.}黄河流域不同地区的典型景观。
}
\label{fig:YRB}
\end{figure*}


\subsection{水资源治理综合指数}

% 将三者合一起,即:
如框架图~\ref{ch4:fig:framework}所示,IWGI结合了水治理的三个方面(压力、目的和分配)。每个维度保持两个方向,本章研究假设水社会循环与其中一个方向对齐,分别为:

\begin{equation}
    Transformation \propto S*P*A
\end{equation}

本章研究选择了一个指标($I_x$, $x=S$, $P$,或$A$,分别对应压力,目的和分配)来有效地量化这些方面。然后将上式转化为自然对数,便于计算:

\begin{equation}
    Transformation \propto ln(I_S) + ln(I_P) + ln(I_A)
\end{equation}

那么,综合水治理指数(IWGI)是标准化指标$I'_x$的平均值:
\begin{equation}
    IWGI = (I'_S + I'_P + I'_A) / 3
\end{equation}

其中:
\begin{equation}
    I'_x = (I_x - I_{x, min}) / (I_{x, max} - I_{x, min})
\end{equation}

\subsubsection*{水资源压力指数}

本章研究采用Qin等人(2019)提出的稀缺性-韧性-可变性(SFV)水胁迫指数来评价水胁迫\cite{qin2019}。这一指标考虑了管理措施(如水库的建设)和用水结构变化对水资源短缺评估的影响。SFV指数考虑了水资源的灵活性和可变性,从发展的角度更关注水资源的动态响应,是衡量水资源压力\cite{qin2019}时间变化的有效指标。
根据黄河流域的水文和经济背景,划分了四个二级区域(源区、上区、中区和下区,见\textit{ Supporting  Information S1})。整个YRB的水分胁迫指标$I_S$为各区域sfv指数的平均值:

\begin{equation}
    I_S = \frac{1}{4} * \sum_{i=1}^4 SFV_{i}
\end{equation}

其中$SFV_i$为区域$i$的SFV指数$SFV_i$,结合了以下三个指标:

首先,对于稀缺性,$A_{i, j}$为区域$i$在第$j$年的耗水量占多年平均径流量的比例(本研将为黄河流域划分为四个子区域,见\ref{ch4:sec:region}\nameref{ch4:sec:region}):

\begin{equation}
    A_{i, j} = \frac{WU_{i,j}}{R_{i, avg}}
\end{equation}

其次,对于灵活性,$B_{i, j}$是第$i$年和第$j$地区的不灵活用水$WU_{inflexible}$(例如能源行业冷却用水或人类和牲畜)占平均多年径流量的比例:

\begin{equation}
    B_{i, j} = \frac{WU_{i, j, inflexible}}{R_{i, avg}}
\end{equation}

最后,易变性(Variability)还考虑了水库容量和蓄水对自然径流波动的积极影响:
\begin{gather}
    C_i = C1_i * (1 - C2_i) \\
    C1_{i, j} = \frac{R_{i, std}}{R_{i, avg}} \\
    C2_{i} = \frac{RC_{i}}{R_{i, avg}}, \ if RC < R_{i, avg} \\
    C2_{i} = 1, \ if RC >= R_{i, avg}
\end{gather}

上式中,$R_{i, avg}$为$i$区域的平均径流量,$RC_i$为$i$区域水库的总库容,$R_{i, std}$为$i$区域径流量的标准差。

最后,假设三个指标(稀缺性、灵活性和可变性)具有相同的权重,我们可以将它们归一化后计算出$SFV$指标:

\begin{gather}
    V = \frac{A_{normalize} + B_{normalize} + C_{normalize}}{3}\\
    a = \frac{1}{V_{max} - V_{min}};\\
    b = \frac{1}{V_{min} - V_{max}} * V_{min}\\
    SFV = a * V + b
\end{gather}


\subsubsection*{水资源供给指标}

为了量化目的$I_P$,本章研究使用了用水的非供应目的份额(NPS)作为指标。供应目的用水($WU_{pro}$)包括家庭用水、灌溉用水和牲畜用水,非供应目的用水($WU_{non-pro}$)包括工业用水和城市服务用水。本章研究计算NPS为:

\begin{equation}
    NPS = \frac{WU_{pro}}{WU_{pro} + WU_{non-pro}}
\end{equation}

在本研究中,本章研究将牲畜用水、城乡生活用水和农业用水作为供应用水,因为它们直接服务于生存。其他的是非供应:服务和工业用水,因为它们主要为经济服务。

\subsubsection*{水资源分配指标}
为了描述分配$I_A$,本章研究设计了一个基于熵的分配度量指标,它衡量水分配的均匀程度:

\begin{equation}
    I_A = CEM = \sum_{i=1}^N -log(p_{i}) * p_{i}
\end{equation}

其中$p_{i}$为区域$i$与整个流域的水量比例(这里,$N=4$考虑了黄河流域的划分区域,见\textit{ Supporting  Information S1})。

\subsection{变化点检测}

本章研究采用Pettitt(1979)提出的的变化点检测方法,在不假设数据分布的情况下,对连续数据的水文时间序列中的单个变化点进行检测\cite{pettitt1979}。
它测试的原假设$H0$是:独立同分布的变量不存在存在变化趋势差异,备择假设则为存在一个变化趋势点。
数学上,将随机变量序列分为$\mathrm{x}_{1}, \mathrm{x}_{1}, \ldots, x_{t_{0}}$和$x_{t_{0}+1}, x_{t_{0}+2}, \ldots, x_{T}$表示的两段,如果每段都有一个共同的分布函数,即$F_1(x)$、$F_2(x)$和$F_1(x) \neq F_2(x)$,则在$t_0$处确定变化点。为实现变化点的识别,定义统计指标$U_{t,T}$如下:

\begin{equation}
    U_{t, T} = \sum_{i=1}^t\sum_{j=t+1}^T sgn(X_i - X_j), 1 \leq t < T
\end{equation}

其中:
\begin{equation}
    \operatorname{sgn}(\theta)= \begin{cases}1 & \text { if } \theta>0 \\ 0 & \text { if } \theta=0 \\ -1 & \text { if } \theta<0\end{cases}
\end{equation}

找到最可能的变化点$\tau$,其值满足$K_{\tau} = max|U_{t, T}|$,与值$K_{\tau}$相关的显著性概率近似计算为:

\begin{equation}
    p=2 \exp \left(\frac{-6 K_{\tau}^{2}}{T^{2}+T^{3}}\right)
\end{equation}
给定某个显著性水平$\alpha$,如果$p < \alpha$,本章研究拒绝原假设,并得出结论,$x_{\tau}$是水平$\alpha$的显著变点。

本章研究使用$\alpha = 0.001$作为p值的阈值水平,这意味着统计上显著的变化点判断有效的概率大于$99.9\%$。本章研究将该序列分为两个,并分别分析每个序列,直到检测到所有重要的变化点。虽然在$\alpha = 0.001$的正文中有两个断点,但是从$0.0005$到$0.05$的阈值并不影响本章研究的结果,并且本章研究识别的断点是鲁棒的(参见图~S3)。

\subsection{数据来源与处理}
为了计算 IWGI ,本章研究需要计算多个指标及子指标,所有使用的数据集都在表\ref{ch4:tab:data_source}中列出。

% Table generated by Excel2LaTeX from sheet '数据集'
\begin{table}[htbp]
    \centering
    \caption{数据分类与来源}
      \begin{tabularx}{\textwidth}{p{4cm}LLLp{3cm}}
      \toprule
      数据集   & 数据类型  & 空间尺度  & 时间尺度  & 数据来源 \\
      \midrule
      行政区水资源利用 & 统计    & 市级行政单元 & $1965-2013$ & 文献\cite{zhou2020} \\
      子流域水资源使用 & 统计    & 二级子流域 & $2003-2019$ & 水资源公报 \\
      水库数据集 & 水文    & 站点数据  & $1949-2015$ & 文献\cite{wang2019f} \\
      实测泾流量 & 水文    & 站点数据  & $1949-2019$ & 文献\cite{wang2019f} \\
      黄河流域相关法律 & 文献    & 流域相关文件 & $1949-2013$ & 黄河流域规划\cite{shuilibuhuangheshuiliweiyuanhui2010} \\
      黄河水利委员会历史沿革 & 文献    & 流域相关文件 & $1949-2002$ & 黄河水利委员会档案馆 \\
      黄河大事件 & 文献    & 流域相关文件 & $1949-2015$ & 黄河水利委员会档案馆 \\
      \bottomrule
      \end{tabularx}%
    \label{ch4:tab:data_source}%
\end{table}%
  

我们使用GDP数据;水资源数据来源于第二次国家水资源评估方案\cite{zhou2020}和统计年鉴\url{http://www.yrcc.gov.cn/other/hhgb/}。
水资源利用数据集由Zhou et al. \cite{zhou2020}发布,该数据集记录了不同部门的水资源利用情况以及地级的社会经济状况。第二次国家水资源评价计划主要提取了2002年国家发展和改革委员会、水利部牵头启动的该数据集(详见ref(1)和\url{http://www.mwr.gov.cn/english/publs/})。从那时起,使用相同标准的调查统计数据与2013年的行政区划进行了补充和协调。

数据涵盖了用水的四个大类下的子类别:农业(IRR)、工业(IND)、城市(URB)和农村(RUR)用水(详见Zhou et al., \cite{zhou2020})。
每个分类用水部门在县尺度上都存在不确定性,但由于校正数据是使用水平衡方法进行统计信息,因此该数据足以用于本研究中使用的区域尺度。

\subsubsection{水资源数据集}
储层数据集由Wang et al. \cite{wang2019c}收集,其中介绍了1949年以来在黄河上游新建的重要储层(图~\ref{fig:reservoirs})。YRCC在其中标注了调控型储层,见\url{http://www.yrcc.gov.cn/hhyl/sngc/})。此外,从水文站测量得来的年径流数据与\cite{wang2019c}和\cite{wang2016e}使用的数据集相同。

\begin{figure}[tb]
    \centering
    \includegraphics[width=0.6\linewidth]{img/ch4/reservoirs.jpg}
    \caption{
        各年新增水库数量.
    }
    \label{fig:reservoirs}
\end{figure}

\subsubsection{政治数据集}

政策数据集收集了\cite{shuilibuhuangheshuiliweiyuanhui}书中所列的黄河流域相关法律政策,由流域级以上部门(如长江水利委)(如国家机关)颁布实施(表~\ref{ch4:tab:policies})。
此外,有些很难分类;黄河水利委员会的黄河大事件记录了黄河流域的许多治理实践,但这并不是一个里程碑;我们从\url{http://www.yrcc.gov.cn/hhyl/hhjs/}中收集它们。

% Table generated by Excel2LaTeX from sheet '黄河流域法律政策'
\begin{table}[htbp]
    \centering
    \caption{黄河流域法律政策}
      \begin{tabularx}{\textwidth}{L p{1cm} L}
      \toprule
      法律或政策名称 & \multicolumn{1}{l}{时间} & 颁布机构 \\
      \midrule
      中华人民共和国水法 & 1,988 & 全国人民代表大会常务委员会 \\
      中华人民共和国水法  修正 & 2,002 & 全国人民代表大会常务委员会 \\
      中华人民共和国水法  第一次修订 & 2,009 & 全国人民代表大会常务委员会 \\
      中华人民共和国水法  第二次修订 & 2,016 & 全国人民代表大会常务委员会 \\
      中华人民共和国水污染防治法 & 1,984 & 全国人民代表大会常务委员会 \\
      中华人民共和国水污染防治法  修正 & 1,996 & 全国人民代表大会常务委员会 \\
      中华人民共和国水污染防治法  第一次修订 & 2,008 & 全国人民代表大会常务委员会 \\
      中华人民共和国水污染防治法  第二次修订 & 2,018 & 全国人民代表大会常务委员会 \\
      取水许可和水资源费征收管理条例 & 2,006 & 中华人民共和国国务院 \\
      取水许可和水资源费征收管理条例  第一次修订 & 2,017 & 中华人民共和国国务院 \\
      黄河水量调度条例 & 2,006 & 中华人民共和国国务院 \\
      黄河可供水量分配方案 & 1,987 & 中华人民共和国国务院 \\
      取水许可管理办法 & 2,008 & 中华人民共和国水利部 \\
      取水许可管理办法  第一次修订 & 2,015 & 中华人民共和国水利部 \\
      取水许可管理办法  第二次修订 & 2,017 & 中华人民共和国水利部 \\
      黄河水量调度条例 & 2,006 & 中华人民共和国国务院 \\
      黄河可供水量年度分配及干流水量调度方案 & 1,998 & 国家发展计划委员会,水利部 \\
      黄河水量调度管理办法 & 1,998 & 国家发展计划委员会,水利部 \\
      黄河水权转换管理实施办法 & 2,004 & 水利部 \\
      取水许可和水资源费征收管理条例 & 2,006 & 中华人民共和国国务院 \\
      取水许可证制度实施办法 & 1,993 & 中华人民共和国国务院 \\
      建设项目水资源论证管理办法 & 2,002 & 国家发展计划委员会,水利部 \\
      水利工程管理体制改革实施意见 & 2,006 & 中华人民共和国国务院 \\
      \bottomrule
      \end{tabularx}%
    \label{ch4:tab:policies}%
  \end{table}%
  


\section{人类活动主导时期人水关系演变过程}\label{ch4:process}
\subsection{综合指标变化过程}\label{ch4:sec:process}

\begin{figure*}[ht!]
	\centering
	\includegraphics[width=\textwidth]{img/ch4/ch4_index.png}
	\caption[IWGI指数反映黄河流域的水治理变化阶段]{IWGI指数反映黄河流域的水治理变化阶段。两个突变点将1965年以来的黄河流域水治理划分为三个阶段,第一阶段(P1): $1965 \sim 1978$,第二阶段(P2): $1979 \sim 2001$,第三阶段(P3): $2002 \sim 2013$。
	\textbf{A,} 检测IWGI的突变点和三个指标的各自贡献:“稀缺情况(S)”、“使用目的(P)”和“分配方式(A)”。1978年和2001年出现了两个显著的变化点($p<0.01$)。
	\textbf{B,}  各阶段的IWGI变化与三个指标各自变化的相关性。
	\textbf{C,} IWGI随时间变化的同时,三个指标贡献比例的组合不断改变,致使水治理向不同方向发生阶段性转移。
	}\label{ch4:fig:IWGI}
\end{figure*}

IWGI在研究时段内存在两个突变点,将1965年以来的黄河流域水治理划分为三个阶段,第一阶段(P1): $1965 \sim 1978$,第二阶段(P2): $1979 \sim 2001$,第三阶段(P3): $2002 \sim 2013$,而“稀缺情况(S)”、“使用目的(P)”和“分配方式(A)”的三个指标在每个阶段的贡献不同(图~\ref{ch4:fig:IWGI}A)。
% 第一阶段
在第一个时期(P1, $1965 \sim 1978$)水资源压力对IWGI的贡献很小,“使用目的”和“分配方式”的指标的贡献更大(平均分别为$49.45\%$和$34.95\%$),但均呈现出显著的下降趋势($p<0.01$,图~\ref{ch4:fig:IWGI}~B),导致此时期IWGI迅速下降。
% 第二阶段
在第二阶段(P2, $1979 \sim 2001$),水资源压力指标的显著增加($p<0.01$)并为IWGI的略微上升做出主要贡献($p<0.01$,图~\ref{ch4:fig:IWGI}~A),而“使用目的”和“分配方式”的指标对IWGI的变化起了消极作用。
% 第三阶段
最后,在第三个时期(P3, $1995 \sim 2013$),尽管水资源压力指标在贡献中保持着$57.11\%$的最突出份额,但其数值已几乎保持不变,反而是“使用目的”的指标的降低和“分配方式”指标的增加,共同推动了综合指标IWGI的变化。
%的整体
综上所述,“稀缺情况(S)”、“使用目的(P)”和“分配方式(A)”的三个指标在不同时期对黄河流域水治理整体特征变化的贡献不同,将其演变历史划分为明显的三个阶段,依据其各自特点可命名为:集中供水时期、治理转变时期、适应增强时期(对应时间阶段分别为$1965 \sim 1978$、$1979 \sim 2001$、$2002 \sim 2013$(图~\ref{ch4:fig:IWGI}~C)。

\subsection{各子指标变化}

\begin{figure}[!ht]
  \centering
  \includegraphics[width=\textwidth]{img/ch4/ch4_indicators.png}
  \caption[各子指标变化趋势]{
	各子指标变化趋势。
	\textbf{A} 稀缺情况(S)
	\textbf{B} 使用目的(P)
	\textbf{C} 分配方式(A)
}\label{ch4:fig:indicators}
\end{figure}


构成稀缺情况的指标(SFV指数)在研究时段(包括三个不同时期)内呈现出先降低、然后迅速增加、最后再次略微降低的变化趋势(图~\ref{ch4:fig:indicators}~A)。
% ,表明水资源压力先减少再迅速增加,后趋于稳定
源区、上游、中游、下游这四个不同的区域中(表~\ref{ch4:tab:sfv_contribution}),源区对三个时段的SFV指标变化几乎没有贡献,下游也仅在治理转变时期和适应增强时期呈现微弱的负向贡献。
对稀缺情况影响最大的是黄河的上游和中游,上游在集中供水时期和治理转变时期都是SFV变化的最大贡献区域,中游则在适应增强时期做出最大贡献。

% Table generated by Excel2LaTeX from sheet 'sfv各区域贡献'
\begin{table}[!ht]
  \centering
  \caption{黄河源区、上中下游对稀缺情况指标(IS)变化的贡献}
    \begin{tabularx}{0.5\textwidth}{lrrr}
    \toprule
          & \multicolumn{1}{l}{$1966-1978$} & \multicolumn{1}{l}{$1978-2001$} & \multicolumn{1}{l}{$2001-2013$} \\
    \midrule
    源区 & 0.00\% & 0.00\% & 0.00\% \\
    上游 & 90.95\% & 89.94\% & -5.28\% \\
    中游 & 9.05\% & 36.42\% & 123.86\% \\
    下游 & 0.00\% & -26.36\% & -18.58\% \\
    \bottomrule
    \end{tabularx}%
  \label{ch4:tab:sfv_contribution}%
\end{table}%


在用水目的上,供给性用水比例在集中供水时期基本保持不变,但在治理转变时期和适应增强时期呈现迅速下降的趋势(图~\ref{ch4:fig:indicators}~B)。
三个时段都是由灌溉用水的变化主导了该比例变化,城市、农村的人居用水、农村牲畜用水等几乎对该比例的变化没有影响(表~\ref{ch4:tab:ip_contribution})。

% Table generated by Excel2LaTeX from sheet 'purpose各区域贡献'
\begin{table}[!ht]
    \centering
    \caption{不同用水部门对使用目的指标(IP)变化的贡献}
      \begin{tabularx}{0.8\textwidth}{lrrr}
      \toprule
            & \multicolumn{1}{l}{P1: $1965-1977$} & \multicolumn{1}{l}{P2: $1978-2000$} & \multicolumn{1}{l}{P3: $2001-2013$} \\
      \midrule
      IP总变化 & 0.45\% & -3.68\% & -3.57\% \\
      灌溉用水   & 45.02\% & 13.59\% & -6.11\% \\
      城市居民用水 & 1.03\% & 0.53\% & -0.61\% \\
      农村居民用水 & 1.86\% & 0.61\% & -0.37\% \\
      农村牲畜用水 & 0.48\% & 0.20\% & -0.16\% \\
      \bottomrule
      \end{tabularx}%
    \label{ch4:tab:ip_contribution}%
  \end{table}%
  

分配方式的指标变化呈现明显“V形”趋势,表明黄河的源区、上游、中游、下游之间水资源呈现先逐渐远离均匀分配,又在2000年后逐渐趋于平均的变化过程(图\ref{ch4:fig:indicators}~C)。


\section{人类活动主导时期人水关系演变机制}\label{ch4:mechanism}

% \subsection{水治理变化的驱动力分析}\label{ch4:sec:mechanism}

\begin{figure}[th!]
	\centering
	\includegraphics[width=\textwidth]{img/ch4/causes.pdf}
	\caption[黄河流域水治理阶段变化的驱动因素]{
		黄河流域水治理阶段变化的驱动因素。
		\textbf{A.}总灌溉面积($A$, 橙色线)和用水强度($WU/A$,用水量除以灌溉面积,绿点线)的变化。
        \textbf{B.}工业和服务业的总增加值(蓝线,$GVA$)变化及其用水强度($WU/GVA$,WU除以GVA,红点线)。
        \textbf{C.}每个水库的完工时间及其所在区域的用水量(Local Water Use, LWU)占水库完工时流域总用水量(Basinal Water Use, BWU)的百分比。红圈为负责黄河流域综合调度的水库。每个圆圈的大小表示其储水能力的大小。
        \textbf{D.}社会转型(红色三角形)和国家层面的治理政策(圆圈,不同颜色表示由不同的国家机构签署,越靠上代表国家机构的登记越高,详见表\ref{ch4:tab:policies})。浅灰色条形图以流域尺度(黄河大事件)计算与黄河流域有关的官方治理文献记录。}\label{ch4:fig:mechanism}
\end{figure}


\begin{figure}[tb]
    \centering
    \includegraphics[width=0.6\linewidth]{img/ch4/ch4_reservoirs.png}
    \caption{黄河流域新增水库数量的时间分布}\label{ch4:fig:reservoirs}
\end{figure}


本节进一步探讨了致使IWGI变化的原因,灌溉区扩张和工业和服务业的经济增长是推动“集中供水时期”和“治理转变时期”两个阶段变化的关键。
黄河流域的用水需求在“集中供水时期”迅速增加,尤其是灌溉农业面积以$0.25*10^6 ha/yr$的速度迅速扩张(图\ref{ch4:fig:mechanism}~A),同时通过建设水库增加供应(图~\ref{ch4:fig:reservoirs})。
进入“治理转变时期”后,尽管灌溉区扩张停滞,但工业和服务业的发展开始增长,共同推动流域用水需求的进一步增加(图\ref{ch4:fig:mechanism}~A和B)。
接下来从“治理转变时期”到“适应增强时期”的演变过程中,水分利用效率变化最明显。
在“适应增强时期”,不仅工业和城市服务也承担了更重要的经济角色(由总增加值GVA表示,图\ref{ch4:fig:mechanism}~B),灌溉面积也恢复了缓慢扩张(图\ref{ch4:fig:mechanism}~A)。
但因用水效率的普遍提高,单位灌溉面积或单位产量的用水量都显著下降(图~\ref{ch4:fig:mechanism}~A和图~\ref{ch4:fig:mechanism}~B),因此部门和地区之间的用水差异在不断缩小,但流域水资源压力总体、持续维持在较高水平,公平合理分配宝贵水资源的压力越来越大(图~\ref{ch4:fig:IWGI}~A)。

最后,环境背景、社会文化、水治理政策等因素为三个时期的指标变化都产生了影响。
本研究首先计算了每个水库的区域用水量和流域用水量之比,较高的比值代表了该水库等潜在作用更有可能是旨在为该流域供水而不是流域调度;此外,流域调度的枢纽水库也以红色进行了标记(图~\ref{ch4:fig:mechanism}~C)。
可以看到,在阶段一的“集中供水时期”,受“人定胜天”的社会环境引领,大部分水库都建在需水量较大的地区,因此区域用水量和流域用水量之比值明显较高($p<0.01$,见图~\ref{ch4:fig:mechanism}~C)。
进入“治理转变时期”之后,新建水库的数量明显减少且多为枢纽水库,但全流域层次的法律法规(包括著名的“八七”分水方案)开始被不断提出,流域内的治理记录也迅速增加,可见此时期层出不穷的流域政策已深刻影响了流域水治理,流域水治理正在进行一场从工程措施向非工程措施的“治理转变”(图~\ref{ch4:fig:mechanism}~D, $p<0.01$和图\ref{ch4:fig:reservoirs})。
最后在“适应增强时期”,持续高位的水压力已成为制约区域发展的瓶颈,亟需通过节水转型和跨区域协调、调水来满足经济发展的用水需求,因此并且“大规模进行环境治理和节水转型”的国家战略指导下,有关部门提出了更多的、级别更高的水治理决策(图~\ref{ch4:fig:mechanism}~D)。
综上所述,从“集中供水时期”到“治理转变时期”的转变与当时水资源供需的增加相一致;而“治理转变时期”到“适应增强时期”的演变过程则是在水资源压力趋于稳定的同时,由社会监管政策和节水转型带来的效率提高所驱动的。

\section{讨论}\label{ch4:discussion}
\section{人水关系变化模式总结}

研究结果显示黄河流域在人类主导时期有三种不同的流域治理模式:集中供水时期(P1: $1965 \sim 1978$)、治理转型时期(P2: $1979 \sim 2001$)和适应转向时期(P3: $2002 \sim 2013$)(图~\ref{ch4:fig:IWGI})。
在黄河流域水资源压力相对较低的集中供水时期($1965 \sim 1978$年),主要的水资源需求是为牲畜和作物等供给服务为目的,水治理也倾向于通过建造水库和引水渠等来增加水资源供给。
然而,正如当时“人定胜天”的口号所暗示的那样,水资源供应的增加并不能促进人-水关系和谐,因为它在不考虑生态保护的情况下急剧增加了用水,且这常常是一种不可逆转的变化\cite{zhou2020}。
在接下来的十年内,灌溉农田和引水设施的迅速扩张使黄河流域超负荷取水,在1972年以来,超过$80\%$的地表水被使用,导致河流频繁枯竭,造成了额外的生态问题,如湿地萎缩和生物多样性下降\cite{wang2019c}。
此外,由于水资源压力也限制了新兴的、更有利可图的工业与服务业的发展,这让流域的水治理模式接近了社会-生态危机的临界点,因为继续增加供应无论如何都是不切实际的\cite{loch2020, wohlfart2016a}。

治理体制转型时期的开始(P2: $1979 \sim 2001$)恰逢“改革开放”后,水资源竞争的持续加剧,但枯竭的黄河已经开始断流。
黄河流域在此时期开始水治理转型,这一结果与理论分析的结果高度一致:在流域总供应稳定的情况下,水资源需求在接近可供给水资源上限后的持续增长,将为流域水治理带来重大转型,流域社会-水文系统将通过制度措施让快速增强社会适应力,以响应该水资源供小于求的临界点\cite{loch2020}。
因此黄河流域在此时期发起了诸多水治理举措,包括控制灌溉面积的增长、倡导节水设施建设、制定全国首个水资源配额制度、并初步制定跨流域调水方案(南水北调)等\cite{wang2019b,long2020,nickum2021},成为了中国九大流域的制度变革先驱。
因此,尽管黄河流域的水资源压力仍然很大,且因径流减少和用水灵活性降低而持续增加,但黄河的断流问题却得到了解决,$1999$年的最后一次断流向世人昭示着此次水治理转型的巨大影响\cite{wang2019b}。

在随后的适应增强时期(P3: $2002 \sim 2013$)为适应稳定在高位的水资源压力,许多国家层面的黄河流域水治理实践都在这一时期提出,以期用最经济地方式,在保护生态的同时实现流域高质量发展。
二十一世纪初提出的“环境整治”和$2011$年提出的“最严格水资源管理”都是对水资源粗放式开发利用的响应,不断使流域水治理模式变得更高效。
这一时期除了中央政府主导的适应举措,地区各部门之间为了追求生产效率,社会经济过程为“水怎么用、水怎么分”的权衡发挥了重要作用。
典型的例子是黄河流域的水权转换工作,许多地区都积极推动了农业节水转型,并将节约的水额供给到经济效益更高的工业服务业之中。
通过不同行业和地区对流域水资源的广泛重建,通过日益强化的社会-经济过程调整以前粗放的发展模式以及刚性的制度约束,这些都是水治理适应性增强的体现,才能在有限的供水条件下优化平衡来自不同地区、不同行业的需求\cite{dalin2015,song2022}

黄河流域的水治理演化过程是“增大供给、治理转型、适应增强”的突出的案例,而其中变化的内在机制已在世界人类主导流域社会-水循环的过程中广泛出现(图~\ref{fig:summary})。
随着社会-经济过程对水循环影响加深,不同时期的流域面临不同的水治理挑战:在前期主要是经济和环境方面,在后期集中于与制度和政策方面。
放眼全球其它流域,以水资源短缺和供水困难为代表的水治理挑战是制约发展中流域的瓶颈\cite{allan2019,speed2013,liu2012a}。
人类社会-经济过程主导的发达流域(特别是跨界河流)则须重点解决结构性挑战,如水纠纷或缺乏公平\cite{mirumachi2015}。
本章研究开发并使用的综合水治理指数(IWGI)可以将流域水治理变化过程与这些挑战联系起来,提供了一种解释人类活动主导下人-水关系变化的方法。

\begin{figure}[htbp!]
	\includegraphics[width=\textwidth]{img/ch4/ch4_transition.png}
	\caption[人类主导下流域系统的水治理阶段过渡]{
		人类主导下流域系统的水治理变化过程。蓝色代表人类活动尚未成为主导的社会-水文循环,红色代表由社会经济过程主导的流域水过程。
        \textbf{A.}随着社会经济系统的发展,用以非供给服务的水需求增加;同时,通过水库等工程措施使人们能够控制水循环并局部缓解水资源压力。
        \textbf{B.}随着人为干预升级,不同地区和部门间的用水权衡愈发突出;流域亟需提升利用效率和调度能力,并组织起的更具适应性的水治理。
        \textbf{C.}在人类主导的流域社会-水文循环模式下,水治理转型常发生在水资源亏缺之后,在社会-经济过程主导下推动适应能力快速增长。在此之前水治理主要面临经济和环境挑战,但随后面临社会和政策挑战。
	}\label{fig:summary}
\end{figure}

\section{研究局限与意义}

通过在数据丰富的黄河流域进行应用,本章研究表明所有的水治理问题都会导致“谁获得水、何时获得水以及如何获得水”的改变,因此监测“有多少水、怎么用、怎么分”三个关键问题对识别流域水治理变化有极大帮助。
但在世界范围内广泛应用该方法的主要局限是缺乏全球尺度的长时间序列数据,这意味着IWGI的缺点是难以推广。
在数据相对不足的流域应用IWGI时,建议不同方面的指标选择可根据现有数据集进行调整,因为底层指标之间关系的变化趋势比精确计算指标更重要。
在当今这个“人类世”,人-水关系由人类活动所主导的情况越来越普遍,应对层出不穷的治理挑战已成为复杂的人-水系统的核心\cite{cumming2018,cumming2014,jaeger2019}。
许多流域仍在不断接近系统随时可能崩溃的“地球界限”\cite{gleeson2020, wang-erlandsson2022},只有更深入地理解流域水治理系统,结合非线性稳态变化和转型的思想,才能维持流域社会-生态系统的韧性并实现可持续、高质量发展\cite{falkenmark2019}。

% Implications
由于大型河流流域是生态系统服务、经济发展和人类福祉的关键来源,水治理逐渐从主要的地方问题成为国家或国际问题\cite{best2019,best2020}。
随着远程耦合在联系日益紧密的世界中提出了更多的水治理挑战,水治理的政权转变与不同的人与水的关系相一致\cite{diaz2019}。
这一过程反映了社会如何通过增强其在水社会循环中的适应能力来改变治理实践的建议,IWGI定量地确定了这一转变\cite{loch2020,turton1999}。
科学家和决策者认识到不断变化的治理挑战是至关重要的,因为在一个阶段下开发的模型、制度、工程和方法不一定适用于另一个治理阶段\cite{reyers2018}。


\section{小结}\label{ch4:summary}
% 本章整合了水文数据、统计数据、历史文献材料等多源数据,开发流域水治理综合指数,利用断点识别方法量化识别了上世纪中期以来的黄河流域治理的阶段变化,并分析了转变发生的驱动因素。

本章研究表明,两个突变点可以将$1965 \sim 2013$ 年的黄河流域水治理演变历史划分为明显的三个阶段,依据其各自特点可命名为:集中供水时期($1965 \sim 1978$,P1)、治理转变时期($1979 \sim 2001$,P2)、适应增强时期($2002 \sim 2013$,P3),而“多少水(S)”、“怎么用(P)”和“怎么分(A)”的三个方面的指标在各阶段为黄河流域水治理的特征变化做出了不同贡献。
进一步探讨致使IWGI变化的原因,灌溉区扩张和工业和服务业的经济增长是推动“集中供水时期”和“治理转变时期”两个阶段变化的主要原因,环境背景、社会文化等因素对不同阶段的指标变化都做出了贡献,水治理政策是将“治理转变时期”推向“适应增强时期”的主要驱动力。

黄河流域的水治理演化过程是“增大供给、治理转变、适应增强”的突出的案例,而其中变化的内在机制已在世界人类主导流域社会-水循环的过程中广泛出现。
理解黄河流域水治理的变化规律,可以为快速变化的大河流域提供实现可持续性的重要理论依据。

% Implications
由于大型河流流域是生态系统服务、经济发展和人类福祉的关键来源,水治理逐渐从主要的地方问题成为国家或国际问题\cite{best2019,best2020}。
随着远程耦合在联系日益紧密的世界中提出了更多的水治理挑战,水治理的政权转变与不同的人与水的关系相一致\cite{diaz2019}。
这一过程反映了社会如何通过增强其在水社会循环中的适应能力来改变治理实践的建议,IWGI定量地确定了这一转变\cite{loch2020,turton1999}。
科学家和决策者认识到不断变化的治理挑战是至关重要的,因为在一个阶段下开发的模型、制度、工程和方法不一定适用于另一个治理阶段\cite{reyers2018}。

第三章研究表明,人为压力和气候变化因素仅触发了黄河泥沙量的稳态转换,黄河流域在千年尺度上是由层级制度主导的“洪泛-响应”关系。除此之外,非中央政府主导的开发利用、生产贸易、综合治理等因素,尚没有显著影响人水关系稳态。而现代黄河流域人水关系出现的关键变化之一就是对黄河流域全面的综合开发、利用、以及治理。黄河流域的战略瓶颈已经由保护黄河泛滥转向应对高质量发展面临的水资源短缺、人水关系不协调问题之上。第三章研究已经表明,这是19世纪以降短短百余年间经济社会高速发展带来的问题,而新中国成立以来,高强度的黄河流域综合治理,使得近六十年间黄河流域人水关系的演变模式与过去千年相比显得截然不同。

首先在流域内部,积累的取水压力让这个位于干旱-半干旱区的流域迅速不堪重负,取水量一度接近年径流的$80\%$,其中近$90\%$均为农业取水。因此,黄河自1970年代开始频繁断流,这引起中央高度重视,开始采取一系列水资源管理的工程、非工程措施来协调水资源供需矛盾。与此同时,大规模的经济建设和城市发展也让宝贵水资源的流域内分配成为了挑战。而跨区域贸易带来了水资源以农产品形式向流域外的转移,让水资源分配的问题变得更加复杂严峻。因此,黄河流域已从过去的调水调沙与防洪等河道内外的“工程问题”,转向了“谁、在何时得到水”的流域层面的“非工程问题”,这是近现代人水关系演变的体现。而这种稳态转变何时发生、如何发生,仍需要定量研究。本章在黄河流域市级统计数据的支持下,利用断点识别、复杂网络分析等方法量化分析了上世纪七十年代以来的黄河人水关系演变过程及驱动因素。

黄河流域是世界上第五大、最富沉积物的河流,由于地质和人类历史的原因,需要综合的水治理
\cite{mostern2021,best2019}。
自20世纪60年代以来,水库、堤坝和保护措施等治理措施已经遏制了数千年来高沉积物负荷所困扰的问题
\cite{wang2016e,song2020a}。
然而,最近出现了新的挑战,如流量减少和水资源枯竭,导致了用水调节和跨流域的水转移——不同的重点水治理策略
\cite{wang2019c}。
目前,还不可能完全解决长江流域水资源压力、生态系统服务之间的权衡、不同区域的不平衡发展问题,使各方都满意
\cite{wohlfart2016a}。
由环境、经济、社会和政治因素引发的治理挑战导致长江流域成为世界上治理最密集的大型流域之一。
因此,确定长江流域内水治理的政权变化,可以为快速变化的大流域以及治理如何应对其可持续性挑战提供重要的见解。
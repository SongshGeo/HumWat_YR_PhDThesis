随着黄河流域进入了当今由人类活动主导的阶段,各级利益相关者通过水资源取用、改变土地覆被、生产贸易等因素对黄河进行了彻底的开发利用。
各级决策者也主导了大型水利工程、流域综合调度、水资源保护制度等流域治理的工程/非工程措施。
因此在现代治黄时期,黄河流域的人水关系演变过程与历史时期截然不同,这种复杂的人-水关系演变过程,何时发生、如何发生、为何发生,仍需要新的方法进行定量研究。

流域水治理是对“有多少水、谁用水、怎么使用”的综合决策考虑,是人类活动主导下人-水关系的决定性因素。
本章在黄河流域市级统计数据的支持下,开发流域水治理综合指数,并利用断点识别方法量化识别了上世纪中期以来的黄河流域治理的阶段变化。
通过进一步整合水文数据、统计数据、历史文献材料等多源数据深入分析其驱动因素。
社会-经济联系日益紧密的流域系统存在诸多水治理挑战,以1960s年代以来人类活动主导的黄河流域为例理解水治理的变化规律,可以为快速变化的大河流域提供实现可持续性的重要理论依据。

第三章研究表明,黄河流域历史时期人-水关系主要是在中央王朝主导下对黄河洪泛等典型问题进行治理的“压力-响应”模式,而不断增长的人为活动压力触发了人-水关系的稳态转换,使流域系统的社会-水二元循环结构摆脱了气候周期桎梏,进入了当今由人类活动主导的状态。
伴随着该演变过程的,是各级利益相关者通过水资源取用、改变土地覆被、生产贸易等因素对黄河进行了彻底的开发利用,而中央政府同时也主导了大型水利工程、流域综合调度、水资源保护制度等流域治理。
新中国成立以降不足百年的时间内,黄河流域的人水关系演变模式与历史时期相比显得截然不同。
而这种自下而上与自上而下过程结合的、由人类活动主导的现代人-水关系演变,正是当今黄河流域的战略瓶颈由“治河”转向高质量发展与水资源短缺的矛盾、人水关系不协调等问题的核心。

仅从宏观描述中,我们也能对近代黄河流域变化之迅速可见一斑。
首先,因流域系统的重要性得到了重视,“治黄先治沙”的科学认识让“水土保持”和“开发利用”成为了新中国早期黄河流域全面的综合治理的重点,渠道、水库、和淤地坝等建设盛极一时,塑造无计其数的良田。
接下来,黄河流域内因人为活动而产生的取水量逐渐接近年径流的$80\%$,其中近$90\%$均为农业取水,水资源压力让这个位于干旱-半干旱区的流域不堪重负。
最后,大规模的经济建设和城市发展,让本就宝贵的水资源如何在流域内进行分配成为了挑战,同时愈加复杂的跨区域贸易网让资源分配挑战变得严峻,恶性争水的事件时常见诸报道。
综上所述,由环境、经济、社会和政治因素引发的治理挑战,让黄河流域成为世界上治理最密集的大型流域之一,国家采取了一系列水资源治理的工程、非工程措施来协调水资源供需矛盾。
而这种复杂的人-水关系演变过程,何时发生、如何发生、为何发生,仍需要新的方法进行定量研究。

流域水治理是对“有多少水、谁用水、怎么使用”的综合决策考虑,是人类活动主导下人-水关系的决定性因素。
本章在黄河流域市级统计数据的支持下,开发流域水治理综合指数,并利用断点识别方法量化识别了上世纪中期以来的黄河流域治理的阶段变化。
通过进一步整合了水文数据、统计数据、历史文献材料等多源数据深入分析了其驱动因素。
理解黄河流域水治理的变化规律,可以为快速变化的大河流域提供实现可持续性的重要理论依据。

% 黄河流域是世界上第五大、最富沉积物的河流,由于地质和人类历史的原因,需要综合的水治理 \cite{best2019}。
% 自20世纪60年代以来,水库、堤坝和保护措施等治理措施已经遏制了数千年来高沉积物负荷所困扰的问题
% \cite{wang2016e}。
% 然而,最近出现了新的挑战,如流量减少和水资源枯竭,导致了用水调节和跨流域的水转移——不同的重点水治理策略
% \cite{wang2019c}。
% 目前,还不可能完全解决长江流域水资源压力、生态系统服务之间的权衡、不同区域的不平衡发展问题,使各方都满意
% \cite{wohlfart2016a}。
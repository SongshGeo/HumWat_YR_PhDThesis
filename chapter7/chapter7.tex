\chapter{结论与展望}

\subsection{主要研究结论}

\subsection{创新点}
(1) 本研究建立了“基于利益相关者的人-水关系分析框架”,该框架能实现跨尺度、跨主题的人-水关系分类与演变阶段划分。利用该框架,对近代基于科学研究黄河人-水关系变化的和基于历史材料的人-水关系变化事件进行了定量的结构性梳理,兼具系统性和科学性。
(2)本研究从在黄河流域受关注较少但更具有普适性的“水的资源属性”切入,着重分析人-水资源关系的演化过程。引入“经济复杂性”、“激励扭曲”、“集体记忆”等社会科学理论,利用“合成控制法”、“博弈论”、“复杂网络”等计量手段,通过多学科交叉方案解决人地系统耦合问题。
(3) 本研究通过建立机制模型,找到流域人-水资源关系演化现象的内在机制,并尝试用“机制解释现象,模型预测影响”的思路,通过基于复杂系统的多主体建模来对这一关键机制进行复现,尝试找到人-水资源关系演变中的因果关系,从而为黄河流域高质量发展提出兼具针对性和科学性的政策建议。

\subsection{研究展望}

% 人水关系 匹配 comment
% Our dynamic matching framework tells scientists and decision makers three key points to focus on to achieve sustainability through human-water matching:
我们的动态匹配框架告诉科学家和决策者,通过人与水的匹配实现可持续性需要关注三个关键点:
% 人水关系 匹配 comment
% 1.	First, it is necessary to recognize the importance of human-water matching and develop a coordinated understanding of human-water matching and sustainable development. Although a well known definition is “development that meets the needs of the present without endangering the ability of future generations to meet their needs ”(1987, our Common future), we are often inaccurate about what the present needs and what the future generations need. Therefore, based on our disagreement about what is sustainable, the path to achieve human-water matching is by no means immutable. But similar to the importance of the SUSTAINABLE Development Goals, we still need to establish a universal understanding of human-water matching.
首先,要认识到人水匹配的重要性,对人水匹配与可持续发展形成协调的认识。尽管一个众所周知的定义是“既满足当代人的需求,又不危及后代满足其需求的能力的发展”(1987年,《我们共同的未来》),但我们常常不准确地了解现在的需求和后代的需要。因此,基于我们对什么是可持续的分歧,实现人与水匹配的路径绝不是不可改变的。但与可持续发展目标的重要性类似,我们仍然需要建立对人水匹配的普遍认识。

% 人水关系 匹配 comment
% 2.	Secondly, theoretical methods should be established to observe, explain and predict the relationship between human and water, so as to find new problems on the path to sustainable development. When we have identified a general paradigm for understanding human-water matching, it will guide us in studying which fundamental issues of the catchment system are critical to achieving matching, and help us determine which human-water relationship changes are alarming. The development of new theories and methods to decouple and predict human-water systems is conducive to answering a series of key questions to achieve human-water matching and narrowing the gap between us and watershed sustainability.
其次,建立观察、解释和预测人与水关系的理论方法,从而在可持续发展的道路上发现新的问题。当我们确定了理解人-水匹配的一般范式时,它将指导我们研究流域系统的哪些基本问题对实现匹配至关重要,并帮助我们确定人-水关系的哪些变化令人担忧。人-水系统解耦和预测的新理论和新方法的发展,有利于回答实现人-水匹配的一系列关键问题,缩小我们与流域可持续发展的差距。

% 人水关系 匹配 comment
% 3.	Finally, the impacts of watershed system changes on the realization of human-water matching are evaluated, the research content is adjusted to adapt to the changes, and the transformation of matching paths is made if necessary. As has been emphasized in this paper, the path to realize human-water matching is constantly changing, which will bring new basic problems for research and solution for the decoupling of human-water system, thus giving birth to new theories, methods and even research paradigm. Moreover, when the new human-water relationship exposes many limitations of the previous paradigm, we should actively adjust our understanding of watershed sustainability and achieve the match under the new-man-water relationship through transformation.
3.最后,评估流域系统变化对人水匹配实现的影响,调整研究内容以适应变化,必要时进行匹配路径转换。正如本文所强调的,人水匹配的实现路径是不断变化的,这将为人水系统解耦的研究和解决带来新的基础问题,从而产生新的理论、方法甚至研究范式。此外,当人水关系新模式暴露出以往范式的诸多局限性时,我们应积极调整对流域可持续性的认识,通过转型实现人水关系新模式下的匹配。

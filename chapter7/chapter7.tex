\chapter{结论与展望}

\section{主要研究结论}

% 研究黄河流域人-水关系演变规律、揭示黄河流域人-水关系演变机制,有助于深化对人地关系地域系统结构特征和耦合机理的认识,为协调黄河流域人-水关系、促进高质量发展提供理论框架和科学依据。
% “人-水关系”是一个尺度敏感的概念,代表了利益相关者与水圈要素过程的联系状态,流域是其理想研究单元,流域系统治理是解析其关系演变机制的重要视角。
本研究以黄河流域为研究区,结合水文气象观测、社会经济统计、历史数据重建、遥感反演等多源数据,借助统计分析和模型仿真等手段,解析了黄河流域人-水关系的演变过程和驱动机制。
本研究定量揭示了历史时期和现代治黄时期黄河流域人-水关系演变过程,分析了不同阶段驱动流域人-水关系演变的主要原因。
本研究着眼于对黄河治理转变至关重要的非工程治理,以水资源配额制度为例,分别从自上而下和自下而上的视角解析了流域人-水关系变化的驱动机制,定量评估了制度变化产生的影响、发展了流域人-水关系演变机制分析的因果推断方法、建立了黄河流域社会-生态系统的多主体模型。
本研究的主要结论如下:

(1)厘清水沙特征变化与历史气候的关系,是明晰黄河流域人-水关系是否发生变化的关键。
本研究通过识别黄河流域在过去两千年中的水沙特征变化,总结历史时期人-水关系变化的发生过程。研究指出历史时期黄河流域存水沙变化的在两个湿润气候驱动期(900AD - 1100AD和1700AD - 1900AD)和两个人类活动驱动期(1350AD - 1650AD 和 1900AD迄今)。其中第一个气候驱动期位于“中世纪暖期”(约900AD - 1100AD),此时黄河的水沙变化变化仍由气候因素主导。随后中游黄土高原地区发生农田与人口的快速扩张,不断增加的人为压力与另一次潮湿气候共同推动水沙特征在1700AD - 1900AD越过变化的临界点。上述结果表明人类活动带来的影响最早追溯至1350AD才开始取代气候变化,逐步成为历史时期主导黄河流域人-水关系的主要因素。

(2)本研究通过开发流域水治理综合指数(IWGI)并对其进行突变点检测,量化识别了二十世纪六十年代以来的黄河流域治理的阶段变化。本研究表明当代治黄时期的流域水治理演变过程可划分为三个阶段,并依据其各自特点命名为:集中供水时期(1965 - 1978年)、治理转变时期(1979 - 2001年)、以及适应增强时期(2002 - 2013年)。灌区扩张和水库修建的放缓,是黄河从集中供水时期向治理转变时期过渡的主要特征。在治理转变时期流域的非工程治理措施迅速增加,过渡至适应增强时期后保持稳定,并提升用水效率。经讨论,上述治理模式转变可能在全世界流域系统中普遍存在,而本研究的分析指出水资源供给趋近极限可能是触发上述转变的关键。

(3)在识别当代黄河的治理转变时期($1978$年至$2001$年)的基础上,本研究重点关注1987年的``八七''分水方案和1998年的“流域统一调度”两次自上而下的水资源分配制度变化,分析它们如何重塑了流域系统社会-生态结构,并定量识别其对流域用水的影响。结果表明这两次制度变化以不同的方式重塑了黄河流域的社会-生态系统结构,因而较政策初衷产生不同的治理效果。其中1987年通过的``八七''分水方案违背制度预期地使黄河流域用水量显著增加约$5.75\%$;而1998年参照该分水方案施行流域统一调度之后,大多数省份地区的用水量迅速得到控制,流域总用水在接下来十年间以每年$6.6$亿立方米的速度显著下降,成功治理了长达二十余年的黄河断流问题。

(4)黄河水资源配额制度在1987年与1998年两次变化对流域用水产生的影响,是用水者主体自下而上响应并适应制度变化的表现,深入理解其人-水关系的变化机制需要考虑系统内利益相关者复杂的、差异化的反馈过程。本研究耦合了反映水资源配额制度的人类社会模块与计算三种主要粮食作物(水稻、玉米、小麦)灌溉用水需求的自然模块,发展了黄河流域农业灌溉用水者响应分水制度变化的多主体模型。模型仿真结果表明黄河流域的粮食生产灌溉需求与水资源配额之间存在时空不匹配,配额在1987至1998年间对农业灌溉用水的约束效果不明显;而1998年后,统一调度制度在成功约束农业取水的同时也没有导致大规模的地下水替代性开采。

\section{创新点}

(1)本研究结合因果推断方法,定量识别揭示了黄河流域人-水关系演变的阶段特征,重点解析了过去常常被忽视的非工程治理措施对人-水关系演变的驱动作用。

(2)本研究开发了黄河流域社会-生态系统的多主体模型,耦合了水资源分配的社会过程与作物灌溉需水的自然过程,自下而上解析了农业主体对制度变化的响应及其用水效应。

\section{研究展望}

由于数据来源和研究方法的限制,本论文不可避免地存在一些局限性,可供未来研究做进一步探讨,例如:
(1)第二章的历史时期研究仍无法明确人-水关系稳态的弹性范围,需要发展弹性评估方法,建立系统动态模型并量化流域系统的弹性,分析稳态无法继续维系的原因。
(2)第三章开发的综合水治理指数依赖于大量实测和统计数据,因此其应用范围受到数据可获得性的限制。未来可以在全世界流域对子指标进行简化,应用指数构建的框架,分析黄河的人-水关系演变过程是否是流域系统的共性规律。
(3)第四章利用社会-生态网络的概念刻画了流域社会-生态系统的结构块,但没能更细致具体地量化其网络结构,未来可利用大数据和网络分析方法量化分析其结构特征,定量推断流域社会-生态系统结构和功能之间的联系。
(4)第五章使用多主体模型探讨了农业用水主体对流域水额分配制度变化的响应,但尚与土壤水、地下水补给、产汇流等水文过程耦合不够紧密。未来可以结合分布式水文模型和系统动力学模型,建立更全面的流域社会-生态系统耦合模型,实现对不同制度情景下黄河流域发展路径的预测。

但本研究以黄河为例在人-水关系变化上开展了系统研究,这个典型的人类活动主导的流域凭其复杂的演变过程与演变机制向科学家和决策者证明:流域治理需要深入解析复杂的人-水关系,未来实现人水和谐与流域可持续仍需强调以下三方面:

(1)深化对人水匹配重要性的认识。“流域可持续”的定义是“既满足当代人的需求,又不损害后代人满足其需求的能力”,但常常难以准确地了解现在的需求和后代的需要。因此,保证人类系统与水圈的结构匹配非常重要,因为未来的水治理应实现的目标由利益相关者共同决定。

(2)建立观察、解释、和预测人-水关系的理论与模型方法。经验观察将指导流域系统的哪些变化对治理至关重要,任何预测流域人-水系统变化的新理论、新方法,都有利于流域人-水关系和谐。本研究表明若不理解复杂的人-水关系反馈过程,制度可能带来违背预期的结果,因此需要开发耦合人类社会过程的模型。

(3)评估流域系统人-水关系变化对可持续发展的影响。人类世的流域社会-生态环境变化迅速,需要不断调整治理策略进行适应,并在必要时推动治理转型。由于流域治理措施的结果常难以预料,当其呈现出违背预期的表现时,应积极调整对流域的治理策略,必要时通过转型来促进流域系统的人-水关系和谐。

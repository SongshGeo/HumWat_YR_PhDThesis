\chapter{结论与展望}

\section{主要研究结论}

“人-水关系”是一个存在尺度敏感性的概念,代表了利益相关者与水圈要素过程的联系状态。本研究将其定义为:“在流域尺度下,主导人水复杂系统“自然-社会二元循环结构”的利益相关者与地球表层系统中水圈要素过程的联系状态”,人-水关系变化是流域系统反馈循环模式的重组,稳态转换是识别这种变化的重要理论基础,识别稳态转换驱动力是理解作用机制的关键。

本研究针对流域这个特定的人地关系地域系统(或社会-生态系统),本研究从流域治理的角度出发,识别了黄河人-水关系的变化过程及其机制,具体包含以下主要研究结论:

(1)$900 \sim 1100AD$的中世纪暖期气候仍主导黄河水沙特征变化;随后黄土高原农田与人口快速扩张;$1700 \sim 1900AD$ 的潮湿期结束后,各指标没有回退且显著增强,标志着人类驱动超越气候影响并主导流域水沙特征,改变了人水关系。

(2)$1965$至$2013$年的黄河流域水治理演变历史划分为明显的三个阶段,依据其各自特点可命名为:集中供水时期($1965 \sim 1978$)、治理转变时期($1979 \sim 2001$)、适应增强时期($2002 \sim 2013$);其中水资源供给逼近极限是触发治理转变的关键。

(3)$1987$年的“八七”分水方案违背制度预期地促使黄河流域用水量显著增加约$5.75\%$;$1998$年流域统一调度后,大多数区域用水量迅速得到控制,流域总用水以每年$6.6$亿立方米的速度显著下降,结束了长达二十余年的黄河断流。

(4)黄河流域水资源配额与灌溉需求之间仍存在时空不匹配,多主体模型也预测部分省份倾向于违背配额制度。农业主体响应分水制度的方式存在区域差异;但蓝/绿水的比例整体呈下降趋势,说明分水制度的主要作用方式是促进农业节水转型。


\section{创新点}

(1) 本研究建立了“基于利益相关者的人-水关系分析框架”,该框架能实现跨尺度、跨主题的人-水关系分类与演变阶段划分。利用该框架,对近代基于科学研究黄河人-水关系变化的和基于历史材料的人-水关系变化事件进行了定量的结构性梳理,兼具系统性和科学性。

(2)本研究从在黄河流域受关注较少但更具有普适性的“水的资源属性”切入,着重分析人-水资源关系的演化过程。引入“经济复杂性”、“激励扭曲”、“集体记忆”等社会科学理论,利用“合成控制法”、“博弈论”、“复杂网络”等计量手段,通过多学科交叉方案解决人地系统耦合问题。

(3) 本研究通过建立机制模型,找到流域人-水资源关系演化现象的内在机制,并尝试用“机制解释现象,模型预测影响”的思路,通过基于复杂系统的多主体建模来对这一关键机制进行复现,尝试找到人-水资源关系演变中的因果关系,从而为黄河流域高质量发展提出兼具针对性和科学性的政策建议。

\section{研究展望}

我们的动态匹配框架告诉科学家和决策者,通过人与水的匹配实现可持续性需要关注三个关键点:

首先,要认识到人水匹配的重要性,对人水匹配与可持续发展形成协调的认识。尽管一个众所周知的定义是“既满足当代人的需求,又不危及后代满足其需求的能力的发展”(1987年,《我们共同的未来》),但我们常常不准确地了解现在的需求和后代的需要。因此,基于我们对什么是可持续的分歧,实现人与水匹配的路径绝不是不可改变的。但与可持续发展目标的重要性类似,我们仍然需要建立对人水匹配的普遍认识。

其次,建立观察、解释和预测人与水关系的理论方法,从而在可持续发展的道路上发现新的问题。当我们确定了理解人-水匹配的一般范式时,它将指导我们研究流域系统的哪些基本问题对实现匹配至关重要,并帮助我们确定人-水关系的哪些变化令人担忧。人-水系统解耦和预测的新理论和新方法的发展,有利于回答实现人-水匹配的一系列关键问题,缩小我们与流域可持续发展的差距。

3.最后,评估流域系统变化对人水匹配实现的影响,调整研究内容以适应变化,必要时进行匹配路径转换。正如本文所强调的,人水匹配的实现路径是不断变化的,这将为人水系统解耦的研究和解决带来新的基础问题,从而产生新的理论、方法甚至研究范式。此外,当人水关系新模式暴露出以往范式的诸多局限性时,我们应积极调整对流域可持续性的认识,通过转型实现人水关系新模式下的匹配。

(1)本研究揭示了黄河流域历史时期长时间尺度的人水关系演变过程,阐明了人类影响逐步超越气候周期,主导黄河水沙特征的历史过程。
但历史时期人水关系稳态的弹性范围尚不明确,需要发展弹性评估方法,建立系统动态模型并量化流域系统的弹性,分析响应洪灾的关系无法继续维系稳态的原因。

(2)揭示了过去六十余年间全面综合治理黄河的人水关系稳态变化过程,指出水资源供给达到极限时是流域治理转型的关键阶段。
在全球各大流域应用本研究开发的综合治理指数,分析黄河的人水关系演变过程是否是流域系统的共性规律


(3)利用社会-生态网络概念自上而下刻画流域社会-生态系统结构,使用差分合成控制法定量识别制度变迁的净效应,指出结构匹配的重要性。利用大数据和网络分析方法具体刻画社会-生态结构特征,定量推断社会-生态系统网络结构和功能之间的联系

(4)使用多主体模型探讨了的农业引水决策对流域社会-生态变化的响应,自下而上分析了人水关系变化机制。建立流域社会-生态系统耦合模型,结合分布式水文模型和系统动力学模型,实现对不同制度情景下黄河流域发展路径的预测。

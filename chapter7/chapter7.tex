\chapter{结论与展望}

\subsection{主要研究结论}

\subsection{创新点}
(1) 本研究建立了“基于利益相关者的人-水关系分析框架”,该框架能实现跨尺度、跨主题的人-水关系分类与演变阶段划分。利用该框架,对近代基于科学研究黄河人-水关系变化的和基于历史材料的人-水关系变化事件进行了定量的结构性梳理,兼具系统性和科学性。
(2)本研究从在黄河流域受关注较少但更具有普适性的“水的资源属性”切入,着重分析人-水资源关系的演化过程。引入“经济复杂性”、“激励扭曲”、“集体记忆”等社会科学理论,利用“合成控制法”、“博弈论”、“复杂网络”等计量手段,通过多学科交叉方案解决人地系统耦合问题。
(3) 本研究通过建立机制模型,找到流域人-水资源关系演化现象的内在机制,并尝试用“机制解释现象,模型预测影响”的思路,通过基于复杂系统的多主体建模来对这一关键机制进行复现,尝试找到人-水资源关系演变中的因果关系,从而为黄河流域高质量发展提出兼具针对性和科学性的政策建议。

\subsection{研究展望}

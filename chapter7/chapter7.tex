\chapter{结论与展望}

\section{主要研究结论}

\section{创新点}
(1) 本研究建立了“基于利益相关者的人-水关系分析框架”,该框架能实现跨尺度、跨主题的人-水关系分类与演变阶段划分。利用该框架,对近代基于科学研究黄河人-水关系变化的和基于历史材料的人-水关系变化事件进行了定量的结构性梳理,兼具系统性和科学性。
(2)本研究从在黄河流域受关注较少但更具有普适性的“水的资源属性”切入,着重分析人-水资源关系的演化过程。引入“经济复杂性”、“激励扭曲”、“集体记忆”等社会科学理论,利用“合成控制法”、“博弈论”、“复杂网络”等计量手段,通过多学科交叉方案解决人地系统耦合问题。
(3) 本研究通过建立机制模型,找到流域人-水资源关系演化现象的内在机制,并尝试用“机制解释现象,模型预测影响”的思路,通过基于复杂系统的多主体建模来对这一关键机制进行复现,尝试找到人-水资源关系演变中的因果关系,从而为黄河流域高质量发展提出兼具针对性和科学性的政策建议。


% % 开题报告
% (1)黄河流域历史人-水关系演变过程的主要阶段。
% 描述人-水关系及其演变是一个复杂的问题,试图寻找某个指标、建立指标体系或者研制出某种定量数学模型描述黄河人水关系也是困难的。此外, 黄河各河段自然和社会经济状况相差很大,很难对各河段的人-水关系进行统一的描述。但是,为不同主题、不同河段之间建立一个较为统一的分析框架是可能的。利用统一的分析框架对不同主题(如水资源利用、水旱灾防治等)、不同河段(源区、上游、中游、下游)进行综合分析后,可以对黄河流域历史人-水关系演化过程的主要阶段进行粗略划分。

% (2)大河流域人-水资源互馈关系变化的核心机制。
% 人类和生态系统的健康都是以液态水的存在为前提的,而人作为陆生动物严重依赖于地表流域的水资源。因此人-水关系中,为了获取、竞争、保护这种宝贵的自然资源,人类社会与水资源形成了复杂的互馈关系。理解互馈关系是人-水系统模型建立的基础,而模型的建立则是预测未来的保障。在人-水关系发生演变的过程中,人与水资源的互馈关系也发生重大变化, 因此找到这种重大变化的发生机制并利用模型进行解释和模拟,再进一步结合未来情景进行预测。


% 创新性:
% (1)通过复杂网络分析方法揭示流域人地系统的结构特征,定量分析结构变化对水沙、生态和经济的影响。
% (2)构建链接水文过程模型与主体模型的流域人地系统耦合模型,识别并分析人地匹配网络结构特征,阐释多主体协同过程形成机制与调控路径。

\section{研究展望}

% 人水关系 匹配 comment
% Our dynamic matching framework tells scientists and decision makers three key points to focus on to achieve sustainability through human-water matching:
我们的动态匹配框架告诉科学家和决策者,通过人与水的匹配实现可持续性需要关注三个关键点:
% 人水关系 匹配 comment
% 1.	First, it is necessary to recognize the importance of human-water matching and develop a coordinated understanding of human-water matching and sustainable development. Although a well known definition is “development that meets the needs of the present without endangering the ability of future generations to meet their needs ”(1987, our Common future), we are often inaccurate about what the present needs and what the future generations need. Therefore, based on our disagreement about what is sustainable, the path to achieve human-water matching is by no means immutable. But similar to the importance of the SUSTAINABLE Development Goals, we still need to establish a universal understanding of human-water matching.
首先,要认识到人水匹配的重要性,对人水匹配与可持续发展形成协调的认识。尽管一个众所周知的定义是“既满足当代人的需求,又不危及后代满足其需求的能力的发展”(1987年,《我们共同的未来》),但我们常常不准确地了解现在的需求和后代的需要。因此,基于我们对什么是可持续的分歧,实现人与水匹配的路径绝不是不可改变的。但与可持续发展目标的重要性类似,我们仍然需要建立对人水匹配的普遍认识。

% 人水关系 匹配 comment
% 2.	Secondly, theoretical methods should be established to observe, explain and predict the relationship between human and water, so as to find new problems on the path to sustainable development. When we have identified a general paradigm for understanding human-water matching, it will guide us in studying which fundamental issues of the catchment system are critical to achieving matching, and help us determine which human-water relationship changes are alarming. The development of new theories and methods to decouple and predict human-water systems is conducive to answering a series of key questions to achieve human-water matching and narrowing the gap between us and watershed sustainability.
其次,建立观察、解释和预测人与水关系的理论方法,从而在可持续发展的道路上发现新的问题。当我们确定了理解人-水匹配的一般范式时,它将指导我们研究流域系统的哪些基本问题对实现匹配至关重要,并帮助我们确定人-水关系的哪些变化令人担忧。人-水系统解耦和预测的新理论和新方法的发展,有利于回答实现人-水匹配的一系列关键问题,缩小我们与流域可持续发展的差距。

% 人水关系 匹配 comment
% 3.	Finally, the impacts of watershed system changes on the realization of human-water matching are evaluated, the research content is adjusted to adapt to the changes, and the transformation of matching paths is made if necessary. As has been emphasized in this paper, the path to realize human-water matching is constantly changing, which will bring new basic problems for research and solution for the decoupling of human-water system, thus giving birth to new theories, methods and even research paradigm. Moreover, when the new human-water relationship exposes many limitations of the previous paradigm, we should actively adjust our understanding of watershed sustainability and achieve the match under the new-man-water relationship through transformation.
3.最后,评估流域系统变化对人水匹配实现的影响,调整研究内容以适应变化,必要时进行匹配路径转换。正如本文所强调的,人水匹配的实现路径是不断变化的,这将为人水系统解耦的研究和解决带来新的基础问题,从而产生新的理论、方法甚至研究范式。此外,当人水关系新模式暴露出以往范式的诸多局限性时,我们应积极调整对流域可持续性的认识,通过转型实现人水关系新模式下的匹配。

% % 人水匹配 comment mydoc.docx
% 作为实现匹配的一个重要途径,制度分析认为,权利、规则和决策等制度可以引起或解决人与环境互动中的问题。因此,建立匹配的制度可以引导系统功能向理想的结果发展。
% 研究表明,制度匹配在很多方面有利于人类水系统的可持续性。
% 例如,设立流域管理机构可以有效避免水事纠纷,建立水权转换制度有利于资源的有效配置,加强监管可以遏制河流污染。
% 因此,许多大流域的综合治理是建立一套制度的尝试,它以实现人水匹配的一系列制度为核心,包括与之密切相关的文化和技术要素,从而保证流域的可持续发展。
% 这样一个体系的建立庞大而重要,需要自然科学和社会科学的交叉,解耦和理解人水系统中复杂的反馈回路,从而通过制度分析实现人水匹配(图1,PATH 1和路径2)。

% 制度分析往往是在特定背景下进行的,因此,人水关系的动态变化使得宏大的人水系统匹配更加复杂。
% 首先,动态变化是人类社会系统和自然水文系统的共同特征,但只有当关键变量达到临界点时,系统功能的稳态转变才会发生。
% 需要匹配的关键系统功能往往是由流域可持续性的需要决定的,这为理解复杂系统动态的观点提供了价值判断(图1,路径3)。
% 另一方面,当一些可能导致系统稳态转变的变化发生时,也需要重新审视如何在新的人水关系下实现匹配。因此,这些超过阈值的变化可能成为成功过渡到人水匹配的关键驱动力(图1,PATH 4)。
% 然而,实现上述动态匹配还需要对人水系统有更高层次的理解,因为这些动态变化为系统解耦提供了新的科学难题,如变化的弹性、变化之间的因果关系、稳态转化及其级联效应等(图1,PATH 5)。
% 只有通过深度解耦和对反馈循环的理解,才能对人类与水系统的动态进行预测(图1,PATH 6),促进机构分析和过渡匹配(图1,路径1和路径3)。
% 一般来说,人-水系统的反馈循环和动态变化是人-水匹配的基础。由于三者是相互联系的,面对不断变化的流域系统,人水匹配需要动态实现。
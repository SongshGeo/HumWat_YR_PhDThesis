\chapter{结论与展望}

\section{主要研究结论}

% 研究黄河流域人水关系演变规律、揭示黄河流域人水关系演变机制,有助于深化对人地关系地域系统结构特征和耦合机理的认识,为协调黄河流域人水关系、促进高质量发展提供理论框架和科学依据。
% “人-水关系”是一个尺度敏感的概念,代表了利益相关者与水圈要素过程的联系状态,流域是其理想研究单元,流域系统治理是解析其关系演变机制的重要视角。
本研究以黄河流域为研究区,结合水文气象观测、社会经济统计、历史数据重建、遥感反演等获取多源数据,借助统计分析和模型模拟等手段,分析黄河流域人水关系的演变过程和驱动机制。
本研究分别在历史时期和现代治黄时期,定揭示了黄河流域人水关系演变过程、分析了不同阶段驱动流域人水关系演变的主要原因。接下来着眼于对现代黄河治理转变意义重大的制度变化驱动力,以对黄河流域影响最为深远的水资源分配制度为例,分别从自上而下和自下而上两个路径分析流域人水关系变化的驱动机制。通过发展流域人水关系演变机制分析的因果推断方法,建立黄河流域社会-生态系统的多主体模型,定量评估制度变化产生的影响。具体主要研究结论如下:

(1)历史时期的黄河流域水沙变化存在两个湿润气候驱动期($900AD\sim1100AD$和$1700AD\sim1900AD$)和两个人类活动驱动期($1350AD \sim 1650AD$ 和 $1900AD$迄今)。其中第一个气候驱动期也被称为中世纪暖期(约$900AD \sim 1100AD$),此时期的综合指标变化仍由气候主导。随后黄土高原农田与人口快速扩张,不断增加的人为压力与潮湿气候共同推动水沙变化在$1700AD \sim 1900AD$的第二个潮湿期已越过临界点,说明人类活动压力最早在$1350AD$后开始取代周期波动的气候,逐步成为主导黄河流域人-水关系的驱动因素。

(2)当代治黄时期,黄河流域的水治理演变过程可划分为三个阶段,依据其各自特点可命名为:集中供水时期($1965 \sim 1978$)、治理转变时期($1979 \sim 2001$)、适应增强时期($2002 \sim 2013$)。灌区扩张和经济增长是推动前两个阶段变化的直接原因,环境背景、社会文化、水治理政策等因素在各阶段都存在一定影响,同时也是驱动流域水治理由第二阶段向第三阶段转变的主要驱动力。这样的治理转变模式在全球流域系统中普遍存在,水资源供给逼近极限可能是触发转变的关键。

(3)黄河流域的地表水资源分配制度两经变化,是上述治理转变期中影响深远的典型政策干预,重塑了黄河流域的社会-生态系统结构,并产生了截然不同的治理效果。其中$1987$年的“八七”分水方案违背制度预期地促使黄河流域用水量显著增加约$5.75\%$;而$1998$年流域统一调度后,大多数区域用水量迅速得到控制,流域总用水以每年$6.6$亿立方米的速度显著下降,结束了长达二十余年的黄河断流。

(4)黄河流域水资源分配制度所规定的地表水额度与流域三种主要粮食作物(水稻、玉米、小麦)的灌溉需求之间存在时空不匹配。通过发展黄河流域粮食灌溉响应分水制度的多主体模型,发现了限制地表取水的水资源分配制度没有直接导致大规模的地下水替代性开采,节水转型是各地区响应制度变化的主要方式。

\section{创新点}

(1) 本研究提出了流域系统人-水关系的演变分析框架,定量识别并划分了黄河流域人水关系演变的主要阶段与过程,重点解析了过去常常被忽视的流域非工程因素驱动人水关系变化的机制。

(2) 本研究发展了流域人水关系演变机制分析的因果推断方法,建立黄河流域社会-生态系统的多主体模型,明晰了推动人水关系变化的关键制度机制,定量评估其产生的影响。

\section{研究展望}

由于数据来源和研究方法的限制,本论文不可避免地存在一些局限性,在未来的研究中可以作进一步深入探讨,例如:
(1)第三章的历史时期研究仍无法明确人水关系稳态的弹性范围,需要发展弹性评估方法,建立系统动态模型并量化流域系统的弹性,分析响应洪灾的关系无法继续维系稳态的原因。
(2)第四章开发的指数需要太多数据做支撑,因此限制了作用范围。未来可以在全球各大流域应用本研究开发的综合治理指数框架,分析黄河的人水关系演变过程是否是流域系统的共性规律。
(3)第五章利用社会-生态网络概念自上而下刻画流域社会-生态系统的抽象结构,但没有办法具体更细致地量化制度结构,未来可以利用大数据和网络分析方法具体刻画社会-生态结构特征,定量推断社会-生态系统网络结构和功能之间的联系。
(4)第六章使用多主体模型探讨了的农业引水决策对流域社会-生态变化的响应,自下而上分析了人水关系变化机制,但模型的模拟范围仅限于农业领域的三种主要作物,和除了降水、蒸散发之外的水文过程联系也不够紧密。未来可以建立更全面的流域社会-生态系统耦合模型,结合分布式水文模型和系统动力学模型,实现对不同制度情景下黄河流域发展路径的预测。

但本研究以黄河为例在人水关系变化上开展了系统研究,这个典型的人类活动主导的流域通过其复杂的演变与机制向科学家和决策者证明:流域治理需要深入认识复杂的人-水关系,未来通过人与水的匹配实现可持续性需要重点强调以下三个方面。

(1)要认识到人水匹配的重要性,对人水匹配与可持续发展形成一致的认识。众所周知的“流域可持续”定义是“既满足当代人的需求,又不危及后代满足其需求的能力的发展”,但我们常常难以准确地了解现在的需求和后代的需要。此时保证人类系统与水圈的结构匹配便显得非常重要,因为只有利益相关者明白,未来对他们意味着什么。

(2)亟需建立观察、解释和预测人与水关系的理论方法,从而在可持续发展的道路上发现新的问题。在确定理解人-水匹配的一般范式后,经验观察将指导观测流域系统的哪些变化对治理至关重要。任何解耦和预测流域人-水系统的新理论、新方法,都有利于流域可持续发展。

(3)评估流域系统人水关系变化对实现可持续发展的影响,调整治理策略以适应变化,并在必要时推动流域治理的转型。人水关系复杂多变,治理实践的结果也难以预料,当人水关系新模式暴露出以往范式的诸多局限性时,我们应积极调整对流域可持续性的认识,通过必要的治理转变推动人水关系在新阶段下实现匹配。

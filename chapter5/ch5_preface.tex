第四章研究表明,在人类活动主导社会-水循环的近百年,黄河流域的水治理模式发生了xxx的转型。
黄河流域的转型过程既有中央决策者自上而下的制度改革,也包含利益相关者自下而上对有限水资源的开发、交易、与利用。
为了深入理解上述转型过程,本章将分析自上而下的制度改革对黄河流域产生的影响及其机制。

流域系统的水治理,通过政策、法律和规范等制度来重塑其SES结构。
% 制度代表了所有相关的治理实践,包括社会行动者、生态单元或社会和生态系统要素之间的相互作用
% \cite{lien2020, bodin2017b}。
% 了解这些复杂的相互作用对于制定有效管理自然资源和增强社会-生态系统恢复力的战略至关重要\cite{kluger2020}。
有效的(或“匹配”“合适”)制度在适当的空间、时间和功能尺度上运作,以管理和平衡人类与水系统之间的不同关系和相互作用,支持(但不保证)SES的可持续性\cite{epstein2015, wang2019d}。
一些制度上的进步已经取得了良好的水治理成果(例如,中国黑河流域的生态调水工程\cite{wang2019d},以及欧洲的协同水治理系统\cite{green2013})。
然而,在一个大而复杂的流域上强加制度变化可能会建立或破坏社会主体和生态单元之间的联系,在那里匹配的社会-生态结构并不普遍。
% 现有制度匹配知识的两个特别弱点包括理解(图~\ref{fig:framework}~\textbf{b}):(i)社会经济体系结构和结果之间的因果联系;(ii)由于制度上缺乏匹配而导致的基本过程的细节,特别是不同参与者的激励。
% 这些弱点限制了对制度设计的理解,阻碍了对制度匹配的研究。

% 为了更好地理解水治理机构如何匹配或不匹配其社会-生态环境,我们以中国黄河流域(YRB)为例(见\textit{\nameref{sec:yrb}}),深入研究SES结构和结果之间的因果关系。
% 具体来说,我们关注了长江流域水资源分配的两个制度转变:1987年的水资源分配方案(87-WAS)和1998年的统一流域条例(98-UBR),后者极大地重新构建了SES结构。
% YRB提供了一个信息丰富的案例,主要有两个原因:
% (1)自上而下的制度变迁导致了社会经济体系结构的急剧变化,使我们能够定量估计其净影响。
% (2)由于很少有大型河流流域经历过一次以上这种根本性的制度转变,本案例研究为理解东南流域结构变化对自然资源的影响提供了可比性的自然实验。

本章利用黄河流域的准自然实验探索了中央政府引领的制度变迁如何改变流域社会-生态系统结构,并最终导致了不同的水治理结果。
具体来说,我们关注了长江流域水资源分配的两个制度转变:1987年的水资源分配方案(87-WAS)和1998年的统一流域条例(98-UBR),两者都极大地重新构建了SES结构。
首先,我们使用了官方文件在两次制度转移后的变化数据,通过将它们抽象为SES结构主题(或构建块)来描述1979年至2008年与YRB相关的SES结构的可比较变化。
然后,我们使用了一种称为“差异综合控制(DSC)”的方法\cite{arkhangelsky2021},该方法考虑了经济增长和自然背景,在没有制度转变的情况下估算理论用水情景。
这种方法使我们能够创建一个反事实,以此来探索黄河流域社会-生态系统结构和结果之间的联系机制,从而更深入地理解机构在全球水治理中的潜在作用。
% 最后,我们进一步开发了一种边际效益分析方法,以SE s结构(\textit{\nameref{sec:model}})为基础来解释不匹配制度的潜在过程。

第四章研究表明,在人类活动主导黄河流域社会-水循环的近百年,社会-经济的水资源需求超越临界点推动其水治理模式发生了深刻转型。
转型过程中既包含利益相关者自下而上对有限水资源的开发、交易、与利用,也存在中央决策者自上而下的制度改革,两者共同触发了当代黄河的人-水关系演变。
本章针对水资源分配制度这一典型的自上而下的制度改革进行研究,分析其对黄河流域产生的影响。

通过政策、法律和规范等制度来重塑社会-生态系统结构是流域人水关系变化的重要原因之一,影响了所有相关社会行动者、生态单元或社会和生态系统要素之间的相互作用\cite{lien2020, bodin2017b}。
% 了解这些复杂的相互作用对于制定有效管理自然资源和增强社会-生态系统恢复力的战略至关重要\cite{kluger2020}。
有效的(或“匹配”的,``fit'')制度在特定的时空和功能尺度上运作,能够平衡人类与生态系统之间的相互作用,支持社会-生态系统的可持续性\cite{epstein2015, wang2019d}。
一些制度上改革已被证明取得了良好的水治理成果,如中国黑河流域的生态调水工程\cite{wang2019d},以及欧洲的协同水治理系统\cite{green2013})。
这种匹配的社会-生态结构并不普遍,因为一个复杂的大河流域推动制度变化会建立或破坏众多社会主体和生态单元之间的联系,产生超乎预料的长期影响。
% 现有制度匹配知识的两个特别弱点包括理解(图~\ref{fig:framework}~\textbf{b}):(i)社会经济体系结构和结果之间的因果联系;(ii)由于制度上缺乏匹配而导致的基本过程的细节,特别是不同参与者的激励。
% 这些弱点限制了对制度设计的理解,阻碍了对制度匹配的研究。

% 为了更好地理解水治理机构如何匹配或不匹配其社会-生态环境,我们以中国黄河流域(YRB)为例(见\textit{\nameref{sec:yrb}}),深入研究SES结构和结果之间的因果关系。
% 具体来说,我们关注了长江流域水资源分配的两个制度转变:1987年的水资源分配方案(87-WAS)和1998年的统一流域条例(98-UBR),后者极大地重新构建了SES结构。
% YRB提供了一个信息丰富的案例,主要有两个原因:
% (1)自上而下的制度变迁导致了社会经济体系结构的急剧变化,使我们能够定量估计其净影响。
% (2)由于很少有大型河流流域经历过一次以上这种根本性的制度转变,本案例研究为理解东南流域结构变化对自然资源的影响提供了可比性的自然实验。

本章利用过去四十年间黄河流域的制度变迁设计准自然实验,分析自上而下的水资源分配制度改革如何重塑流域社会-生态系统结构,产生不同的治理效果并最终影响了流域可持续性。
具体来说,我们关注了1987年的水资源分配改革(“八七”分水方案)和1998年的“流域统一调度”,两者都曾极大地重构了黄河流域的社会-生态系统联系。
我们首先通过抽象社会-生态系统构件(building blocks)的方法,从官方文件中勾勒出两次制度前后的系统结构变化。
接下来,使用名为“差分合成控制法(Differenced Synthetic Control, DSC)”的因果推断方法\cite{arkhangelsky2021},考虑经济增长和气候条件的变化构建“没有发生制度转变”的反事实情景,估算此情景下的理论用水量,分析上述系统结构变化和用水量间的因果联系。
最后,我们基于不同时期的流域社会-生态系统结构提出了一个边际效益分析模型,对上述估算的用水量变化做出潜在机制解释。

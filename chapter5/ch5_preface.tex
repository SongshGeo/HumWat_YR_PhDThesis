% 第四章研究表明,近百年来,人类活动对黄河流域社会\textendash{}水文循环的主导,使得社会\textendash{}经济的水资源需求超出了临界点,从而导致其水治理模式发生了深刻的转型。这个转型过程既包含了从下到上的利益相关者对有限水资源的开发、交易和利用,也包含了从上到下的中央决策者的制度变化。两者共同促使了当代黄河的人\textendash{}水关系的演变。

% 本章专门研究了水资源分配制度这一典型的从上到下的制度变化,分析了它对黄河流域产生的影响。
% 通过政策、法律和规范等制度来重塑社会\textendash{}生态系统结构,是导致流域人\textendash{}水关系变化的重要原因之一,它影响了所有相关社会行动者、生态单元或社会和生态系统要素之间的相互作用\cite{lien2020, bodin2017b}。
% 有效的(或“匹配”的)制度能够在特定的时空和功能尺度上运作,平衡人类与生态系统之间的相互作用,支持社会\textendash{}生态系统的可持续性\cite{epstein2015, wang2019c}。
% 已有证据表明,制度上的改良能够取得良好的水治理成果,如中国黑河流域的生态调水工程\cite{wang2019c},以及欧洲的协同水治理系统\cite{green2013})。
% 但社会\textendash{}生态结构的匹配不是很普遍,因为在复杂的大河流域中,推动制度变化会建立或破坏很多社会主体和生态单元之间的关系,从而导致意想不到的长期影响。

本章通过$1980$至$2010$年间黄河流域的制度变迁设计了一个准自然实验,分析了自上而下的水资源分配制度变化如何重塑流域社会\textendash{}生态系统结构,产生的不同治理效果以及对流域可持续性的影响。
本研究关注了1987年的``八七''分水方案和1998年的“流域统一调度”,通过抽象社会\textendash{}生态系统构件的方法,从官方文件中勾勒出了制度前后的系统结构变化。
接下来,本章使用主成分分析法对影响区域用水量的经济增长、基础设施建设、和气候条件等特征进行降维;最后使用差分合成控制法(DSC)\cite{arkhangelsky2021}估算了没有发生制度转变的理论用水量,分析了系统结构变化和用水量间的因果联系。

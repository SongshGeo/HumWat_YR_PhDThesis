% discussion-1:
% 用水量的上升、下降-结果解读
% 制度对社会生态系统的结果产生影响在世界范围内都很普遍,
\section{自上而下的制度驱动机制}

\subsection{违背预期的分水制度}

制度对社会-生态系统结果的影响在世界范围内被广泛报道,但很少有人试图量化其净效应\cite{cumming2020a}。
本章研究结果表明,$1998$年的流域统一调度如预期那般降低了黄河流域的用水量。
这与许多前人研究的结论类似,即$1998$年流域统一调度制度减少了地表水竞争,也是导致黄河从断流中恢复的关键\cite{chen2021,huangang2002,an2007}。
但是$1987$年的“八七”分水方案却远远偏离预期,在制度出台后的十年内反而促使流域总用水量增加了$5.75\%$。
之前的分析和评论中通常认为“八七”分水方案的“效果不明显”,但本文的结果对这一结果对提出了挑战,因为黄河流域的实际用水与合成用水之间存在显著差异,且实际用水高于模型预测\cite{abadie2015,hill2021},这表明即使控制了环境和经济变量,“八七”分水方案的政策干预还会导致流域用水量增加。
与许多其他社会生态系统治理失败的情况相似,“八七”分水方案在$1987 \sim 1998$年间结果与目标的背道而驰,再一次证明了不匹配的社会生态结构会加剧公共资源的的掠夺式开采\cite{kellenberg2009,cai2016,barnes2019}。
近年来,全国范围内开始实施类似的水配额政策,这改变了社会行动者与资源单位之间的关系。由于本研究中的两种制度变革都在社会-生态系统内部引发了意料之外的变化或级联效应,所以更好的管理需要对耦合的人类和自然系统进行更多的制度分析。

% 实证研究表明,在社会经济系统中,这种广泛存在的构建模块是结构功能的关键。基于网络的方法是将实体之间的连接抽象为链接和节点\cite{bodin2017a,kluger2020,guerrero2015}。
%
% discussion-2: 87的增加
% 结果与制度的目的完全相反的“八七”分水方案与许多其它的SES失败治理案例类似,表明不匹配的社会生态结构可能促进了对公共资源的掠夺式开采。
\subsection{制度的结构匹配机制}

在“八七”分水方案实施后,用水量的增加与一些省份频繁争夺水资源的担忧是一致的,其效果不理想的原因已经得到广泛的讨论(如执行力度和可行性不足、分水方案有失公平、忽视经济发展等),但仍缺乏对流域社会-生态结构的关注。
各省面对制度变迁的响应差异表明,“八七”分水方案实施后的十年间各省的用水量的增减,与其分水方案提出时的当前用水量存在正相关(图\ref{fig:regulating}),这种“用水量大的人倾向于使用更多的水资源”的模式支持这样的假设:独立的利益相关者(如个别省份)将通过最大化效用来适应制度变化。
这一理论分析得到了两个事实的支持:
(1) “八七”分水方案的配额在敲定前,在1982年到1987年间经历了数次会议的漫长讨论,各个利益相关者(即各个省份)试图展示在未来的发展潜力,证明他们需要更多水资源配额,即“讨价还价”的过程\cite{wang2019a, wang2019d}。
这种“讨价还价”过程是水资源配额与经济预期之间的匹配,但可分配水量严重不足的客观条件决定最后不得不由自上而下的制度干预确定配额。
有研究便指出,如果只考虑经济潜力的话,当时主要用水大省(如山东和河南)需要的水资源一定会远远超出了他们最终争得的配额\cite{zuo2020}。
(2) 由于决策者与利益相关者之间存在信息不对称,以当前用水量作参考分配未来的用水量是可行的方案,因此目前用水量较高的省份在水资源分配中具有更强的“议价”能力。
因此,用水量较大的利益相关者有更大的动机来防止水配额阻碍他们的经济发展,这一点也可以从他们$1987$年配额敲定后,仍不断向中央政府要求更多水资源的游说中得到应证\cite{wang2019a, wang2019d}。

% 这种机制还得到

相比之下,在$1998$年实施流域统一调度之后,由黄河水利委员会开始协调各利益相关者的黄河干流水资源分配,以简单直接的方式限制了各省黄河干流地表水的使用,许多证据表明这一制度转变可能正导致更广泛的级联效应。
例如,由于各省自主引水并请求提高配额的动力大大降低,内部节水创新的动力大大提升,以提高用水效率(如大量增设节水灌溉设备),表明这种流域尺度的制度匹配对于节约用水是重要的\cite{krieger1955, ostrom1990}。
又如流域统一调度后,许多灌区由于水需求无法得到地表水满足,导致地下水取水量增加,黄河流域成为地下水下降的热点\cite{sun2022b},但因少有关于地下水使用的合格数据,难以用数据分析的方法开展进一步研究。

本章利用的社会-生态系统结构模式(图~\ref{fig:structure}~C)在全球范围内的其他社会-生态系统中也有报道。
流域统一调度之前的社会-生态结构,是独立的社会行为主体与相关联的生态系统单元相连接,已有研究指出这更可能导致不协调的结果出现,因为孤立的行为主体通常难以从整体角度出发考虑互相关联的生态系统过程\cite{sayles2017,sayles2019,cai2016,bergsten2019}。
流域统一调度后,制度调整将黄河水利委员会(流域尺度管理者)调整为用水的直接责任人,在黄河流域的水资源供应与需求之间建立了空间尺度的连接,提高了流域社会-生态系统的匹配度,并在解决断流问题上取得了更好的效果\cite{cumming2020a,wang2019d}。
这再次证明,在耦合的人类-自然系统中寻求双赢局面非常困难\cite{hegwood2022},良好的流域治理需要更深入了解其社会-生态结构\cite{bergsten2019, sayles2019}。
黄河流域用水的案例可以更深入地解释先前的社会-生态系统结构模式的匹配/失配问题,而本研究的因果推断方法能够将这一潜在变化和社会-生态系统的结果相联系,更好地理解制度变化对流域人水关系的深远影响。

% \subsection{Limitation, insights and implications}
\section{对流域治理制度的启示}

本研究的结果对于理解流域社会-生态系统结构匹配、改善流域社会-生态系统治理等方面有具有重要启示。
流域统一调度制度的成功从理论和实践上证明了社会-生态匹配的重要性,为社会-生态系统结构匹配在水管理中的应用提供了因果明晰的典型案例,指出与流域尺度匹配的责任部门在治理中的重要性~\cite{bodin2017b, ostrom2009, reyers2018}。
全球各地、各类的生态-社会系统中,类似结构模式都是影响系统演变的关键因素,因此本研究强调的社会-生态匹配机制对于治理人地耦合系统至关重要,为了流域可持续发展,有必要进一步加强利益相关者与流域管理部门之间的联系。
% 如果用边际效应分析的话,加入以下内容
% 从这些角度来看,基于边际效益分析(见\textit{\nameref{secS5}})的两种情况可以启发如何减少不匹配的制度设计。
% 例如,水权转让可能是在利益相关者之间建立横向联系的另一种方式,这也有可能导致更好的水治理。
% 此外,政策制定者可以提出更有活力和更灵活的制度,以增加利益相关者对不断变化的社会经济环境的适应\cite{reyers2018}。

随着黄河流域生态保护和高质量发展重大国家战略的深入实施,黄河流域社会经济从“高速发展”进入了“高质量发展”阶段,需从过去对水资源的过度开发和粗放式利用转向产业结构优化调整下的节约集约利用。
随着不断变化的环境背景,旧有的水资源分配制度已经不能满足可持续发展的要求,重新设计黄河流域水分配制度的呼声之高,也说明了制度对流域可持续性的重要意义,改革这一历史悠久的水资源分配制度已迫在眉睫\cite{wang2019a}。
本研究的结果警示着,如果社会-生态系统结构失配,制度可能会导致意想不到的结果,保持强调利益相关者与流域生态系统之间尺度匹配非常重要,通过深入理解这一过程,黄河流域的制度变化经验将能为全球生态-社会系统管理提供借鉴和指导~\cite{hegwood2022, muneepeerakul2017, leslie2015}。

% % 江恩慧 2022本子
面对复杂的流域社会-生态系统,本研究的方法必然存在一些局限性。
首先,尽管控制了经济和环境因素,但由于相互交织的因果关系,仍很难分辨经济增长速率变化和制度干预的贡献。
本章研究使用合成控制法背后暗含的假设是,黄河流域经济、环境因素与用水量的关系,无论政策处理前后都与中国其它地区的关系是变化趋势一致的,因此经济增速的变化(如各省因水资源限制放缓经济增长)与制度促进的节水(如增设更多节水灌溉设施)都被认为是政策干预的后果,无法对其做出进一步区分。
其次,各省都会在每一年推出一些新的改革措施,但在使用DSC方法时,不可能排除同一时间断点(1987年或1998年)上其他政策的影响。
尽管存在上述不足,但本研究的实验方法仍能证明黄河流域的制度变革对利益相关各省的用水模式产生了深远影响,结果的显著性对制度在流域人水系统变化中的关键驱动作用提供了有力证据。

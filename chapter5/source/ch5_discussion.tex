\label{discussion-1}
% discussion-1:
% 用水量的上升、下降-结果解读
% 制度对社会生态系统的结果产生影响在世界范围内都很普遍,
制度对社会-生态系统(SE)结果的影响在世界范围内被广泛报道,但很少有人试图量化其净效应\cite{cumming2020a}。
结果表明,98-UBR降低了YRB的用水量,而87-WAS增加了$8.57\%$倍。
该结果对之前的分析提出了挑战(即,表明87-WAS“几乎没有实际影响”),因为从理论上讲,如果没有影响,那么YRB的实际用水和合成用水之间应该没有什么差距\cite{abadie2015,hill2021}。
然而,我们的分析表明,87-WAS的显著净效应表明,即使在控制了环境和经济变量之后,也会有更多的用水量。
相反,98-UBR减少了地表水竞争,因此许多研究将河流流量恢复主要归因于该制度\cite{chen2021,huangang2002,an2007}的成功引入。

\label{discussion-2}
% discussion-2: 87的增加
% 结果与制度的目的完全相反的87-WAS与许多其它的SES失败治理案例类似,表明不匹配的社会生态结构可能促进了对公共资源的掠夺式开采。
上述比较表明,87-WAS的结果与该制度的目的相反,与许多其他SES治理失败相似,支持不匹配的社会生态结构会恶化共同资源\cite{kellenberg2009,cai2016,barnes2019}。
87-WAS之后用水量的增加与在此期间一些省份频繁争夺水资源的担忧相一致。
尽管87-WAS效果不理想的原因已经被广泛讨论(如执行、可行性和公平性),然而,结构变化受到的关注有限。
我们的结果表明,87-WAS后,当前用水量与变化用水量(增加或减少)之间的相关性显著(图\ref{fig:regulating})。
这种“主要用户使用更多”的模式支持这样一个假设,即分离的利益相关者(个别省份)将通过最大化效用来响应结构(在我们基于结构的模型中解释,参见图~\ref{fig:model})。

我们的理论分析的有效性得到两个事实的支持:
(1) 87-WAS的水配额(或初始水权)经历了利益相关者之间的“讨价还价”阶段(1982年至1987年)\cite{wang2019a, wang2019d},每个省份都试图展示其与水利用相关的发展潜力。
议价也是一个将水份额与经济总量相匹配的过程,因为主要用水户(如山东和河南)需要的水超过了他们原来的配额(如果在设计机构时仅考虑经济潜力)\cite{zuo2020}。
(2)由于决策者与利益相关者之间的信息不对称,当前用水量较高的省份在水资源分配中具有更强的议价能力。
因此,利益相关者有相当大的动机来防止水配额阻碍他们的经济潜力,这与他们向更高一级的中央政府要求更大份额的呼吁一致\cite{wang2019a, wang2019d}。

\label{discussion-3}
另一方面,社会-生态匹配也可以得到结构效应的支持。
98-UBR后,长江流域水利枢纽可根据气候条件调整整个长江流域地表水利用指标。
当YRCC开始在利益相关者之间进行协调时,各省外部对提高配额的呼吁变成了内部提高用水效率的创新(例如,大幅增加节水设备)。
\cite{krieger1955, ostrom1990}。
在此期间,各省用水量按比例减少表明预期的河流治理效果(见\nameref{result-3})。
同时,我们的模型也表明,在这种情况下,统一的规模匹配机构对于减少用水是不可或缺的。
然而,由于98-UBR仅规范地表水的使用,许多线索表明,由于水需求得不到满足,制度转变可能会导致更广泛的影响(级联效应)。
例如,文献估计,在许多集约灌区,98-UBR后地下水取水量增加,尽管很少有关于地下水使用和利益相关者的合格数据\cite{sun2022b}。
自21世纪以来,类似的水配额政策开始在全国范围内实施,并改变了社会行动者与资源单位之间的关系。
由于本文研究的两种制度变迁都在SE内部引发了意想不到的变化或级联效应,更好的治理要求在未来对耦合的人类和自然系统进行更多的制度分析。

我们在这里描述的结构构建块(或motif)(图~\ref{fig:structure})也在世界各地的其他SE中报道过。
在98-UBR之前,SES结构(即与独立社会行为体相关联的片段生态单元)更有可能不匹配,因为孤立的行为体通常难以维持相互关联的生态系统整体\cite{sayles2017,sayles2019,cai2016,bergsten2019}。
98-UBR以来的制度调整提高了YRCC的权威,并帮助其与YRB的资源供应规模相匹配,从而增强了社会-生态适应性,并在径流恢复方面取得了更好的结果\cite{cumming2020a,wang2019d}。
这一比较再次证明了在耦合的人类-自然系统\cite{hegwood2022}中寻找双赢局面的挑战,以及更深入地理解社会-生态结构\cite{bergsten2019, sayles2019}的作用的必要性。
因此,YRB案例可以进一步解释先前SES构建模块的匹配和不匹配,通过合理的因果关系和潜在过程将SES结构和结果联系起来。


% \subsection{LIMITATION, INSIGHTS AND IMPLICATIONS}
% \subsection{Limitation, insights and implications}
% \label{discussion-4}
% discussion-3:

\subsection{启示、未来的展望}
我们的方法有一些不可避免的局限性。
首先,由于相互交织的因果关系,经济增长和制度变迁的贡献难以区分(制度变迁也会影响相关的经济变量);
其次,当应用DSC方法时,很难排除其他政策在相同时间断点(1987年和1998年)上的影响。
尽管如此,我们的准实验方法提供了支持以下观点的证据,即在YRB独特的制度转变之后,水的使用轨迹发生了变化,并为水治理提供了深刻的见解(特别是拥有一个规模匹配的、全流域的水分配解决方案权威的重要性\cite{bodin2017b, ostrom2009, reyers2018})。
此外,98-UBR制度变迁的最终成功从理论上和实践上证明了社会-生态契合的重要性。
因此,为了未来的可持续性,有必要强调与生态系统规模相一致的主体加强利益相关者之间联系的必要性。
从这些角度来看,基于边际效益分析(见\textit{\nameref{secS5}})的两种情况可以启发如何减少不匹配的制度设计。
例如,水权转让可能是在利益相关者之间建立横向联系的另一种方式,这也有可能导致更好的水治理。
此外,政策制定者可以提出更有活力和更灵活的制度,以增加利益相关者对不断变化的社会经济环境的适应\cite{reyers2018}。

导致不同结果的结构构件是全球SE中反复出现的主题,因此我们提出的机制对于治理这种耦合系统至关重要。
近年来在黄河流域重新设计水分配机构的呼吁也说明了机构解决方案对可持续性的重要性(见\textit{\nameref{secS1}})。
鉴于不断变化的环境背景,过时和不灵活的水配额已不能满足可持续发展的要求\cite{wang2019a}。
因此,中国政府已经开始计划重新设计其已有数十年历史的水资源分配制度(见\textit{\nameref{secS1}})。
我们的分析表明,这些举措可能会导致扭曲,因为在发展新机构\cite{bodin2017b}时,构建模块不匹配。
因此,我们的研究为机构如何改变不正当激励提供了一个警世故事\cite{hegwood2022},而YRB的见解可以为全球SE管理提供指导\cite{muneepeerakul2017, leslie2015}。

% discussion-1:
% 用水量的上升、下降-结果解读
% 制度对社会生态系统的结果产生影响在世界范围内都很普遍,
制度对社会-生态系统(SE)结果的影响在世界范围内被广泛报道,但很少有人试图量化其净效应\cite{cumming2020a}。
结果表明,流域统一调度降低了黄河流域的用水量,而“八七”分水方案增加了$8.57\%$。这一结果对之前的分析提出了挑战,即表明“八七”分水方案“几乎没有实际影响”,因为从理论上讲,如果没有影响,那么黄河流域的实际用水和合成用水之间应该没有什么差距\cite{abadie2015,hill2021}。
然而,我们的分析表明,“八七”分水方案的显著净效应表明,即使在控制了环境和经济变量之后,也会有更多的用水量。
相反,流域统一调度减少了地表水竞争,因此许多研究将河流流量的恢复主要归因于该制度的成功引入\cite{chen2021,huangang2002,an2007}。

% 实证研究表明,在社会经济系统中,这种广泛存在的构建模块是结构功能的关键。基于网络的方法是将实体之间的连接抽象为链接和节点\cite{bodin2017a,kluger2020,guerrero2015}。
%
% discussion-2: 87的增加
% 结果与制度的目的完全相反的“八七”分水方案与许多其它的SES失败治理案例类似,表明不匹配的社会生态结构可能促进了对公共资源的掠夺式开采。
“八七”分水方案的结果与该制度的目标相悖,与许多其他社会生态系统治理失败的情况相似,证明不匹配的社会生态结构会加剧共同资源的恶化\cite{kellenberg2009,cai2016,barnes2019}。
在“八七”分水方案实施后,用水量的增加与一些省份频繁争夺水资源的担忧是一致的。
尽管“八七”分水方案效果不理想的原因已经得到广泛的讨论(如执行、可行性和公平性),但是结构变化的关注却有限。
我们的结果表明,在“八七”分水方案实施后,目前的用水量与变化用水量(增加或减少)之间存在显著的相关性(图\ref{fig:regulating})。
这种“主要用户使用更多”的模式支持这样的假设:独立的利益相关者(如个别省份)将通过最大化效用来适应结构变化。

我们的理论分析得到了两个事实的支持:
(1) 在“八七”分水方案的配额讨论中,各个利益相关者(如各个省份)试图展示他们与水资源利用相关的发展潜力,因此在1982年到1987年间进行了“讨价还价”的过程\cite{wang2019a, wang2019d}。
这个“讨价还价”的过程也是一个将水资源配额与经济总量相匹配的过程,因为如果只考虑经济潜力的话,主要用水户(如山东和河南)需要的水资源超过了他们原先的配额\cite{zuo2020}。
(2)由于决策者与利益相关者之间的信息不对称,目前用水量较高的省份在水资源分配中具有更强的讨价能力。因此,利益相关者有相当大的动机来防止水配额阻碍他们的经济潜力,这与他们向中央政府要求更多水资源的呼吁是一致的\cite{wang2019a, wang2019d}。

本研究的结果对于改善流域社会-生态系统的治理具有重要启示。
通过流域统一调度,黄河流域水利枢纽可根据水文情况调整整个流域地表水利用指标。
当黄河水利委员会开始协调利益相关者时,各省外部对提高配额的请求变为内部创新,以提高用水效率(例如,大量增加节水设施)\cite{krieger1955, ostrom1990},研究结果表明这种流域尺度的制度匹配对于减少用水是必要的。
然而,由于流域统一调度仅规范地表水的使用,有许多证据表明,由于水需求得不到满足,制度转变可能导致更广泛的级联效应。
例如,文献估计,在许多灌区,流域统一调度后地下水取水量增加,可惜少有关于地下水使用的合格数据,难以用数据分析的方法开展进一步研究\cite{sun2022b}。
近年来,全国范围内开始实施类似的水配额政策,这改变了社会行动者与资源单位之间的关系。由于本研究中的两种制度变革都在社会-生态系统内部引发了意料之外的变化或级联效应,所以更好的管理需要对耦合的人类和自然系统进行更多的制度分析。

我们这里描述的结构构建单元(或图模型)(参见图~\ref{fig:structure})在全球范围内的其他社会-生态系统中也有报道。
在流域统一调度之前,SES结构(即独立的社会行为主体与生态系统片段相关联的单元)更有可能不协调,因为孤立的行为主体通常难以保持生态系统整体的相互关联\cite{sayles2017,sayles2019,cai2016,bergsten2019}。
流域统一调度后,制度调整增强了黄河水利委员会的权威,帮助其与黄河流域的资源供应规模相匹配,从而提高了社会-生态的适应性,并在改善径流方面取得了更好的效果\cite{cumming2020a,wang2019d}。
这再次证明了在耦合的人类-自然系统中寻求双赢局面的困难\cite{hegwood2022},以及更深入了解社会-生态结构的必要性\cite{bergsten2019, sayles2019}。
因此,黄河流域的案例可以更深入地解释先前的SES结构构建单元的协调和不协调,通过合理的因果关系和潜在过程将SES结构和结果联系起来。

% \subsection{Limitation, insights and implications}
\subsection{启示、未来的展望}
% % 江恩慧 2022本子
% 随着黄河流域生态保护和高质量发展重大国家战略的深入实施,黄河流域社会经济从“高速发展”进入了“高质量发展”阶段,需从过去对水资源的过度开发和粗放式利用转向产业结构优化调整下的节约集约利用,对流域水资源配置提出了更高要求。然而,黄河水资源开发利用程度高达80\%,甚至在西北部分地区达到了100\%,已触及水资源供给的“天花板”(王建华等,2020);沿黄城市发展高度依赖黄河水资源配置,且已有90\%以上城市处于水资源超载状态(李原园等,2021);未来水资源供需矛盾有可能进一步加剧。

我们的方法存在一些必然的局限性。
首先,由于因果关系的相互交织,很难分辨经济增长和制度变革的贡献。其次,在使用DSC方法时,很难排除其他政策在同一时间断点(1987年和1998年)上的影响。
尽管如此,我们的实验方法提供了证据,证明在黄河流域的独特制度变革后,水的使用模式发生了变化,并对水管理提供了深刻的见解(特别是关于拥有一个与流域规模匹配的、全流域的水分配解决方案的权威重要性)\cite{bodin2017b, ostrom2009, reyers2018}。
此外,流域统一调度制度变革的最终成功从理论和实践上证明了社会-生态匹配的重要性。
因此,为了未来的可持续性,有必要强调与生态系统规模相一致的主体加强利益相关者之间的联系的必要性。
% 如果用边际效应分析的话,加入以下内容
% 从这些角度来看,基于边际效益分析(见\textit{\nameref{secS5}})的两种情况可以启发如何减少不匹配的制度设计。
% 例如,水权转让可能是在利益相关者之间建立横向联系的另一种方式,这也有可能导致更好的水治理。
% 此外,政策制定者可以提出更有活力和更灵活的制度,以增加利益相关者对不断变化的社会经济环境的适应\cite{reyers2018}。

全球生态-社会系统的结构模式是影响结果的关键因素,因此我们提出的机制对于治理这种耦合系统至关重要。
最近,黄河流域重新设计水分配制度的呼声也说明了制度解决方案对可持续性的关键性。
随着不断变化的环境背景,旧有的水资源分配制度已经不能满足可持续发展的要求,因此中国政府已经开始规划重新设计其历史悠久的水资源分配制度\cite{wang2019a}。
我们的分析表明,如果模式不匹配,新制度的建立可能会导致系统扭曲。
因此,我们的研究为如何解决不正当激励问题提供了一个重要的教训,而黄河流域的经验可以为全球生态-社会系统的管理提供指导\cite{hegwood2022, muneepeerakul2017, leslie2015}。

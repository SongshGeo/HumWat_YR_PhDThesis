通过制度(如政策、法律、和社会规范)来重塑社会-生态结构,是流域系统人水关系变化的重要驱动力,时空匹配的制度能够达到预期的水治理目标,支持流域高质量、可持续发展。
本章重点关注1987年的“八七”分水方案和1998年的“流域统一调度”两次自上而下的水资源分配制度变化,分析它们重塑流域系统社会-生态结构的过程,定量识别其对流域用水的影响。

我们首先抽象社会-生态系统的结构模式,从官方文件中勾勒出两次制度前后流域系统的社会-生态结构,发现流域尺度的利益相关者——黄河水利委员会在制度改革后,引入了与生态节点和其它社会节点之间的新联系,形成了不同的社会-生态系统结构模式。
接下来使用“差分合成控制法”的因果推断模型进行反事实推断,考虑经济增长和气候条件,估算“假设没有制度干预”情景下的理论用水量,与实际的流域用水量对比后分析制度干预的净效应。
研究表明“八七”分水方案的制度改革显著($p<0.05$)促使黄河流域在接下来十年内多用了约$5.75\%$的水资源;而流域统一调度制度后,流域总用水量以每年$6.6$亿立方米的速度下降,远小于模型预测的每年$5.5$亿立方米的增速,并结束了长达二十余年的河流断流。

本章研究表明如果社会-生态系统结构失配,制度可能会导致意想不到的结果,保持强调利益相关者与流域生态系统之间尺度匹配非常重要,而黄河流域的制度改革经验将能为全球生态-社会系统管理提供借鉴和指导。

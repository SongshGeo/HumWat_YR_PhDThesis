通过政策、法律和规范等制度来重塑社会-生态系统结构是流域人水关系变化的重要原因之一,匹配的水治理制度能够支持流域社会-生态系统的可持续性。
本章重点关注了1987年的水资源分配改革(“八七”分水方案)和1998年的“流域统一调度”两次自上而下的水资源分配制度改革,分析它们如何重塑流域社会-生态系统结构,产生不同的治理效果并最终影响了流域可持续性。

我们首先通过抽象社会-生态系统构件(building blocks)的方法,从官方文件中勾勒出两次制度前后的系统结构变化,发现流域尺度机构黄河水利委员会在两次制度改革后都在社会与生态节点之间引入了新的联系,但形成的具一般性的社会-生态系统结构块却略有不同。
接下来使用“差分合成控制法”的因果推断模型,考虑经济增长和气候条件的变化构建“没有发生制度转变”的反事实情景,估算此情景下的理论用水量。
研究表明“八七”分水方案的制度改革促使黄河流域在接下来十年内多取用了约$6.57\%$的水资源,而流域统一调度制度后,总用水量以每年$-4.9$亿立方米的速度减少,比模型预测的$+8.2$亿立方米的增速大大降低,也立竿见影地结束了长达二十余年的河流断流。
研究表明了在自上而下驱动流域水治理变化时制度匹配的重要性,模型分析结果也可供近年来黄河流域分水制度的进一步改革提供参考。

通过制度(如政策、法律、和社会规范)来重塑社会-生态结构,是流域系统人水关系变化的重要驱动力,时空匹配的制度能够达到预期的水治理目标,支持流域高质量、可持续发展。
本章重点关注$1987$年的“八七”分水方案和$1998$年的“流域统一调度”两次自上而下的水资源分配制度变化,分析它们重塑流域系统社会-生态结构的过程,定量识别其对流域用水的影响。

本章厘清了两次水资源分配制度变化前后黄河流域的社会-生态结构,发现流域尺度的黄河水利委员会在制度变化后,引入了与生态节点和其它社会节点之间的新联系,形成了不同的结构模式。
使用“差分合成控制法”的因果推断模型进行反事实推断,考虑经济增长和气候条件,估算“假设没有制度干预”情景下的理论用水量,表明“八七”分水方案的制度变化显著($p<0.05$)促使黄河流域在接下来十年内多用了约$5.75\%$的水资源;而流域统一调度制度后,流域总用水量以每年$6.6$亿立方米的速度下降,远小于模型预测的每年$5.5$亿立方米的增速,展现出良好的治理效果。

本章研究表明社会-生态系统结构失配可能使会制度产生背离预期的结果,强调了保持治理系统尺度匹配的重要性,本章黄河流域分水制度的治理案例可为全球大河流域提供借鉴和指导。

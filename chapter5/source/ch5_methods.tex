%! Author = songshgeo
%! Date = 2022/3/10

% 现有制度匹配知识的两个特别弱点包括理解(图~\ref{fig:framework}~\textbf{b}):(i)社会经济体系结构和结果之间的因果联系;(ii)由于制度上缺乏匹配而导致的基本过程的细节,特别是不同参与者的激励。
% 这些弱点限制了对制度设计的理解,阻碍了对制度匹配的研究。

为了更好地理解自上而下的制度改革对黄河流域产生的影响,本章以黄河流域的水资源分配制度为例,重点关注两次重大制度转变:1987年提出的“八七”分水方案和1998年的提出的流域统一调度。
研究选择水资源分配制度为切入点的主要理由如下:
(1) 本章关注自上而下的制度变迁对人-水关系的影响,而黄河的水资源配额制度是典型的由中央主导的改革,利益相关者(相关各省)之间没有相互作用。
(2) 很少有大型流域经历这种制度转变一次以上,1987和1998前后两次制度变革具有难得的可对比性,提供了绝妙的准自然实验。
(3) 两次制度改革之间维持的时间周期足够长,便于对数据进行时间序列分析,且该研究时段完全覆盖了第三章中识别的“治理转型时期”。

% 为了量化制度变迁为黄河流域用水带来的影响,我们按附图1所示的技术路线执行了分析过程
本章研究重点关注上述两次水资源配额制度改革对1979年至2008年黄河流域的用水变化,两次制度变迁将这段研究时期分为均匀的三个时间序列:$1979-1987$,$1988-1998$,$1999-2008$。
我们首先详细介绍了两次制度改革的背景及特点,以及刻画其社会-生态系统结构模块的方法。
接下来收集用以构建反事实推断模型的数据集并使用主成分分析(PCA)方法对其进行降维。
最后,我们利用差分合成控制法(DSC)构建了反事实推断模型\cite{arkhangelsky2021},估算了“如果没有发生制度转变”时的用水量,从而分析两种制度变迁对黄河流域各省份用水量变化的净影响。
% 最后,在理论讨论方面,我们基于已确定的SES结构进行了边际效益分析,为观测到的用水变化模式提供了理论解释。

\subsection{制度背景介绍}\label{sec:yrb}

20世纪80年代,因大量地表水取用以及水库拦蓄等其他形式的人类干预(约占黄河地表径流量的$80\%$),黄河发生了连续的断流问题,造成严重的生态、经济和社会危机(如湿地萎缩、作物缺水、和地区间水资源争夺)。
为此,黄河流域率先实施了几项雄心勃勃的治理措施以缓解水资源压力,如水库联合调度、南水北调工程、以及本研究所关注的水资源配额制度\cite{long2020, wang2019d}。
这些努力共同恢复了黄河的生态,黄河已二十余年不再断流,这被认为是一项重大的管理成就和世界河流恢复的奇迹。
虽然已有研究仔细评估了南水北调和水库建设等工程解决方案产生的影响\cite{long2020,wang2019c},但被认为对遏制断流问题帮助最大的水资源配额制度却缺乏评估\cite{wang2019b}。

水资源分配制度在全世界范围内都是普遍存在对流域管理制度,而黄河流域是我国首次实施此类制度的流域,其提出、实施、发展、改革等过程可概括如下\cite{wang2019b}:% todo 这里可以根据obsidian里的笔记做补充
\begin{itemize}
    % 1978年:因黄河断流,整理意见
    \item 1982年:水利部要求沿黄各省和黄河水利委员会制定黄河水资源规划\cite{wang2019, wang2019a}。
    \item 1983年:各省提出规划需求,首次拟定配额方案的额度。
    \item 1987年:分配额度调整并确定,分配计划正式实施。
    \item 1998年:流域统一调度改革实施。
    \item 2008年:要求各省制定新的水利规划,进一步细化水资源分配\cite{wang2019,wang2019a}
    % 详见\url{http://www.ccgp.gov.cn/cggg/zygg/gkzb/202107/t20210721_16591901.htm
    \item 2021:呼吁对水资源分配制度进行调整。
\end{itemize}

由于2008年的文件标志着该方案成熟并进入下一阶段,开始建立流域级、省级、地级市、区级的多级水配额制度,且配额会根据水资源丰枯情况进行月尺度动态调整,本章结合数据可获得性将研究范围选定在1978年至2008年。
在此期间,使用1987年和1998年两次制度改革的官方文件为基础,进行接下来的分析。

% \href{http://www.gov.cn/zhengce/content/2011-03/30/content_3138.htm#}{http://www.mwr.gov.cn},最后访问:\today
1987年的官方文件传达了以下要点:

\begin{itemize}
	\item 该政策面向的目标是利益相关者(沿黄省或地区),流域尺度的黄河水利委员会没有在文件中被提及。
	\item 政策制定的首要考虑是解决断流问题。
	\item 各省被鼓励在此配额下制定自己的用水计划。
	\item 水资源供给无法满足需求对相关省(地区)是普遍现象。
\end{itemize}

% (\href{http://www.mwr.gov.cn/ztpd/2013ztbd/2013fxkh/fxkhswcbcs/cs/flfg/201304/t20130411_433489.html}{http://www.mwr.gov.cn}, last access: \today)
1998年“流域统一调度”的官方文件传达了以下要点:

\begin{itemize}
	\item 除了说明政策针对的各省区之外(\textbf{第一章第三条}),明确指出其用水需要黄河水利委员会进行申报,并由其组织和监管 (\textbf{第三章第十一条,第五章至第七章})。
	\item 优先考虑上、中、下游用水的总体规划(\textbf{第一章第一条})。
	\item 在与1987年政策相同的配额下,鼓励各省进一步将配额分配给下级行政部门(\textbf{第二章第六条、第八章第四十一条})。
	\item 强调“以水量决定需求”,各省用水理论上不能超过1987年分配的额度 (\textbf{第一章至第二章}).
\end{itemize}

\subsection{社会-生态系统结构模式的抽象}\label{sec:structures}

本章研究在对以上官方文件的基础上,对两次制度变迁后的流域水资源利用的社会-生态结构模式按框架图\ref{fig:framework}进行了抽象。
在社会-生态系统中广泛存在的结构模块,常被表示为社会要素与生态要素共同构成的局部网络,通过抽象系统中存在相互关系为节点和链接来刻画它们~\cite{bodin2017a,kluger2020,guerrero2015}。
我们参考 Wang 等人在黑河流域制度变化分析中对该方法的应用\cite{wang2019d},
从官方文件的叙述中将可供取水的河段作为生态单位、分水制度涉及的行政单元(黄河水利委员会)和利益相关者(沿黄各省)之间的关系抽象为一般的结构模块(见图\ref{fig:framework})。
所有社会节点(省份和黄河水利委员会)与河段之间因水资源的取用或监管形成生态连接,它们因黄河水资源相关的过程产生相互作用被总结为社会连接。
% 19“八七”分水方案要求黄河水利委员会监测每条河流的范围,而19流域统一调度要求黄河水利委员会和各省之间直接互动(通过用水许可证)。
% 因此,我们将黄河水利委员会单元与“八七”分水方案之后的每个生态单元和流域统一调度之后的每个省份单元联系起来。
% 我们测试了关注社会经济体系结构而非制度细节是否能够合理解释黄河流域中由制度转变引起的差异。

\begin{figure}[htb] % use float package if you want it here
    \centering
    \includegraphics[width=\textwidth]{img/ch5/framework.png}
    \caption[用于理解社会经济体系结构和结果之间联系的框架]{用于理解社会经济体系结构和结果之间联系的框架。\textbf{a.}分析社会-生态系统的通用框架(改编自 Ostrom, 2009~\cite{ostrom2009})内,制度改革可以通过改变系统内部的相互作用重塑结构。\textbf{b.}通过构建反事实推断,可以分析匹配/不匹配SES结构如何影响系统演变。}\label{fig:framework}
\end{figure}

\subsection{数据与预处理}\label{sec:dataset}

为构建反事实推断模型,需要选择表征实际用水情况的因变量,以及能够用以预测地区用水量的自变量。
研究从国家水利部开展的水资源调查数据中获取省级行政单元的实际用水量\cite{zhou2020},从不同来源获取了农业、工业、服务业、居民生活、环境五个类别共计$39$个影响区域用水的社会-经济-环境因素作为自变量(详见表~\ref{ch5:tab:data_source})。

% Table generated by Excel2LaTeX from sheet '数据来源'
\begin{table}[htbp]
    \caption{Add caption}
      \begin{tabularx}{\textwidth}{LLLL}
      \toprule
      \multicolumn{1}{l}{部门} & 分类    & 单位    & 描述 \\
      \midrule
      \multicolumn{1}{l}{农业$^1$} & 灌溉面积  & 千公顷   & 装配了灌溉设施的不同作物面积 \\
      \multicolumn{1}{l}{工业$^2$} & 产值    & 千百万元  & 工业各产业的总增加值 \\
            & 循环用水比例 & \%    & 工业循环用水占总用水比例 \\
      \multicolumn{1}{l}{服务业} & 服务业总增加值 & 百万元   & 服务业的总增加值 \\
      \multicolumn{1}{l}{居民生活} & 城市人口  & 百万人   &  \\
            & 农村人口  & 百万人   &  \\
            & 牲畜数量  & 十亿千焦  & 牲畜卡路里总和 \\
      \multicolumn{1}{l}{环境} & 气温    & K     & 近地表气温 \\
            & 降水量   & mm    & 年累计降水量 \\
      \bottomrule
      \end{tabularx}\label{ch5:tab:data_source}%
      \footnotesize
      1. 包括以下作物类型:水稻、小麦、玉米、水果、其它
      2. 包括以下产业:纺织、造纸、石油化工、冶金、采矿、粮食生产、水泥、机械、电子、电力、其它
\end{table}%
  

在“八七”分水方案和流域统一调度均涉及的$10$个省份或地区中,四川、天津和北京因为从黄河流域取用水占该地区总用水量不足$5\%$而被剔除(见表~\ref{ch5:tab:quota})。
Bayan(2021)\cite{bayani2021}已证明了将PCA与DSC相结合可以提高因果推理的鲁棒性,研究将所有数据归一化后使用主成分分析(PCA)对各省的多年平均值进行降维,并用肘部法确定主成分的个数,从而降低输入反事实推断模型的自变量数目。
% 通过$5$个主成分\textbf{ Appendix~\nameref{secS2}}捕获$89.63\%$个解释方差(图~\ref{fig:elbow})。

\subsection{差分合成控制法}\label{sec:DSC}

将经上述处理的数据集分为反事实推断所需的“实验组”和“对照组”数据,其中水配额制度涉及的省份数据作为目标组$(n=8)$数据,其他省份则作为对照组$(n=20)$数据。
本章研究选用差分合成控制法(Differenced Synthetic Control, DSC)构建反事实推断模型,估算不受制度改革影响情景下的流域用水量。
DSC 方法是对合成控制法(Synthetic Control, SC)在鲁棒性上的改进\cite{billmeier2013, smith2015},这一类因果推断方法可以构建一个可比的控制单元来估计历史事件或政策干预对真实单元(如城市、地区和国家)的影响及其变化趋势\cite{abadie2010, abadie2015, hill2021}。
该方法旨在评估政策变化的影响,而这些政策变化集中在其中一些单元上,在本研究中即为分水制度涉及的省份属于政策影响的单元,称之为“处理单元”,其用水特征作为“真实值”使用“目标组”数据的因变量进行表达。
合成控制法通过重新加权非处理单元构造其凸组合作为控制组,并匹配处理单元和控制单元在政策干预前的变化趋势构成反事实推断,这一来处理单元和控制单元在政策干预后的因变量差异便可估计为政策干预的净效应。

本章研究中,所有处理单元(即制度涉及的诸省)都在1987年和1998年先后受到两次不同的制度干预,因此每个政策干预前后十年都被视为单独的分析时期,即一次构建反事实推断模型的研究时段可能为 $1979 \sim 1998$(1987年为政策干预时点)或 $1987 \sim 2008$(1998年为政策干预时点)。
接下来将黄河流域分水制度涉及的每个省份分别作为处理单元($n=8$,见\textbf{\nameref{sec:dataset}}),其余的$J=20$个控制单元均来自流域外未受“政策干预”的省份构建模型,而这种多处理单元的合成控制法也已广泛应用\cite{abadie2021}。
% 然后,我们认为在第$T = {1,2 \ldots , T}$时间段观测到的第$J+1$个单位。
我们定义$T_0$为政策干预前的周期数(年份,$1,\ldots,t_0$),定义$T_1$为政策干预后的周期数($t_0,\ldots,T$),使得$T = T_0+ T_1$,每个处理单元在政策干预后的时间$T_1$内都面临制度变迁,而在之前的$T_0$时期不受政策干预的影响。
这一来,控制单元的任何加权组合都是一个潜在的合成控制模型,可以用一个权重为$\mathbf{W} = (w_{1},\ldots,w_{J}), w_j \in (0, 1)$的向量$J * 1$来表示。
通过引入$k * k$的对角线半定矩阵$\mathbf{V}$表示各协变量相对重要性,那么在所有潜在合成控制模型中求解最优合成控制权重$W$的过程为:

\begin{equation}
    \mathbf{W^{*}(V)}=\underset{\mathbf{W} \in \mathcal{W}}{\operatorname{minimize}}\left(\mathbf{X}_{\mathbf{1}}-\mathbf{X}_{\mathbf{0}} \mathbf{W}\right)^{\prime} \mathbf{V}\left(\mathbf{X}_{\mathbf{1}}-\mathbf{X}_{\mathbf{0}} \mathbf{W}\right)
\end{equation}

给定$\mathbf{V}$的情况下,$\mathbf{W}$是矩阵$\mathbf{W}^{*}(V)$的权重向量,使处理单元与控制单元之间的特征差异最小。
这意味着$\mathbf{W^{*}}$取决于对$\mathbf{V}$的选择,因此可以用$\mathbf{W*(V)}$表示。
差分合成控制法选择$\mathbf{V^{*}}$作为优化$\mathbf{W*(V)}$的$\mathbf{V}$,通过让政策干预单元和合成控制法构建的对照单元在政策干预前的结果差异最小化:

\begin{equation}
    \mathbf{V}^{*}=\underset{\mathbf{V} \in \mathcal{V}}{\operatorname{argmin}}\left(\mathbf{Z}_{1}-\mathbf{Z}_{0} \mathbf{W}^{*}(\mathbf{V})\right)^{\prime}\left(\mathbf{Z}_{1}-\mathbf{Z}_{0} \mathbf{W}^{*}(\mathbf{V})\right)
\end{equation}

其中$\mathbf{Z}_{1}$是一个$1*T_0$的矩阵,包含政策干预单元在干预前的每一个观察样本。
类似地,让$\mathbf{Z}_{0}$作为一个$k * T_0$的矩阵,其中包含了政策干预前期控制单元每个的观察结果,其中$k$为数据集中变量的数目。

% \subsection{Marginal benefits analysis}\label{sec:model}

% 为了推断结果背后的机制,我们开发了基于边际收入的边际效益分析,以分析制度转变如何导致用水差异。

% (water -dependent production)由于不可替代,假设水是唯一具有两种生产效率类型的生产函数输入。
% (生态成本分配)假设生态是整个流域的单一实体,用水成本平均分配到各省份。

% (多时段设置)有多个设定时段,对未来用水的预期有一个恒定的折现因子。

% 在上述简化假设下,我们展示了三个案例——对应于抽象的SES结构(图~\ref{fig:structure}~C),推理了SES结构如何改变各省决策的预期边际效益和成本。
% 作为对SES结构与制度效应因果关系的可能解释之一,基于上述三个假设的模型推导可以在\textbf{ Appendix~\nameref{secS4}}中找到,\textbf{ Appendix~\nameref{secS5}}中涉及到一些简单的基于模型的扩展。

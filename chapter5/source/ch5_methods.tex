%! Author = songshgeo
%! Date = 2022/3/10

% 为了量化制度变迁为黄河流域用水带来的影响,我们按附图1所示的技术路线执行了分析过程
1979年至2008年的YRB,其中两次制度变迁将这段时期分为三个部分。
为了处理数据,我们使用主成分分析(PCA)方法来降低影响总用水量的变量的维数。
然后,我们利用差异综合控制(DSC)方法\cite{arkhangelsky2021}估算了两种制度变迁对黄河流域各省份用水总量、变化趋势和差异的净影响。
最后,在理论讨论方面,我们基于已确定的SES结构进行了边际效益分析,为观测到的用水变化模式提供了理论解释。

\subsection{研究范围}\label{sec:yrb}

黄河流域是世界第五大流域,在中国社会经济发展中发挥着至关重要的作用。
% 它支撑着中国$35.63\%$的灌溉和$30\%$的人口,却只拥有中国 $2.66\%$的水资源(数据来自\href{http://www.yrcc.gov.cn}{http://www.yrcc.gov.cn},最后访问:\today)。
20世纪80年代,大量用水(约占黄河地表径流量的$80\%$),加上其他形式的人类干预(如水土保持和水利工程),造成了连续的干旱事件和严重的生态、经济和社会危机(如湿地萎缩、农业萎缩和水资源争夺)。
作为回应,中国当局在长江流域实施了几项雄心勃勃的水资源管理措施,以缓解水资源压力,如水库调控、南水北调工程(WDP)、1987年水资源分配方案(87-WAS)和1998年流域统一条例(98-UBR) \cite{long2020, wang2019d}。
这些努力导致了湿地和河口三角洲的生态恢复。干旱已经避免了20多年,这被广泛认为是一项重大的管理成就。
87-WAS(为黄河流域的各省分配了用水配额)和98-UBR(各省必须从黄河水利委员会(YRCC,流域一级的权威机构)获得许可)等制度战略主要集中在限制对水的需求上,而不是依靠工程来增加供水 \cite{bouckaert2022, speed2013}。
虽然研究人员仔细评估和量化了工程解决方案对供水的影响\cite{long2020},,但很少有人试图评估机构对黄河流域成功水治理的贡献。

\subsection{Portraying structures}\label{sec:structures}
% 制度结构关系抽象
SES中广泛存在的构建模块是结构功能的关键,基于网络的描述是一种广泛使用的方法,通过抽象链接和节点来描述它们~\cite{bodin2017a,kluger2020,guerrero2015}。
我们应用网络方法\cite{bodin2017b}从官方文件中将生态单位(河流流域)、利益相关者(省份)和行政单位(YRCC)之间的关系抽象为一般的构建模块(或主题)(见图\ref{framework})来描绘SES结构。
实证研究表明,在社会经济系统中,这种广泛存在的构建模块是结构功能的关键。基于网络的方法是将实体之间的连接抽象为链接和节点\cite{bodin2017a,kluger2020,guerrero2015}。
在本研究中,我们研究了两种关注制度转移的官方文件
% TODO (87-WAS和98-UBR,详见\textit{ Appendix \nameref{secS1}})。
除了生态连接的河段外,代理人(省份和YRCC)被抽象为节点,它们在用水方面所需的相互作用被总结为链接。
1987-WAS要求YRCC监测每条河流的范围,而1998-UBR要求YRCC和各省之间直接互动(通过用水许可证)。
因此,我们将YRCC单元与87-WAS之后的每个生态单元和98-UBR之后的每个省份单元联系起来。
我们测试了关注社会经济体系结构而非制度细节是否能够合理解释YRB中由制度转变引起的差异。

\subsection{Dataset and preprocessing}\label{sec:dataset}

我们选择数据集和变量来比较黄河流域的实际用水量和估算用水量。
实际用水量可从国家水资源利用调查的中国省级年度用水量数据集中获取,其详细信息可从Zhou (2020) \cite{zhou2020}中获取。
为了通过假设没有制度变迁的影响来估算黄河流域的用水量,我们关注了五个类别的变量(环境、经济、家庭和技术)用水因素。它们的具体项目和来源列在表~\ref{ch5:tab:data_source}中。

% Table generated by Excel2LaTeX from sheet '数据来源'
\begin{table}[htbp]
    \caption{推断地区用水量的自变量数据}
      \begin{tabularx}{\textwidth}{LLLL}
      \toprule
      \multicolumn{1}{l}{部门} & 分类    & 单位    & 描述 \\
      \midrule
      \multicolumn{1}{l}{农业$^1$} & 灌溉面积  & 千公顷   & 装配了灌溉设施的不同作物面积 \\
      \multicolumn{1}{l}{工业$^2$} & 产值    & 千百万元  & 工业各产业的总增加值 \\
            & 循环用水比例 & \%    & 工业循环用水占总用水比例 \\
      \multicolumn{1}{l}{服务业} & 服务业总增加值 & 百万元   & 服务业的总增加值 \\
      \multicolumn{1}{l}{居民生活} & 城市人口  & 百万人   &  \\
            & 农村人口  & 百万人   &  \\
            & 牲畜数量  & 十亿千焦  & 牲畜卡路里总和 \\
      \multicolumn{1}{l}{环境} & 气温    & K     & 近地表气温 \\
            & 降水量   & mm    & 年累计降水量 \\
      \bottomrule
      \end{tabularx}\label{ch5:tab:data_source}%
      \footnotesize
      1. 包括以下作物类型:水稻、小麦、玉米、水果、其它
      2. 包括以下产业:纺织、造纸、石油化工、冶金、采矿、粮食生产、水泥、机械、电子、电力、其它
\end{table}%
  

在87-WAS和98-UBR分配的$31$个数据可访问省份(或地区)中,我们剔除了四川、天津和北京,因为他们从YRB用水很少(见\textit{ Appendix}~表~\ref{tab:quota})。然后,我们将数据集分为“目标组”和“对照组”,将涉及水配额的省份作为目标组$(n=8)$,将其他省份作为对照组$(n=20)$进行DSC应用。

使用所有变量的归一化数据,我们进行PCA归一化,通过$5$个主成分\textit{ Appendix~\nameref{secS2}}捕获$89.63\%$个解释方差。
Bayan证明了将PCA与DSC相结合可以提高因果推理\cite{bayani2021}的鲁棒性。
我们首先应用了零均值归一化(单位方差),因为变量的单位相差甚远。然后,我们将主成分分析应用于各省的多年平均值,用肘关节法确定主成分的个数(\textit{ Appendix~\nameref{secS2}~图~\ref{fig:elbow}})。
最后,对数据集进行转换,并将降维后的结果输入DSC模型。

\subsection{Differenced Synthetic Control}\label{sec:DSC}

利用差分综合控制(DSC)方法,我们估算了不受制度转移影响的用水量。
DSC方法是一种有效的识别策略,可以通过构建一个可比控制单元来估计历史事件或政策干预对总体单位(如城市、地区和国家)的净影响 \cite{abadie2010, abadie2015, hill2021}。

该方法旨在评估政策变化的影响,这些政策变化在各个单位之间不是随机的,而是集中在其中一些单位上(即,这里的YRB的制度变化)。
通过重新加权单元以匹配处理单元和控制单元的预趋势,DSC方法通过构建处理单元的合成版本等于控制单元的凸组合,为处理单元注入处理后的控制结果。
因此,综合版本和实际版本的差异可以作为处理单元的净效应来估计。

在实践中,所有被处理的单位(即省份)都在1987年和1998年受到制度转移的影响,每个都被视为两个单独分析时期(1979-1998)的“转移”时间 $T$:$1987 \sim 2008$。
我们将YRB ($n=8$,见\textit{\nameref{sec:dataset}})中的每个省份分别作为处理单元,因为多处理单元方法已广泛应用\cite{abadie2021}。
然后,我们认为在第$T = {1,2 \cdots , T}$时间段观测到的第$J+1$个单位,其余的$J=20$个单位是来自外部的未经处理的省份。
我们定义$T_0$为治疗前的周期数($1,\cdots,t_0$),定义$T_1$为治疗后的周期数($t_0,\cdots,T$),使得$T = T_0+ T_1$。
接受治疗的单位在政策处理后的时间$T_1$内都面临制度变迁,而不受之前所有时期$T_0$的制度影响。
然后,控制单元的任何加权平均都是一个合成控制,可以用一个($J * 1$)权重为$\mathbf{W} = (w_{1},...,w_{J})$和$w_j \in (0, 1)$的向量表示。
其中,通过引入$k * k$的对角矩阵,表示各协变量相对重要性的对角线半定矩阵$\mathbf{V}$,求最优综合控制($W$)的DSC方法过程为:

\begin{equation}
    \mathbf{W^{*}(V)}=\underset{\mathbf{W} \in \mathcal{W}}{\operatorname{minimize}}\left(\mathbf{X}_{\mathbf{1}}-\mathbf{X}_{\mathbf{0}} \mathbf{W}\right)^{\prime} \mathbf{V}\left(\mathbf{X}_{\mathbf{1}}-\mathbf{X}_{\mathbf{0}} \mathbf{W}\right)
\end{equation}

其中$\mathbf{W}^{*}(V)$是权重向量$\mathbf{W}$,使处理单元的预处理特征与合成对照之间的差异最小化,给定$\mathbf{V}$。也就是说,$\mathbf{W^{*}}$取决于对$\mathbf{V}$的选择,因此用$\mathbf{W*(V)}$表示。因此,我们选择$\mathbf{V^{*}}$作为导致$\mathbf{W*(V)}$最小化以下表达式的$\mathbf{V}$:

\begin{equation}
    \mathbf{V}^{*}=\underset{\mathbf{V} \in \mathcal{V}}{\operatorname{argmin}}\left(\mathbf{Z}_{1}-\mathbf{Z}_{0} \mathbf{W}^{*}(\mathbf{V})\right)^{\prime}\left(\mathbf{Z}_{1}-\mathbf{Z}_{0} \mathbf{W}^{*}(\mathbf{V})\right)
\end{equation}

这是治疗单位与综合对照在治疗前期间的结果之间的最小差异,其中$\mathbf{Z}_{1}$是一个($1*T_0$)矩阵,包含治疗单位在治疗前期间的每一个观察结果。类似地,设$J+1$0为一个($k * T_0$)矩阵,其中包含预处理期间每个控制单元的结果,$k$为数据集中变量的数量。
DSC方法推广了差中差估计量,并允许时变个体特定的未观察异质性,具有双重鲁棒性\cite{billmeier2013, smith2015}。

% \subsection{Marginal benefits analysis}\label{sec:model}

% 为了推断结果背后的机制,我们开发了基于边际收入的边际效益分析,以分析制度转变如何导致用水差异。

% (water -dependent production)由于不可替代,假设水是唯一具有两种生产效率类型的生产函数输入。
% (生态成本分配)假设生态是整个流域的单一实体,用水成本平均分配到各省份。

% (多时段设置)有多个设定时段,对未来用水的预期有一个恒定的折现因子。

% 在上述简化假设下,我们展示了三个案例——对应于抽象的SES结构(图~\ref{fig:structure}~C),推理了SES结构如何改变各省决策的预期边际效益和成本。
% 作为对SES结构与制度效应因果关系的可能解释之一,基于上述三个假设的模型推导可以在\textit{ Appendix~\nameref{secS4}}中找到,\textit{ Appendix~\nameref{secS5}}中涉及到一些简单的基于模型的扩展。

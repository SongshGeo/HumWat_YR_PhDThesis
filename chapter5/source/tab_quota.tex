% Table generated by Excel2LaTeX from sheet '八七分水方案配额'
\begin{table}[htbp]
    \caption{八七分水方案水资源配额}
      \begin{tabularx}{\textwidth}{LLLLLLLLLLL}
      \toprule
            & \multicolumn{1}{l}{青海} & \multicolumn{1}{l}{四川$^b$} & \multicolumn{1}{l}{甘肃} & \multicolumn{1}{l}{宁夏} & \multicolumn{1}{l}{内蒙古} & \multicolumn{1}{l}{山西} & \multicolumn{1}{l}{陕西} & \multicolumn{1}{l}{河南} & \multicolumn{1}{l}{山东} & \multicolumn{1}{l}{津冀$^b$} \\
      \midrule
      规划需求  & 35.7  & 0     & 73.5  & 60.5  & 148.9 & 115   & 60.8  & 111.8 & 84    & 6 \\
      1983年方案 & 14    & 0     & 30    & 40    & 62    & 43    & 52    & 58    & 75    & 0 \\
      1987年方案 & 14.1  & 0.4   & 30.4  & 40    & 58.6  & 38    & 43.1  & 55.4  & 70    & 20 \\
      多年平均耗水$^a$ & 12.03 & 0.25  & 25.8  & 36.58 & 61.97 & 21.16 & 11.97 & 34.3  & 77.87 & 5.85 \\
      黄河水在地区总用水的占比 & 48.12\% & 0.10\% & 30.79\% & 58.45\% & 47.82\% & 73.55\% & 44.39\% & 24.77\% & 34.41\% & 3.11\% \\
      \bottomrule
      \end{tabularx}\label{ch5:tab:quota}%
      \footnotesize
      $a$ 使用1987年到2008年数据计算,四川因数据不足,使用2004到2017年数据计算。
      $b$ 由于取自黄河流域的水资源占省(或地区)总用水量的比值太小($< 5\%$),不在本研究中进行考虑。
\end{table}%
